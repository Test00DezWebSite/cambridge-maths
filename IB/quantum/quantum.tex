\documentclass[a4paper]{article}
\usepackage{amsmath,amsfonts,fullpage}

\newcommand{\R}{\mathbb{R}}
\newcommand{\C}{\mathbb{C}}
\newcommand{\RRR}{\mathbb{R}^{3}}
\newcommand{\Z}{\mathbb{Z}}
\newcommand{\eps}{\epsilon_0}
\newcommand{\pd}[2]{\frac{\partial #1}{\partial #2}}
\newcommand{\schrod}{Schr{\"o}dinger }
\newcommand{\abs}[1]{\left| #1 \right|}
\newcommand{\mom}{\mathbf{p}}
\newcommand{\posn}{\mathbf{x}}
\newcommand{\amom}{\mathbf{L}}
\newcommand{\com}[2]{\left[ #1,#2 \right]}

\begin{document}

\title{Quantum Mechanics}
\author{Dr.~N.S.~Manton\thanks{\LaTeX ed by Paul Metcalfe.  Comments and
corrections to \texttt{soc-archim-notes@lists.cam.ac.uk}}}
\date{Mich\ae lmas Term 1996}
\maketitle

\tableofcontents

\section{Introduction}

This section is mandated by the schedules (and was lectured), but it's 
all A-level physics.  So I'll just summarize the salient points 
\emph{extremely} briefly.

\emph{Planck} hypothesised that light of angular frequency $\omega$ exists
in packets of energy $\hbar \omega$.  $\hbar$ is a fundamental constant of
nature, and equals $1.05 \times 10^{-34} Js$.  Einstein's 
\emph{photoelectric effect} confirmed Planck's idea, the crucial formula
being $\hbar \omega = E + W$.  Photons are massless particles, and so move at
$c$.

\emph{The Bohr atom.}  To explain atomic spectra, Bohr suggested that angular 
momentum was quantized in units of $\hbar$, so using
\[
m v R = N \hbar
\] and
\[
\frac{m v^2}{R} = \frac{1}{4\pi \eps}\frac{e^2}{R^2}
\] gives the allowed radii.  In good agreement with experiment, but has two
major flaws.  Firstly, can only be used for hydrogen atom and secondly is 
a load of rubbish.

De Broglie suggested that associated with a particle of momentum $\mathbf{p}$
is a wave with wave vector $\mathbf{k}=\frac{\mathbf{p}}{\hbar}$.

\section{The \schrod Equation}

This is not relativistic and only works for a single particle in a potential
$U(\mathbf{x},t)$.

\[
i\hbar \pd{\psi}{t} = -\frac{\hbar^2}{2m}\nabla^2 \psi + U(\mathbf{x},t)\psi
\]

It is, however, the remainder of this course...

For a free particle with $U\equiv 0$, the solution is 
\[
\psi(\mathbf{x},t) = A e^{i(\mathbf{k}.\mathbf{x} - \omega t)}.
\]

We interpret $\hbar \mathbf{k}$ as the momentum of the particle and
$\hbar\omega$ as the kinetic energy.

Note that
\begin{itemize}
\item $\psi$ is complex,
\item \schrod equation is linear,
\item $\psi$ is called the state of the particle and
\item the \schrod equation governs how the state evolves in time.
\end{itemize}

It is postulated that any solution of the \schrod equation is an 
allowed physical state.

\subsection{Probabilistic Interpretation of $\psi$}

The probability of finding the particle in an infinitesimal region $dV$
centered on $\mathbf{x}$ is postulated to be $\abs{\psi(\mathbf{x},t)}^2 dV$.

$\psi$ must be normalised, i.e.
\[
\int_{\RRR} \abs{\psi(\mathbf{x},t)}^2 dV = 1.
\]

There is still freedom to multiply by $e^{i\alpha}$ -- this has no
physical consequences.  Some wavefunctions are not normalisable, and are
regarded as idealisations of physically realisable ones.

\subsubsection{Probability Flux and Conservation of probability}

If the potential is real, then consider the \schrod equation and it's 
complex conjugate
\[
i\hbar \pd{\psi}{t} = -\frac{\hbar^2}{2m}\nabla^2 \psi + U(\mathbf{x},t)\psi
\]
\[
i\hbar \pd{\psi^*}{t} = -\frac{\hbar^2}{2m}\nabla^2 \psi^* + U(\mathbf{x},t)\psi^*,
\] and calculate $\pd{\psi\psi^*}{t}$ to get
\[
\pd{\psi\psi^*}{t} + \nabla . (-\frac{i\hbar}{2m}(\psi^* \nabla \psi -
\nabla \psi^* \psi)) = 0,
\] an analogue of conservation of charge.
\[\mathbf{J} = -\frac{i\hbar}{2m}(\psi^* \nabla \psi - \nabla \psi^* \psi)
\] is called the probability current.

\subsection{Stationary States}

If $U$ is independent of time, we can separate the \schrod equation to get
\[
\psi(\mathbf{x},t) = \chi(\mathbf{x})e^{-i\omega t},
\] with
\[
H\chi = E\chi,
\] where $E$ is the energy $\hbar \omega$ and $H$ is the Hamiltonian operator
\[
H = -\frac{\hbar^2}{2 m}\nabla^2 + U.
\]

This wavefunction is called a stationary state with energy $E$. $\psi$ is
normalised iff $\chi$ is normalised.  In a stationary state, the probability
density is $\abs{\chi}^2$ and depends only on position.  Since $H$ is real,
$E$ is real.  If $\chi$ is real, then $\mathbf{J}=0$.

The general solution of the \schrod equation with a static potential is a
superposition of stationary states, i.e. suppose $H$ has eigenvalues
$E_1, E_2, \dots$ with corresponding eigenfunctions $\chi_1, \chi_2, \dots$.
Then
\[
\psi(\mathbf{x},t) = \sum_{n=1}^{\infty} a_n\chi_n(\mathbf{x}) e^{-\frac{i E_n t}{\hbar}}.
\]  $\psi$ is normalised iff
\[
\sum_{n=1}^{\infty} \abs{a_n}^2 = 1.
\]

This can all be generalised to non-normalisable states and continuum 
and degenerate eigenvalues.

\subsection{Gaussian Wave Packet}

We superpose plane wave solutions to get a localised particle.

\[
\psi(x,t) = \int_{\R} e^{-\frac{\sigma (k-k_0)^2}{2}} e^{i(k x - 
\frac{\hbar k^2}{2m}t)} dk
\]

This integral ``simplifies'' (honestly!) to give
\[
\psi(x,t) = \frac{const}{\sqrt{\sigma + \frac{i \hbar t}{m}}}
\exp \frac{1}{2}\left( \frac{k_0 t + i x}{\sigma + \frac{i \hbar t}{m}}
\right)^2.
\]

\[
\psi \psi^* = \frac{const'}{\sqrt{\sigma^2 + \frac{\hbar^2 t^2}{m^2}}}
\exp{-\sigma
\frac{\left( x - \frac{\hbar k_0}{m}t
\right)^2}
{\sigma^2 + \frac{\hbar^2 t^2}{m^2}}}, 
\] which is a Gaussian in $x$, centered at $\frac{\hbar k_0}{m}t$.  $\psi$ is
normalised if $const' = \sqrt{\frac{\sigma}{m}}$.

Particle moves along at a speed $\frac{\hbar k_0}{m}=\frac{\text{average momentum}}{\text{mass}}$.  The width of the packet changes with time and is narrowest
at $t=0$.

\subsection{Particle in infinite potential well (1--D)}

This has a potential
\[
U(x)=
\begin{cases}
0       & 0 < x < a; \\
\infty  & \text{otherwise.}
\end{cases}
\]

Look for stationary states, i.e. solutions of
\[
\frac{-\hbar^2}{2m}\pd{^2\chi}{x^2}=E\chi ,
\]
with $\chi(0)=\chi(a)=0$ since $\chi=0$ where $U$ is infinite.  Not too hard to
get
\[
\chi_n(x)=\sqrt{\frac{2}{a}}\sin \frac{n \pi x}{a}
\]
with
\[
E_n = \frac{\hbar^2}{2m}\frac{n^2 \pi^2}{a^2}.
\]

\subsection{Remarks on bound states}

The stationary state equation is

\[
-\frac{\hbar^2}{2m}\pd{^2\chi}{x^2} + U(x)\chi = E\chi,
\]
with $U \rightarrow U_0$ as $x \rightarrow \pm \infty$.  This equation
has 2 linearly independent solutions.  If $E > U_0$, both of this are
oscillatory as $x \rightarrow \pm \infty$ and there is no bound state.
If $E < U_0$, one soln decays exponentially one way and grows exponentially
the other way.  For special $E$, the eigenvalues, this doesn't happen and
there is one solution which decays exponentially as $x \rightarrow \pm \infty$.
This solution is unique up to normalisation.  The other solution is not
normalisable.  The eigenvalues correspond to bound states with $E<U_0$.

\subsection{Remarks on continuity}

If $U$ is smooth, so is $\chi$.  If $U$ has a finite discontinuity, then $\chi$
and $\chi'$ are continuous, but $\chi''$ is discontinuous.  If $U$ has an
infinite discontinuity, usually $\chi$ is continuous, $\chi'$ discontinuous
and $\chi = 0$ where $U$ infinite.

\subsection{Remarks on parity}

If $U(x)=U(-x)$, then bound states are either even or odd, since if $\chi(x)$
is a solution with energy $E$, so is $\chi(-x)$ and hence is a multiple of
$\chi(x)$.

\subsection{Finite Potential Well (1D)}

\[
U(x) =
\begin{cases}
0       & -a < x < a \\
U_0     & \text{otherwise.}
\end{cases}
\]

Look for even parity bound states.  Let $k=\sqrt{\frac{2 m E}{\hbar^2}}$ and
$\kappa = \sqrt{\frac{2 m (U_0 - E)}{\hbar^2}}$.  Then we have to solve
\[
\pd{^2\chi}{x^2} + k^2 \chi = 0 \text{ if } \abs{x} < a,
\]
and
\[
\pd{^2\chi}{x^2} - \kappa^2 \chi = 0 \text{ if } \abs{x}>a,
\]
with $\chi$ and $\chi'$ continuous at $x=\pm a$.  Easily obtain
\[
\chi(x)=
\begin{cases}
A \cos k x      & \abs{x} < a; \\
B e^{-\kappa \abs{x}}   & \abs{x} > a.
\end{cases}
\]

Now impose continuity on $\frac{1}{\chi}\pd{\chi}{x}$ at $x=\pm a$ to get
\[
k \tan k a = \kappa.
\]

We also have
\[
k^2 + \kappa^2 = \frac{2 m U_0}{\hbar^2}.
\]
Solve numerically or graphically.  Number of solutions increases with $k a$
and there is always one solution is $U_0 > 0$.  The energy is
\[
E = \frac{\hbar^2 k^2}{2 m}.
\]

\subsection{Scattering problems}

Ideally, we would like to study the scattering of wavepackets on some
potential.  The evolution of such scattering is complicated, and finding the
probabilities for reflection and transmission is not nice...  So we look for
stationary states.

\subsubsection{Potential step}

\[
U(x) =
\begin{cases}
0 & x < 0; \\
U_0 &   x > 0.
\end{cases}
\]

If $E<U_0$, let $k=\sqrt{\frac{2 m E}{\hbar^2}}$ and 
$\kappa = \sqrt{\frac{2 m (U_0 - E)}{\hbar^2}}$.  Solve appropriate ode's
and impose continuity at $x=0$ to get
\[
\chi(x) =
\begin{cases}
e^{i k x} + A e^{-i k x}        & x < 0, \\
B e^{-\kappa x} & x > 0.
\end{cases}
\]
with $A = \frac{k-i\kappa}{k+i\kappa}$ and $B = \frac{2 k}{k + i \kappa}$.
$\abs{A}^2$ is the probability of reflection, and equals 1.

If $E>U_0$, let $k=\sqrt{\frac{2 m E}{\hbar^2}}$ and 
$\kappa = \sqrt{\frac{2 m (E-U_0)}{\hbar^2}}$.  Solve appropriate ode's
and impose continuity at $x=0$ to get
\[
\chi(x) =
\begin{cases}
e^{i k x} + A e^{-i k x}        & x < 0, \\
B e^{i \kappa x}        & x > 0.
\end{cases}
\]
with $A = \frac{k-\kappa}{k+\kappa}$ and $B = \frac{2 k}{k + \kappa}$.
$\abs{A}^2$ is the probability of reflection, and if $E \gg U_0$,
$\abs{A}^2 \approx 0$.  This is the classical limit.

$\abs{B}^2$ is the probability of finding a particle in $x>0$.  The decrease
in speed bunches up the particles and we need to compensate for this.  The
transition probability is $\frac{\kappa}{k}\abs{B}^2$, which takes account
of the change in momentum.

\subsubsection{Quantum Tunneling}

\[
U(x) = 
\begin{cases}
U_0     & 0<x<a; \\
0       & \text{otherwise.}
\end{cases}
\]

Suppose $E<U_0$ and let $k=\sqrt{\frac{2 m E}{\hbar^2}}$ and 
$\kappa = \sqrt{\frac{2 m (U_0-E)}{\hbar^2}}$.  Get

\[
\chi(x) =
\begin{cases}
e^{i k x} + A e^{-i k x}        & x < 0; \\
B e^{- \kappa x} + C e^{\kappa x}       & 0 \le x \le a; \\
D e^{i k x}     & x > a.
\end{cases}
\]

Impose continuity and do painful algebra to get...
\[
D = \frac{-4 i k \kappa}{(\kappa - i k)^2 e^{(i k + \kappa)a}-(\kappa + i k)^2
e^{(i k - \kappa) a}}
\]
and
\begin{align*}
\text{probability of tunneling} &= \abs{D}^2 \\
& = \frac{4 k^2 \kappa^2}{(k^2+\kappa^2)^2 \sinh^2 \kappa a +4 k^2 \kappa^2},
\end{align*}
which although exponentially small if $k a$ is large, is non-zero.  Tunneling
probabilities in real systems exhibit an enormous range, e.g. the half-lives
for $\alpha$ decay range from $10^{-8}$ seconds to $10^{10}$ years.

\subsection{The Quantum Harmonic Oscillator}

\[
U(x)=\frac{1}{2}m \omega^2 x^2
\]

Solve
\[
\frac{-\hbar^2}{2 m} \pd{^2 \chi}{x^2} + \frac{1}{2} m \omega^2 x^2 \chi = E\chi.
\]

Put $\xi = \sqrt{\frac{m \omega}{\hbar}}x$ and $\epsilon = \frac{2 E}{\hbar \omega}$.  Now must solve
\[
-\pd{^2\chi}{\xi^2} + \xi^2 \chi = \epsilon \chi.
\]

Try $\chi(\xi) = f(\xi)e^{-\frac{1}{2}\xi^2}$.  Solve for $f$ using power
series to get
\[
f(\xi) = \sum_{n=0}^{\infty}a_n\xi^n
\]
and
\[
a_{n+2}=\frac{2 n + 1 - \epsilon}{(n+2)(n+1)}a_n.
\]
A non-terminating series is unacceptable, since then $\chi \sim e^{\frac{1}{2}\xi^2}$.  So series terminates, with implies $\epsilon$ is odd. $f_N(\xi)$
is the $N^{th}$ Hermite polynomial $H_N(\xi)$.  $E_N = (N + \frac{1}{2})\hbar \omega$.  $\frac{\hbar \omega}{2}$ is called the zero-point energy of the
oscillator.

\begin{center}
\begin{tabular}{ c | c | c}
$N$     & $E_N$ & $\chi_N(\xi)$ \\ \hline \hline
0       & $\frac{1}{2} \hbar \omega$    & $e^{-\frac{1}{2}\xi^2}$ \\
1       & $\frac{3}{2} \hbar \omega$    & $\xi e^{-\frac{1}{2}\xi^2}$ \\
2       & $\frac{5}{2} \hbar \omega$    & $(1-2 \xi^2)e^{-\frac{1}{2}\xi^2}$ \\
3       & $\frac{7}{2} \hbar \omega$    & $(\xi - \frac{2}{3}\xi^3)e^{-\frac{1}{2}\xi^2}$ \\
\end{tabular}
\end{center}

\section{Observables and Operators}

In quantum mechanics, physical numbers such as position, velocity etc. are
represented by operators.  These are chosen such that a state with a definite
value for the quantity is an eigenfunction of the operator, with the value
being the eigenvalue of the operator.

The operators look the same as their values.

Energy is represented by the Hamiltonian
\[
H = \frac{-\hbar^2}{2 m} \nabla^2 + U.
\]

Momentum is represented by 
\[
\mom = - i \hbar \mathbf{\nabla}.
\]

Note that 
\[
H = \frac{1}{2 m}\mom . \mom + U(\posn),
\]
which is reassuring.

The position operators $x_1, x_2, x_3$ act by multiplication, so $f$ is an
eigenfunction of $\posn$ if
\[
\posn f(x) = \mathbf{X} f(x)
\]
for constant $\mathbf{X}$.  $f$ must be a delta function, and
$\delta(x_1 - X_1)\delta(x_2 - X_2) \delta(x_3 - X_3)$ is an eigenfunction of
$\posn$ with eigenvalue $\mathbf{X}$.

Angular momentum is defined as expected
\begin{align*}
\amom &= \posn \wedge \mom \\
&=-i \hbar \posn \wedge \mathbf{\nabla}.
\end{align*}

In general, a state is not an eigenfunction.

\subsection{Canonical Commutation Relations}

Operators do not necessarily commute.

The \emph{commutator} of $O_1$ and $O_2$ is $O_1 O_2 - O_2 O_1$ and is written
$\com{O_1}{O_2}$.  $O_1$ and $O_2$ commute iff $\com{O_1}{O_2}=0$.

\begin{itemize}
\item $\com{x_i}{x_j} = 0$
\item $\com{p_i}{p_j} = 0$
\item $\com{x_i}{p_j} = i \hbar \delta_{ij} 1$.
\end{itemize}

These are easily proven.

\subsection{Hermitian Operators}

An operator $O$ is Hermitian if

\[
\int_{\RRR} \mathbf{v}^{*}(\posn) (O\mathbf{u})(\posn)d^3x =
\int_{\RRR} ( (O\mathbf{v})(\posn))^* \mathbf{u}(\posn)d^3x.
\]
for all $\mathbf{u}, \mathbf{v}$ decaying at infinity.

All the operators we have seen so far are Hermitian (easily proven).

\subsubsection{Eigenvalues and Eigenfunctions of Hermitian operators}

Suppose $O$ is a Hermitian operator, with eigenvalues $\lambda_n$ and
normalised eigenfunctions $U_n(\posn)$.  Then

\begin{align*}
\lambda_m \int y_n^* y_m &= \int y_n^* O y_m \\
&= \lambda_n^* \int y_m^* y_n.
\end{align*}

Putting $m=n$ gives $\lambda_n$ real, and $m \ne n$ gives that
\[
\int y_n y_m^* = 0.
\]

Any decaying $U(\posn)$ can be written as a linear combination of the $U_n$.

It is postulated that :-

\begin{itemize}
\item Each dynamical variable is represented by a Hermitian operator,
\item If the normalised wavefunction at a given time is
$\psi(\posn) = \sum a_n U_n(\posn)$, with the $U_n$ normalised eigenfunctions,
then the probability that the particle is in state $n$ is $\abs{a_n}^2$.
\item If an experiment is carried out and the particle is found to be in
state $n$, then immediately afterwards $\psi(\posn) = U_n(\posn)$.  Further
evolution of the system is governed by the \schrod equation.
\end{itemize}

We can simultaneously measure observables whose operators commute, since
such operators have a complete set of simultaneous eigenfunctions.

\subsection{Expection}

A measurement of $O$ gives outcome $\lambda_n$ with probability $\abs{a_n}^2$.
The expectation of $O$, $\langle O \rangle = \sum \lambda_n \abs{a_n}^2$.

It is easy to prove that $\langle O \rangle = \int \psi^* O\psi$.

In a stationary state, $\langle \posn \rangle$ and $\langle \mom \rangle$ are
constant.  

It is easy to see that $\langle E \rangle$ is independent of time, and this
can be interpreted as conservation of energy.

\subsection{Uncertainty Principle}

The uncertainty of the observable $O$ is $\Delta O$, where $(\Delta O)^2= \langle (O - \langle O \rangle )^2 \rangle$.  There are wavefunctions which have
$\Delta O  = 0$.  Given two observables with commuting operators, we can make
$\Delta O_1$ and $\Delta O_2$ arbitrarily small by choosing simultaneous
eigenfunctions.  We have an inequality
\[
(\Delta O_1)^2 (\Delta O_2)^2 \ge \frac{1}{4} \langle i \com{O_1}{O_2} \rangle^2,
\]
which gives the Heisenberg Uncertainty principle
\[
\Delta x \Delta p \ge \frac{1}{2} \hbar.
\]

The Gaussian achieves equality.

\section{\schrod Equation in Three Dimensions}

Assume a spherically symmetric potential $U(r)$ and look at the spherically
symmetric stationary states $\chi(r)$.  Get

\[
\frac{-\hbar^2}{2 m r} \pd{^2 r \chi}{r^2} + U(r)\chi = E\chi.
\]

with $\chi(0)$ finite and $\lim_{r \rightarrow 0} \chi = 0$ for a bound state.
Let $\sigma(r) = r\chi(r)$ to get something familiar...
 \[
\frac{-\hbar^2}{2 m} \pd{^2 \sigma}{r^2} + U(r)\sigma = E\sigma,
\]
$\sigma(0)=0$, which gives the 1-D \schrod equation on the whole line, with
a reflection symmetric potential and an odd parity $\sigma$.

The spherical harmonic oscillator ($U(r) = \frac{1}{2}m \omega^2 r^2$) and
potential well ($U(r) = 0 (r < a), U_0 (r>a)$) follow through, with the proviso
that no bound states need exist for the potential well.

\subsection{Spherically symmetric bound states for Hydrogen atom}

Let $U(r) = \frac{-\beta}{r}$, with $\beta = \frac{e^2}{4 \pi \eps}$ and
solve
\[
\frac{-\hbar^2}{2 m_e}(\pd{^2\chi}{r^2}+\frac{2}{r}\pd{\chi}{r})
- \frac{\beta \chi}{r} = E \chi.
\]
Let $\nu^2 = \frac{-2 m_e E}{\hbar^2}$ and $\alpha = - \frac{2 m_e \beta}{\hbar^2}$.

\[
\pd{^2 \chi}{r^2} + \frac{2}{r}\pd{\chi}{r} + \frac{\alpha}{r} \chi - 
\nu^2 \chi = 0.
\]

Asymptotically, $\chi \sim e^{-\nu r}$, so put $\chi(r) = f(r) e^{-\nu r}$ and
try a power series solution for $f$ to get
\[
\frac{a_n}{a_{n-1}} = \frac{2 \nu n - \alpha}{n(n+1)}.
\]
The series must terminate, otherwise $\chi \sim e^{\nu r}$, so $\alpha = 2 \nu n$ for some $n$.
\[
E_N = \frac{-m_e \beta^2}{2 N^2 \hbar^2} =
\frac{-m_e e^4}{32 \pi^2 \eps^2 \hbar^2} \frac{1}{N^2},
\]
as in the Bohr model.

$\chi_N = L_N(\nu r)e^{- \nu r}$, where $L_N$ is the $N^{th}$ Laguerre polynomial.  For normalisation,
\[
4 \pi \int_0^\infty \chi^2(r) r^2 dr = 1.
\]

The Bohr radius is $R_0 = \frac{2}{\alpha} = \frac{\hbar^2}{m_e \beta}$.  In
the $N^{th}$ state, $\langle r \rangle = \frac{3}{2}N^2 R_0$.  Get spectral
lines etc..

\subsection{Angular Momentum Operators}

\[
\amom = \posn \wedge \mom
\]

\[
L_j = i \hbar \epsilon_{jkl}x_k \pd{}{x_l}
\]

It can be shown by expanding out that

\[
\com{L_j}{L_k} = i \hbar \epsilon_{jkl}L_l.
\]

Define $\amom^2 = L_1^2 + L_2^2 + L_3^2$.  Now $\amom^2$ commutes with all
the $L_i$ (use $\com{A}{B^2}=\com{A}{B}B + B\com{A}{B}$).

Now
\[
\com{L_i}{x_j} = i \hbar \epsilon_{ijk}x_k,
\]
\[
\com{L_i}{p_j} = i \hbar \epsilon_{ijk}p_k
\]
and
\[
\com{L_i}{U(r)} = 0 \text{ for spherically symmetric U.}
\]

It is now easy to show that $\com{L_i}{\nabla^2} = 0$ and so $\com{L_i}{H} = 0$
and $\com{L^2}{H} = 0$.  Thus $H, \amom^2 \text{ and } L_3$ commute if the
potential is spherically symmetric.  $L_3$ is the conventional choice of
component.  Thus the simultaneous eigenfunctions of $H, \amom^2 
\text{ and } L_3$ form a complete set of functions.

\subsection{Eigenfunctions of $\amom^2$ and $L_3$}

For $L_3$,
\[
-i \hbar \pd{f}{\phi} = \lambda f
\]
gives 
\[
f(\phi) = e^{i m \phi}.
\]
Since the eigenfunction should be unchanged by $\phi \rightarrow \phi + 2 \pi$,
$m \in \Z$.  The possible values of $L_3$ are $\text{an integer }\times \hbar$,
which looks like Bohr's hypothesis.  The simultaneous eigenfunctions of $L_3$
and $\amom^2$ are the spherical harmonics, which, un-normalised, are
\[
Y_{l,m}(\theta,\phi) = P_{l,m}(\theta)e^{i m \phi}, l \ge 0 \text{ and }
-l \le m \le l.
\]

$P_{l,m}$ is an associated Legendre function
$P_{l,m}(\theta) = (\sin \theta)^{\abs{m}}
\pd{^{\abs{m}} P_l \left(\cos \theta \right)}{\theta^{\abs{m}}}
$ and $P_l$ is the 
$l^{th}$ Legendre polynomial.

Or, $P_{l,m}$ is the solution to
\[
\left(
\frac{-1}{\sin \theta}
\pd{}{\theta}
\left(
\sin{\theta}
\pd{}{\theta}
\right)
- \frac{m^2}{\sin^2 \theta}
\right)
P_{l,m}(\theta)
= l(l+1)P_{l,m}(\theta).
\]

The eigenvalues are $L_3 Y_{l,m} = \hbar m Y_{l,m}$ and $\amom^2 Y_{l,m}=\hbar^2 l(l+1)$.

\subsection{\schrod Equation with spherically symmetric potential}

\[
\frac{-\hbar^2}{2 M}
\left(
\pd{^2 \chi}{r^2} + \frac{2}{r}\pd{\chi}{r}
+ \frac{1}{r^2}
\left(
\frac{1}{\sin \theta} \pd{}{\theta}
\left(
\sin \theta \pd{\chi}{\theta}
\right)
+\frac{1}{\sin^2 \theta}\pd{^2 \chi}{\phi^2}
\right)
\right)
+U(r) \chi = E \chi
\]

Separate variables to put $\chi(r,\theta,\phi) = g(r) Y_{l,m}(\theta,\phi)$.
This simplifies the above equation to
\[
\frac{-\hbar^2}{2 M}
\left(
\pd{^2 g}{r^2} + \frac{2}{r} \pd{g}{r}
\right)
+ \left(
U(r) + \frac{\hbar^2}{2 M} \frac{l(l+1)}{r^2}
\right)g = E g.
\]

There is an effective potential of $U(r) + \frac{\hbar^2}{2 M}\frac{l(l+1)}{r^2}$, giving a centrifugal repulsion.  We seek bound states with $E < 0$.

Put $\alpha = \frac{2 m_e \beta}{\hbar^2}$ and $\nu^2 = \frac{-2 m_e E}{\hbar^2}$.  Asymptotically $g(r)=e^{-\nu r}$, so try $g(r) = e^{- \nu r}f(r)$.  The
equation now simplifies to
\[
\pd{^2f}{r^2} + 
\left(
\frac{2}{r} - 2 \nu
\right)
\pd{f}{r}
-\frac{l(l+1)}{r^2}f
+ \frac{1}{r}
\left(
\alpha - 2 \nu
\right)
f = 0.
\]
Try a series solution $f(r) = \sum_{n=0}^{\infty}a_n r^{n+\sigma}$.  The
indicial equation gives $\sigma = l$, and the rest reduces to
\[
\frac{a_n}{a_{n-1}}=\frac{(n+l)2\nu - \alpha}{n(n+2l+1)}
\].  This must terminate, so $\nu = \frac{\alpha}{2(n+l)}$, or
$\alpha = 2\nu N$.  $E_N = \frac{-m_e e^4}{32 \pi^2 \eps^2 \hbar^2}\frac{1}{N^2}$ is familiar.  The complete unnormalised wavefunction
\[
\chi_{N,l,m}(r,\theta,\phi) = r^l L_{N,l}(\frac{\alpha r}{2 N})
e^{\frac{-\alpha r}{2 N}} Y_{l,m}(\theta,\phi).
\]
$L_{N,l}$ is a generalised Laguerre polynomial.

\subsection{Energy Levels}

Energy only depends on $N$ (the principal quantum number).
\[
E_N = \frac{-m_e e^4}{32 \pi^2 \eps^2 \hbar^2}\frac{1}{N^2}
\]

The allowed values for $l, m$ for given $N$ are $0 \le l < N$ and $-l \le m
\le l$.  The total degeneracy of $E_N$ is $\sum_{l=0}^{N-1}(2l+1) = N^2$.
This large degeneracy is a feature of the Coulomb potential.

$N$ is called the principal quantum number, $l$ is the total angular momentum
quantum number and $m$ is the magnetic quantum number (a magnetic field along
$3^{rd}$ axis removes this degeneracy.

\subsection{Relation with Bohr orbit}

The Bohr picture emerges if $N$ is large, $l \approx N$.  If $m=l \approx N$
then the electron has an angular momentum component about the $3^{rd}$ axis
$\approx \hbar N$ and the total angular momentum $\approx \hbar N$.

Consider the radial part of the wavefunction with $l = N-1$, $g(r) = r^l
e^{\frac{-\alpha r}{2 N}}$.  The radial probability density is proportional
to $r^2 g^2(r) \approx r^{2 N} e^{-\alpha r}{N}$, which has a maximum at
$r = \frac{2 N^2}{\alpha} = N^2 R_0$, agreeing with the Bohr model.

{\noindent \hrulefill}

The syllabus ends here.  The lecturer went on to discuss atoms with more 
than one electron.  It was covered at approximately A-level Chemistry standard
(although considerably quicker).

\end{document}

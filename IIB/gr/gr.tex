\documentclass{notes}

\usepackage{varioref}

\newcommand{\dif}[1]{\frac{\ud\phantom{#1}}{\ud#1}}
\newcommand{\absd}[1]{\frac{\uD\phantom{#1}}{\ud#1}}
\newcommand{\absdf}[2]{\frac{\uD#1}{\ud#2}}
\newcommand{\ch}[2]{\genfrac{\{}{\}}{0pt}{}{#1}{#2}}
\newcommand{\del}{\nabla}
\renewcommand{\v}{\vect{v}}
\renewcommand{\Box}{\square}
\newcommand{\cO}{\mathcal{O}}
\newcommand{\cL}{\mathcal{L}}

\begin{document}

\frontmatter

\title{General Relativity}

\lecturer{Dr.~P.~D.~D'Eath}
\maintainer{Paul Metcalfe}
\date{Lent 1998} \maketitle

\thispagestyle{empty}

\noindent\verb$Revision: 2.9 $\hfill\\
\noindent\verb$Date: 2001/11/01 14:49:47 $\hfill

\vspace{1.5in}

The following people have maintained these notes.

\begin{center}
\begin{tabular}{ r  l}
-- date & Paul Metcalfe
\end{tabular}
\end{center}

\tableofcontents

\chapter{Introduction}

These notes are based on the course ``General Relativity'' given by
Dr.~P.~D.~D'Eath in Cambridge in the Lent Term 1998.  These typeset
notes are totally unconnected with Dr.~D'Eath.  The recommended books
for this course are discussed in the bibliography.

\alsoavailable
\archimcopyright

\mainmatter

\chapter{Outline of the theory}

\section{Curved spaces}

Consider a two dimensional curved surface in Euclidean $\R^3$, for
instance with the defining equation $z = z(x,y)$.  We distinguish
between the extrinsic and intrinsic properties of such a surface.

The extrinsic properties describe the relation between the surface and the
surrounding 3 dimensional space, for instance the extrinsic
curvature if $z = z(x,y)$ is not a plane.

The intrinsic properties refer to quantities such as distance, angle
and area measured within the surface.  For instance, the
Euclidean metric $\ud s^2 = \ud x^2 + \ud y^2 + \ud z^2$ gives
the distance between nearby points in the surface as
\[
\ud s^2 = \ud x^2 + \ud y^2 + \left( \pd{z}{x} \ud x + \pd{z}{y} \ud y
\right)^2.
\]

The surface has an intrinsic Riemannian (positive definite) metric of the form
\[
\ud s^2 = A(x,y)\, \ud x^2 + 2 B(x,y)\, \ud x \ud y + C(x,y)\, \ud y^2.
\]

The metric at a point $P$ on the surface describes the geometry (distances,
angles, etc.) on the plane tangent to the surface at $P$.

We have the freedom to change co-ordinates.  If $x,y \mapsto
x'(x,y), y'(x,y)$ then the metric becomes
\[
\ud s^2 = A'(x',y')\, \ud x'{}^2 + 2 B'(x',y')\, \ud x' \ud y' + C'(x',y')\,
\ud y'{}^2,
\]

where $A'$, $B'$ and $C'$ can be calculated.  The geometry of the
surface is the same however the co-ordinate lines are painted on ---
as an example take $\R^2$ with the metric ${\ud s^2 = \ud x^2 + \ud y^2
= \ud r^2 + r^2\, \ud \theta^2}$.

The cylinder in $\R^3$, $x^2 + y^2 = R^2$ has the intrinsic
metric $\ud s^2 = \ud z^2 + R^2\, \ud \phi^2$ (using cylindrical polars).
We can locally make the co-ordinate change $x = z$, $y = R \phi$ and
we get the flat space metric.  The intrinsic geometry of the cylinder is
that of a plane, although the cylinder has extrinsic curvature.

We can do the same sort of thing with the 2-sphere of radius $a$ in
$\R^3$. This has the intrinsic metric $\ud s^2 = a^2 \left( \ud
  \theta^2 + \sin^2 \theta\, \ud \phi^2 \right)$ (where $\theta$ and
$\phi$ are the usual spherical polars).  Letting $r = a \theta$ we get
the metric
\begin{align*}
\ud s^2 &= \ud r^2 + \sin^2 \frac{r}{a}\ \ud \phi^2 \\
& \approx \ud r^2 + \left( r^2 - \frac{r^4}{3 a^2} + \dots \right)\, \ud \phi^2
\quad \text{near the North pole.}
\end{align*}

These extra terms are the effects of the intrinsic curvature of the sphere,
which is $K = a^{-2}$.

The circumference of the circle at constant $r$ is
\[
C = 2 \pi \left(r - \frac{r^3}{6 a^2} + \dots \right) = \oint \ud s,
\]

and the area within the circle is
\[
A = \pi r^2 - \frac{\pi r^4}{12 a^2} + \dots = \int C \ud r.
\]

We note that
\begin{equation}\label{eq:curv}
K = \lim_{r \to 0} \frac{3}{\pi} \frac{2 \pi r - C}{r^3}
= \lim_{r \to 0} \frac{12}{\pi} \frac{\pi r^2 - A}{r^4}.
\end{equation}

\subsection{Geodesics}

These are the generalisation of straight lines in a flat space.  If
points are not too far apart we can find the geodesic $\gamma$ by
extremizing the length
$\int_\gamma \ud s$.

Geodesics are intrinsic to the surface (they depend on the metric).
As an example, great circles on the sphere are geodesics.

We can find the intrinsic curvature of any surface at a point $P$ by
drawing all geodesics from $P$ out to a distance $r$.  We evaluate
the circumference $C(r)$ and thus area $A(r)$ and use \eqref{eq:curv}
to define the curvature $K$ at that point.  $K$ can be negative
(for instance at a saddle).

Let $\xi(r)$ (the geodesic deviation) be the distance between the ends
of two nearby geodesics of length $r$ from a point $P$.  On $S^2$,
$\xi(r) = a \sin \tfrac{r}{a} \delta \phi$ and we note that $\xi$
satisfies the \emph{geodesic deviation equation}:

\[
\diff{^2 \xi}{r^2} = - K \xi.
\]

A similar equation holds in any curved space.  Thus if $K > 0$ two
neighbouring geodesics recross (eventually), if $K = 0$ the geodesics
are straight lines and if $K < 0$ the geodesics separate exponentially.

\section{The principle of equivalence}

\subsection{Uniqueness of free fall}

Consider the (Newtonian) dynamics of a single particle under gravity,
\[
\mathcal{M} \ddot{\x} = - M \nabla \phi, 
\]

where $\mathcal{M}$, the inertial mass, equals $M$, the passive
gravitational mass.  Motion under gravity is independent of mass
and composition.

If a gravitational field $\vect{g} = - \nabla \phi$ is constant in
space and time then all particles have a constant acceleration
$\vect{a} = \vect{g}$ superimposed on the gravity-free motion, $\x =
\x' + \tfrac{1}{2} \vect{a} t^2$, where $\x'$ could be regarded as the
position in an inertial frame with no gravitational field.
Conversely, uniform acceleration $\vect{a}$ applied to the
co-ordinates gives the illusion of a uniform gravitational field
$\vect{a}$.  Uniform gravitational fields are ``fictitious'' --- they can
be eliminated by a change of co-ordinates.

In \emph{any} gravitational field, if an observer falls freely in
a non-rotating laboratory, he%
\footnote{More properly, (s)he.  Sex will be assigned at random.}
sees objects in the laboratory moving essentially on straight lines
--- the local gravitational field has been eliminated.

A freely falling non-rotating laboratory provides a \emph{local
  inertial frame} allowing inertial co-ordinates $(\x,t)$ to be set up
near the laboratory.

There are limitations on local inertial frames.  Nearby particles
at $\x$ and $\x + \boldsymbol{\xi}$ have relative tidal acceleration
\[
\diff{^2 \xi^i}{t^2} = - \phi_{,ij} \xi^j.
\]

In a ``true'' non-uniform gravitational field tidal forces cannot
be eliminated by co-ordinate transforms and there are many different local
inertial frames with relative accelerations.

\subsection{Equivalence principle}

\begin{center}
\parbox{4in}{\itshape
All local inertial frames are equivalent for the performance
of all experiments.  All non-gravitational laws of physics take
their special-relativistic forms in local inertial frames (by the
usual arguments of special relativity).
}
\end{center}

We can thus do things like fluid dynamics, quantum mechanics
and electromagnetic theory in a gravitational field by using the
special-relativistic laws and local inertial frames.

The speed of light is therefore $c$ and distances and times are
measured by the Minkowski metric $\ud s^2 = \ud x^2 + \ud y^2 + \ud z^2 -
c^2 \ud t^2$.

\subsection{Consequences for light propagation}

The obvious consequence is that light can be deflected by gravitational
fields (just like ordinary matter) because light moves in straight lines
in local inertial frames which accelerate with respect to global
co-ordinates.

There is also a gravitational frequency shift.  Consider a lift of height
$h$ accelerating downwards at a rate $g$ with respect to the earth.
The lift has speed $0$ at $t=0$ and a light ray of frequency $\nu$
is emitted from the base of the lift at $t = 0$.  At $t = h c^{-1}$ 
the light ray is at the top of the lift and has an observed frequency
$\nu$ in the lift frame (by the equivalence principle).  The lift
then has speed $\tfrac{gh}{c}$ and so the light has
a Doppler shifted frequency $\nu \left(1 - \frac{gh}{c^2} \right)$
measured from the earth frame.  Note that

\begin{equation}\label{eq:gravtim}
\frac{\ud \nu}{\nu} = - \frac{\ud \phi}{c^2}.
\end{equation}

The same thing happens for light emitted in the other direction.%
\footnote{This aspect of the equivalence principle is confirmed
by the Pound-Rebka experiment.}

We can integrate \eqref{eq:gravtim} to find

\[
\frac{\nu}{\nu_0} = \exp \left(\frac{\phi_0 - \phi}{c^2}\right)
\]

for a photon emitted at $P_0$ with a frequency $\nu_0$ and observed
with a frequency $\nu$ at $P$.

We expect that clocks in a potential well will appear to go slow
(gravitational time dilation).  This is observed for spectral lines
in some white dwarf stars, but is not a big effect.

\subsection{Special relativity and gravitation}

Can we fix up special relativity so that it holds over an extended
region containing gravitational fields?

Gravitational time dilation implies that a good clock at rest
measures a time $t_m = t_c \exp \frac{\phi}{c^2}$,
where $(x_c, y_c, z_c,t_c)$ are special relativistic co-ordinates.
We can only make the theory Lorentz invariant if all measurements
obey
\[
\ud s_m^2 = \exp \left(\frac{\phi}{c^2}\right)\, \ud s^2_c.
\]

This is completely equivalent to a special type of curved spacetime
theory (\emph{not} GR) with metric
\[
\ud s^2 = \exp \left(\frac{\phi}{c^2}\right)\, \left(\ud x^2 + \ud y^2 +
\ud z^2 - c^2 \ud t^2\right),
\]

and this is a more natural viewpoint.  We see that attempts at
combining special relativity and Newtonian gravity lead naturally
to curved spacetimes.

\section{Outline of general relativity}

Our arguments have led us to a curved spacetime with four
co-ordinates $x^a$ ($a = 1 \dots 4$) and a metric
$\ud s^2 = g_{a b}\, \ud x^a\, \ud x^b$, where
$g_{a b} = g_{b a}$ depends on position.

At any event $P$ in spacetime one can find a local inertial frame ---
one can make a co-ordinate change such that
\[
g_{a b} = \eta_{a b}
= \begin{pmatrix}
1 & 0 & 0 & 0 \\
0 & 1 & 0 & 0 \\
0 & 0 & 1 & 0 \\
0 & 0 & 0 & -c^2
\end{pmatrix}.
\]

$\eta_{a b}$ is the \emph{Minkowski metric} (here defined oppositely
to Electrodynamics).

The metric $g_{a b}$ has the canonical form $+ + + -$ at each point.

Recall that intervals with $\ud s^2 < 0$ are \emph{timelike},
intervals with $\ud s^2 > 0$ are \emph{spacelike} and intervals with
$\ud s^2 = 0$ are \emph{null}.

Ordinary massive bodies move on timelike paths $x^a(\lambda)$ with
$\diff{x^a}{\lambda}$ timelike and light travels on null paths
with $\diff{x^a}{\lambda}$ null.

We also have the \emph{clocks hypothesis}: the metric determines
the time measured by a standard clock moving on a timelike path
$x^a(\lambda)$ from event $A$ to event $B$:

\[
\tau = \frac{1}{c} \int_A^B \abs{\ud s}.
\]

The \emph{length hypothesis} is that standard rods measure the length
$\ud s$.

As for the equations of motion, we know that particles move on essentially
straight lines in local inertial frames and this suggests the \emph{geodesic
hypothesis}, that freely falling particles move on geodesics.  The
path of a particle moving from $A$ to $B$ extremizes
\[
\int_A^B \abs{\ud s}.
\]

We guess that massive particles move on timelike geodesiscs and
that massless particles move on null geodesics.

\section{Static spacetime and Newtonian gravity}

We compare general relativity with Newtonian gravity for low
speeds and weak fields.

\subsection{Static metrics}

Consider a time independent gravitational field produced
by a system of bodies at rest.  We have co-ordinates
$x^a = (x^i,t)$ ($i =1,2,3$) and the metric tensor
$g_{a b}$ is a function of the $x^i$ alone.  This is a stationary metric,
and allows us to synchronize clocks.

\begin{flushright}
\parbox{3in}{
Observer 1 at $x_1^i$ bounces light rays off observer 2 at $x_2^i$.
Since the metric is independent of time the photon paths
$x^i(\lambda)$ in space, and hence the
time $t$ elapsed must be the same for each bounce.

Observer 1 sees the photon returning regularly after a proper time
interval $\tau_1$ and observer 2 sees the photon returning regularly
after a proper time interval $\tau_2$.  Thus observer 2 can measure
time by defining $t_2 = \tfrac{k_1}{k_2} \tau_2 + \text{const}$.
Observers at all points in space can do the same thing --- ensure that
time passes at the same rate.
}
\end{flushright}

However, can we actually synchronize the origin of time for different
observers?  We can do this (as shown) if the metric is \emph{static}:

\[
\ud s^2 = g_{ij}(x^k)\, \ud x^i\, \ud x^j - A(x^k)\, \ud t^2.
\]

In this case the metric is symmetric under time reversal and so the
time reverse of a photon path is also a photon path.

If the field is produced by matter at rest then the matter distribution
is invariant under time reversal and so the metric should also be symmetric
--- i.e. static.

Stationary metrics are produced by steadily moving distributions of
matter --- for example rotating stars.

From gravitational time dilation we see that $A = c^2 \exp \left(
\tfrac{2 \phi}{c^2} \right)$.

\subsection{Newtonian limit}

Take a static metric and a weak field, that is $\abs{\tfrac{\phi}{c^2}}$
small.  Then $A \approx c^2 + 2 \phi$ and we expect
$g_{ij} = \delta_{ij} \cO(\tfrac{\phi}{c^2})$.

Consider a slowly moving particle with $v^i = \diff{x^i}{t}$ such
that $v^2 \ll c^2$.  We thus have
\[
\ud s^2 = g_{a b}\, \ud x^a\, \ud x^b
= \left(v^2 + \cO(v^2 \tfrac{\phi}{c^2})
- c^2 - 2 \phi + \cO(\tfrac{\phi^2}{c^2}) \right) \ud t^2
\quad \text{on the path.}
\]

Thus
\[
\int \abs{\ud s} \approx \int c^2 + \phi - \tfrac{1}{2} v^2\, \ud t
\]

and the Euler-Lagrange equations yield $\diff{v^i}{t} = - \pd{\phi}{x^i}$.

\chapter{Metric differential geometry}

We need a formulation of physics valid in arbitrary co-ordinate
systems.  Physical quantities must have existence independent of
particular co-ordinates being used -- hence must transform properly
under co-ordinate transforms.  They should be represented by tensors.

\section{Basic tensors}
Consider the co-ordinate change $x^a \mapsto x^{a'}(x^b)$ on spacetime,
with inverse $x^{a'} \to x^a(x^{b'})$.

Define
\begin{align*}
p^{a'}_a =& \pd{x^{a'}}{x^a},\\
p^a_{a'} =& \pd{x^a}{x^{a'}}.
\end{align*}

Note that $p^{a'}_a p^{b}_{a'} = \pd{x^{a'}}{x^a}
\pd{x^b}{x^{a'}} =
\delta^{\phantom{a}b}_a$, by using the chain rule, where
\[
\delta^{\phantom{a}b}_a= \begin{cases} 1 & a=b \\
0& a \neq b \end{cases}
\]
is the Kronecker delta.

Under repeated co-ordinate change $x^a \mapsto x^{a'} \mapsto x^{a''}$, 
we have the group property, using the chain rule,
\[
p^{a'}_a p^{a''}_{a'} = p^{a''}_a.
\]

A \emph{covariant tensor} of $n^{\text{th}}$ rank, with components $T_{a
  \cdots b}$ with respect to co-ordinates $x^a$, at a point $P$
has transformation law
\[
T_{a \cdots b} \to T_{a' \cdots b'}=p^{a}_{a'} \cdots
p^{b}_{b'} T_{a \cdots b}.
\]
Note that the group property shows that the components $T_{a' \cdots b'}$
are uniquely defined with respect to any co-ordinate system if they are
fixed in one system $x^a$; this provides a way of constructing all tensors.

A \emph{contravariant tensor} $T^{a \dots b}$ transforms as
\[
T^{a \cdots b} \to T^{a' \cdots b'}= p^{a'}_{a}
\dots p^{b'}_{b}  T^{a \cdots b}
\]
Similarly for \emph{mixed tensors}, for example
\[
T^{\phantom{a}b}_a \to T^{\phantom{a'}b'}_{a'} 
= p^a_{a'} p^{b'}_b T^{\phantom{a}b}_a
\]
It is important to keep the order of the indices the same.

A \emph{scalar} is a tensor with no indices, invariant under 
co-ordinate change, for example the mass of a particle.

A \emph{scalar field} $\phi(x^a)$ is a scalar function for example
pressure or particle density in a fluid.

A \emph{covariant vector field} $v_a(x^b)$ is a vector function
of position.  For example if $\phi (x^b)$ is a scalar field then
$v_a = \phi_{,a}:=\pd{\phi}{x^a}$
is a covariant vector field, since 
\[
v_{a'}= \pd{\phi}{x^{a'}} = \pd{x^a}{x^{a'}} \pd{\phi}{x^a}.
\]
A further example is a pressure gradient $p_{,a}$ in a fluid.

Suppose a curve $x^a(\lambda)$
parametrised by $\lambda$, has a tangent of $v^a=\diff{x^a}{\lambda}$
at the point $P$.  $v^a$ is a \emph{contravariant vector}.  This follows
because
\[
v^{a'}= \diff{x^{a'}}{\lambda} =
\diff{x^{a'}}{x^a}\diff{x^a}{\lambda}.
\]

Other examples are the $4$-velocity of an observer, $u^a=\diff{x^a}{\tau}$,
where $\tau$ is proper time.

\subsection{Examples of tensors}

\begin{itemize}
\item The Kronecker delta is a tensor:
\[
p^a_{a'} p^{b'}_{b} \delta^{\phantom{a}b}_a = 
p^a_{a'} p^{b'}_a = \delta^{\phantom{a'}b'}_{a'}.
\]
\item The metric $g_{ab}$ is a tensor, since the invariant $\ud s^2$
can be written
\[
\ud s^2= g_{ab} \ud x^a \ud x^b = g_{ab}
\pd{x^a}{x^{a'}} \ud x^{a'} \pd{x^b}{x^{b'}} \ud x^{b'} =
g_{a'b'} \ud x^{a'} \ud x^{b'}
\]
where $g_{a'b'}=g_{ab} p^a_{a'} p^b_{b'}$.
\end{itemize}

Further examples will be provided by the curvature and energy momentum
tensors.

\subsection{Operations preserving tensor property}

\begin{itemize}
\item Addition: $T_{ab}+W_{ab}$ is a tensor if $T_{ab}$
  and $W_{ab}$ are tensors.
\item Scalar multiplication: $fT_{ab}$ is a tensor if $T_{ab}$ is a tensor.
\item Outer products: $v^a T_{bc}$ transforms
as
\[
v^{a'} T_{b' c'} = p^{a'}_a p^b_{b'} p^c_{c'}
v^a T_{bc}.
\]
\item Contraction of one covariant with one contravariant index: if
  $T^a_{\phantom{a}bc}$ is a tensor, define
  $v_c=T^a_{\phantom{a}ac}$, transforming as
\begin{align*}
  v_{c'} &= T^{a'}_{\phantom{a'}a'c'} \\
  &= p^{a'}_a p^b_{a'} p^c_{c'}
  T^{a}_{\phantom{a} bc} = \delta^{\phantom{a}b}_a
  p^{c}_{c'}
  T^{a}_{\phantom{a}bc}\\
  &= p^{c}_{c'} T^a_{\phantom{a}ac}
  = p^{c}_{c'} v_c.
\end{align*}

\item Interchange of indices: $T_{ab}$ (a tensor) $\mapsto$ $T_{ba}$
which is also a tensor.  Similarly symmetrisation and anti-symmetrisation
\begin{align*}
T_{(ab)} &= \frac{1}{2!} (T_{ab}+T_{ba})\\
T_{[ab]} &= \frac{1}{2!} (T_{ab}-T_{ba})
\end{align*}
\end{itemize}

The above can readily be generalised to more indices.

\subsection{Quotient theorem}

Suppose $U^a=T^{ab}V_b$ is a vector for all vectors $V_b$.  Then
$p^{a'}_a U^a=p_a^{a'} T^{ab} V_b$, and 
\[
U^{a'}=p^{a'}_a U^a= T^{a'b'} V_{b'}=
T^{a'b'}p^b_{b'} V_b.
\]

Subtracting these last two yields,

\begin{align*}
  (T^{a'b'} p^b_{b'} - p^{a'}_a T^{ab})V_b
  &= 0 \qquad \forall V_b \text{ and so}\\
  T^{a'b'}p^b_{b'} &= p^{a'}_a T^{ab}.
\end{align*}

Multiplying both sides by $p_b^{c'}$ yields
\[
T^{a'b'} p^b_{b'} p^{c'}_b =T^{a'b'}
\delta^{\phantom{b'}c'}_{b'}
= T^{a'c'}= p^{c'}_b p^{a'}_a T^{ab}.
\]

Hence $T^{a}$ is a tensor.

\subsection{Inverse metric tensor}

Define $g^{ab} (=g^{ba})$ to be the matrix inverse of
$g_{ab}$, i.e.  such that $g_{ac}
g^{cb}=\delta_{a}^{\phantom{a}b}$.  Now for any vector
$V^a$ can define a vector $U_a=g_{ab} V^b$.  Note that there
is a one to one correspondance between $U_a$ and $V^a$ since
$g_{ab}$ is non-singular and so we can construct all vectors $U_b$
in this way.  The quotient theorem implies that $g^{ab}$ is a tensor.

\subsection{Raising and lowering of indices}

We can use $g^{ab}$ to raise any covariant index.  For example
$T_{ab}$ gives a tensor $T^a_{\phantom{a}b}= g^{ac}
T_{c b}$ if the first index is raised.  Similarly we can use
$g_{ab}$ to lower any index, for example $W^{ab}$ gives
$W^{\phantom{a}b}_a = g_{ac} W^{cb}$. The index
ordering must be carefully maintained.

Raising and lowering are inverse operations.  One normally regards
e.g. $T_{ab}$, $T_a^{\phantom{a}b}$,
$T^a_{\phantom{a}b}$ and $T^{ab}$ as different versions of
the same object.

\subsection{Partial derivatives of tensors}

Partial derivatives of tensors are not tensors in
general.  For example suppose $v_a$ is a vector field.
Then $v_{a'}= \pd{x^a}{x^{a'}} v_a$ and so
\begin{align*}
  \pd{v_{a'}}{x^{b'}} =\pd{^2x^a}{x^{a'} \partial x^{b'}}
  v_a + \pd{x^a}{x^{a'}}\pd{x^b}{x^{b'}} \pd{v_a}{x^b}\\
  & \neq p^a_{a'} p^b_{b'} \pd{v_a}{x^b} \qquad \text{in general.}
\end{align*}

The only exception is $\phi_{,a}$ as mentioned earlier.

\section{Lengths and geodesics}

The squared magnitude of a vector $v_a$ or $v^a$ is defined to be
$v_a v^a= v_a v_b g^{ab}= v^a v^b g_{ab}$ and is
invariant under co-ordinate transformations.

It can be evaluated in a local inertial frame where
$g_{ab}=\eta_{ab}$.

\[
v_a \text{ is } \begin{cases}
\text{spacelike if $v_a v^a > 0$,} \\
\text{null if $v_a v^a = 0$,}\\
\text{timelike if $v_a v^a < 0$.}
\end{cases}
\]

As before, if $v^a$ is spacelike we can find a Lorentz transformation
in the local inertial frame making $v^a=(v^i,0)$.  Then $v_a v^a=
(v^1)^2+ (v^2)^2 +(v^3)^2= \abs{\v}^2$, which is the physically
measured squared magnitude of $v^a$ in that frame.  If $v^a$ is
timelike we can make $v^a=(0,v^4)$ and then $v_a v^a = -c^2 (v^4)^2$.

\section{Angles between vectors}
Suppose that $v^a$ and $w^a$ are both spacelike and that we have
chosen a local inertial frame such that $v^a=(v^i,0)$,
$w^a=(w^i,0)$.  Then the angle $\theta$ between $v^a$ and $w^b$
is defined by

\begin{align*}
\cos \theta &= \frac{\v \cdot \vect{w}}{\abs{\v} \abs{\vect{w}}} \qquad
\text{Cartesian notation in LIF, or}\\
&= \frac{(g_{a b} v^a w^b)}{(v_d v^d)^{1/2}
(w_c w^c)^{1/2} } \qquad
\text{invariant definition.}
\end{align*}

\section{Lengths of curves}

If $x^a(\lambda)$ describes a spacelike curve $\gamma$, which is
parameterised by $\lambda$ (i.e. if $v^a=\diff{x^a}{\lambda}$ is spacelike
along $\gamma$), the length of $\gamma$ from $A$ to $B$ is
\[
\int_A^B \ud s= \int_A^B \left( g_{a b} \diff{x^a}{\lambda}\diff{x^b}
{\lambda} \right)^{1/2} \ud\lambda.
\]

If $x^a(\lambda)$ gives a timelike curve $\gamma$ (i.e.
$v^a=\diff{x^a}{\lambda}$ is timelike along $\gamma$), then the time
elapsed along $\gamma$ from $A$ to $B$ is
\[
\frac{1}{c}\int_A^B \abs{\ud s} =\frac{1}{c} \int_A^B \left( - g_{a
    b} \diff{x^a}{\lambda}\diff{x^b} {\lambda} \right)^{1/2}
\ud\lambda.
\]

\section{Geodesics}

A geodesic $\gamma$ from $A$ to $B$ extremises 
\[
\int_A^B \abs{\ud s} = \int_A^B \abs{g_{a b}  \diff{x^a}{\lambda}
\diff{x^b}{\lambda}}^{1/2} \ud \lambda =
\int_A^B L(x^a(\lambda), \dot x^b(\lambda))\, \ud \lambda,
\]
where
$\dot x^a(\lambda)=\diff{x^a}{\lambda}$, subject to fixed
endpoints: $x^a(\lambda_1)$ are the co-ordinates of $A$, and
$x^a(\lambda_2)$ are the co-ordinates of $B$. 

For example consider a spacelike geodesic.  Then
\begin{align*}
\pd{L}{\dot x^a} &= \frac{g_{a b} \dot x^b}{L}\cr
\pd{L}{x^a} &= \frac{g_{bc,a} \dot x^b \dot x^c}{2L}
\end{align*}

Using the Euler-Lagrange equations
\[
\dif{\lambda} \left ( \pd{L}{\dot x^a} \right)-\pd{L}{x^a} = 0
\]

gives 
\[
L^{-1} \left [ g_{a b} \ddot x^b + (g_{a b,c}-\frac{1}{2}
  g_{b c,a}) \dot x^b \dot x^c \right] =
L^{-2} \diff{L}{\lambda} g_{a b} \dot x^b.
\]
Using our freedom to reparametrise the curve we can choose
$\lambda=s$, the distance along $\gamma$.  Then $L=1$
and $\diff{L}{\lambda}=0$ along $\gamma$.  Therefore

\begin{align*}
0 &= g_{a b}\ddot x^b + ( g_{a b,c} -\frac{1}{2} g_{bc,
a}) \dot x^b \dot x^c \\
&=  g_{a b}\ddot x^b + \frac{1}{2} ( g_{a b,c}+
g_{a c,b}-g_{c b,a}) \dot x^b \dot x^c.
\end{align*}

Raising index $a$ yields the \emph{geodesic equation}
\[
\frac{\ud^2 x^a}{\ud s^2} + \ch{a}{bc}
\diff{x^b}{s}\diff{x^c}{s} = 0,
\]
where
\[
\ch{a}{b c} = \frac{1}{2} g^{ad} ( g_{bd,c}+
g_{cd,b}-g_{b c,d} ).
\]

The same equation is obtained for timelike geodesics and we have the
equations of motion for a test particle in a gravitational field.

The expression $\ch{a}{b c}$ involves the ``derivatives of
gravitational potential'' and corresponds to $\phi_{,i}$ in Newtonian
gravity.  It is possible to rederive Newtonian dynamics in this way --
see later.

The geodesic equation is a second order ordinary differential equation
and so a geodesic is uniquely specified once the starting point
$x^a(0)$ and an initial tangent direction $\dot x^a(0)$ are
chosen.

\section{Covariant differentiation and Christoffel symbols}

Physical laws involve partial derivatives. We need a generalisation
$\del_a$ of $\partial_a := \pd{}{x^a}$ which preserves tensorial
properties.  We want the covariant derivative operator to

\begin{itemize}
\item keep $\del_a \phi=\partial_a \phi$ for scalar
fields $\phi$, since $\partial_a \phi$ is already a 
covariant vector field.

\item look like $\del_b v_a = \partial_b v_a -
  \Gamma^c_{b a} v_c$ acting on covariant vector fields,
  where $\Gamma^c_{b a}$ is a (non-tensorial) collection of
  $4^3$ numbers to be constructed out of the metric and its first
  derivatives, and $-\Gamma^c_{b a} v_c$ is designed to
  cancel out the bad transformation properties of $\partial_b v_a$.

\item commute with addition:
\[
\del_d ( T^{a \dots b}_{\phantom{a \dots b}
d \dots c} +
U^{a \dots b}_{\phantom{a \dots b} d \dots c} ) =
\del_d T^{\dots}_{\phantom{\dots}\dots} + \del_d U^{\dots}_{
\phantom{\dots}\dots}.
\]

\item obey the Leibniz rule
\[
\del_a \left( T^{\dots}_{\phantom{\dots}\dots}
  U^{\dots}_{\phantom{\dots}\dots} \right) = \left(\del_a
  T^{\dots}_{\phantom{\dots}\dots}\right)
U^{\dots}_{\phantom{\dots}\dots} + T^{\dots}_{\phantom{\dots}\dots}
\left(\del_a U^{\dots}_{\phantom{\dots}\dots}\right)
\]

\item satisfy $\nabla_d g_{a b} = 0$, $\nabla_d g^{a b}
  = 0$ and $\nabla_d \delta^a_{\phantom{a}b} = 0$.

\item commute with index contraction:
\[
\del_a \left( T^{\cdots b \cdots}_{\phantom{\cdots b \cdots}\cdots
    b \cdots}\right) = \delta^d_{\phantom{d} c} \del_a \left(
T^{\cdots c \cdots}_{\phantom{\cdots c \cdots}\cdots
    d \cdots}
\right).
\]
\end{itemize}

These properties imply that $\del_a$ commutes with the operations of
raising/lowering indices.

We want to find the $\Gamma$'s, which we will do using the zero
covariant derivative of the metric.  First note that for any covariant
vector fields $u_a$ and $v_b$,

\begin{align*}
\del_d (u_a v_b) &= u_a \del_d v_b + v_b \del_d
u_a \\
&= \partial_d (u_a v_b) - \Gamma^c_{d b} u_a
v_c - \Gamma^c_{d a} u_c v_b.
\end{align*}

Now any tensor field $T_{a b}$ can be built by adding tensors of
the form $u_a v_b$, so using linearity
\[
\del_d T_{a b} = \partial_d T_{a b} -
\Gamma^c_{d b} T_{a c} - \Gamma^c_{d a}
T_{c b}.
\]

for any tensor $T_{a b}$.  We apply this to the metric tensor
$g_{a b}$ to get

\begin{align}
\del_d g_{a b} &= g_{a b,d} - \Gamma^c_{d b}
g_{a c} - \Gamma^c_{d a} g_{c b} = 0.
\label{eq:cvdm1} \\
\intertext{Permuting the indices cyclically, we get}
\del_a g_{b d} &= g_{b d,a} - \Gamma^c_{a d}
g_{b c} - \Gamma^c_{a b} g_{c d} = 0.
\label{eq:cvdm2} \\
\del_b g_{d a} &= g_{d a,b} - \Gamma^c_{b a}
g_{d c} - \Gamma^c_{b d} g_{c a} = 0.
\label{eq:cvdm3}
\end{align}

We make the further simplifying assumption of symmetry:
$\Gamma^c_{a b} = \Gamma^c_{b a}$.  Now take
$\eqref{eq:cvdm3} - \eqref{eq:cvdm1} - \eqref{eq:cvdm2}$ and adjust
the indices to get
\[
2 \Gamma^d_{bc} g_{ad} = - g_{bc,a} + g_{ac,b} + g_{ab,c}.
\]

Raise the index $a$ to get
\begin{equation}\label{eq:chrsym}
\Gamma^a_{bc} = \tfrac{1}{2} g^{ad} ( g_{bd,c} + g_{cd,b} - g_{bc,d}).
\end{equation} 

These are the \emph{Christoffel symbols} for the metric $g_{ab}$ and
define the metric connection $\del$ on spacetime.

\subsection{Transformation properties of Christoffel symbol}

We start from $g_{a'b'} = p^a_{a'} p^b_{b'} g_{ab}$, so that
\[
g_{a'b',c'} = p^a_{a'} p^b_{b'} p^c_{c'} g_{ab,c} + g_{ab}
\partial_{c'} \left( \pd{x^a}{x^{a'}} \pd{x^b}{x^{b'}}\right). 
\]

\begin{multline*}
g_{a'b',c'} + g_{a' c',b'} - g_{b' c',a'} =p^a_{a'} p^b_{b'} p^c_{c'}
\left( g_{ab,c} + g_{ac,b} - g_{bc,a} \right) \\
+ g_{ab}
\left( \partial_{c'} \left( \pd{x^a}{x^{a'}} \pd{x^b}{x^{b'}}\right)
+ \partial_{b'} \left(\pd{x^a}{x^{a'}} \pd{x^c}{x^{c'}} \right)
- \partial_{a'} \left( \pd{x^b}{x^{b'}} \pd{x^c}{x^{c'}}\right) \right) 
\end{multline*}

Putting all of this together we find 
\begin{equation}\label{eq:gamtran}
\Gamma^{a'}_{b' c'} = p^{a'}_a p^b_{b'} p^c_{c'} \Gamma^a_{bc}
+ \pd{x^{a'}}{x^a} \pd{^2 x^a}{x^{b'} \partial x^{c'}}.
\end{equation}

\eqref{eq:gamtran} can be used to verify that $\del_b v_a$ is a
tensor.

\subsection{Covariant differentiation of other types of tensor}

For instance:  what is $\del_b u^a$?  Take an arbitrary covariant
vector field $v_a$ and consider $\del_b \left(v_a u^a\right)
= \partial_b \left( v_a u^a \right)$.  Then

\begin{align*}
\del_b \left( v_a u^a \right) &= u^a \del_b v_a + v_a \del_b u^a \\
&= u^a \partial_b v_a - \Gamma^c_{ba} v_c u^a + v_a \del_b u^a \\
&= u^a \partial_b v_a + v_a \partial_b u^a.
\end{align*}

This is true for all $v_a$, so that
\[
\del_b u^a = \partial_b u^a + \Gamma^a_{bc} u^c.
\]

In general we get a $+$ sign for each contravariant index and a $-$
sign for each covariant index, that is
\[
\del_b T_a^{\phantom{a} c} = \partial_b T_a^{\phantom{a}c}
- \Gamma^d_{ba} T_d^{\phantom{d} c} + \Gamma^c_{bd} T_a^{\phantom{a} d}.
\]

We write $\del_b\left(\ \right)$ as $\left(\ \right)_{;b}$.

\section{Differentiation along a curve: geodesics}

We need a geometrical description of the rate of change of a physical
quantity seen by an observer moving along a path $x^a(\lambda)$.  This
is the \emph{absolute derivative}, given by
\begin{align*}
\absd{\lambda} v^a &= \diff{x^b}{\lambda} \del_b v^a
= \diff{x^b}{\lambda} \pd{v^a}{x^b} + \Gamma^a_{bc}
\diff{x^b}{\lambda} v^c \\
&= \diff{v^a}{\lambda} + \Gamma^a_{bc} \diff{x^b}{\lambda} v^c.
\end{align*}

Note that we only need to know $v^a$ along the path.  We can similarly
define $\absd{\lambda}$ on other fields.  The absolute derivative of a
tensor is again a tensor.

A field $v^a$ is said to be \emph{parallelly transported} along a curve
$x^a(\lambda)$ iff $\absdf{v^a}{\lambda} = 0$ (and similarly for other
types of tensor).

Note that parallel transport preserves lengths and angles.  If $v^a$
and $w^a$ are parallelly transported, then

\begin{align*}
\dif{\lambda} \left( v_a w^a\right)  &= \diff{x^b}{\lambda} \del_b
\left( g_{cd} v^c w^d \right) \\
&= \diff{x^b}{\lambda} g_{cd;b} + g_{cd} w^d \absdf{v^c}{\lambda}
+ g_{cd} v^c \absdf{w^c}{\lambda}\\
& = 0.
\end{align*} 

We can apply the notation of absolute derivative to the tangent vector
$\diff{x^a}{\lambda}$.  A curve $x^a(\lambda)$ is said to be
\emph{autoparallel} iff
\[
\absd{\lambda} \diff{x^a}{\lambda} = 0,
\]

that the tangent vector is parallelly transported along the curve.
This is equivalent to
\[
\diff{^2 x^a}{\lambda^2} + \Gamma^a_{bc} \diff{x^b}{\lambda}
\diff{x^c}{\lambda} = 0,
\]

which, since $\Gamma^a_{bc} = \ch{a}{bc}$, is the geodesic equation.
This gives an alternative characterisation of geodesics, and $\lambda$
is called an \emph{affine parameter} along the geodesic.

If $\gamma$ is a geodesic with affine parameter $\lambda$ then
\[
\dif{\lambda} \left( g_{ab} \diff{x^a}{\lambda}
  \diff{x^b}{\lambda}\right) = 0,
\]

so that $g_{ab} \diff{x^a}{\lambda} \diff{x^b}{\lambda}$ is a constant
along $\gamma$ and $\lambda$ is proportional to length $s$ (or proper
time $\tau$) along $\gamma$.

The acceleration (vector) of a timelike curve $x^a(\tau)$ with
4-velocity $u^b = \diff{x^a}{\tau}$ is
\[
a^b = \absdf{u^b}{\tau} = \diff{^2 x^b}{\tau^2} + \Gamma^b_{cd}
\diff{x^c}{\tau} \diff{x^d}{\tau}.
\]

and so geodesics are unaccelerated curves (free fall).

\section{Local inertial frames}

We can now make our definition of a LIF more precise.  We want to
choose locally inertial co-ordinates $x^a$ near an event $P$ ($x^a =
0$) such that $g_{ab} = \eta_{ab}$ at $P$ and that particles moving
through $P$ under gravity have no co-ordinate acceleration.  We want
to arrange $\Gamma^a_{bc} = 0$ (or equivalently $g_{ab,c} = 0$) at $P$

In a LIF

\begin{itemize}
\item the metric looks as much as possible like the flat space metric
\item geodesics become straight lines
\item parallel transport, acceleration etc. acquire usual flat-space
  interpretations
\item covariant derivatives become partial derivatives.
\end{itemize}

To find inertial co-ordinates near $P$ we translate to put $x^a  = 0$
and then use a linear transformation to give $g_{ab} = \eta_{ab}$ at
$P$.   Define ${}^0\Gamma^a_{bc} = \left.\Gamma^a_{bc}\right|_P$.
Then use the transformation $x^a \to y^a$ with quadratic inverse
\[
x^a = y^a - \tfrac{1}{2} {}^0\Gamma^a_{bc} y^b y^c.
\]

In the new co-ordinates,
\begin{align*}
g^{\text{new}}_{ab} &= \pd{x^c}{y^a} \pd{x^d}{y^b} g^{\text{old}}_{cd}
\\
&= \left(\delta_a^{\phantom{a}c} - {}^0 \Gamma^c_{ae} y^e\right)
\left(\delta_b^{\phantom{b}d} - {}^0\Gamma^d_{bf} y^f \right)
\left(\eta_{cd} + g^{\text{old}}_{cd,g} y^g + \dots\right).
\end{align*}

The terms linear in $y^c$ are
\[
\left(- {}^0\Gamma^d_{ac} \eta_{bd} - {}^0\Gamma^d_{bc} \eta_{ad} +
g^{\text{old}}_{ab,c} \right) y^c = 0.
\]

Hence $g^{\text{new}}_{ab} = \eta_{ab} + \text{quadratic in $y^c$}$
--- the co-ordinates $y^a$ provide a LIF near $P$.

\section{Curvature}

The curvature of spacetime measures the non-commutation of covariant
derivatives.  For a scalar field $\phi$, $\phi_{;ab} = \phi_{;ba}$,
but for a vector field $v^a$,

\begin{align*}
v^a_{;bc} - v^a_{;cb} &= \left( \Gamma^a_{be,c} - \Gamma^a_{ce,b}
+ \Gamma^a_{cd} \Gamma^d_{be} - \Gamma^a_{bd} \Gamma^d_{ce}
\right) v^e \\
&= R^a_{\phantom{a} e c b} v^e,
\end{align*}

where

\[
R^a_{\phantom{a} e c b} = \Gamma^a_{be,c} - \Gamma^a_{ce,b}
+ \Gamma^a_{cd} \Gamma^d_{be} - \Gamma^a_{bd} \Gamma^d_{ce}.
\]

$R^a_{\phantom{a} e c b}$ is a tensor (by the quotient theorem) and is
called the the \emph{Riemann curvature tensor}.  It is constructed
from the metric and its first and second covariant derivatives.  If
the spacetime is flat we can choose Minkowskian co-ordinates to get
$g_{ab} = \eta_{ab}$ so that $R^a_{\phantom{a}bcd} = 0$.  Therefore
$R^a_{\phantom{a}bcd} = 0$ in all co-ordinates.  The converse can be
proved: if $R^a_{\phantom{a}bcd} = 0$ then the spacetime is flat.

In a LIF at $P$,
\[
R_{abcd} = \tfrac{1}{2} \left( g_{ad,bc} + g_{bc,ad} - g_{ac,bd} -
  g_{bd,ac} \right).
\]

This gives the symmetry properties

\begin{itemize}
\item $R_{abcd} = R_{[ab]cd} := \tfrac{1}{2} \left(R_{abcd} -
    R_{bacd}\right)$
\item $R_{abcd} = R_{ab[cd]}$
\item $R_{abcd} = R_{cdab}$
\item $R_{a[bcd]} := \tfrac{1}{3!} \left(R_{abcd} + R_{acdb} +
    R_{adbc} - R_{acbd} - R_{abdc} - R_{adcb}\right) = 0$.  Using the other
    symmetries of $R_{abcd}$, this can be equivalently written as
\[
R_{abcd} + R_{acdb} + R_{adbc} = 0.
\] 
\end{itemize}

Since symmetries of tensors are preserved by co-ordinate
transformations, these hold at any point $P$ in any co-ordinates.
These symmetries imply that $R_{abcd}$ has only 20 free components.

The \emph{Ricci tensor} is $R_{bd} = R^a_{\phantom{a}bad}$.  Note that
\[
R_{bd} = g^{ac} R_{abcd} = g^{ac} R_{cdab} = R_{db}.
\]

$R_{bd}$ therefore has only ten free components.  The \emph{Ricci
  scalar} is $R = g^{bd} R_{bd}$.

\section{Geodesic deviation}

Spacetime curvature produces relative acceleration of nearby test
particles moving on geodesics.  For convenience in the derivation we
replace ``2 nearby test particles'' with ``1 parameter family of
geodesics''. Each geodesic is labelled by a parameter $s$.  We label
points on a given geodesic by proper time $\tau$ measured from the
origin.

\vspace{1.5in}

Write $u^a = \pd{}{\tau} x^a(\tau,s)$: the 4-velocity on the geodesic
labelled by $s$.  The geodesic equation is
\[
\frac{\uD}{\partial\tau} u^a := \pd{x^b}{\tau} \del_b u^a = u^b \del_b u^a = 0.
\]

Define $\xi^a = \pd{}{s} x^a(\tau,s)$.  Then for small $\Delta s$,
$\Delta s \xi^a$ is a separation vector from the geodesic labelled by
$s$ to the geodesic labelled by $s+\Delta s$.

Note that
\[
\pd{u^a}{s} = \pd{^2 x^a}{s \partial \tau} = \pd{\xi^a}{\tau},
\]

and so

\begin{align*}
\xi^b \del_b u^a &= \pd{x^b}{s} \del_b u^a = \pd{u^a}{s} +
\Gamma^a_{bc} \xi^b u^c \\
&= \pd{\xi^a}{\tau} + \Gamma^a_{bc} u^b \xi^c = \pd{x^b}{\tau} \del_b \xi^a
\\
&= u^b \del_b \xi^a. 
\end{align*}

We now prove (and then use!) the \emph{curvature identity}, which is
valid for any vector fields $X^a$, $Y^b$ and $Z^c$:

\begin{multline*}
Y^b \del_b \left( Z^c \del_c X^a \right) - Z^c \del_c \left(Y^b \del_b
  X^a\right) = \\
Y^b \left( \del_b Z^c\right) \left( \del_c X^a\right)
+ Y^b Z^c \del_b \del_c X^a \\
- Z^c \left( \del_c Y^b \right) \left( \del_b X^a \right)
- Z^c Y^b \del_c \del_b X^a \\
=\left(Y^c \del_c Z^b - Z^c \del_c Y^b \right)\del_b X^a
+ Y^b Z^c R^a_{\phantom{a} dbc} X^d.
\end{multline*}

Now take $X^a = u^a$, $Y^b = u^b$ and $Z^c = \xi^c$, so that
\begin{align*}
Z^c \del_c X^a &= \xi^c \del_c u^a = u^c \del_c \xi^a \\
Y^b \del_b X^a &= u^b \del_b u^a = 0 \\
Y^c \del_c Z^b - Z^c \del_c Y^b &= u^c \del_c \xi^b - \xi^c \del_c u^b
= 0.
\end{align*}

Substituting these into the curvature identity we get
\[
u^b \del_b \left( u^c \del_c \xi^a \right) = R^a_{\phantom{a}dbc} u^d
u^b \xi^c,
\]
or
\begin{equation}\label{eq:geodev}
\frac{\uD^2\phantom{\tau}}{\partial \tau^2} \xi^a
= R^a_{\phantom{a}dbc} u^d u^b \xi^c.
\end{equation}

This is the \emph{equation of geodesic deviation}.  It shows that the
relative acceleration is proportional to separation for two nearby
test bodies.  We have a \emph{true} gravitational field iff we have
relative accelerations, iff $R^a_{\phantom{a}bcd} \neq 0$, iff
spacetime is curved.

\chapter{Vacuum gravitational fields}

\section{The vacuum field equations}

We need to guess the field equations of General Relativity.  We will use
the Newtonian limit to suggest the vacuum GR field equations and then
compare the predictions of these equations in the non-Newtonian case.

In the Newtonian limit (weak fields and low speeds), $\abs{\tfrac{\phi}{c^2}}
\ll 1$ and $\tfrac{V}{c} \ll 1$ (where $V$ is a ``typical'' speed).  We
will use co-ordinates $x^a = (x,y,z,ct)$.  From the equivalence principle,
$g_{44} = -1 - \tfrac{2 \phi}{c^2} + \dots$ and it is reasonable to
expect that \emph{all} deviations from flatness are of order
$\tfrac{\phi}{c^2}$.

A geodesic in Newtonian gravity has $\tau \approx t$,
$\abs{\diff{x^i}{t}} \ll c$, and the spatial component of the
geodesic equation is
\[
0 = \diff{^2 x^i}{\tau^2} + \Gamma^i_{a b} \diff{x^a}{\tau}
\diff{x^b}{\tau} \approx \diff{^2 x_i}{t^2} + \Gamma^i_{44} c^2.
\]

Now
\[
\Gamma^i_{44} = \tfrac{1}{2} g^{i 4} g_{44,4} + \tfrac{1}{2} g^{ij}
(g_{j4,4} - g_{44,j}) \approx - \tfrac{1}{2} \delta^{ij} g_{44,j}
\]

as $g^{ij} \approx \delta^{ij}$ and the derivative $\partial_4$ should be
smaller than derivatives $\partial_i$ by a factor of order
$\cO(\tfrac{v}{c}) \ll 1$.  Thus $\Gamma^i_{44} \approx c^{-2} \phi_{,i}$
and the geodesic equation is
\[
\diff{^2 x^i}{t^2} \approx - \phi_{,i},
\]

which is probably a good thing.  $\Gamma^i_{44}$ are the only
Christoffel symbols significant for Newtonian gravity.

We now consider geodesic deviation in the Newtonian limit.  Take a
(spatial) separation vector $\xi^a = (\xi^i,0)$.  Now
$u^a = \diff{x^a}{\tau} = (0,c)$ at low speeds and the geodesic
deviation equation 

\[
\frac{\uD^2 \xi^a}{\ud t^2} = R^a_{\phantom{a}bdc}
u^b u^d \xi^c
\]

gives $\diff{^2 \xi^i}{t^2} \approx c^2 R^i_{\phantom{i}44j} \xi^j$.
Therefore $R^i_{\phantom{i}44j}$ are the only components of the Riemann
tensor which are significant for Newtonian gravity.  In the
Newtonian limit

\begin{align*}
R^a_{\phantom{a}bdc} &= \Gamma^a_{bc,d}
- \Gamma^a_{bd,c}
+ \Gamma^a_{d\alpha} \Gamma^\alpha_{bc}
- \Gamma^a_{c\alpha} \Gamma^\alpha_{bd} \\
&\approx \Gamma^a_{bc,d} - \Gamma^a_{bd,c}.
\end{align*}

Therefore $R^i_{\phantom{i}44j} \approx \Gamma^i_{4j,4}
- \Gamma^i_{44,j}$.  If $L$ is a typical lengthscale of the system then
all the $\Gamma^a_{bd}$ are $\cO(\tfrac{\phi}{c^2 L})$ and so
$\Gamma^i_{4j,4} = \cO( \tfrac{\phi}{c^2 L^2}\tfrac{V}{c})$ but
$\Gamma^i_{44,j} = \cO( \tfrac{\phi}{c^2 L^2})$.  Thus
$R^i_{\phantom{i}44j} \approx - \Gamma^i_{44,j} = - c^{-2}
\phi_{,ij}$.  The Newtonian geodesic deviation equation is therefore

\[
\diff{^2 \xi^i}{t^2} \approx - \phi_{,ij} \xi^j.
\]

We base the vacuum GR equations on that for a Newtonian field,
$\phi_{,ii} = 0$.  In the Newtonian limit we find that
\[
R_{44} = R^i_{\phantom{i}4i4} \approx c^{-2} \phi_{,ii} = 0 \text{ in vacuum.}
\]

Since we want \emph{tensor} field equations valid in all co-ordinate
systems this suggests

\begin{equation}\label{eq:einstein}
R_{a b} = 0
\end{equation}

for the vacuum field equations (vacuum Einstein equations).  Since
$R_{a b}$ is symmetric we have 10 field equations, second order
in the ``gravitational potential'' $g_{a b}$.  A highly nonlinear
gravitational field can act as its own source.

\section{The Schwarzschild metric}

We look for a solution of the vacuum Einstein equations describing the
gravitational field outside a spherically symmetric body at rest.
(It can be shown that) A static spherically symmetric metric has the
form
\[
\ud s^2 = e^{\alpha(r)} \ud r^2 + r^2 \left( \ud \theta^2
+ \sin^2 \theta \ud \phi^2 \right) - e^{\gamma(r)}\ud t^2
\]
in suitable co-ordinates.  There are no cross terms $\ud t \ud (
\text{space})$ since the metric must be invariant under $t \to -t$.

The radial co-ordinate $r$ is chosen for simplicity --- such that each
sphere with $t$, $r$ constant has the intrinsic metric
$r^2 \left( \ud \theta^2 + \sin^2 \theta \ud \phi^2 \right)$.  We
can change $r$ to $r'(r)$, but the metric loses its simplicity in this
case.

The spherical symmetry forbids cross terms $\ud r \ud \theta$ (etc)
and makes $g_{rr}$ a function of $r$ only.

To impose the vacuum Einstein equations $R_{a b} = 0$ we need to
find the Christoffel symbols.  It is most convenient to find them
via the geodesic equations.  We use an alternate Lagrangian for
the geodesics,
\[
\delta \int_A^B g_{a b} \Dot{x}^a \Dot{x}^b\, \ud \lambda = 0,
\]

where $\Dot{x}^a \equiv \diff{x^a}{\lambda}$.  It is easy to
use the Euler-Lagrange equations to show that this gives the geodesic
equations.  We find that $\lambda$ must be a multiple of $s$ or $t$
along extremal curves.

In this spherically symmetric metric

\begin{equation}\label{eq:schwarzL}
\cL = e^{\alpha(r)} \Dot{r}^2 + r^2 \left( \Dot{\theta}^2 + \sin^2 \theta
\Dot{\phi}^2 \right) - e^{\gamma(r)} \Dot{t}^2.
\end{equation}

The Euler-Lagrange equations give

\begin{gather}
  2 e^\alpha \Ddot{r} + e^\alpha \alpha' \Dot{r}^2 - 2 r \left(
    \Dot{\theta}^2 + \sin^2 \theta \Dot{\phi}^2 \right) + e^\gamma
  \gamma' \Dot{t}^2 = 0, \\
  2 r^2 \Ddot{\theta} + 4 r \Dot{r} \Dot{\theta} - 2 r^2 \sin \theta
  \cos \theta \Dot{\phi}^2 = 0, \label{eq:ELschwartz2}\\
  2 r^2 \sin^2 \theta \Ddot{\phi} + 4 r \sin^2 \theta \Dot{\phi}
  \Dot{r} + 4 r^2 \sin \theta \cos \theta \Dot{\theta} \Dot{\phi} = 0, \\
  -2 e^\gamma \Ddot{t} - 2 \gamma' e^{\gamma} \Dot{r} \Dot{t} = 0.
\end{gather}

The only non-zero Christoffel symbols (where $x^a =
(r,\theta,\phi,t)$) are:
\[
\begin{array}{l l l}
\Gamma^1_{11} = \tfrac{1}{2} \alpha' & \Gamma^1_{22} = -r e^{-\alpha}
& \Gamma^1_{33} = - r e^{-\alpha} \sin^2 \theta \\
\Gamma^1_{44} = \tfrac{1}{2} \gamma' e^{\gamma - \alpha} &
\Gamma^2_{12} = r^{-1} & \Gamma^2_{33} = - \sin \theta \cos \theta \\
\Gamma^3_{13} = r^{-1} & \Gamma^3_{23} = \cot \theta
& \Gamma^4_{14} = \tfrac{1}{2} \gamma'
\end{array}
\]
and the transposes $\Gamma^a_{bc} = \Gamma^a_{cb}$.  We can now
find the Ricci tensor, which has non-zero components

\begin{align}
R_{11} &= - \tfrac{1}{2} \gamma'' + \tfrac{1}{4} \alpha' \gamma'
+ \tfrac{1}{4} \gamma'{}^2 + r^{-1} \alpha' \label{eq:ricciS1} \\
R_{22} &= e^{-\alpha} \left( \tfrac{1}{2} r (\alpha' - \gamma') - 1
\right) + 1 \label{eq:ricciS2} \\
R_{33} &= \sin^2 \theta R_{22} \label{eq:ricciS3} \\
R_{44} &= e^{\gamma - \alpha} \left( \tfrac{1}{2} \gamma''
- \tfrac{1}{2} \alpha' \gamma' + \tfrac{1}{4} \gamma'{}^2
+ r^{-1} \gamma'\right) \label{eq:ricciS4}.
\end{align}

Equations \eqref{eq:ricciS1} and \eqref{eq:ricciS4}
give us $\alpha + \gamma = \kappa$ (a constant).  Substituting into
\eqref{eq:ricciS2} we get $e^{-\alpha} = 1 - \tfrac{a}{r}$
(and we can check that this is consistent with \eqref{eq:ricciS1}).
Thus
\[
\ud s^2 = \frac{\ud r^2}{1 - \tfrac{a}{r}} + r^2 \left(
\ud \theta^2 + \sin^2 \theta \ud \phi^2 \right) - e^{\kappa}
\left( 1 - \tfrac{a}{r} \right) \ud t^2.
\]

We normalize $t$ to ordinary time as $r \to \infty$, so that
$e^\kappa = c^2$.  In the far field, $g_{tt} = - c^2 + \tfrac{2 G M}{r}$
if the body has mass $M$.  Thus $a = \tfrac{2 G M}{c^2}$ and we
arrive at the Schwarzschild metric

\begin{equation}\label{eq:Schwarz}
\ud s^2 = \frac{\ud r^2}{1 - \tfrac{2 G M}{c^2 r}} + r^2 \left(
\ud \theta^2 + \sin^2 \theta \ud \phi^2 \right) - c^2
\left( 1 - \tfrac{2 G M}{c^2 r} \right) \ud t^2
\end{equation}
(in Schwarzschild co-ordinates $(r,\theta,\phi,t)$).

This is only defined in the vacuum outside the body, but we can smoothly
join it onto a different solution in the region containing matter.
There is an apparent singularity at $r = r_S = \tfrac{2 G M}{c^2}$,
the Schwarzschild radius.

Usually the radius of matter is much greater than the Schwarzschild
radius (for the Sun, $r_S = 3 \mathrm{km}$), but if there is vacuum down
to $r = r_S$ we have a black hole.

It can be proven that spherical symmetry and the Einstein equations
imply the static Schwarzschild solution (even allowing time dependence).
This is called Birkhoff's theorem.

Direct computation shows that $R^a_{\phantom{a}b dc}$
has non-zero components and so this spacetime is genuinely curved.
The space part has metric
\[
\ud s^2 = \frac{\ud r^2}{1 - \frac{2 G M}{c^2 r}} + r^2 \left(
\ud \theta^2 + \sin^2 \theta \ud \phi^2\right)
\]

and is also curved.

As expected the corrections to the flat space metric are
$\cO(\tfrac{\phi}{c^2})$ in the far field.

\section[Gravitational redshift]%
{Gravitational redshift in the Schwarzschild metric}

\vspace{1.5in}

The proper frequency as measured by the emitter is
$b_1 = \tfrac{2 \pi}{\ud \tau_1} = \tfrac{2 \pi c}{p_1
\left( - g_{tt}(r_1) \right)^{\frac{1}{2}}}$.  The
proper frequency measured by the reciever is
$b_2 = \tfrac{2 \pi}{\ud \tau_2} = \tfrac{2 \pi c}{p_1
\left( - g_{tt}(r_2) \right)^{\frac{1}{2}}}$.

The ratio
\[
\frac{b_2}{b_1} = \sqrt{\frac{g_{tt}(r_1)}{g_{tt}(r_2)}}
= \left( 1 - \frac{2 G M}{c^2 r_1} \right)^{\frac{1}{2}}
\left( 1 - \frac{2 G M}{c^2 r_2} \right)^{-\frac{1}{2}}
\]

gives the gravitational redshift.  This is observed for many white
dwarf stars.

\section[Particle and photon paths]%
{Particle paths in the Schwarzschild metric}

In this section we use geometrical units in which $c = G = 1$.
We can obtain the geodesics from the Lagrangian $\cL$ in
\eqref{eq:schwarzL}.  We do not attempt to solve the geodesic
equations directly but instead seek first integrals of the motion.

$\pd{L}{\phi} = 0$ and so $r^2 \sin^2 \theta \Dot{\phi} = h$,
a constant.  For a massive particle ($\lambda = \tau$) in
the far field ($\tau \approx t$), we see that this is
just the angular momentum (per unit mass) about the $\theta = 0$ axis.

$\pd{L}{t} = 0$, so that $\left( 1 - \tfrac{2 M}{r}\right)
\Dot{t} = E$ is a constant.  This is the energy per unit mass.

For a slow moving massive particle, $\diff{t}{\tau}
\approx \left( 1 - v^2 \right)^{-\frac{1}{2}} \left( 1 + \tfrac{M}{r}\right)
\approx 1 + \tfrac{M}{r} + \tfrac{1}{2} v^2$.  Thus in the far
field, for a slow moving massive particle we see that
$E \approx 1 + \tfrac{1}{2} v^2 - \tfrac{M}{r}$, and so it is reasonable
to associate $E$ with the energy.

Finally, as particle paths are geodesics,
\[
\frac{\uD\phantom{\lambda}}{\ud\lambda} \diff{x^a}{\lambda} = 0
\]

and so
\[
\dif{\lambda} \left( g_{a b} \diff{x^a}{\lambda}
\diff{x^b}{\lambda} \right) = 0.
\]

Thus $\cL$ is conserved.  In fact, for a timelike geodesic $\cL = - 1$
and for a spacelike geodesic $\cL = 0$.

We can further simplify the problem by taking the motion only
in the equatorial plane.  We can initially arrange
$\theta = \tfrac{\pi}{2}$ and $\Dot{\theta} = 0$ by rotating
the co-ordinates.  Note that \eqref{eq:ELschwartz2} is
now automatically satisfied.

Using the conserved quantities we get the radial equation
\[
- \frac{E^2}{1 - \frac{2 M}{r}} + \frac{\Dot{r}^2}{1-\frac{2 M}{r}}
+ \frac{h^2}{r^2} = \begin{cases}
-1 & \text{massive particle} \\
0 & \text{massless particle.}
\end{cases}
\]

In principle we can integrate this to get $\lambda(r)$, $\phi(r)$
and $t(r)$.  For spatial orbits we use $u = r^{-1}$, so that
$\Dot{t} = -h \diff{u}{\phi}$.

For a massive particle the radial equation becomes
\[
h^2 \left( \diff{u}{\phi} \right)^2
= E^2 - (1 + h^2 u^2)(1- 2 M u).
\]

Taking $\dif{\phi}$ of this and dividing by $\diff{u}{\phi}$ we get
\[
\diff{^2 u}{\phi^2} + u = \frac{M}{h^2} + 3 M u^2.
\]

The massless version of this is
\[
\diff{^2 u}{\phi^2} + u = 3 M u^2.
\]

\section{Perihelion advance}

Consider bound orbits of a slow massive particle at large $r$ ($r \gg
M$).  We seek to solve the equation

\[
\diff{^2 u}{\phi^2} + u = \frac{M}{h^2} + 3 M u^2,
\]

which we will do by perturbation methods.  The zeroth approximation is
\[
u = \frac{1 + e \cos \phi}{l}, \qquad l = \frac{h^2}{m}.
\]

We iterate this: the next approximation to $u(\phi)$ satisfies
\[
\diff{^2 u}{\phi^2} + u = \frac{1}{l} + \frac{3 h^2}{l^3}
\left( 1+\tfrac{1}{2} e^2 +2 e \cos \phi + \tfrac{1}{2} e^2 \cos 2 \phi
\right).
\]

It will be a better approximation if $h \ll l$.  The solution of this
equation is
\[
l u = 1 + \tfrac{3 h^2}{l^2} (1 + \tfrac{1}{2} e^2)
+ \frac{h^2 e}{l} \left(3 \phi \sin \phi - \tfrac{1}{2} e \cos 2 \phi\right)
+ e \cos \phi.
\]

The $e \cos \phi$ comes from the zeroth order solution.  The aperiodic
$\phi \sin \phi$ term corresponds to an altered periodicity.  Note that
\[
e \cos \phi + \tfrac{3 h^2 e}{l^2} \phi \sin \phi
\approx e \cos \left( 1- \tfrac{3 h^2}{l^2}\right)\phi,
\]

and the periodicity in $\phi$ is approximately
$2 \pi \left( 1+ \tfrac{3 h^2}{l^2}\right)$.  If $a$ is
the semi-major axis with
\[
2 a = \frac{1}{u_{\text{min}}} + \frac{1}{u_{\text{max}}}
= \frac{2 l}{1-e^2}
\]
we can write the perihelion advance
\[
\frac{6 \pi h^2}{l^2} = \frac{6 \pi M}{l}
= \frac{6 \pi M}{a (1-e^2)},
\]
which is $\tfrac{6 \pi G M}{c^2 a(1-e^2)}$ in MKS units.

The orbit is approximately elliptical, but is slowly rotating (precessing).
The perihelion advance is $\tfrac{6 \pi M}{a (1-e^2)}$ per orbit.

In the solar system the largest effect is on Mercury --- the residual
precession (that not accounted for by $n$-body Newtonian effects) 
of $43''$ per century measured agrees with this calculated result.

In a binary pulsar this effect is much bigger --- about
$4^\circ$ per year.

\section{Light deflection}

Consider a particle on a null geodesic, satisfying the equation
\[
\diff{^2 u}{\phi^2} + u = 3 M u^2.
\]

\vspace{1.5in}

The zeroth order approximation for the light path is
$u = \frac{\sin \phi}{R}$.  The next approximation is
\[
u = \frac{\sin \phi + \tfrac{M}{2 R} (3 +\cos 2 \phi)}{R},
\]
keeping the symmetry about $\phi = \tfrac{\pi}{2}$.

The light path is bent and we need to find $\epsilon$.  We set $u = 0$
and $\sin \epsilon \approx \epsilon$, $\cos 2 \epsilon \approx 1$, so
that $\epsilon \approx \tfrac{2 M}{R}$.  This is
$\tfrac{2 G M}{c^2 R}$ in MKS units.

This is observed in the solar system for light from stars which passes
close to the sun at eclipses.

More detailed analysis of light deflection shows that the
$\cO(\tfrac{\phi}{c^2})$ corrections in $g_{ij}$ and $g_{tt}$
produce comparable contributions.  In other theories of matter
they combine differently, for instance Nordstr\"om's theory
predicts no light deflection.

\section{Black holes and the event horizon}

Consider the vacuum Schwarzschild metric near $r=2 M$ and look at
particles/photons falling towards $r = 2M$.  For radial infall we have
$(\theta,\phi)$ constant and $h=0$.  We want to solve the equations

\[
\left( 1- \frac{2 M}{r} \right) \diff{t}{\lambda} = E
\]

and
\[
\left( \diff{r}{\lambda} \right)^2 - E^2 = \begin{cases}
- \left( 1 - \frac{2 M}{r} \right) & \text{massive} \\
0 & \text{massless.}
\end{cases}
\]

In the massive case $\lambda = \tau$ and so
\[
\ud \tau = - \frac{\ud r}{\left( E^2 - 1 + \tfrac{2 M}{r}
\right)^{\frac{1}{2}}}.
\]

We clearly need $E^2 > 1$.  We see that $r \to 2 M$ in a finite proper
time
\[
\tau = - \int \frac{\ud r}{\left( E^2 - 1 + \tfrac{2 M}{r}
\right)^{\frac{1}{2}}}.
\]

The co-ordinate time is nastier.  We have
\[
\ud t = - \frac{E \ud r}{(1 - \frac{2 M}{r}) \left( E^2 - 1 +
    \tfrac{2 M}{r} \right)^{\frac{1}{2}}}
\]

and so $t \to \infty$ as $r \to 2 M$.  Something similar happens for
photons.  We conclude that $t$ is not a good co-ordinate for
analysing the metric near $r = 2 M$.  Instead we use a co-ordinate
tied to the incoming particles.  It is simplest to do photons.

Consider radially infalling photons, which satisfy the equation
\[
\diff{r}{t} = - \left( 1 - \frac{2 M}{r} \right).
\]

We can integrate this to find $t = -r - 2 M \log (r - 2M) + v$, where
$v$ is constant on photon paths.  We change co-ordinates from
$(r,\theta,\phi,t)$ to $(r,\theta,\phi,v)$, \emph{ingoing
  Eddington-Finkelstein co-ordinates}.  The Schwarzschild metric
becomes
\[
\ud s^2 = - \left( 1- \frac{2 M}{r} \right)\, \ud v^2 + 2 \ud v \ud r
+ r^2 \left( \ud \theta^2 + \sin^2 \theta\, \ud \phi^2\right).
\]

This metric is well-behaved down to $r = 0$ (except for
a trivial polar co-ordinate singularity at $\theta = (0,\pi)$).  It
has $\det g_{a b} < 0$ and canonical form $+++-$ everywhere in
$v > 0$.  It provides the ingoing extension of the Schwarzschild
metric through $r = 2 M$.  This is a simple \emph{co-ordinate singularity}.

However we find the curvature invariant
$R_{abdc}R^{a b d c} = \tfrac{48 M^2}{r^6}$
and so there is a genuine singularity of the spacetime at $r=0$.
This is a \emph{curvature singularity} and constitutes a boundary
of the spacetime.

On any worldline we need $\ud s^2 \le 0$ (equality iff photons).  Thus
\[
- \left( 1 - \frac{2 M}{r} \right)\ud v^2 + 2 \ud v \ud r \le 0
\]
with equality iff we have photons with $\ud \theta = \ud \phi = 0$.
In $r > 2 M$ the future light cone is defined by
\[
\ud v \ge \frac{2 \ud r}{1 - \frac{2 M}{r}} \qquad \ud v \ge 0.
\]

\vspace{2in}

In $r < 2 M$ we have
\[
\ud r \le \left( 1 - \frac{2 M}{r} \right)\ud v \qquad \ud v \ge 0
\]

and so $\ud r \le 0$.  Thus any particle in $r < 2 M$ inevitably
has $r$ decreasing to zero. Light or particles cannot escape from
$r < 2M$, but \emph{can} clearly escape from $r > 2 M$.  The
region $r < 2M$ is a \emph{black hole} and the boundary surface
$r = 2 M$ is its \emph{event horizon}.

\chapter{Matter in General Relativity}

Our final aim is to formulate the nongravitational laws of physics in
curved spacetime and to find the field equations of GR in the
presence of matter.

\section{Physical laws}

The equivalence principle means that all laws have their usual special
relativistic forms in any LIFs.  Moreover, the formulation of the laws
should be the same in any reference frame --- tensorial.  Therefore to
find physical laws we take the special relativistic laws and use them
at the centre of a LIF to find the curved space covariant law.

In a LIF at $P$, $g_{a b} = \eta_{a b}$, $g_{a b,d} =
0$ and $\Gamma^a_{b d} = 0$ and so covariant derivatives reduce
to partial derivatives.  Therefore to make a special relativistic law
covariant we replace partial derivatives with covariant derivatives
and $\eta_{a b}$ with $g_{a b}$.  This is \emph{minimal
  coupling} --- we do not make unnecessary changes to the flat space
laws.

For instance, consider free particle motion, which satisfies the
equation $\diff{^2 x^a}{\tau^2} = 0$ in a local inertial frame.
This becomes
\[
\diff{^2 x^a}{\tau^2} + \Gamma^a_{b d} \diff{x^b}{\tau}
\diff{x^d}{\tau} = \frac{\uD}{\ud \tau} \left( \diff{x^a}{\tau}\right)
= 0,
\]

which is the geodesic law.

A scalar field $\psi$ satisfying the wave equation $\Box \psi =
\eta^{a b} \psi_{,a b} = 0$ in flat spacetime becomes $g^{a
  b} \psi_{;a b} = 0$ in curved spacetime.

\section{Energy-momentum tensors}

The matter content of spacetime is described by an energy-momentum
tensor $T^{a b}$.

Consider a continuous medium of density $\rho$, without pressure
(``dust'').  $\rho$ is the proper density measured in the local
inertial rest frame.  Let $T^{a b} = \rho u^a u^b = T^{b
  a}$.  In a local inertial frame we have $T^{a b}_{\phantom{a
    b},b} = 0$ (by Navier-Stokes and the continuity equation) and
so the equations of motion in general co-ordinates are $T^{a
  b}_{\phantom{a b};b} = 0$.  In the Newtonian limit, with
gravity, the space parts of this give Navier-Stokes and the time part
gives the continuity equation.

All forms of matter have symmetric energy-momentum tensors $T^{a
  b}$ obeying $T^{a b}_{\phantom{a b};b}$.  This is
  ultimately because all quantum fields have a Lagrangian from which
  one can construct an energy-momentum tensor which is automatically
  conserved.

\section{The Einstein field equations}

We wish to generalise the vacuum Einstein equations $R_{a b} = 0$
to include matter sources and reproduce $\phi_{,ii} = 4 \pi G \rho$ in
the Newtonian limit.

\subsection{The Bianchi identities}

In a local inertial frame centred at $x^a=0$ we have $g_{a b} =
\eta_{a b} + \text{quadratic}$ and $\Gamma^a_{b d}=
\text{linear}$ in a Taylor expansion about $x^a = 0$.  Then
\[
R^a_{\phantom{a} b d c} = \Gamma^a_{b c, d}
- \Gamma^a_{b d,c} + \text{quadratic}
\]
and so
\[
R^a_{\phantom{a} b d c;e} = \Gamma^a_{b c, d e}
- \Gamma^a_{b d,c e} + \text{quadratic}.
\]

Hence at the origin of a local inertial frame,
$R^a_{\phantom{a} b [d c;e]} = \Gamma^a_{b
  [c, d e]} - \Gamma^a_{b [d,c e]} = 0$.  But
this is a tensorial equation, so $R^a_{\phantom{a} b [d
  c;\xi]} = 0$ everywhere.  These are the \emph{Bianchi
  identities}.

They can be equivalently written (using the symmetries of
$R^a_{\phantom{a} b d c}$) as
\[
R^a_{\phantom{a} b d c; e}
+ R^a_{\phantom{a} b c e; d}
+ R^a_{\phantom{a} b e d; c} = 0.
\]

contracting on $a$ and $c$, multiplying by $g^{b e}$ and
renaming the indices gives
\[
-R^a_{\phantom{a} b; a} + R_{; b} - R^a_{\phantom{a}b;
 a} = 0.
\]

We thus obtain the \emph{contracted Bianchi identities},
\[
\nabla_b\left(R^{a b} - \tfrac{1}{2} g^{a b} R \right) = 0
\]

\subsection{Field equations}

The contracted Bianchi identities suggest taking the field equations
\[
R^{a b} - \tfrac{1}{2} R g^{a b} = \kappa T^{a b}.
\]

These are the \emph{Einstein field equations}.  The Bianchi identities
then imply the conservation of energy-momentum automatically.  In
fact it can be shown that the left hand side of the Einstein equations
is the only possible tensorial expression linear in $g_{a b,d
  c}$, not involving higher derivatives, vanishing in flat
spacetime and with identically zero divergence.

One can verify the Newtonian limit: it turns out that $\kappa =
\frac{8 \pi G}{c^4}$ (Einstein's constant of gravitation).

Gravitation is nonlinear.  The gravitational field must carry energy,
although this can never be localised, since the geometry near any
point $P$ looks Minkowskian in a local inertial frame at $P$.

\backmatter

\begin{thebibliography}{9}

\bibitem{LandL} Landau and Lifschitz, \emph{The Classical Theory of
    Fields}, Fourth ed., Butterworth-Heinemann, 1975.
  
  {\sffamily \small The more I read this book the better I think it
    is.  The style of presentation is very different from that used in
    this course with least action principles being used whenever
    possible.  This style is more to my liking but you may have
    a different opinion.}

\bibitem{ElectroNotes} \emph{Electrodynamics}, unpublished, 1997.
  
  {\sffamily \small Another field theory which the interested reader
    may like to take a look at.  It was presented in a more quickfire
    manner than this course.}

\end{thebibliography}

\end{document}

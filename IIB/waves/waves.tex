\documentclass{notes}

\usepackage{varioref}

\newcommand{\te}[1]{\underline{\underline{#1}}}
\newcommand{\vk}{\vect{k}}
\newcommand{\bs}[1]{\boldsymbol{#1}}
\newcommand{\cO}{\mathcal{O}}
\newcommand{\cE}{\mathcal{E}}
\newcommand{\cA}{\mathcal{A}}
\newcommand{\cL}{\mathcal{L}}
\renewcommand{\Box}{\square}
\DeclareMathOperator{\dive}{div}
\DeclareMathOperator{\curl}{curl}
\DeclareMathOperator{\grad}{grad}

\begin{document}

\frontmatter

\title{Waves in Fluid and Solid Media}

\lecturer{Prof.~D.~G.~Crighton}
\maintainer{Paul Metcalfe}
\date{Lent 1998} \maketitle

\thispagestyle{empty}

\noindent\verb$Revision: 2.2 $\hfill\\
\noindent\verb$Date: 2002/01/17 15:27:59 $\hfill

\vspace{1.5in}

The following people have maintained these notes.

\begin{center}
\begin{tabular}{ r  l}
-- date & Paul Metcalfe
\end{tabular}
\end{center}

\tableofcontents

\chapter{Introduction}

These notes are based on the course ``Waves in Fluid and Solid Media''
given by Prof.~D.~G.~Crighton in Cambridge in the Lent Term 1998.
These typeset notes are totally unconnected with Prof.~Crighton.  The
recommended books for this course are discussed in the bibliography.

\alsoavailable
\archimcopyright

\mainmatter

\chapter{Sound waves}

\section{Equations of motion of a compressible fluid}

Assume the fluid is both homogeneous and non-dissipative.  We will
denote the pressure, density and fluid velocity at $(\vx,t)$ by
$p(\vx,t)$, $\rho(\vx,t)$ and $\vu(\vx,t)$ respectively.

The governing equations are derived from conservation of mass,
conservation of momentum and the laws of thermodynamics.

\subsection{Conservation of mass}

Consider an arbitrary fixed volume $V$ with surface $S$ and outward
normal $\vn$.  Then

\begin{align*}
\text{rate of increase of mass in $V$}
&= \text{mass flux across $S$} \\ &
+ \text{rate of creation of mass within $V$} \\
\ddt{} \int \rho \ud V & = - \int \rho \vu \cdot \vn\, \ud S
+ \int Q\, \ud V \\
& = - \int \dive \rho \vu\, \ud V + \int Q\, \ud V.
\end{align*}

We can make $V$ arbitrarily small to derive the continuity equation

\begin{equation}\label{eq:contin}
\pddt{\rho} + \dive \rho \vu = Q.
\end{equation}

\subsection{Conservation of momentum}

We use the same sort of argument to get

\begin{align*}
\ddt{} \int \rho \vu\, \ud V & = - \int \rho \vu (\vu \cdot \vn)
+ p \cdot \vn\, \ud S + \int Q \vu + \rho \vF\, \ud V \quad
\text{or in tensor form} \\
\ddt{} \int \rho u_i\, \ud V &= - \int \rho u_i u_j n_j + p n_i\, \ud S
+ \int Q u_i + \rho F_i\, \ud V \\
&= - \int \pd{}{x_j} \left( \rho u_i u_j \right) + \pd{p}{x_i}\, \ud V
+ \int Q u_i + \rho F_i\, \ud V.
\end{align*}

Making $V$ arbitrarily small we get

\begin{equation}\label{eq:consmom}
\pddt{} \rho u_i + \pd{}{x_j} \rho u_i u_j + \pd{p}{x_i} = Q u_i + \rho F_i
\end{equation}

and subtracting $u_i \times \eqref{eq:contin}$ from this we derive the
Euler equation

\begin{equation}\label{eq:euler}
\rho \DDt{\vu} = - \nabla p + \rho \vF.
\end{equation}

We now have five quantities ($\rho$, $p$ and $\vu$) to solve for
and only four equations.  The final equation comes from thermodynamics.

\subsection{Thermodynamics}

A perfect gas with specific volume $V = \rho^{-1}$ satisfies
\emph{Boyle's law}, $PV = R T$ and if $T$ is constant then this gives us
the fifth equation.  This is not overly realistic and we need something
further.

When a system is in thermodynamic equilibrium we have the functions
of state (for instance $p$, $\rho$ and $T$), any two of which serve to
define a simple fluid via an equation of state (for instance
$f(p,\rho,T) = 0$).

\subsubsection*{The first law}

\begin{center}
\parbox{3in}{ \itshape
Every thermodynamic system has a function of state $e$, the internal
energy per unit mass.  $e$ can be changed by either adding heat $Q$
and/or doing work $W$.  If $e$ is changed, then
\[
\ud e = \delta Q + \delta W.
\]
}
\end{center}

$Q$ and $W$ are not functions of state and they depend on how the system
was changed into its state.

An \emph{adiabatic process} is one in which $\delta Q = 0$.  A reversible
process is one which is very slow, and (ahem) reversible.

For a perfect gas we find that $e = C_V T$.  We also have an equation 
for $\delta W$, ${\delta W = - p \ud V}$.

\subsubsection*{The second law}

\begin{center}
\parbox{3in}{ \itshape
All thermodynamic systems have a function of state, the entropy
(per unit mass) $S$ such that $\delta Q = T \ud S$ in any reversible
process (thus defining $T$).  This can also be stated as
``$T^{-1}$ is an integrating factor for $\delta Q$''.
}
\end{center}

We can solve the equations we have to find

\begin{equation}\label{eq:entropy}
S = C_V \log \frac{p}{\rho^\gamma} + \text{const},
\end{equation}

where $\gamma = \frac{R + C_V}{C_V}$ is the specific heat ratio.
$R + C_V = C_P$, the specific heat at constant pressure, whereas $C_V$
is the specific heat at constant volume.

In an ideal fluid there is no heat transfer between particles
(no diffusion of momentum by viscosity, no diffusion of heat by
thermal conductivity) and so $\delta Q = 0$.  Thus we have the final
equation we need,

\begin{equation}\label{eq:consent}
\DDt{S} = 0.
\end{equation}

Processes with $\DDt{S} = 0$ are called \emph{isentropic} and processes
with $S$ the same constant for all particles are called \emph{homentropic}.

In summary, for a perfect gas:

\begin{equation}\label{eq:gassum}
p = \rho R T \quad \rho = C_V T \quad \frac{p}{\rho^\gamma}
= \text{const} \times \exp \frac{S}{C_V}.
\end{equation}

If the system is homentropic then

\begin{equation}\label{eq:homent}
\frac{p}{p_0} = \left(\frac{\rho}{\rho_0}\right)^\gamma \quad
\text{and} \quad \diff{p}{\rho} = \frac{\gamma p}{\rho}.
\end{equation}

For a monatomic gas $\gamma = \tfrac{5}{3}$ and for a diatomic
gas $\gamma = \tfrac{7}{5}$.  Air is approximately diatomic.  To fit
a curve of this form to water, $\gamma \sim 7$.

\section{Linear acoustic waves}

From $\pddt{} \eqref{eq:contin} + \pd{}{x_i} \eqref{eq:consmom}$
we get

\begin{equation}\label{eq:fullwave}
\pd{^2 \rho}{t^2} - \nabla^2 p = \pddt{Q} - \pd{}{x_i}
\left( \rho F_i + Q u_i \right) + \pd{^2}{x_i x_j}
\left( \rho u_i u_j \right).
\end{equation}

This has an exact solution, the rest state $\vu = 0$, $p=p_0$,
$\rho = \rho_0$, $Q = 0$ and $\vF = 0$.  We make small perturbations

\begin{align*}
\rho &= \rho_0 + \Tilde{\rho} &
p &= p_0 + \Tilde{p} &
\vu &= \Tilde{u} \\
Q &= \Tilde{Q} &
\vF &= \Tilde{\vF} \\
\end{align*}
with $\abs{\Tilde{\rho}} \ll \abs{\rho_0}$,
$\abs{\Tilde{p}} \ll \abs{p_0}$, and $\Tilde{\vu}$, $\Tilde{Q}$
and $\Tilde{\vF}$ all small.

Then to first order, $\rho F_i = \rho_0 F_i$ and
$Q u_i$ and $\rho u_i u_j$ can be neglected.%
\footnote{About 6 lectures of a Part III course have just vanished
  with the casual neglect of $\rho u_i u_j$.  If this is included we
  get Lighthill's model of aerodynamic noise.}
The linear approximation to \eqref{eq:fullwave} is

\begin{equation}\label{eq:linwave}
\pd{^2 \Tilde{\rho}}{t^2}
- \nabla^2 \Tilde{p} = \pddt{\Tilde{Q}} - \rho_0 \dive \vF.
\end{equation}

$S$ is uniform, so $p = p(\rho,S_0)$.  Thus
\[
p_0 + \Tilde{p} = p_0 + \Tilde{\rho} \diff{p}{\rho} (\rho_0,S_0)
\]

and hence $\Tilde{p} = c_0^2 \Tilde{\rho}$, with
$c_0^2 = \diff{p}{\rho} (\rho_0,S_0) = \tfrac{\gamma p_0}{\rho_0}$.

\subsection{The wave equation}

\eqref{eq:linwave} becomes

\begin{equation}\label{eq:waveqn}
\left( \nabla^2 - \frac{1}{c_0^2} \pd{^2}{t^2} \right) \Tilde{p}
= - \pddt{\Tilde{Q}} + \rho_0 \dive \vF.
\end{equation}

The operator $\nabla^2 - \frac{1}{c_0^2} \pd{^2}{t^2}$ is written
$\Box^2$.

Assuming no $Q$ and $\vF$, the governing equations \eqref{eq:contin}
and \eqref{eq:euler} linearise to give

\begin{equation}\label{eq:lingov}
\pddt{\Tilde{\rho}} + \rho_0 \dive \vu = 0 \qquad
\rho_0 \pddt{\vu} = - \nabla p.
\end{equation}

We find that $\Box^2 \dive \Tilde{\vu} = 0$ and 
$\Box^2 \pddt{\vu} = 0$.  Thus $\Box^2 \vu = \bs{\alpha}(\vx)$ 
and $\dive \bs{\alpha} = 0$.  If the motion starts from rest then
$\bs{\alpha} \equiv 0$.  We conclude that
$\Box^2 (\Tilde{p}, \Tilde{\rho}, \Tilde{\vu}) = 0$.

\subsection{Potential}

We solve these equations by a potential method.  Taking the curl
of \eqref{eq:lingov} we obtain the vorticity equation,
$\pddt{\bs{\omega}} = 0$.  Thus $\vu = \nabla \phi + \Tilde{\vu}_0(\vx)$,
and if the motion starts from rest then $\Tilde{\vu}_0 \equiv 0$.

Then \eqref{eq:lingov} gives
\[
\Tilde{\rho} = \frac{\Tilde{p}}{c_0^2} = - \frac{\rho_0}{c_0^2}
\pddt{\phi} + \beta(t).
\]

By redefining $\phi$ we can take $\beta \equiv 0$ and so $\Box^2 \phi = 0$.

All of $\phi$, $\Tilde{\vu}$, $\Tilde{\rho}$ and $\Tilde{p}$
satisfy the wave equation with propagation speed

\begin{equation}\label{eq:wavespeed}
c_0 = \left( \pd{\rho}{p} \right)^{\frac{1}{2}}_{S_0}
= \left( \frac{\gamma p_0}{\rho_0} \right)^{\frac{1}{2}}.
\end{equation}

$c_0 \approx 340 ms^{-1}$ in air at sea level and $\approx 1500 ms^{-1}$
in water.

\subsection{Energy density, energy flux}

We use equations \eqref{eq:lingov} to derive an energy equation,

\begin{equation}\label{eq:energy}
\pddt{} \left( E_k + E_p \right)  + \dive \vect{I} = 0,
\end{equation}

where the kinetic energy density $E_k = \tfrac{1}{2} \rho_0
\abs{\Tilde{u}}^2$, the potential energy density
$E_p = \tfrac{1}{2} \frac{c_0^2 \Tilde{\rho^2}}{\rho_0}$
and the energy flux vector (or acoustic intensity)
$\vect{I} = \Tilde{p} \Tilde{\vu}$.  Integrating \eqref{eq:energy}
over a fixed volume $V$ we see that the rate of increase of
energy in $V$ equals the rate of working of pressure forces on $\partial V$.

\subsection{Plane waves}

We seek to solve the wave equation $\Box^2 \phi = 0$.  $\phi
= f(\vk \cdot \vx - \omega t)$ is a solution for any (sufficiently smooth)
$f$ if $c_0^2 k^2 = \omega^2$.  This is a dispersion relation, and relates
the space and time parameters of the solution.

By Fourier's theorem, any $G(\vx,t)$ can be expressed as

\begin{align*}
G(\vx,t) &= \int \ud^3 k\, \ud \omega\, \Hat{G}(\vk,\omega) e^{-\imath
( \vk \cdot \vx - \omega t)} \quad \text{where} \\
\Hat{G}(\vk,\omega) &= \frac{1}{\left(2 \pi\right)^4} \int \ud^3 x\, \ud t\,
G(\vx,t) e^{\imath( \vk \cdot \vx - \omega t)}.
\end{align*}

That is, any $G$ can be written as a sum of harmonic plane
components $e^{-\imath( \vk \cdot \vx - \omega t)}$ and plane wave solutions
can be added to generate more general solutions.

At any $t$, $\phi$ is constant on planes $\vk \cdot \vx = \text{const}$.
$\vk$ is called the \emph{propagation vector}.  If we take axes parallel
to $\vk$ with co-ordinate $X$ then $\phi = f(k X - \omega t)$ with
$\omega = \pm k c_0$.  Thus

\begin{equation}\label{eq:planewave}
\phi = f(k X - k c_0 t) + g(k X + k c_0 t).
\end{equation}

\vspace{1in}

\subsection{Properties of plane waves}

We have
\[
\Tilde{\vu} = \nabla \phi = \vk \left( f'(k X - k c_0 t) +
g'(k X + k c_0 t) \right).
\]

$\Tilde{\vu}$ is parallel to $\vk$.  These are longitudinal waves of
compression and rarefaction.  We also have
\[
\Tilde{p} = - \rho_0 \phi_t = \rho_0 c_0 k \left(
f'(k X - k c_0 t) - g'(k X + k c_0 t) \right).
\]

We see that for a single wave with $\Tilde{\vu} = \Hat{\vk} u$
we have $\Tilde{p} = \rho_0 c_0 u$ if $g = 0$ and $\Tilde{p}
= - \rho_0 c_0 u$ if $f=0$.  $\rho_0 c_0$ is called the \emph{specific
acoustic impedance}.

For a single wave with (for example) $g=0$, $E_k = Ep = \tfrac{1}{2}
\rho_0 k^2 f'{}^2$.  There is thus \emph{instananeous equipartition
of energy}.  The energy flux vector is
\[
\vect{I} = \left( E_k + E_p \right) c_0 \Hat{\vk}.
\]

Energy is transported at a speed $c_0$ in direction $\Hat{\vk}$.

\subsubsection*{Plane harmonic waves}

For a plane harmonic wave we have
\[
\phi = \left( \Re \text{ or } \Im\right) \left( A
e^{\imath \left(\vk \cdot \vx - \omega t \right)} \right).
\]

$A$ is called the \emph{complex amplitude}, $\vk \cdot \vx - \omega t$
is the \emph{phase}, $\omega$ is the \emph{angular frequency}
and $\vk$ is the \emph{wavevector} or \emph{wavenumber}.  $k$
is also called the \emph{wavenumber}.

The wavelength $\lambda = \frac{2 \pi}{k}$ and the period
is $T = \frac{2 \pi}{\omega}$.  The dispersion relation for
acoustics is
\[
\omega^2 - c_0^2 k^2 = 0.
\]

\section[Nonlinear acoustics]%
{Finite amplitude soundwaves (Nonlinear acoustics)}

We want to see what the effect of nonlinearity is on soundwaves.  We
work in a one dimensional homentropic system.  We have the governing
equations

\begin{gather}
\pddt{\rho} + \pd{}{x} \rho u = 0 \label{eq:1dmass} \\
\pddt{u} + u \pd{u}{x} = - \frac{1}{\rho} \pd{\rho}{x} \label{eq:1deuler} \\
\pd{p}{x} = c^2(\rho) \pd{\rho}{x} \label{eq:1dpress},
\end{gather}

where $c^2(\rho) = \diff{p}{\rho}$.

\subsection{Riemann analysis}

We try to arrange \eqref{eq:1dmass} and \eqref{eq:1deuler} in a more
symmetric form.  We try $\lambda \times \eqref{eq:1dmass}
+ \eqref{eq:1deuler}$ and get
\[
\left( \pddt{} + ( u+ \lambda \rho) \pd{}{x}\right) u + \lambda \left(
\pddt{} + \left( u + \frac{c^2}{\lambda \rho} \right) \pd{}{x}
\right) \rho = 0.
\]

Letting $\lambda = \pm \frac{c}{\rho}$ we get
\[
\left(\pddt{} + ( u \pm c) \pd{}{x}
\right) u \pm \frac{c}{\rho} \left( \pddt{} + ( u \pm c) \pd{}{x}
\right) \rho = 0.
\]

Defining

\begin{equation}\label{eq:Q}
Q(\rho) = \int_{\rho_0}^\rho \frac{c(\rho')}{\rho'}\, \ud \rho',
\end{equation}

which satisfies $Q_t = \frac{c}{\rho}\rho_t$ and $Q_x = \frac{c}{\rho}
\rho_x$ the governing equations become

\begin{equation}\label{eq:Riemann}
\left(\pddt{} + \left( u \pm c \right) \pd{}{x} \right)
\left( u \pm Q \right) = 0.
\end{equation}

We define the \emph{Riemann invariants}, $R_\pm = u \pm Q$ and make a
change of co-ordinates $(x,y) \mapsto (\xi,\eta)$, where the curves
$\xi$ constant are the $C_+$ characteristics defined by $\diff{x}{t} =
u + c$ and the curves $\eta$ constant are the $C_-$ characteristics
defined by $\diff{x}{t} = u - c$.  On each member of the $C_+$ family,
\[
\diff{R_+}{t} = \pd{R_+}{t} + \pd{R_+}{x} \left.\diff{x}{t}
\right|_{\xi = \text{const}} = 0 \text{ by \eqref{eq:Riemann}.}
\]

Thus $R_+$ is constant on each $C_+$, so $R_+ = R_+(\xi)$ only.
Similarly $R_-$ is constant on each $C_-$ and $R_- = R_-(\eta)$ only.
We can find $u$ and $Q$ from $R_\pm$, as
\[
u = \frac{R_+(\xi) + R_-(\eta)}{2} \qquad
Q = \frac{R_+(\xi) - R_-(\eta)}{2}.
\]

Waves carrying constant values of $R_\pm$ propagate along the $C_\pm$
curves at speed $u \pm c$, that is at speeds $\pm c$ relative to the
local velocity $u$.

\subsection{Perfect gas}

This has $\tfrac{p}{p_0} = \left( \tfrac{\rho}{\rho_0}
\right)^\gamma$, and so $c^2 = \tfrac{\gamma p}{\rho}$.  Thus
$\tfrac{c}{c_0} = \left( \tfrac{\rho}{\rho_0}
\right)^{\tfrac{\gamma-1}{2}}$.  We can also find $Q$,

\[
Q = \int_{\rho_0}^\rho \frac{c(\rho')}{\rho'}\, \ud \rho'
= \left( \frac{2}{\gamma - 1} \right) \left( c - c_0 \right)
\]

Thus
\begin{equation}\label{eq:charC}
R_\pm = u \pm \tfrac{2 (c-c_0)}{\gamma - 1},
\end{equation}
if $R_\pm = 0$ when $u = 0$ and $c = c_0$.

\subsection[Cauchy problem]{Cauchy problem --- solution by characteristics}

Suppose $u(x,0) = U(x)$, $\rho = \rho_0$ and $P = P(x)$, or equivalently
$c = c_0 + C(x)$.  Assume that the initial disturbance has compact
support, $u=0$ and $C = 0$ for $x < x_B$ and $x > x_F$.  We want
the solution at $(X,\delta t)$ for $\delta t \ll 1$.

\vspace{1in}

Draw characteristics through $(X,\delta t)$ back to $t=0$.  Approximately,
$C_+$ is

\begin{align*}
x &= X + \left[ u(X,\delta t) + c(X,\delta t) \right] (t-\delta t)
+ \cO(\delta t)^2 \\
&=  X - \left[ U(X) + c_0 + C(x) \right] \delta t + \cO(\delta t)^2.
\end{align*}

$R_+$ is constant on $C_+$ and so

\begin{align*}
R_+ &= R_+\left(X - \left\{U + c_0 + C \right\} \delta t \right) +
\cO(\delta t)^2 \qquad \text{and similarly} \\
R_- &= R_+\left(X - \left\{U - c_0 - C \right\} \delta t \right) +
\cO(\delta t)^2.
\end{align*}

Thus $u$ and $c$ are now known at $(X,\delta t)$ if known at
$(x,0)$.

\subsubsection*{Numerical recipe}

\begin{enumerate}
\item At $t=0$ we know $u$ and $c$.
\item Calculate $R_\pm$ at $t=0$ using \eqref{eq:charC}.
\item Calculate $R_\pm$ at $t = \delta t$ using above approximation.
\item Calculate $u$, $c$ at $t = \delta t$ using \eqref{eq:charC}.
\end{enumerate}

Repeat as necessary.

\subsection{General picture}

\vspace{1.5in}

\begin{itemize}
\item On characteristics that pass through $x < x_B$ or $x > x_F$ at
$t=0$, $u=0$, $c=c_0$ and $Q=0$.  Hence for $\xi < \xi_B$ and
$\xi > \xi_F$, $R_+ = 0$ and for $\eta < \eta_B$ and $\eta > \eta_F$
$R_- = 0$.  Thus in regions 1,2 and 3, $R_+ = R_- = 0$, $u = 0$,
$c = c_0$ and $Q = 0$.  This is undisturbed flow and all characteristics are
straight lines.

\item  In region 4, $R_- = u - Q = 0$ and so $u = Q$.  On $C_+$,
$R_+ = u + Q$ is constant and so $u$, $\rho$ and $c$ are constant on $C_+$
curves.  Also, $\diff{x}{t} = u+c$ is constant and so characteristics
$C_+$ are straight lines.  This corresponds to a right-running simple wave.

A similar analysis applies to region 5, except that we have a left-running
simple wave and the $C_-$ are straight.

\item Region 6 is the region of compound flow where neither $C_+$
nor $C_-$ are straight.  At $t=t_c$ this region disintegrates.
\end{itemize}

\subsection{Simple waves}

A \emph{simple wave} is one where one of $R_\pm$ is uniformly constant.
WLOG take $R_- = 0$ and so $c = c_0 + \tfrac{1}{2} (\gamma - 1)u$,
$u = Q$.  Thus $R_+ = 2 u$.  On $C_+$ characteristics
$\diff{x}{t} = u + c = c_0 + \tfrac{1}{2}(\gamma + 1)u$.  Thus
$C_+$ are straight lines.

The problem we have to solve equivalent to either

\begin{gather}
\diff{u}{t} = 0 \text{ on } \diff{x}{t} = c_0 + \tfrac{1}{2} (\gamma + 1) u
\label{eq:simpl1} \qquad \text{or} \\
\left( \pddt{} + \left( c_0 + \tfrac{1}{2} (\gamma + 1)u\right)
\pd{}{x} \right)u = 0. \label{eq:simpl2}
\end{gather}

We know that $u(x,t) = f(\xi)$ where from initial data, $f(\xi) = u(\xi,0)$.

Hence, from \eqref{eq:simpl1} and imposing $x=\xi$ at $t=0$ we have

\begin{equation}\label{eq:simpsoln}
x = \left(c_0 + \tfrac{1}{2} (\gamma +1)f(\xi) \right)t + \xi.
\end{equation}

For given $x$, $t$ we can in principle solve for $\xi$ and then find
$u = f(\xi,0)$.  $u$ gives $c$ and $\rho$, $c = c_0 + \tfrac{1}{2}(\gamma - 1)
u$ and
\[
\frac{\rho}{\rho_0} = \left(1 + \frac{(\gamma - 1)u}{c_0}
\right)^{\frac{2}{\gamma-1}}.
\]

\subsection{Generic form}

Equation \eqref{eq:simpl2} can be written in a more suggestive form.
Let $X = x-c_0 t$ and introduce the \emph{excess wavespeed}
$v = \left( \tfrac{\gamma + 1}{2} \right)u = u + c - c_0$.
Then \eqref{eq:simpl2} becomes

\begin{equation}\label{eq:kinwave}
\pddt{v} + v \pd{v}{X} = 0.
\end{equation}

This is a special case of the kinematic wave equation.  \eqref{eq:kinwave}
has $C_+$ characteristics $X = \xi + v(\xi,0) t$ and so
\begin{equation}\label{eq:kwsoln}
v(X,t) = v(\xi,0)= v(X - v(\xi,0)t,0),
\end{equation}

and $\rho(X,t) = \rho(X - v(\xi,0)t,0)$.  Thus density propagates at
speed $v$ (in a frame moving with velocity $c_0$).

\vspace{2in}

Each element of the waveform propagates at speed $v$.  Deformation
is pure distortion and there are no new values of $v$ (or $\rho$).

\subsection{Wave steepening}

We calculate the slope of the wave, $\pd{v}{X}
= \pd{v(\xi,0)}{\xi} \pd{\xi}{X}$.  From \eqref{eq:kwsoln} we have
\[
1 = \pd{\xi}{X} \left( 1 + \pd{v(\xi,0)}{\xi} \right)
\]
and so we have $\pd{v}{X} = \frac{v_\xi(\xi,0)}{1 + v_\xi(\xi,0) t}$.

Parts of the profile with $v_\xi(\xi,0) < 0$ and parts of the profile with
$v_\xi(\xi,0) > 0$ get flatter.  If $v_\xi(\xi,0) < 0$ somewhere then
a triple-valued waveform will form at some finite time.  This is unphysical.
The moment at which the slope becomes infinite is referred to as the
\emph{shock formation time}.

The profile becomes multivalued at the first time that
$t = - \left( v_\xi(\xi,0) \right)^{-1}$ and so the shock formation
time is
\[
t_s = \min_{\xi} - \frac{1}{v_\xi(\xi,0)}
= \left( \frac{2}{\gamma+1} \right) \frac{1}{\max_\xi - u_\xi(\xi,0)}.
\]

If $-u_\xi(\xi,0)$ is maximised at $\xi = \xi_s$ then the position at
which the shock forms is $X_s = \xi_s + v(\xi_s,0)t_s$ and
the \emph{position of shock formation} is $x_s = X_s + c_0 t_s$.

\section{Piston problems}

\vspace{1in}

At $t=0$, $u=0$ and $c=c_0$ in $x \ge 0$.  The piston moves with
$x = X_p(t)$ and $X_p(0) = 0$.

\subsection{Piston moving away from gas}

\vspace{2in}

Assume all characteristics $C_-$ start in $x>0$ at $t=0$.  Then
$R_-=0$ everywhere and we have simple flow.  The $C_+$ are straight lines.

In region 1, $C_+$ meets $t=0$ in $x > 0$ and so $R_+ = 0$.  Thus
$u = 0$ and $c=c_0$.

In region 2, $C_+$ meets the piston path at $t=\tau$ and $x=X_p(\tau)$.
The boundary conditions give $u=\dot{X_p}(\tau)$ at this point. $C_+$
has the equation

\begin{equation}\label{eq:pist1}
x - X_p(\tau) = \left( c_0 + \tfrac{1}{2}(\gamma-1)\right)(t - \tau).
\end{equation}

$\tau$ plays the same r\^ole as $\xi$.  Given $x$, $t$ in region 2
we can solve \eqref{eq:pist1} (numerically) to find $\tau$.  Then
$u(x,t) = \dot{X_p}(\tau)$.  We also find

\[
\rho = \rho_0 \left(1 + \frac{\gamma - 1}{2} \frac{\dot{X_p}(\tau)}{c_0}
\right)^{\frac{2}{\gamma-1}}
\]

and hence $\rho = 0$ if $\dot{X}_p = - \frac{2 c_0}{\gamma - 1}$.
$\tfrac{2 c_0}{\gamma - 1}$ is the maximum speed of the piston for contact
with the gas to be maintained.  It is the speed of expansion of a gas
into a vacuum (escape speed).

We also find
\[
\frac{c}{c_0} = 1 + \frac{\gamma + 1}{2} \frac{\dot{X_p}(\tau)}{c_0}.
\]

\subsection{Piston moves into gas}

\vspace{2in}

The equation of the $C_+$ characteristics is,
\[
x = X_p(\tau) + \left( c_0 + \frac{\gamma - 1}{2} \dot{X_p}(\tau) \right)
(t - \tau).
\]

Characteristics touch when $\pd{x}{\tau} = 0$, that is, when
\[
\tfrac{1}{2}\left(\gamma + 1 \right) \ddot{X_p} t
= \tfrac{1}{2} \left( \gamma + 1 \right) \ddot{X_p} \tau + c_0
+ \tfrac{1}{2} (\gamma - 1) \dot{X_p}.
\]

If we have uniform acceleration, $X_p(\tau) = \tfrac{1}{2} f \tau^2$ then
characteristics touch when
\[
t=t_0 = \frac{2 ( \gamma f \tau + c_0 )}{(\gamma + 1) f}
\]

and the minimum overlap time (shock formation time) is
\[
t_s = \min_{\tau \ge 0} t_0 = \frac{2 c_0}{(\gamma + 1)f} \qquad \text{at }
\tau = 0.
\]

\section{Shock waves}

A triple-valued region is unphysical.  In practice it is prevented
by mechanisms so far neglected, like dissipation.  When the waveform
is steep, viscous and heat conduction terms which were small become large.

The simple wave equation \eqref{eq:simpl2} can be modified to
\begin{equation}\label{eq:viscwave}
u_t + \left(c_0 + \tfrac{1}{2} (\gamma + 1) u\right) u_x = \nu u_{xx}. 
\end{equation}

The extra term is proportional to viscosity.  Analytic solutions to this
equation are known (see Part 3 course).  The triple valued region is avoided by
a region of rapid change, the thickness of which is proportional to $\nu$.

\vspace{1.5in}

Outside the so-called ``shock'' region we can use the characteristic
solution of the simple wave equation.  For real problems we cannot find
exact solutions to \eqref{eq:viscwave} and we try to treat the shock
as a discontinuity when $\nu$ is small.

\subsection{Rankine-Hugoniot relations}

Assume we have a steady discontinuity separating 2 uniform flows.
Take co-ordinates moving with the shock.

\vspace{1in}

Apply mass conservation across the shock to get

\begin{equation}\label{eq:rankmass}
\rho_1 u_1 = \rho_2 u_2.
\end{equation}

Momentum conservation gives

\begin{equation}\label{eq:rankmom}
p_1 + \rho_1 u_1^2 = p_2 + \rho_2 u_2^2.
\end{equation}

Finally, we apply energy conservation to get

\begin{equation}\label{eq:ranken}
\left( \tfrac{1}{2} \rho_1 u_1^2 + \rho_1 e_1 \right) u_1 + p_1 u_1 =
\left( \tfrac{1}{2} \rho_2 u_2^2 + \rho_2 e_2 \right) u_2 + p_2 u_2,
\end{equation}

where $e(\rho,S)$ is the internal energy per unit mass.  \eqref{eq:ranken}
simplifies using \eqref{eq:rankmass} to give

\begin{equation}\label{eq:ranken2}
\tfrac{1}{2} u_1^2 + e_1 + \frac{p_1}{\rho_1} =
\tfrac{1}{2} u_2^2 + e_2 + \frac{p_2}{\rho_2}.
\end{equation}

These are the \emph{Rankine-Hugoniot relations}.

\subsection{Hugoniot adiabatic}

From \eqref{eq:rankmass} and \eqref{eq:rankmom} we get

\[
u_1^2 = \frac{p_2 - p_1}{\rho_1} + \frac{\left( \rho_2 u_2 \right)^2}{\rho_1
\rho_2} = \frac{p_2 - p_1}{\rho_1} + \frac{\rho_1 u_1^2}{\rho_2}.
\]

Solving this, we get
\begin{equation}\label{eq:shockU}
u_1^2 = \frac{\rho_2 (p_2 - p_1)}{\rho_1(\rho_2 - \rho_1)}
\end{equation}

and so if $p_2 > p_1$, $\rho_2 > \rho_1$.  Similarly if $p_2 < p_1$ then
$\rho_2 < \rho_1$.

Substituting for $u_1^2$ and $u_2^2$ into \eqref{eq:ranken2} we get

\begin{equation}\label{eq:hugon}
e_2 - e_1 = \tfrac{1}{2} \left(p_1 +p_2 \right) \left(\frac{1}{\rho_1}
- \frac{1}{\rho_2} \right).
\end{equation}

This is the \emph{Hugoniot adiabat(ic)} and relates the thermodynamic
properties across the shock.

\subsection{Entropy jump}

For a given mass of gas, plot $p$ against $V$ ($=\rho^{-1}$, the
\emph{specific volume}) at constant $S$.  Assume that $\pd{^2 p}{V^2} >0$,
which is true for most normal fluids.

\vspace{2in}

Suppose that $\rho_2 > \rho_1$ and $p_2 > p_1$ and so $V_1 > V_2$.  If 
entropy is constant then $p_1$ and $p_2$ are given by $A$ and $B$.  Writing
down the first law, $\ud e = T \ud S - p \ud V$ and putting $\ud S =0$
we see that
\[
e_2 - e_1 = - \int_{V_1}^{V_2} p \ud V
= \text{area under curve $AB$.}
\]

But $e_2 - e_1$ is the area under the \emph{chord} $AB$
(using \eqref{eq:hugon}), and so $\ud S \neq 0$ and $A$ and
$B$ must have different values of $S$.  If additionally
$\pd{e}{p} > 0$ then \eqref{eq:hugon} can only be satisfied if
$p_2 > p_1$. Now
$\pd{e}{p} = T \pd{S}{p} > 0$ and so the entropy at $p_2$ is greater
than the entropy at $p_1$. As entropy increases with time, $p_2$
must be the downstream pressure.  For such a fluid shocks are compressive ---
the downstream $p_2$, $\rho_2$ are greater than the upstream
$p_1$, $\rho_1$.

Some strange hydrocarbons (for instance Freon-13) have
$\pd{^2 p}{V^2} < 0$ and have rarefaction shocks, but entropy \emph{always}
increases downstream.

\subsection{Upstream/downstream speeds}

\vspace{1.5in}

Using $\rho_2 > \rho_1$ and \eqref{eq:shockU} we have
\begin{align*}
u_1^2 & = \frac{\rho_2 (p_2 - p_1)}{\rho_1 (\rho_2 - \rho_1)} \\
& > \frac{p_2 - p_1}{\rho_2 - \rho_1} \\
& = \text{gradient of $CD$} \\
& > \text{gradient of tangent at $D$} = c_1^2.
\end{align*}

Thus the upstream flow is supersonic.  $M_1$, the upstream Mach number,
is greater than $1$.  It is harder to show that $M_2$ (the downstream
Mach number) is less than $1$ (see Landau and Lifschitz).

\subsection{Perfect gas}

This has

\begin{equation}\label{eq:pgshock}
c^2 = \frac{\gamma p}{\rho} \quad
p = \rho R T \quad e = C_V T = \frac{1}{\gamma - 1} \frac{p}{\rho}.
\end{equation}

We define the shock strength $\beta = \frac{p_2 - p_1}{p_1}$.  From
\eqref{eq:hugon} and \eqref{eq:pgshock} we have
\begin{equation}\label{eq:pgshock2}
\frac{\rho_2}{\rho_1} = \frac{2 \gamma + (\gamma + 1) \beta}{
2 \gamma + (\gamma - 1) \beta}.
\end{equation}

We can also get
\begin{equation}\label{eq:shockmach}
\begin{gathered}
  M_1^2 = \left(\frac{u_1}{c_1}\right)^2 = 1 + \frac{\gamma +1 }{2
    \gamma} \beta > 1 \\
  M_2^2 = \left(\frac{u_2}{c_2}\right)^2 = \frac{1}{1 + \beta} \left(
    1 + \frac{\gamma -1}{2 \gamma} \beta \right) < 1.
\end{gathered}
\end{equation}

\subsection{Strong shocks}

In the limit as $\beta \to \infty$
\[
\frac{\rho_2}{\rho_1} \nearrow \frac{\gamma + 1}{\gamma - 1}.
\]

For air, with $\gamma = 1.4$, the maximum compression by a shock is
by a factor of $6$.  We also find
\[
M_1^2 \sim \left(\frac{\gamma + 1}{2 \gamma} \right) \beta \quad \text{and}
\quad M_2^2 \to \frac{\gamma -1}{2 \gamma}.
\]

\subsection{Weak shocks}

As $\beta \to 0$, we find

\begin{align*}
\frac{\rho_2}{\rho_1} &= 1 + \frac{\beta}{\gamma} + \cO(\beta^2) \\
M_1^2 & = 1+ \left( \frac{\gamma +1}{2 \gamma}\right) \beta
+ \cO(\beta^2) \\
M_2^2 & = 1 - \left( \frac{\gamma +1}{2 \gamma}\right) \beta
+ \cO(\beta^2).
\end{align*}

The variations are all $\cO(\beta)$.  However, the variation in the
entropy, $S=C_V \log \frac{p}{\rho^\gamma}$ is
\[
\frac{S_2 - S_1}{C_V} = \log \left(1 + \beta\right)- \gamma
\log \left( \frac{2 \gamma + (\gamma + 1)\beta}{2 \gamma + (\gamma-1) \beta}
\right) = \left( \frac{\gamma^2 - 1}{12 \gamma^2}\right) \beta^3
+ \cO(\beta^4).
\]

This is \emph{much} smaller than the jumps in $e$, $\rho$, $p$, $M$
(etc.)

\subsection{Moving shocks}

\vspace{1.5in}

$u_1 = v_1 - U$ and $u_2 = v_2 -U$.  If $U > v_2$ then $p_2$, $\rho_2$
and $u_2$ become the ``upstream'' variables.  We need to take care
with the square roots in \eqref{eq:shockmach}.

\subsection{Weak shocks in right running simple waves}

In general the entropy is different behind and in front of a shock and we
thus expect jumps in the Riemann invariants.  However, in weak shocks the
jumps in $\rho$, $p$ and $u$ are all $\cO(\beta)$ and the jump in
$S$ is $\cO(\beta^3)$.  It can be shown that if $U > v_2$ (in the notation
of the last section),
\[
\left[ R_- \right]_1^2 = \cO(\beta^3) \text{ also.}
\]

Thus in weakly nonlinear right running simple waves we can still take
$S=S_0$ and $R_- \equiv 0$ everywhere, even if shocks are formed.

We work with the excess wavespeed $v = u + c - c_0 = \frac{\gamma+1}{2} u$
and let $X = x - c_0 t$.  Then (as before) we must solve
\[
\pddt{v} + v \pd{v}{X} = 0.
\]

After the shock forms at $t = t_s$, we want to know where it is located
on the graph of $v$ against $X$.

\vspace{1.5in}

Mass conservation gives $\int_{-\infty}^\infty \rho\, \ud x$ constant,
and so the shock cuts off lobes of equal area on the graph of $\rho$.
However,
\[
\frac{\rho}{\rho_0} = \left( \frac{c}{c_0} \right)^{\frac{2}{\gamma-1}}
= \left( 1 + \frac{\gamma-1}{2} \frac{u}{c_0} \right) =
1+ \frac{2}{\gamma+1} \frac{v}{c_0} +\cO(\tfrac{v}{c_0})^2.
\]

If $\tfrac{v}{c_0} \ll 1$, then $\int_{-\infty}^\infty v\, \ud X$ is
constant, which also gives conservation of momentum and the ``equal areas
rule'' also applies to the graph of $v$ against $X$.

\subsection{Long time limit}

Suppose $v(X,0)$ is a single pulse of any shape with area $A$.

\vspace{2in}

At large times the wave is approximately triangular, so we try
$v = Xf(t)$ as a solution, which gives $V \sim \tfrac{X}{t}$ as
$t \to \infty$.

If the shock is at $X = X_s(t)$ then the area of the triangle
$A = \tfrac{1}{2} X_s v(X_s)$ and so $X_s = \sqrt{2 A t}$.

Regardless of initial conditions we have the long-time solution
\[
v = \begin{cases}
\frac{X}{t} & 0 < X < X_s(t) \\
0 & \text{otherwise.}
\end{cases}
\]

We can also write down a differential equation governing $X_s(t)$.
The kinematic wave equation \eqref{eq:kinwave} gives
\[
\diff{}{t} \int_{X_1}^{X_2} v\, \ud X
+ \left[ \tfrac{1}{2} v^2 \right]_{X_1}^{X_2} = 0.
\]

If there is a shock at $X_s(t)$, $X_1 < X_s < X_2$ then
\begin{multline*}
\diff{}{t} \left\{ \int_{X_1}^{X_s^-} + \int_{X_s^+}^{X_2}
v\, \ud X \right\} = 
\underbrace{\left\{ \int_{X_1}^{X_s^-} + \int_{X_s^+}^{X_2}
\pddt{v} \, \ud X \right\}}_{\to 0 \text{ if $\pddt{v}$ bounded}} \\
+ \dot{X_s} \left( v(X_s^-) - v(X_s^+) \right).
\end{multline*}

Let $X_2 \to X_s^+$ and $X_1 \to X_s^-$, so if $v(X_s^+) \neq v(X_s^-)$ we get
\[
\dot X_s = \frac{v(X_s^+) + v(X_s^-) }{2}.
\]

For instance consider a sinusoidal initial wave $v = v_0 \sin k x$.

\vspace{2in}

There are fixed shocks at $x= \pm \tfrac{\pi}{k}, \pm \tfrac{3 \pi}{k},\dots$.

In $-\tfrac{\pi}{k} < X < \tfrac{\pi}{k}$ we have the solution
$v = \tfrac{X}{t}$, whose odd periodic continuation gives the full solution.

Now $v(X_s^+) = - \tfrac{\pi}{2 k t}$ and $v(X_s^-) = \tfrac{\pi}{2 k t}$
and so $\dot X_s = 0$ and the \emph{shock jump} $\Delta v = - \tfrac{\pi}{k t}
\to 0$ as $t \to \infty$.  The whole wave decays to $0$.

\chapter[Elastic waves]{Linear elastic waves in solids}

\renewcommand{\t}{\vect{t}}

\section{The governing equations}

As before we make the \emph{continuum hypothesis}.

\subsection{Stress}

Consider a small (plane) element of surface $\vect{dS} = \vn \ud S$
within some material.  Assume that the material on the side of $\ud S$
to which $\vn$ points acts on the material on the other side of $\ud S$
with a force proportional to $\ud S$ and zero moment.

This force is $\t(\vx,t;\vn) \ud S$, and $\t$ is called the \emph{traction
vector}.  We write

\begin{equation}\label{eq:stresstens}
\sigma_{i j} = t_j(\vx,t;\vect{e}_i),
\end{equation}

where $\vect{e}_i$ is the unit vector in the $i$ direction.  We apply
conservation of momentum to an arbitrary material volume $V$ to get

\begin{equation}\label{eq:solid:mom}
\int_V \rho(\vx,t) \left[ \vect{a}(\vx,t) - \vF(\vx,t) \right]\, \ud V
= \int_S \t(\vx,t;\vn)\, \ud S.
\end{equation}

Conservation of angular momentum gives

\begin{equation}\label{eq:solid:angmom}
\int_V \rho(\vx,t) \vx \wedge \left( \vect{a} - \vF \right)\, \ud V
= \int_S \vx \wedge \t\, \ud S.
\end{equation}

We apply the momentum balance \eqref{eq:solid:mom} to a
tetrahedron as drawn.  Suppose that the area of the slant face of the
tetrahedron is $\epsilon^2$.  Then the area of the face
perpendicular to $\vect{e}_i$ is $n_i \epsilon^2$.

Now the left hand side of \eqref{eq:solid:mom} is $\cO(\epsilon^3)$
(if $\vect{a}$ and $\vF$ are bounded) and the right hand side is
\[
\epsilon^2 \left(-n_1 \t(\vx,t;\vect{e}_1) - n_2 \t(\vx,t;\vect{e}_2) -n_3
  \t(\vx,t;\vect{e}_3) + \t(\vx,t;\vn) \right) + \cO(\epsilon^3).
\]

Since the left hand side and right hand side must balance at leading order
we have

\begin{equation}\label{eq:solid:traction}
t_j(\vn) = \sigma_{ij} n_i.
\end{equation}

The quotient theorem implies that $\sigma_{ij}$ is a tensor (as $\vn$
is arbitrary).  It is called the \emph{Cauchy stress tensor}.
\eqref{eq:solid:mom} can now be converted into a differential
equation:

\begin{equation}\label{eq:solid:cauchy}
\pd{\sigma_{ij}}{x_i} = \rho (a_j - F_j).
\end{equation}

We take the $i^{\text{th}}$ component of \eqref{eq:solid:angmom}
and apply \eqref{eq:solid:traction} and \eqref{eq:solid:cauchy} to get

\begin{align*}
\int_V \epsilon_{ijk} x_j \pd{\sigma_{lk}}{x_l}\, \ud V
&= \int_S \epsilon_{ijk} x_j \sigma_{lk} n_l\, \ud S \\
&= \int_V \epsilon_{ijk} \pd{}{x_l}\left(x_j \sigma_{lk}\right)\, \ud V \\
&= \int_V \epsilon_{ijk} x_j \pd{\sigma_{lk}}{x_l}\, \ud V
+ \int_V \epsilon_{ilk} \sigma_{lk}\, \ud V.
\end{align*}

As $V$ is arbitrary, $ \epsilon_{ilk} \sigma_{lk} = 0$ and so
$\sigma_{ij}$ is a symmetric tensor.

\subsection{Strain}

When a solid is deformed the particle at $\bs \xi$ goes to $\vx$ (say),
with \emph{displacement} $\vu = \vx - \bs \xi$.  Under a deformation
two nearby particles suffer a relative displacement
\[
\vect{dS} = \vect{dx} - \vect{d\bs\xi} = \vu(\vx + \vect{dx},t)
- \vu(\vx,t).
\]

Taking the $i^{\text{th}}$ component,
$\ud S_i = \pd{u_i}{x_j} \ud x_j + \cO(\vect{dx}^2)$.  If we let

\begin{equation}\label{eq:solid:strain}
e_{ij} = \frac{1}{2} \left( \pd{u_i}{x_j} + \pd{u_j}{x_i} \right) \qquad
\text{and} \qquad
\omega_{ij} = \frac{1}{2} \left( \pd{u_i}{x_j} - \pd{u_j}{x_i} \right),
\end{equation}

then we can write $\ud S_i = e_{ij}\ud x_j + \omega_{ij} \ud x_j$
(neglecting higher order terms).
As $\omega_{ij} = -\tfrac{1}{2} \epsilon_{ijk} \Omega_k$,
where $\bs \Omega = \nabla \wedge \vu$,
\[
\omega_{ij}\ud x_j = \left(\tfrac{1}{2} \bs \Omega \wedge \vect{dx} \right)_i.
\]

This is what we get from a rigid rotation at rate $\tfrac{1}{2} \bs \Omega$
and so cannot be connected with local states of stress.  The stress
must therefore be associated with the symmetric part $e_{ij}$ of the
local deformation tensor $\pd{u_i}{x_j}$.  The dimensionless
tensor $e_{ij}$ is called the \emph{Cauchy stress tensor}.

\subsection{The constitutive relation}

For an elastic body we take $\sigma = \sigma(e)$ only (not depending
on the history of $e$).  For small $e_{ij}$ we can expand this in
Taylor series to get

\[
\sigma_{ij} = C_{ijkl} e_{kl}
\]

for some fourth rank tensor $C_{ijkl}$. A general fourth rank tensor
has 81 components, but we can use the symmetries
$\sigma_{ij} = \sigma_{ji}$ and $e_{kl} = e_{lk}$ to get
\[
C_{ijkl} = C_{jikl} = C_{ijlk}.
\]

This reduces the number of independent components to 36 for a general
anisotropic medium.  We now assume isotropy, which means that
$C_{ijkl}$ is an isotropic tensor,
\[
C_{ijkl} = \lambda \delta_{ij} \delta_{kl}
+ \mu \delta_{ik} \delta_{jl}
+ \nu \delta_{il} \delta_{jk}.
\]

The symmetries of $C$ mean that $\mu = \nu$, and so

\begin{equation}\label{eq:solid:constit}
\sigma_{ij} = \lambda \delta_{ij} e_{kk} + 2 \mu e_{ij}
\end{equation}

$\lambda$ and $\mu$ are the \emph{Lam\'e constants}.

\subsection{Shear and dilatation}

Consider a shear

\vspace{1.5in}

\begin{align*}
x_1 &= \xi_1 + \delta \xi_2 \\
x_2 &= \xi_2 \\
x_3 &= \xi_3.
\end{align*}

Then $\vu = \vx - \bs \xi = (\delta \xi_2,0,0)$.  $e_{ij}$ is constant,
with $e_{12} = e_{21} = \tfrac{1}{2} \delta$ and the other $e_{ij}$ are
zero.  Now $\sigma_{12} = \sigma_{21} = \mu \delta = S$, with
the other $\sigma_{ij}$ zero.  The angle of the shear, $\gamma$, is
given by $\tan \gamma = \delta$.  Usually $\gamma \ll 1$ so
$\gamma \approx \delta$, and
\[
\mu \approx \frac{S}{\gamma} = \frac{\text{shear force}}{\text{angle of shear}}
\]

is the \emph{shear modulus} or \emph{modulus of rigidity}.  When $\mu = 0$
we have a perfect fluid, but $\mu > 0$ otherwise.

As for dilatation,
\begin{align*}
\Delta V &= \int \vn \cdot \vu\, \ud S \\
&= \int \dive \vu\, \ud V.
\end{align*}

We call $\dive \vu = e_{kk} = \theta$ the dilatation, which is the
change in volume per unit volume.

A special state of stress is that of hydrostatic pressure, with
$\sigma_{ij} = - p \delta_{ij}$ ($p$ is the pressure).  This gives
$\t(\vn) = - p \vn$ and so
\[
p = - \tfrac{1}{3}\sigma_{kk} = - \left( \lambda + \frac{2 \mu}{3} \right)
e_{kk} = -K \theta.
\]

$K = \lambda + \tfrac{2}{3} \mu > 0$ is the modulus of compressibility,
and is the ratio of pressure to the change of volume.

We apply the constitutive relation \eqref{eq:solid:constit} to the
Cauchy equation \eqref{eq:solid:cauchy} to get

\begin{equation}\label{eq:solid:gov1}
\rho a_j = \rho F_j + \pd{}{x_j} \left( \lambda \pd{u_k}{x_k} \right)
+ \pd{}{x_i} \left( \mu \left( \pd{u_i}{x_j} + \pd{u_j}{x_i} \right) \right).
\end{equation}

In linear theory, the acceleration $a_j = \pd{^2 u_j}{t^2}$ and
$\rho(\vx,t) = \rho(\xi,0)$ (the density in the undeformed state).  We
take $\lambda$ and $\mu$ independent of position (a \emph{homogeneous} body)
to get

\begin{equation}\label{eq:solid:gov2}
\begin{aligned}
\rho \pd{^2 \vu}{t^2} &= \rho \vF + \left( \lambda + \mu \right) \grad \dive
\vu + \mu \nabla^2 \vu \\
&=\rho \vF + \left( \lambda + 2 \mu \right) \grad \dive \vu
- \mu \curl \curl \vu.
\end{aligned}
\end{equation}

\section{Compressional and shear waves}

\newcommand{\vort}{\bs\omega}

Taking the divergence of \eqref{eq:solid:gov2} we get

\begin{equation}\label{eq:solid:compwave}
\pd{^2 \theta}{t^2} = c_P^2 \nabla^2 \theta,
\end{equation}

where $\theta = \dive \vu$ and $c_P^2 = \tfrac{\lambda + 2 \mu}{\rho}$.

$c_P$ is the dilatational or compression wave speed.

Taking the curl of \eqref{eq:solid:gov2} we get

\begin{equation}\label{eq:solid:shear}
\pd{^2 \vort}{t^2} = c_S^2 \nabla^2 \vort,
\end{equation}

where $\vort = \curl \vu$ and $c_S^2 = \tfrac{\mu}{\rho} < c_P^2$.

For steel, $c_P \approx 6 \times 10^3 ms^{-1}$ and
$c_S \approx 3 \times 10^3 ms^{-1}$.%
\footnote{P stands for \emph{primary} - these are the waves which arrive
first in an earthquake.  S stands for secondary.  Guess when these arrive!}

\subsection{Dilatational and shear potentials}

We can always write $\vu = \nabla \phi + \curl \bs \psi$ with $\dive
\bs \psi = 0$.

\begin{proof}
Define $\bs \psi$ by $\nabla^2 \bs\psi = - \curl \vu$, so that
\[
\bs \psi = \frac{1}{4 \pi} \int \frac{\curl \vu(\vect y)}{\abs{\vx-\vect y}}
\, \ud^3 y.
\]

(This also gives $\dive \bs\psi = 0$.)  Now let $\vect w = \vu - \curl\bs\psi$,
so that $\curl \vect w = \curl \vu - \left( \grad\dive\bs\psi - \nabla^2 \bs
\psi\right)$.  Thus $\exists \phi$ such that $\vect w = \nabla \phi$ and
$\vu = \nabla \phi + \curl\bs\psi$.
\end{proof}

$\phi$ and $\bs\psi$ are the \emph{elastodynamic potentials}.  We find that
\eqref{eq:solid:gov2} is satisfied if

\begin{gather}
\phi_{tt} - c_P^2 \nabla^2 \phi = 0\stepcounter{equation}
\label{eq:solid:seisphi}\tag{\theequation{}a}\\
\bs\psi_{tt} - c_S^2 \nabla^2 \bs\psi = 0\tag{\theequation{}b}
\label{eq:solid:seispsi}.
\end{gather}

These are the equations of seismology.

\subsection{Plane waves}

The plane wave solution to \eqref{eq:solid:seisphi} is
\[
\phi = \phi(t - \tfrac{\Hat\vk\cdot\vx}{c_P}),
\]

with $\Hat\vk$ a constant vector.  Thus $\vu = - \tfrac{\Hat\vk}{c_P}\phi'$
and $\vu$ is parallel to $\Hat\vk$ --- these are longitudinal waves.

\eqref{eq:solid:seispsi} also has plane wave solutions,
\[
\bs\psi = \bs\psi(t - \tfrac{\Hat\vk\cdot\vx}{c_S}),
\]
and in this case $\vu = - \frac{\Hat\vk \wedge \bs\psi'}{c_S}$.  This
is perpendicular to $\vk$ --- these are transverse waves.

We can also evaluate the stress tensor

\begin{gather*}
  \sigma_{ij} = \left( \lambda \delta_{ij} + 2 \mu \Hat k_i \Hat k_j
  \right)
  \frac{\phi''}{c_P^2} \qquad \text{for P waves} \\
  \sigma_{ij} = \mu \left( \epsilon_{ilm} \Hat k_l \Hat k_j +
    \epsilon_{jlm} \Hat k_l \Hat k_i\right) \frac{\psi_m''}{c_S^2}
  \qquad \text{for S waves.}
\end{gather*}

The energy flux $\vect I$ is given by $I_i = -\sigma_{ij} \Dot u_j$,
which equals $\rho c_P \Dot{u}^2 \Hat \vk$ for P waves and
$\rho c_S \Dot{u}^2 \Hat \vk$ for S waves.

The energy density
\begin{align*}
E_k + E_p &= \tfrac{1}{2} \rho \Dot{u}^2 + \tfrac{1}{2} \sigma_{ij} e_{ij}\\
&= \rho \Dot{u}^2 \qquad \text{for both types of waves separately.}
\end{align*}

For either type of wave (separately), the energy flux vector
$\vect I = (E_k + E_p) \vect c$.  Energy propagates at velocity $\vect c$
in the direction of the waves ($\vk$).

\subsection{Energy equation}

\renewcommand{\v}{\vect{v}}

Consider the rate of change of internal energy (assuming isentropy),

\[
\ddt{} \int_V \rho \cE \, \ud V = \int_V \rho \vF\cdot \v\, \ud V
+ \int_S \t \cdot \v\, \ud S - \ddt{} \int_V \tfrac{1}{2} \rho
v^2\, \ud V,
\]

where $V$ is a material volume.  Hence
\[
\ddt{} \int \rho\left( \tfrac{1}{2} v^2 + \cE \right)\, \ud V
= \int \left( \rho \vF \cdot \v + \pd{}{x_i} \left(\sigma_{ij} v_j\right)
\right)\, \ud V,
\]

or

\begin{equation}\label{eq:enbalance}
\rho \v \cdot \vect{a} + \rho \Dot \cE = \rho \vF \cdot v
+ v_j \pd{\sigma_{ij}}{x_i} + \sigma_{ij} \pd{v_j}{x_i}.
\end{equation}

The momentum equation thus gives

\begin{equation}\label{eq:enb1}
\rho \Dot \cE = \sigma_{ij} \pd{v_j}{x_i}.
\end{equation}

For small displacements and an isotropic material,
\begin{align*}
\pddt{} \tfrac{1}{2} \sigma_{ij} e_{ij} &= \tfrac{1}{2} C_{ijkl}
\left(e_{kl} \Dot{e}_{ij} + \Dot{e}_{kl} e_{ij}\right) \\
&= C_{ijkl} e_{kl} \Dot{e}_{ij} \\
&= \tfrac{1}{2} \sigma_{ij} \left( \pd{v_i}{x_j} + \pd{v_j}{x_i}\right) \\
&= \sigma_{ij} \pd{v_i}{x_j}.
\end{align*}

Assuming there is no strain energy when undeformed, then
\begin{equation}\label{eq:strainen}
\cE = \tfrac{1}{2 \rho} \sigma_{ij} e_{ij}
\end{equation}

is the strain energy density per unit mass.  Now
\begin{align*}
\rho \cE &= \tfrac{1}{2} \left( \lambda e_{kk} e_{jj} + 2 \mu e_{ij} e_{ij}
\right)\\
&= \tfrac{1}{2} \left( \lambda + \frac{2 \mu}{3}\right) \left( e_{kk} e_{jj}
\right) + \mu \left( e_{ij} - \tfrac{1}{3} e_{kk} \delta_{ij}\right)
\left(e_{ij} - \tfrac{1}{3} e_{ll} \delta_{ij} \right).
\end{align*}

Hence $\cE$ is positive definite iff $\lambda + \tfrac{2}{3} \mu > 0$
and $\mu > 0$.  These are the same restrictions as obtained earlier by
physical reasoning.

The linearised version of \eqref{eq:enbalance} can be written
\begin{equation}\label{eq:solid:strainen2}
\pddt{} \left( E_k + E_p \right) + \pd{I_j}{x_j} = \rho \vF \cdot \Dot \vu,
\end{equation}

where $E_k = \tfrac{1}{2} \rho \Dot{\vu} \cdot \Dot{\vu}$,
$E_p = \tfrac{1}{2} \sigma_{ij} e_{ij}$ and $I_i = - \sigma_{ij} \Dot{u}_j$.

We can interpret this physically by considering a small disk with surface
$\vn \ud S$. The rate of working by the material on the $+$ side on the
material on the $-$ side is
\[
\t \cdot \v\, \ud S = n_i \sigma_{ij} \pddt{u_j}\, \ud S,
\]

and so the flux of energy in direction $\vn$ is $\vect{I} = - \te{\sigma}
\cdot \pddt{\vu}$.

\subsection{Rayleigh waves}

\vspace{1.5in}

Consider the situation shown, with a disturbed interface at
$y = \eta(x,z,t)$.  There is no traction at the interface, so
$\sigma_{ij}n_j = 0$ at $y = \eta(x,y,t)$.  Now
\[
\Hat \vn = \frac{(-\eta_x,1,-\eta_z)}{\left( 1+ \eta_x^2 + \eta_z^2
\right)^{\frac{1}{2}}} \approx (0,1,0) \text{ in linear theory.}
\]

We also apply the boundary conditions at $y=0$ instead of $y = \eta$.
We seek a two dimensional solution with $\vu$ in the $(x,y)$ plane.  The
boundary conditions thus become
\[
\sigma_{xy} = \sigma_{yy} = 0 \text{ on } y = 0.
\]

There are two boundary conditions and so we guess that we need both
P and S wave solutions.  Try $\phi = f(y) E$ and $\bs \psi
= g(y) E \Hat{\vect z}$, where $E = e^{\imath(k x-\omega t)}$.  The
wave equation gives
\begin{gather*}
f'' - \left( k^2 - \frac{\omega^2}{c_P^2} \right) f = 0\\
g'' - \left( k^2 - \frac{\omega^2}{c_S^2} \right) f = 0.
\end{gather*}

We assume (and check later) that $k > \frac{\omega}{c_S}$, and so
(imposing boundedness at $y=-\infty$) we get
\[
f= A e^{\alpha y} \qquad g = B e^{\beta y},
\]

where $\alpha = \sqrt{k^2 - \tfrac{\omega^2}{c_P^2}}$ and
$\beta = \sqrt{k^2 - \tfrac{\omega^2}{c_S^2}}$.

Thus $\vu = (\imath k A e^{\alpha y} + \beta B e^{\beta y},
\alpha A e^{\alpha y} - \imath k B e^{\beta y},0) E$.  We now impose the
boundary conditions
\begin{align*}
\left.\sigma_{xy}\right|_{y=0}
&= \mu\left( \pd{u_1}{y} + \pd{u_2}{x} \right) \\
&= \mu E \left( 2 \imath \alpha A + \left( 2 k^2 -
\frac{\omega^2}{c_S^2} \right) \right) = 0 \\
\left.\sigma_{yy}\right|_{y=0}
&= \lambda \pd{u_1}{x} + \left( \lambda + 2 \mu \right)
\pd{u_2}{y} \\
&= \mu E \left( \left(2 k^2 - \frac{\omega^2}{c_S^2} \right) A
- 2 \imath k \beta B\right) = 0.
\end{align*}

These are two homogeneous equations for $A$ and $B$, which have non-trivial
solutions iff
\begin{equation}\label{eq:Rayldisp1}
\left( 2 k^2 - \frac{\omega^2}{c_S^2} \right)^2 = 4 k^2 \alpha \beta.
\end{equation} 

If we define $c_R = \tfrac{\omega}{k}$, the phase velocity of Rayleigh
waves, then \eqref{eq:Rayldisp1} becomes

\begin{equation}\label{eq:Rayldisp2}
\left( 2- \frac{c_R^2}{c_S^2} \right)^2 - 4 \left(1 - \frac{c_R^2}{c_P^2}
\right)^{\frac{1}{2}}
\left( 1 - \frac{c_R^2}{c_S^2} \right)^{\frac{1}{2}} = 0.
\end{equation}

Equation \eqref{eq:Rayldisp1} (equivalently \eqref{eq:Rayldisp2}) is the
is the dispersion relation for Rayleigh waves.  From \eqref{eq:Rayldisp2}
we see that $c_R$ is the same for all $k$ and Rayleigh waves are
non-dispersive.

Let $\xi = \tfrac{c_R^2}{c_S^2}$, so the dispersion relation can
be rewritten
\[
f(\xi) = \xi^3 - 8 \left( 3 - 2 \tfrac{c_S^2}{c_P^2}\right) \xi
- 16 \left(1 - \tfrac{c_S^2}{c_P^2}\right) = 0.
\] 

Now $f(0) < 0$ and $f(1) > 1$ so there is at least one real root of
$f$ in $[0,1]$ and $0 < c_R^2 < c_S^2$ (as was assumed earlier).  In fact
it can be shown that $f$ has precisely one real root.

This Rayleigh wave is non-dispersive and is a \emph{trapped} or
\emph{surface} wave.  $u_1, u_2 \sim e^{\alpha y}, e^{\beta y}$ and
so the disturbance is confined to a layer on the surface of thickness
$y \sim \max \{\alpha^{-1}, \beta^{-1}\}$.

Rayleigh waves are important in seismology.  As they are confined to
the surface they only fall off like $r^{-\frac{1}{2}}$ rather than the
$r^{-1}$ of P and S waves.  At large distances from the initial
disturbance the Rayleigh wave dominates.%
\footnote{And knocks your house down.  They also have technological
uses other than mass demolition --- acoustic microscopy.}

\section{Wave reflection and transmission at interfaces}

\vspace{1.5in}

\subsection{Interface conditions}

Let the interface be at $y = \eta(x,z,t)$.  We assume that there is
no fracture and the two materials remain bonded together, so that
\[
\left[ \vu(x,y=\eta,z,t) \right] = 0.
\]

We assume small displacements, and so this linearises to give

\begin{equation}\label{eq:interface1}
\left[ \vu(x,0,z,t) \right] = 0.
\end{equation}

We also have continuity of traction, so that
$\left[ \sigma_{ij} n_j \right] = 0$ on $y = \eta$, or (linearised)

\begin{equation}\label{eq:interface2}
\left[ \sigma_{iy} \right] = 0 \text{ at } y = 0.
\end{equation}

\subsection{S wave polarisation}

Without loss of generality we can take the propagation vector of all waves
to lie in the $(x,y)$ plane.

The incident P wave has $\vu$ parallel to $\Hat{\vk}_I$, and so
\[
\vu_I = A_I \Hat{\vk}_I e^{\imath\left(\vk_I \cdot \vx - \omega t\right)}.
\]

The incident S wave has $\vu$ perpendicular to $\Hat{\vk}_I$ and so
\[
\vu_I = \Hat{\vk}_I \wedge \vect{B}_I e^{\imath\left(\vk_I \cdot \vx
- \omega t\right)}.
\]

We resolve $\Hat{\vk}_I \wedge \vect{B}_I$ in a component parallel
to $\Hat{\vect{z}}$ (a horizontally polarised SH wave) and a
component parallel to $\Hat{\vect{z}} \wedge \vk_I$ (a vertically polarised
SV wave).

\subsection{Reflection and refraction of SH waves}

\vspace{1.5in}

The incident SH wave has

\begin{equation}\label{eq:SHinc}
\vu_I = (0,0,1) e^{\imath\left(\vk_I \cdot \vx - \omega t\right)}
\qquad \vk_I = \frac{\omega}{c_S} (\sin \theta,\cos \theta,0).
\end{equation}

The $1$ in the $z$-component of $\vu_I$ is a free choice and the factor
$\tfrac{\omega}{c_S}$ is fixed by the wave equation.

The reflected SH wave is

\begin{equation}\label{eq:SHref}
\vu_R = (0,0,R) e^{\imath\left(\vk_R \cdot \vx - \Omega t\right)} \qquad
\vk_R = \frac{\Omega}{c_S} (\sin \Theta,-\cos \Theta,0).
\end{equation}

$R$ is called the \emph{reflection coefficient}.

Finally, the transmitted SH wave is

\begin{equation}\label{eq:SHtrans}
\vu_T = (0,0,T) e^{\imath\left(\vk_T \cdot \vx - \Bar\omega t\right)} \qquad
\vk_T = \frac{\Bar \omega}{\Bar{c}_S} (\sin \Bar \theta, \cos \Bar \theta,0).
\end{equation}

$T$ is the \emph{transmission coefficient}.

Now we impose the boundary conditions $\vu_I + \vu_R = \vu_T$
and $\left(\sigma_{yz}\right)_I + \left(\sigma_{yz}\right)_R
= \left(\sigma_{yz}\right)_T$.

For these expressions to be the same at all times they must have the
same time-dependence, and so $\omega = \Omega = \Bar{\omega}$.  They must
similarly have the same space dependence, and so

\begin{equation}\label{eq:snell}
\frac{\sin \theta}{c_S} = \frac{\sin \Theta}{c_S}
 = \frac{\sin \Bar{\theta}}{\Bar{c}_S}.
\end{equation}

This equation captures both $\theta = \Theta$ (``angle of incidence equals
angle of reflection'') and \emph{Snell's law}.  Feeding all of this
back into the boundary conditions finally gives
\[
1 + R = T \qquad \text{and} \qquad \frac{\mu}{c_S} (1-R) \cos \theta
= \frac{T \Bar{\mu}}{\Bar{c}_S} \cos \Bar{\theta}.
\]

We can solve these by putting

\[
R = \frac{\Bar{z} - z}{\Bar{z} + z} \qquad T = \frac{2 \Bar{z}}{\Bar{z}+z},
\]

where $z = \tfrac{c_S}{\mu \cos \theta}$ and $\Bar{z} = \frac{\Bar{c}_S}{
\Bar{\mu}\cos \Bar{\theta}}$.

We can find the energy flux $\vect{I} = - \te{\sigma} \cdot \Dot{\vu}$,
and we see that
\[
\langle I_z \rangle_I + \langle I_z \rangle_R = \langle I_z \rangle_T,
\]
and the incident energy is all either transmitted or reflected.  If
$\Bar{z} = z$ there is total transmission.

If $\Bar{c}_S < c_S$ then $\sin \Bar{\theta} < \sin \theta$ ($\Bar \theta
< \theta$).  As $\theta$ increases to $\tfrac{\pi}{2}$ then
$\Bar{\theta}$ increases to $\Bar{\theta}_{\text{max}} =
\arcsin \tfrac{\Bar{c}_S}{c_S}$.  No waves propagate into
$\Bar{\theta} > \Bar{\theta}_{\text{max}}$ --- this is the so-called
``quiet zone''.

If $\Bar{c}_S > c_S$ then $\Bar{\theta} > \theta$ and $\exists
\theta_{\text{max}} = \arcsin \tfrac{c_S}{\Bar{c}_S}$ such that the
transmitted wave just grazes the boundary.  If $\theta >
\theta_{\text{max}}$ then we need to seek a solution in $y > 0$ of the
form

\[
\vu_T = (0,0,T) \exp \left(\imath \tfrac{\omega}{\Bar{c}_S}
- \beta y - \imath \omega t \right).
\]

We need $\beta = \tfrac{\omega}{c_S} \left( \sin^2 \theta
- \tfrac{c_s^2}{\Bar{c}_S^2} \right)^{\frac{1}{2}}$, which
is real and positive in the region $\theta > \theta_{\text{max}}$.
$\vu_T$ is an \emph{evanescent} wave confined to $y \lessapprox \beta^{-1}$.

We find that $\langle I_z \rangle_T = 0$ and that $R$ is now a complex
quantity with modulus 1 and when $\theta > \theta_{\text{max}}$ the
incident wave is totally internally reflected.

\subsection{Reflection of P waves}

\vspace{1.5in}

This has the boundary condition $\vu = 0$ at $y = 0$.  As before, all
$\omega$ are the same and $\theta = \Theta$.

The incident P wave is

\[
\vu_I = (\sin \theta,\cos \theta,0) e^{\imath(\vk_I\cdot\vx - \omega t)}
\qquad \vk_I = (\sin \theta, \cos \theta,0) \tfrac{\omega}{c_P}.
\]

The reflected P wave is
\[
\vu_R = R (\sin \Theta, - \cos \Theta,0) e^{\imath(\vk_R\cdot\vx - \omega t)}
\qquad \vk_R = (\sin \Theta, - \cos \Theta,0) \tfrac{\omega}{c_P} 
\]

The reflected SV wave is
\[
\vu_{\Bar{R}} = \Bar{R} (\cos \Bar{\theta}, \sin \Bar{\theta},0)
e^{\imath(\vk_{\Bar{R}}\cdot\vx - \omega t)}
\qquad \vk_{\Bar{R}} = (\sin \Bar{\theta}, - \cos \Bar{\theta},0)
\tfrac{\omega}{c_S}.
\]

We match phases at $y=0$, so that $\theta = \Theta$
and
\[
\Bar{\theta} = \arcsin \left( \tfrac{c_S}{c_P} \sin \theta \right) < \theta
\text{ as } \Bar{\theta} < \theta.
\]

Finally, imposing $\vu = 0$ on $y=0$ we find
\[
R = \frac{\cos (\theta + \Bar{\theta})}{\cos (\theta - \Bar{\theta})}
\quad \text{and} \quad
\Bar{R} = - \frac{\sin 2 \theta}{\cos (\theta - \Bar{\theta})}.
\]

Note that $\Bar{R} \neq 0$ usually and so there is usually at least
partial ``mode conversion'' --- an incident P wave becomes a reflected
P wave and a reflected SV wave.

If $\theta = 0$ (normal incidence) then $\Bar{R} = 0$ and there is
no SV mode.  If $\theta + \Bar{\theta} = \tfrac{\pi}{2}$ then
$R = 0$ --- there is total mode conversion.

\chapter{Dispersive waves}

In any linear continuous system with no explicit dependence on $(x,t)$
we can look for a solution
\begin{equation}\label{eq:disp:gensol}
\phi = \underbrace{e^{\imath k x - \imath \omega t}}_{\text{plane wave
in $x,t$}}
\underbrace{f(y,z)}_{\text{shape in $y,z$}}.
\end{equation}

$\omega$ can be found as $\omega = \omega(k)$, a dispersion relation.
Usually $\tfrac{\omega}{k} = c_p$ (the phase speed) is not constant.
Waves of different $k,\omega$ travel at different speeds and so
disperse.

\section{Geometric dispersion in ducts/tubes}

\vspace{1.5in}

Consider a duct $-\infty < x < \infty$ and hard walls at $0 < y < h$
and $0 < z < b$.  The wave equation for $\phi$ is
\[
\Box^2 \phi = \nabla^2 \phi - \frac{1}{c_0^2} \pd{^2}{t^2} \phi = 0,
\]

where $c_0$ is the sound speed.  The boundary conditions are
$\phi_y = 0$ on $y=0$, $y=h$ and $\phi_z = 0$ on $z=0$, $z=b$.

We try a solution of the form \eqref{eq:disp:gensol},
$\phi = e^{\imath k x - \imath \omega t} f(y,z)$ and so
\[
f_{yy} + f_{zz} + \left( \frac{\omega^2}{c_0^2} - k^2 \right) f = 0,
\]

with $f_y = 0$ on $y=0$, $y=h$ and $f_z = 0$ on $z=0$, $z=b$.  We
try a solution of the form $f = f_{mn} = A_{mn} \cos \tfrac{m \pi y}{h}
\cos \tfrac{n \pi z}{b}$.  This satisfies the boundary conditions,
and satisfies the PDE if
\[
\frac{m^2 \pi^2}{h^2} + \frac{n^2 \pi^2}{b^2} = \frac{\omega^2}{c_0^2}
- k^2.
\]

We get the dispersion relation

\begin{equation}\label{eq:disp:ductdisp}
\omega^2 = k^2 c_0^2 + \omega_{mn}^2,
\end{equation}

where $\omega_{mn}^2 = \left( \frac{m^2 \pi^2}{h^2} + \frac{n^2
    \pi^2}{b^2}\right) c_0^2$.

There is a set of independent ``modes'' labelled by $(m,n)$.
The $(m,n)$ mode is dispersive unless $m=n=0$, when
$\phi = A_{00} e^{\imath k x - \imath \omega t}$.

\subsection{Cut-off frequency}

A mode of given $\omega$ will propagate only if $k$ is real ---
that is $\omega > \omega_{mn}$.  For given $\omega$ only a finite
number of modes with $\omega_{mn} < \omega$ can propagate.  The others
have $k$ imaginary and are decaying evanescent waves.  $\omega_{mn}$
is the \emph{cut-off frequency} for the mode $(m,n)$.

The plane wave mode $(0,0)$ is always \emph{cut-on} and
(always) propagates.   If $h$ and $b$ are very small then other modes
will be cut-off.

\subsection{Phase and group velocity}

The phase velocity $c_p = \pm \left( c_0^2 + \frac{\omega_{mn}^2}{k^2}
\right)^{\frac{1}{2}}$.  Note that $c_p \ge c_0$ (equality only
if $m=n=0$), $c_p \to \pm c_0$ as $k \to \infty$ (short waves)
and $c_p \to \pm \infty$ as $k \to 0$ (long waves).

We define the \emph{group velocity}
$c_g = \diff{\omega}{k}$.

\subsection{Energy propagation}

Consider a single mode with $\phi = A_{mn} \cos \tfrac{m \pi y}{h}
\cos \tfrac{n \pi z}{b}$.  It has

\[
\langle \text{KE} \rangle = \int_0^b \ud z \int_0^h \ud y
\tfrac{1}{2} \rho_0 \tfrac{1}{2} \Re \left[ \left(\nabla \phi\right) \cdot
\left(\nabla \phi\right)^\ast\right]
= \frac{\rho_0 \abs{A_{mn}}^2 b h \omega^2}{16 c_0^2}.
\]

The time averaged potential energy
\[
\langle \text{PE} \rangle = \int_0^b \ud z \int_0^h \ud y
\tfrac{1}{2} \Re \left( \frac{c_0^2 \rho \rho^\ast}{2 \rho_0}\right)
= \langle \text{KE} \rangle.
\]

We can also compute
\[
\langle I_x \rangle = \int_0^b\ud z \int_0^h \ud y \tfrac{1}{2}
\Re \left( p \phi_x^\ast\right)
= \frac{\omega \rho_0 k \abs{A_{mn}}^2 bh}{8}.
\]

Define the mean energy propagation velocity
$U(k)$ for a wave with wavenumber $k$ by
\[
\langle I_x \rangle = U(k) \langle \text{KE} + \text{PE} \rangle.
\]

We see that $U = \tfrac{k c_0^2}{\omega}$, which is the
same as the group velocity $c_g$.  This is an example of a
general result  --- ``energy travels at the group velocity''.

\section{The Cauchy problem}

Consider the following PDE%
\footnote{which models small deflections of an elastic beam},

\[
\phi_{tt} + \gamma^2 \phi_{xxxx} = 0.
\]

Suppose that $\phi(x,0) = \phi_0(x)$ and $\phi_t(x,0) = v_0(x)$.
By Fourier transforming we see that
\begin{equation}\label{eq:gencauchy}
\phi(x,t) = \int_{-\infty}^\infty A(k) e^{\imath(k x - \gamma k^2 t)}\,
\ud k + \int_{-\infty}^\infty B(k) e^{\imath(k x + \gamma k^2 t)}\,
\ud k,
\end{equation}

where
\[
\phi_0(x) = \int_{-\infty}^\infty (A + B) e^{\imath k x}\, \ud k
\quad \text{and} \quad
v_0(x) = - \imath \int_{-\infty}^\infty \gamma k^2 ( A - B ) e^{\imath k x}\,
\ud k.
\]

We can find $A$ and $B$ by Fourier transform.  If $v_0 \equiv 0$
then $A \equiv B$.  We need
\[
I(x,t) = \int_{-\infty}^\infty A(k) e^{\imath (k x \pm \gamma k^2 t)}\,
\ud k.
\]

\subsection{Exact solution for Gaussian wavepacket}

If
\[
\phi_0(x) = e^{- \tfrac{x^2}{\lambda^2} + \imath \alpha x} \quad \text{and}
\quad v_0 \equiv 0
\]

we can do all of the calculations exactly.  First
we find
\begin{align*}
A(k) &= \frac{1}{4\pi} \int_{-\infty}^\infty
e^{- \tfrac{x^2}{\lambda^2} + \imath \alpha x - \imath k x}\, \ud x\\
&= \frac{\lambda e^{- \frac{\lambda^2(k-\alpha)^2}{4}}}{4 \sqrt{\pi}}.
\end{align*}

We now use this to find $I(x,t)$ as
\[
I(x,t) = \frac{\lambda}{2 \left(\lambda^2 + 4 \imath \gamma t
\right)^{\frac{1}{2}}} \exp \left\{ \imath \alpha x - \imath \gamma
\alpha^2 t - \frac{\left( x - 2 \gamma \alpha t\right)^2}{\lambda^2 +
4 \imath \gamma t} \right\}.
\]

The total $\phi$ is $I + $ a similar thing from $\omega = - \gamma
k^2$ (or $t \to - t$).  For large $\lambda$

\[
I \sim \tfrac{1}{2} \exp \left\{ \imath \alpha (x - c_p t)
- \frac{\left( x - c_g t\right)^2}{\lambda^2 +
4 \imath \gamma t} \right\},
\]

where $c_p = \gamma \alpha$ and $c_g = 2 \gamma \alpha$.  The wavepacket
as a whole moves at a speed $c_g > c_p$.  Wavecrests and troughs are
created at the front of the wavepacket and destroyed at the back.

It is sometimes convenient to formalise the separation of
scales in terms of fast co-ordinates $(x,t)$ (for the details)
and slow co-ordinates $(X=\tfrac{x}{\lambda},\tau = \tfrac{t}{\lambda})$
Then
\[
I \sim \tfrac{1}{2} \underbrace{e^{\imath \alpha( x - c_p t)}}_{\text{
fast phase oscillation}}\ \times\
 \underbrace{ e^{-( X - c_g \tau)^2}}_{\text{slow scale modulation}}.
\]

\subsection{Modulated wavetrains}

Suppose we have initial data
\[
\phi_0(x) = e^{\imath \alpha x} M(\tfrac{x}{\lambda}) \qquad \alpha \lambda
\gg 1,
\]

where $e^{\imath \alpha x}$ is a carrier wave with short wavelength
$2 \pi \alpha^{-1}$ and $M$ is a modulation envelope of a large
scale $\lambda$.

% script A
\newcommand{\sA}{\mathcal{A}}

Now

\begin{align*}
A(k) &= \frac{1}{4 \pi} \int_{-\infty}^\infty M(\tfrac{x}{\lambda})
e^{- \imath(k-\alpha)}\, \ud x \\
&= \frac{\lambda}{4 \pi} \int_{-\infty}^\infty M(X)
e^{- \imath \beta X}\, \ud X\\
&= \sA(\beta),
\end{align*}

where $\epsilon = \lambda^{-1}$, $\beta = \tfrac{k - \alpha}{\epsilon}$,
$X = \epsilon x$ and $\tau = \epsilon t$.  Now

\begin{align*}
I &= \int_{-\infty}^\infty \sA(\beta) e^{\imath \alpha x
+ \imath \epsilon \beta x - \imath \omega(\alpha + \epsilon \beta)t}
\epsilon\, \ud \beta. \\
\intertext{If we Taylor expand $\omega$ about $\epsilon = 0$, we get}
I &\sim \epsilon e^{\imath \alpha x - \imath \omega(\alpha) t}
\int_{-\infty}^\infty \sA(\beta) \exp\left( \imath \epsilon \beta(
x - c_g t)\right)\, \ud \beta \\
&= \epsilon e^{\imath \alpha (x - c_p t)}
\int_{-\infty}^\infty \sA(\beta) e^{\imath \beta(X - c_g \tau)}\, \ud \beta\\
&= \tfrac{1}{2} e^{\imath \alpha (x - c_p t)} M(X - c_g \tau).
\end{align*}

The complete solution is

\begin{align*}
\phi = &\tfrac{1}{2} e^{\imath \alpha (x - c_p t)}
M(X - c_g \tau) \qquad \text{right moving wave} \\
&+ \tfrac{1}{2} e^{\imath \alpha (x + c_p t)}
M(X + c_g \tau) \qquad \text{left moving wave.}
\end{align*}

\subsection{Large time behaviour}

We want to find the long time behaviour of
\[
\int_{-\infty}^\infty A(k) e^{\imath (k x - \omega t)}\, \ud k.
\]

For fixed $x$ we can integrate this by parts to get
\begin{align*}
I &= \left[ \frac{A(k)}{-\imath \omega'} \frac{e^{\imath(k x -\omega t)}}{t}
\right]_{-\infty}^\infty -
\frac{1}{t}\int_{-\infty}^\infty e^{\imath(k x -\omega t)}
\diff{}{k} \left( \frac{A(k)}{-\imath \omega'} \right)\, \ud k \\
&= \cO(t^{-1}) + \frac{1}{t} \int_{-\infty}^\infty B(k)
e^{\imath(kx - \omega t)}\, \ud x.
\end{align*}

Therefore $I = \cO(t^{-1})$ unless $\omega' = 0$ for some $k$.  Things
are more interesting if we consider instead an observer moving at
speed $V = \tfrac{x}{t}$.  We wish to approximate
\[
I = \int_{-\infty}^\infty A(k) e^{\imath t \psi(k)}\, \ud k
\]

for large $t$, where $\psi(k) = k V - \omega(k)$.  Integration by parts
gives $I = \cO(t^{-1})$ as before unless $\psi'(\alpha) = 0$ for
some $\alpha$.  Such points $\alpha$ are called \emph{points of stationary
phase}.

These points $\alpha$ satisfy $V = c_g(\alpha)$ --- the points of
stationary phase have group velocity equal to the observer velocity.
Away from $k=\alpha$ the contributions to $I$ almost cancel
because $e^{\imath t \psi}$ oscillates very rapidly when $t \to \infty$.
This cancellation is much reduced when $k = \alpha$.

We approximate $A(k) \approx A(\alpha)$ and
$\psi(k) \approx \psi(\alpha) + \tfrac{1}{2} (k-\alpha)^2 \psi''(\alpha)$
(if $\psi''(\alpha) \neq 0$).  Then for each point of
stationary phase the contribution to $I$ is asymptotically

\[
A(\alpha) e^{\imath t \psi(\alpha)} \int_{\text{nhd of $\alpha$}}
e^{\imath t \tfrac{(k-\alpha)^2}{2} \psi''(\alpha)}\, \ud k
 \sim A(\alpha) e^{\imath t \psi(\alpha)} \int_{-\infty}^\infty
e^{\imath t \tfrac{(k-\alpha)^2}{2} \psi''(\alpha)}\, \ud k.
\]

Thus for one point of stationary phase,

\begin{equation}\label{eq:statphase}
I \sim \left( \frac{2 \pi}{\abs{\psi''(\alpha)} t} \right)^{\frac{1}{2}}
A(\alpha) e^{\imath t \psi(\alpha) + \imath \sigma \tfrac{\pi}{4}},
\end{equation}

where $\sigma = 1$ if $\psi''(\alpha) > 0$ and $\sigma = -1$
if $\psi''(\alpha) < 0$.  To deal with several isolated stationary points
just add the contributions from each.  If $A(\alpha) = 0$ then
approximate $A(k)$ as $(k - \alpha) A'(\alpha)$ and similarly for $\psi$.

Thus given $(x,t)$, solve the equation $c_g(\alpha) = \tfrac{x}{t}$ for
$\alpha$ and then substitute in \eqref{eq:statphase}.

The wavenumbers seen by an observer moving at speed $V$ have the
same group velocity and amplitude proportional to
$t^{-\frac{1}{2}}$.  This is consistent with energy conservation.

\vspace{1.5in}

The length of the wavetrain increases like $t$ and so conservation
of energy implies that the amplitude varies as $t^{-\frac{1}{2}}$.

We also have the \emph{radiation condition}: energy must flow
away from a source and hence waves will only be found where their
group velocity is directed away from the source.%
\footnote{This is not entirely true...}

\subsection{Dispersion in waveguides}

Recall the dispersion relation \eqref{eq:disp:ductdisp} which we
rewrite here as

\[
\omega = \pm \left( \omega_0^2 + k^2 c_0^2 \right)^{\frac{1}{2}}
= \pm \Omega(k),
\]

where $\omega_0 = \omega_{mn}$ for some $(m,n)$.  The
phase is $\psi_\pm(k) = k V \mp \Omega(k)$ and we have
stationary phase at
\[
V = \pm \Omega' = \pm \frac{k c_0^2}{(\omega_0^2 + k^2 c_0^2)^{\frac{1}{2}}}.
\]

For definiteness take $V > 0$.  There are two points of stationary phase
(one on each branch) at
\[
k = \pm k_0 = \pm \frac{\omega_0 V}{c_0 (c_0^2 - V^2)^{\frac{1}{2}}}.
\]

We see that we need $V < c_0$.  From \eqref{eq:gencauchy}
(and imposing $A(k) = B(k)$) we have

\begin{align*}
\phi(x,t) &= \int_{-\infty}^\infty A(k) \left[
e^{\imath \psi_+ t} + e^{\imath \psi_- t}
\right]\, \ud k \\
&\sim \left( \frac{2 \pi \Omega(k_0)^3}{\omega_0^2 c_0^2 t}
\right)^{\frac{1}{2}} \left[ A(k_0) e^{\imath k_0 x
- \imath \Omega(k_0) t - \imath \tfrac{\pi}{4}}
+ A(-k_0) e^{-\imath k_0 x
+ \imath \Omega(-k_0) t + \imath \tfrac{\pi}{4} }
\right].
\end{align*}

As $\phi$ is real we have $A(-k) = A^\ast(k)$ and we also
know $\Omega(-k) = \Omega(k)$.  Therefore

\[
\phi \sim  \left( \frac{2 \pi \Omega(k_0)^3}{\omega_0^2 c_0^2 t}
\right)^{\frac{1}{2}}
2 \Re \left\{ A(k_0) e^{\imath k_0 x
- \imath \Omega(k_0) t - \imath \tfrac{\pi}{4}} \right\}.
\]

\subsection{Continuity of phase}

We make the $t\to \infty$ assumption explicit by introducing
a scaling parameter $\lambda$, fast scales $(x,t)$ and
slow scales $(X = \tfrac{x}{\lambda}, \tau = \tfrac{t}{\lambda})$.
Then \eqref{eq:statphase} becomes
\[
I \sim \frac{1}{\lambda^{\frac{1}{2}}}
\left( \frac{2 \pi}{\abs{\psi''(\alpha)} \tau} \right)^{\frac{1}{2}}
A(\alpha) e^{\imath \lambda ( \alpha X - \omega(\alpha) \tau)
 + \imath \sigma \tfrac{\pi}{4}},
\]

where $\alpha$ is such that $c_g(\alpha) = \tfrac{X}{\tau}$. We have
put the equation into the form
\begin{equation}\label{eq:Ilim}
I \sim \frac{1}{\lambda^{\frac{1}{2}}} \cA(X,\tau)
e^{\imath \lambda \theta(X,\tau)},
\end{equation}

where
\[
\cA = \left( \frac{2 \pi}{ \abs{\psi''(\alpha)} \tau} \right)^{\frac{1}{2}}
A(\alpha) e^{\pm \tfrac{\imath \pi}{4}}
\]

and the phase function
\begin{equation}\label{eq:phs}
\theta(X,\tau) = \alpha X - \omega(\alpha) \tau,
\end{equation}

where $\alpha$ is determined by $c_g(\alpha) = \tfrac{X}{\tau}$.

Rapid changes are consigned to the phase $\lambda \theta(X,\tau)$ and
$\cA$, $\alpha$ and $\theta$ are all functions of the slow co-ordinates
$(X,\tau)$.

Expressions like \eqref{eq:Ilim} arise in many contexts
(for instance in non-uniform media and nonlinear waves) but then
it does not have the specific forms for $\cA$ and $\theta$.

In general, when we have expressions like
\eqref{eq:Ilim} we can define a \emph{local} wavenumber $k$ and frequency
$\omega$ by
\begin{equation}
k = \pd{\lambda \theta}{x} = \pd{\theta}{X} \quad \text{and} \quad
\omega = -\pd{\lambda \theta}{t} = -\pd{\theta}{\tau}.
\end{equation}

This agrees for the single wave where $\theta = k X - \omega \tau$ and
also agrees for \eqref{eq:phs} (as $X - c_g(\alpha) \tau = 0$).
Since $\theta_{X \tau} = \theta_{\tau X}$ we have

\begin{equation}\label{eq:phasecont}
\pddt{k} + \pd{\omega}{X} = 0.
\end{equation}

This is a continuity equation for $k$.  We thus interpret $k$ as
a wavenumber density and $\omega$ as the flux of wavenumber.
If $\omega = \Omega(k)$ then \eqref{eq:phasecont} becomes

\[
\pd{k}{t} + c_g(k) \pd{k}{X} = 0
\]

and so $k$ satisfies a nonlinear hyperbolic partial differential equation.
We introduce characteristics defined by $\diff{X}{\tau} = c_g(k)$
and have dynamics $\diff{k}{\tau} = 0$ on characteristics.
Thus $k$ is constant on characteristics and the situation is the same as
nonlinear gas dynamics.

\section{Water waves}

\vspace{1.5in}

Consider an incompressible invisicid fluid, so that $\vu = \nabla \phi$.
Incompressibility gives the Laplace equation $\nabla^2 \phi = 0$,
which we want to solve subject to the boundary conditions

\begin{gather}
\phi_y = 0 \text{ on } y = -h,\label{eq:water:bc1} \\
\DDt{} \left( y - \eta(x,z,t) \right) = 0 \text{ on } y = \eta \\
\text{which linearises to }\
\phi_y = \pddt{\eta} \text{ on } y = 0,\label{eq:water:bc2} \\
p - p_{\text{atm}} = T \left(\kappa_1 + \kappa_2\right)
\approx - T \left( \pd{^2 \eta}{x^2} + \pd{^2 \eta}{z^2} \right) \text{ on }
y = 0 \label{eq:water:bc3}.
\end{gather}

Bernoulli's equation for the pressure linearises to
\[
p - p_{\text{atm}} = - \rho g \eta - \rho \phi_t \text{ on } y = 0.
\]

We try a solution of the form
$\phi = f(y) e^{\imath k_1 x + \imath k_3 x - \imath \omega t}$.
By choice of axes we can take $k_3 = 0$ and $k_1 = 0$.  Laplace's equation
and \eqref{eq:water:bc1} give
$f(y) = \cA \cosh k(y+h)$.

For any hope that the equations balance we must have
$\eta = A e^{\imath k x - \imath \omega t}$ and then \eqref{eq:water:bc2}
gives
\[
A = \frac{\imath k \cA}{\omega} \sinh k h.
\]

Finally the Bernoulli equation and \eqref{eq:water:bc3} give the
dispersion relation

\begin{equation}\label{eq:water:disp}
\omega^2 = g k \left( 1 + \frac{T k^2}{\rho g} \right) \tanh k h.
\end{equation}

\subsection{Limiting cases}

When there is no surface tension we have \emph{gravity waves}.
The dispersion relation gives
\[
c_g = \frac{\omega}{2 k} \left( 1+ \frac{2 k h}{\sinh 2 k h}\right)
< \frac{\omega}{k} = c_p.
\]

Thus new waves are created at the back of the wavepacket and travel to
the front.

In the limit $kh \ll 1$ we have \emph{shallow water waves}.  The
dispersion relation is
\[
\omega^2 = g k^2 h \left(1 + \beta k^2\right),
\]

where $\beta = \tfrac{T}{\rho g}$.  The group velocity is
\[
c_g = \frac{\omega}{k} \frac{ 1 + 2 \beta k^2}{1 + \beta k^2} > c_p.
\]

The \emph{long wave} case is when $kh \to 0$, so that
$\omega^2 = g h k^2 = c_0^2 k^2$ and the waves are non-dispersive.

We get \emph{deep water capillary-gravity waves} when $k h \to \infty$,
so that the dispersion relation is
\[
\omega^2 = gk \left( 1+ \beta k^2 \right).
\] 

Thus $\omega \sim \sqrt{gk}$ as $k \to 0$ and $\omega \sim \sqrt{g
  \beta k^3}$ as $k \to \infty$.  The group velocity is
\[
c_g = \frac{\omega}{k} \frac{1 + 3 \beta k^2}{2(1+ \beta k^2)}
\to \begin{cases}
\frac{1}{2} c_p & k \to 0 \\
\frac{3}{2} c_p & k \to \infty.
\end{cases}
\]

We can draw a graph of $\omega$ against $k$.

\vspace{2in}

Suppose we have an initial disturbance with
$A(k) = 0$ for $k < k_0$ and $k > k_3$ with
$k_0 < k_1 < k_2 < k_3$ and $c_g(k_0) > c_g(k_3)$.

\vspace{1.5in}

The changes from $0$ waves to $2$ waves at $x= c_g(k_1)t$, from
$2$ waves to $1$ wave at $x = c_g(k_3) t$ and from $1$ wave to $0$
waves at $x = c_g(k_0) t$ are continuous, but the method of stationary
phase does not apply at any of these points (degeneracies).

\chapter{Ray theory}

This is a generalisation of the WKB method used for ODEs.%
\footnote{See the Methods of Math. Phys. notes for an introduction to
  this.}  It applies to slowly varying wavetrains in uniform and
non-uniform media.  These variations must be slow on the scale of the
wavelength / period.  We will neglect dissipation, nonlinear effects,
rapid variations (boundaries, interfaces, focussing points) and
diffraction.

\section{Ray tracing equations}

We seek slowly varying solutions of the partial differential equation
\begin{equation}\label{eq:pdoper}
\cL(\partial_t,\partial_{x_1}, \dots ; \vx,t) \phi = 0,
\end{equation}

for instance the equation modelling sound in the atmosphere in
which $\rho_0$ and $c_0$ vary with $z$:

\[
\cL \Tilde{p} = \pd{^2 \Tilde{p}}{t^2}
- \rho_0 c_0^2 \pd{}{z} \left( \frac{1}{\rho_0} \pd{\Tilde{p}}{z}\right)
- c_0^2 \pd{^2 \Tilde{p}}{x^2} - c_0^2 \pd{^2 \Tilde{p}}{z^2} = 0.
\]

We try a solution of the form

\begin{equation}\label{eq:pdetrial}
\phi = A(\vx,t;\epsilon) e^{\frac{\imath \theta(\vx,t)}{\epsilon}},
\end{equation}

where $\epsilon$ is a small parameter.  There is rapid phase fluctuation
but only slow change in $A$ and $\theta$.  In fact
\[
\phi_t = \frac{\imath \theta_t \phi}{\epsilon}
+ \frac{A_t \phi}{A}, \text{ and similarly for $\phi_x$ etc.}
\]

We substitute in \eqref{eq:pdoper} and extract the leading order terms
to get a dispersion relation
\begin{equation}\label{eq:raydisperse}
\cL(-\imath \omega, \imath k_1, \imath k_2, \imath k_3;\vx,t) = 0,
\end{equation}

where $\omega = - \frac{\theta_t}{\epsilon}$ and
$k_\alpha = \frac{\theta_{x_\alpha}}{\epsilon}$.  This gives
us the local frequency and wavenumber.

Now $\nabla \phi_t = \pddt{}\nabla \phi$ and so
\begin{equation}\label{eq:raytr1}
\nabla_\vx \omega = -\pddt{\vk}.
\end{equation}

We also know that $\pd{k_i}{x_j} = \epsilon^{-1} \pd{}{x_j}
\pd{\theta}{x_i} = \pd{k_j}{x_i}$ and we can therefore write 
\eqref{eq:raytr1} as
\[
\pddt{k_j} + \pd{\Omega}{k_i} \pd{k_j}{x_i} + \pd{\Omega}{x_j} = 0,
\]

where the dispersion relation is $\omega = \Omega(\vk;\vx,t)$.  If
we introduce the group velocity vector $\vect{c}_g = \nabla_\vk \Omega$
we can write these equatons as
\[
\pddt{\vk} + \vect{c}_g \cdot \nabla_\vx \vk = - \nabla_\vx \Omega.
\]

This has the characteristic form

\begin{equation}\label{eq:raytrace}
\ddt{\vk} = - \nabla_\vx \Omega \text{ on characteristics }
\ddt{\vx} = \vect{c}_g.
\end{equation}

A \emph{ray} is a path along the characteristics $\ddt{\vx} =
\vect{c}_g$ traversed at the group velocity and \eqref{eq:raytrace}
are the \emph{ray tracing equations}.

\subsection{Special cases}

In a homogeneous medium $\nabla_\vx \Omega = 0$ and therefore
$\vk$ is constant on each ray.  If all rays emanate from the same
uniform medium then $\vk$ is the same constant everywhere.  If (say)
$\Omega$ is independent of $x_\alpha$ then $k_\alpha$ is constant on
each ray.

The time-independent case is slightly harder.  In general
\begin{align*}
\ddt{\omega} &= \pd{\Omega}{k_j} \ddt{k_j} + \pd{\Omega}{x_j}
\ddt{x_j} + \pddt{\Omega} \\
&= \pd{\Omega}{k_j} \left( - \pd{\Omega}{x_j} \right)
+ \pd{\Omega}{x_j} \left( \pd{\Omega}{k_j} \right) + \pddt{\Omega}
\quad \text{(using the ray tracing equations)} \\
& = \pddt{\Omega}.
\end{align*}

If the medium is time-independent then $\Omega$ is constant on rays.
If both $\pddt{\Omega} = 0$ and $\nabla_\vx \Omega = 0$ then
$\omega = \Omega(\vk)$, $\vect{c}_g = \vect{c}_g(\vk)$ and $\vk$ is
constant on rays, which are straight.

\subsection{Hamilton's equations}

The ray tracing equations \eqref{eq:raytrace} can be written
\[
\ddt{k_j} = - \pd{\Omega}{x_j} \qquad \ddt{x_j}  = \pd{\Omega}{k_j}.
\]

These are Hamilton's equations for a Hamiltonian $\Omega$, generalised
co-ordinates $x_j$ and generalised momenta $k_j$.  Waves travel like
particles at the group velocity.

From the dispersion relation we get

\[
\pddt{\theta} + \epsilon \Omega(\epsilon^{-1} \nabla \theta; \vx,t) = 0.
\]

This is the Hamilton-Jacobi equation with the phase function as the
action.

\subsection{Wave action}

The energy density $E$ is not usually conserved as waves propagate,
although the wave action $I = E \omega^{-1}$ is.  In fact
\[
\pddt{I} + \dive_\vx \left(\vect{c}_g I\right) = 0.
\]

This follows from taking the next term of $\cL \phi = 0$, which
involves the amplitude $A$.  If $\omega$ is constant on rays
then $E$ is conserved.

\section{Gravity waves approaching a beach}

\vspace{2in}

Consider a sloping beach with shoreline at $x=0$ and a water depth
$h(x,z)$.  Assume that $h \to \infty$ and
$\vk = (k_1,k_3) \to k_\infty (\cos \phi_\infty,\sin \phi_\infty)$ as
$x \to - \infty$.

The dispersion relation for gravity waves is
$\omega^2 = \Omega^2(\vk;\vx) = g k \tanh k h$.  As $\Omega$ is
time independent, $\omega$ is constant on rays, and as all
rays originate from $x = -\infty$, $\omega^2 = g k_\infty$ everywhere.

Thus
\begin{equation}\label{eq:beach1}
\tanh kh = \tfrac{k_\infty}{k}
\end{equation}
and measuring $k$ is equivalent to measuring $h$.

Wave crests are given by $\theta = \text{const}$ and so
\[
\theta_x\, \ud x + \theta_z\,  \ud z = 0 \quad \Rightarrow
k_1\, \ud x + k_3\, \ud z \quad \Rightarrow \diff{x}{z} = - \frac{k_3}{k_1}.
\] 

Water waves are an example of \emph{isotropic dispersion}, that is
$\Omega = \Omega(k;\vx)$.  In such cases $\vect{c}_g = \Hat{\vk}
\pd{\Omega}{k}$ and the rays are parallel to $\vk$.  The rays are given
by the equation
\[
\diff{x}{z} = \frac{k_1}{k_3}.
\]

If $h = h(x)$ (independent of $z$) then $\Omega = \Omega(k;x)$ and
$\ddt{k_3} = 0$.  Therefore $k_3 = k_\infty \sin \phi_\infty$
everywhere.

As $h \to 0$, $\tanh k h \sim k h$ and so \eqref{eq:beach1} becomes
$k^2 \sim \tfrac{k_\infty}{h} \to \infty$ as $h \to 0$.  Now $k_3$
is fixed, so that $k_1 \to \infty$ as $h \to 0$.  We have proved that
waves hit the beach normal to the shoreline.

\section{Fermat's Principle}

The ray tracing equations are equivalent to the variational principle
\begin{equation}\label{eq:variwave}
\delta \int_{t_1}^{t_2} \Phi(\vx,\Dot{\vx}, \vk,t)\, \ud t = 0,
\end{equation}

where $\Phi = \vk \cdot \Dot{\vx} - \Omega(\vk;\vx,t)$ and $\delta \vx$
and $\delta \vk$ vary independently with $\delta \vx (t_1) = \delta \vx
(t_2) = 0$.\footnote{Proof is trivial.}

\subsection{Time independent case}

Suppose that $\pd{\Omega}{t} = 0$ and $\omega = \Omega$ is a uniform
constant.  We need to use a restricted class of variations $\delta
\vx$, $\delta \vk$ such that $\Omega(\vk + \delta \vk;\vx + \delta \vx) =
\Omega(\vk;\vx)$.  Then

\[
\delta \int_{t_1}^{t_2} \Omega\, \ud t = 0
\]

automatically and the variational principle \eqref{eq:variwave}
becomes
\[
\delta \int_{t_1}^{t_2}\, \vk \cdot \Dot{\vx}\, \ud t
= \delta \int_{\vx_1}^{\vx_2}\, \vk \cdot \ud \vx.
\]

If we further assume isotropy then $\vk$ is parallel to $\nabla_\vk
\Omega$ and hence on a ray $\vk$ is parallel to $\ud \vx$, so that
\[
\int_{\vx_1}^{\vx_2}\, \vk \cdot \ud \vx = \int_{\vx_1}^{\vx_2}\, k\, \ud s
= \omega \int_{\vx_1}^{\vx_2}\, \frac{\ud s}{c},
\]

where $\ud s$ is the element of arc length.  Since $\omega$ is a
constant,

\[
\delta \int_{\vx_1}^{\vx_2}\, \frac{\ud s}{c} = 0.
\]

This is the principle of least time --- \emph{Fermat's principle}.

\subsection{Snell's Law}

Suppose that $\Omega = \Omega(\vk;x_3)$ so that
$k_1$, $k_2$ and $\omega$ are constant on rays.

\vspace{1in}

Then
\[
\sin \alpha = \frac{\sqrt{k_1^2 + k_2^2}}{k} =
\frac{c \sqrt{k_1^2 + k_2^2}}{\omega}
\]

and so
\begin{equation}\label{eq:snellray}
\frac{\sin \alpha}{c} = \frac{\sin \alpha_0}{c_0} = \text{constant.}
\end{equation}

\subsection{Sound waves in inhomogeneous media}

Consider a ray in the $(x,z)$ plane and suppose that
$\pd{\Omega}{x} = \pd{\Omega}{y} = 0$, so that
$\omega = \Omega(k;x) = k c(z)$.  We want to know if there is a ray
between two points on the same horizontal plane.

\vspace{1.5in}

Either use the condition

\[
\delta \int_{-a}^a \frac{\ud s}{c} = 0 \quad \Rightarrow \quad
\diff{z}{x} = \pm \left( \frac{c_0^2}{c(z)^2} - 1
\right)^{\frac{1}{2}} \text{ by Euler-Lagrange equations,}
\]

\vspace{1in}

or use Snell's Law:

\[
\frac{\sin \alpha}{c} = \text{constant} \quad \Rightarrow \quad
\frac{\ud x}{\left( \ud x^2 + \ud z^2 \right)^{\frac{1}{2}}}
 = \frac{c(z)}{c_0},
\]

which gives the same ODE as before.  We take the positive root of our
ODE and note that initially $c > c_0$.  If $c(z)$ decreases as $z$
increases then $\diff{z}{x} > 0$ always and the ray can't bend back
down.  If $c$ increases with $z$ then there exists $z_0$ at which
$\diff{z}{x} = 0$ and the path curves back down.

For instance we can solve explicitly if $c = c_0 \left( 1+
  \tfrac{z}{\lambda} \right)$ and we see that the rays are given by
$(x-x_0)^2 + (z + \lambda)^2 = \lambda^2$. 

\section{Media with flow}

Consider a source of frequency $\omega_s$ moving with a uniform
velocity $\vect U$ in a medium at rest.  In a frame fixed to the
fluid, $(\vx,t)$ we suppose the dispersion relation is $\omega_r
= \Omega(\vk)$.  In a frame $(\vect X,t)$ fixed to the source,
$\vect X = \vx - \vect U t$ the fluid has velocity $- \vect U$.  Now
\[
\left. \pd{}{x_\alpha} \right|_t = \left. \pd{}{X_\alpha} \right|_t
\quad \text{and} \quad
\left. \pddt{} \right|_\vx = \left. \pddt{}\right|_{\vect X}
 - U_j \pd{}{X_j},
\]

and so $- \imath \omega_r = - \imath \omega_s - \imath \vk \cdot \vect U$,
which gives $\omega_s = \Omega_r(\vk) - \vk \cdot \vect{U}$.

\subsubsection*{The Doppler effect}

\vspace{1in}

Consider a source emitting sound waves at frequency $\omega_r = c k$.
Now $\vk \cdot \vect U = k U \cos \theta = \frac{\omega_r}{c} U \cos
\theta$.  Hence
\[
\omega_r = \frac{\omega_s}{1 - \frac{U}{c} \cos \theta},
\]

and (if $\tfrac{U}{c} < 1$), $\omega_r > \omega_s$ if $\cos \theta >
0$ (ahead of the source) and $\omega_r < \omega_s$ if $\cos \theta <
0$ (behind the source).

\subsection{Stationary capillary-gravity waves}

Consider a steady disturbance moving at a velocity $(U,0,0)$
generating a steady one dimensional wave pattern.  Now
$\omega_s = \omega_r - k U$, where
\[
\omega_r = \pm \left( g k + \frac{T k^3}{\rho} \right)^{\frac{1}{2}}
\text{ (deep water).}
\]

Steady waves satisfy $\omega_s = 0$, so that
\[
k U = \pm \left( g k + \frac{T k^3}{\rho} \right)^{\frac{1}{2}}.
\]

For simplicity we take units such that $g = \tfrac{T}{\rho} = 1$, so
that $k U = \pm \left( k + k^3 \right)^{\frac{1}{2}}$.

\vspace{2in}

If $U < \sqrt{2}$ then there is no solution (barring the trivial one
at the origin) and no steady waves are possible.  If $U > \sqrt{2}$
there are solutions: if $k > k_2$ then $c_g > U$ and if $k < k_2$,
$c_g < U$.  Since the group velocity relative to the flow must be
positive, short (capillary) waves are only found ahead of the
disturbance and long (gravity) waves are only found behind it.

\subsection{Ship waves}

We consider two dimensional steady waves in a mean flow.
The approximations we make are that a ship is a point disturbance, the
water is deep and there is no surface tension.  Then $\Omega_r =
\sqrt{g k}$ and we have the condition
$( \omega_s + \vk \cdot \vect U)^2 = gk$ (for steady waves), giving
\begin{equation}\label{eq:shipdisp}
\omega_s = \pm g^{\frac{1}{2}} \left( k_1^2 + k_3^2
\right)^{\frac{1}{4}} - k_1 U.
\end{equation}

This is an anisotropic dispersion relation, so that $\vect{c}_g$ is not
parallel to $\vk$.  We seek a steady wave pattern with $\omega_s = 0$.
According as $k_1 \gtrless 0$ we need the $\pm$ signs in
\eqref{eq:shipdisp}, so that
\[
k_1 U = \pm g^{\frac{1}{2}} \left( k_1^2 + k_3^2 \right)^{\frac{1}{4}}
\]

and

\[
\pd{\omega}{k_1} = - \frac{U (k^2 + k_3^2)}{2 k^2} < 0.
\]

This means that the waves are behind the ship.  Since $\Omega_t = 0$
and $\nabla_{\vx} \Omega = 0$ the rays are straight.  Write
$\diff{z}{x} = \tan \psi$ on such rays.

Now
\[
\diff{z}{x} = \frac{\diff{z}{t}}{\diff{x}{t}} = - \frac{k_1 k_3}{k^2 + k_3^2}.
\]

Writing $\vk = k(\cos \phi, \sin \phi)$ gives
\begin{equation}\label{eq:shipk}
k^{\frac{1}{2}} = \pm \frac{g^{\frac{1}{2}}}{U \cos \phi} \qquad
\text{and} \qquad
\tan \psi = - \frac{\tan \phi}{1 + 2 \tan^2 \phi}.
\end{equation}

We need $\abs{\pi - \psi} < \tfrac{\pi}{2}$ in order to have waves
behind the ship.  We restrict (WLOG) to $-\tfrac{\pi}{2} < \phi <
\tfrac{\pi}{2}$, and note that there is a maximum value of $\abs{\pi -
  \psi}$ when $\tan^2 \phi = \tfrac{1}{2}$ and $\tan \psi = -
\frac{1}{2 \sqrt{2}}$.

Waves are thus confined to a wedge of semi-angle $19
\tfrac{1}{2}^\circ$.  The phase can be calculated by

\begin{align*}
\theta &= \int_{\text{any path}} \nabla \theta \cdot \vect{\ud x} \\
&= \vk \cdot \vx \qquad \text{choosing the path as the ray $\psi =
  \text{const}$.} \\
&= k r \cos (\psi - \phi) \\
&= k r \frac{\cos \psi}{\cos \phi} \left( \frac{1}{1+ 2 \tan^2 \phi} \right).
\end{align*}

Hence
\[
x = r \cos \psi = \frac{ \theta \cos \phi \left( 1+2 \tan^2 \phi \right)}{k}
\]

and using \eqref{eq:shipk} we obtain a parametric equation for the
wave crest shape,

\begin{align*}
x &= \frac{\theta U^2}{g} \cos^3 \phi ( 1+ 2 \tan^2 \phi) \\
z &= - \frac{\theta U^2}{g} \cos^2 \phi \sin \phi.
\end{align*}

\section[Internal gravity waves]
{Internal gravity waves in a stratified incompressible medium}

Consider a fluid medium in which the mean pressure $p_0(z)$ and the
density $\rho_0(z)$ are in hydrostatic balance, i.e.

\[
\diff{p_0}{z} = - \rho_0 g
\]

when there is no motion.  Assume that the vertical lengthscale for
$\cO(1)$ changes in $\rho_0$ is $L$.  Since the fluid is
incompressible,
\[
\DDt{\rho} = 0 \qquad \text{and} \qquad \nabla \cdot \vu = 0.
\]

Further, the motion is governed by the momentum equation
\[
\rho \DDt{\vu} = - \nabla p - \rho g \Hat{\vect{z}}.
\]

We linearise the motion about the mean state, $p = p_0 + \Tilde{p}$
and $\rho = \rho_0 + \Tilde{\rho}$, so that (writing $\vu = (u,v,w)$)
we have

\begin{align}
\Tilde{\rho}_t + w \rho_0'(z) &= 0, \label{eq:ig1} \\
u_x + v_y + w_z &= 0, \label{eq:ig2} \\
\rho_0 u_t &= - \Tilde{p}_x, \label{eq:ig3} \\
\rho_0 v_t &= - \Tilde{p}_y, \label{eq:ig4} \\
\rho_0 w_t &= - \Tilde{p}_z - \Tilde{\rho} g. \label{eq:ig5}
\end{align}

Equations \eqref{eq:ig1} and \eqref{eq:ig5} give
\[
\rho_0 w_{tt} = - \Tilde{p}_{zt} + \rho_0' g w, 
\]

and equations \eqref{eq:ig2}, \eqref{eq:ig3} and \eqref{eq:ig4} give
\[
\rho_0 w_{zt} = \Tilde{p}_{xx} + \Tilde{p}_{yy}.
\]

Hence
\[
\left(\rho_0 w_{zz} + \rho_0' w_z \right)_{tt} -
\left(\partial^2_x + \partial^2_y \right) \left( \rho_0' g - \rho_0
  \partial^2_t \right) w = 0,
\]

or $\cL(\partial_t, \partial_x, \partial_y, \partial_z; z) w = 0$.
We can make a slowly varying assumption if a plane harmonic wave has a
relatively short vertical lengthscale: $\tfrac{2 \pi}{k_3} \ll L$
where $\tfrac{\rho_0}{\rho_0'} = \cO(L)$, giving $\rho_0' w_z \ll
\rho_0 w_{zz}$.  The leading order dispersion relation is
\[
- \omega^2 \left( - \rho_0 k_3^2 \right) + (k_1^2 + k_2^2) (\rho_0' g
  + \rho_0 \omega^2) = 0,
\]
or equivalently
\begin{equation}\label{eq:igwdisp}
\omega^2 = \Omega^2(\vk,z) \equiv \frac{N^2(z) (k_1^2 + k_2^2)}{k_1^2 +
  k_2^2 + k_3^2},
\end{equation}

where
\[
N^2(z) = - \frac{g \rho_0'}{\rho_0} > 0.
\]

$N$ is the \emph{Brunt-V\"ais\"al\"a frequency}.

Note that

\begin{itemize}
\item No solutions to \eqref{eq:igwdisp} are possible if $\omega^2 > N^2$.
\item The group and phase velocities are perpendicular.
\item $\omega$, $k_1$ and $k_2$ are always constant on rays, and $k_3$
  is only constant if $N' = 0$, that is $\rho_0 = \Bar{\rho}
  e^{-z/L}$, where $L = N^{-2}g$ and $\Bar{\rho}$ is a constant.
\end{itemize}

Consider an experiment where waves are generated by a two-dimensional
oscillating cylinder ($\omega < N$).  Then if $\vk = k(\cos \phi,0,\sin
\phi)$, $\cos \theta = \pm \tfrac{\omega}{N}$ and
\[
\vect{c}_g = \frac{1}{k} \left( \pm N \sin^2 \phi,0,-\omega \sin \phi\right).
\]

Assuming that $\vk$ is constant (and hence all rays straight) we see
that downward propagating waves imply upward propagating energy.

\backmatter

\begin{thebibliography}{9}

\bibitem{Courant} Courant \& Friedrichs, \emph{Supersonic Flow and
    Shock Waves}, Springer-Verlag, 1976.

  {\sffamily \small This is quite good for the nonlinear waves part of
  the course.  You might be lucky to find a copy of this --- I found
  it second hand in Hay-on-Wye and have never seen it anywhere else.}

\bibitem{FluidsNotes} \emph{Fluid Dynamics II}, unpublished, 1997.
  
  {\sffamily \small I'm biased. Somebody send me a review.}

\bibitem{Hinch} E.J.~Hinch, \emph{Perturbation Methods}, CUP, 1991.
  
  {\sffamily \small Reasonably interesting book on a potentially
    extremely dull subject.}

\bibitem{MMPnotes} \emph{Methods of Mathematical Physics},
  unpublished, 1997.

  {\sffamily \small Again, I'm biased.}

\bibitem{VanDyke} M. van Dyke, \emph{An Album of Fluid Motion}, The
  Parabolic Press, 1982.
  
  {\sffamily \small Lots of pictures of flows.  An excellent book.  Go
    out and buy it.  Now.}

\end{thebibliography}

I only found one book I liked for this course.  There \emph{must} be
other good ones.  The MMP notes and Fluids notes are useful
background, as is Hinch.

\end{document}
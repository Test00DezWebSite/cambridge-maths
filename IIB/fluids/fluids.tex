\documentclass{notes}

\usepackage{varioref}

\newcommand{\cO}{\mathcal{O}}
\newcommand{\om}{\boldsymbol{\omega}}
\newcommand{\bt}{\boldsymbol{\tau}}
\newcommand{\bo}{\boldsymbol{\Omega}}
\newcommand{\Rey}{\mathrm{Re}}
\newcommand{\Ca}{\mathrm{Ca}}
\newcommand{\bs}{\boldsymbol{\sigma}}
\newcommand{\e}{\mathbf{e}}

\theoremstyle{plain}
\newtheorem*{theorem}{Theorem}

\begin{document}

\frontmatter

\title{Fluid Dynamics 2}

\lecturer{Dr.~J.~M.~Rallison}
\maintainer{Paul Metcalfe}
\date{Mich\ae lmas 1997} \maketitle

\thispagestyle{empty}
\noindent\verb$Revision: 2.6 $\hfill\\
\noindent\verb$Date: 2002/10/02 09:12:17 $\hfill

\vspace{1.5in}

The following people have maintained these notes.

\begin{center}
\begin{tabular}{ r  l}
-- date & Paul Metcalfe
\end{tabular}
\end{center}

I have included some of EJH's updates to these notes and will include
more as time progresses.

\tableofcontents

\chapter{Introduction}

These notes are based on the course ``Fluid Dynamics 2'' given by
Dr.~J.M.~Rallison in Cambridge in the Mich\ae lmas Term 1997.  These
typeset notes are totally unconnected with Dr.~Rallison.  Recommended
books will be discussed at the end.

\alsoavailable
\archimcopyright

\mainmatter

\chapter{Review of inviscid fluids}

\section{Continuum hypothesis}

We assume that at each point $\vx$ of the fluid we can define, by
averaging over a small volume,
properties like density $\rho(\vx,t)$, velocity $\vu(\vx,t)$ and
pressure $p(\vx,t)$ and that these vary smoothly over the fluid.  We
do not deal with the dynamics of individual molecules.

\section{Time derivatives}

A \emph{fluid particle}, sometimes called a \emph{material element} or
\emph{Lagrangian point}, is one that moves with the fluid,
so that its position $\vx(t)$ satisfies $\dot{\vx} = \vu(\vx,t)$.

The rate of change of some quantity moving with the fluid is written
$\DDt{}$; the chain rule gives
\begin{equation}\label{eq:matder}
\DDt{} = \pd{}{t} + \vu\cdot\nabla.
\end{equation}

In particular, the acceleration of a fluid particle is
$\DDt{\vu} = \pd{\vu}{t} + \vu \cdot \nabla \vu$.

\section{Mass conservation}

Since mass is conserved the mass density $\rho$ satisfies
$\pd{\rho}{t} + \nabla \cdot (\rho \vu) = 0$.  The quantity
$\rho \vu$ is called the \emph{mass flux}.

For an \emph{incompressible} fluid the density of each material element
is constant, and so $\DDt{\rho} = 0$.  Thus $\nabla \cdot \vu = 0$.

In this course we will restrict to fluids which are incompressible and
have uniform density so that $\rho$ is independent of both $\vx$ and
$t$.

For planar flows, the condition $\nabla \cdot \vu = 0$ is
automatically satisfied if we have $\vu = \nabla \times (0,0,\psi(x,y))$,
so that $\vu = (\psi_y,-\psi_x,0)$.  $\psi(x,y)$ is called the streamfunction.

\section{Kinematic boundary condition}

Applying mass conservation to a region close to a boundary $S$
we get $\vn\cdot\vu_+ = \vn \cdot \vu_-$ at $S$.

\vspace{1in}

This states that the normal component of velocity is continuous across $S$.
In particular, if $S$ is fixed we have $\vn.\vu = 0$ at $S$.

The kinematic boundary condition can be written a different way.  Suppose
the boundary of a fluid is given by $F(\vx,t) = 0$.  Then since the
surface consists of material points $\DDt{F} = 0$.   This is
sometimes more convenient for free surface problems.

\section{Momentum conservation}\label{sec:momc}

\emph{Assuming} that the only force acting across a material surface
$\vn\, \ud S$ is given via a pressure $p(\vx,t)$ as $-p \vn\, \ud S$ then we
obtain Euler's equation:

\begin{equation}\label{eq:euler}
\rho \DDt{\vu} = - \nabla p + \rho \vF(\vx,t),
\end{equation}

where $\vF(\vx,t)$ is the body force per unit mass (for instance
gravity) that acts on the fluid.

\section{Dynamic boundary condition}\label{sec:dynbc}

On the same assumption, applying momentum conservation to a region close to
a boundary $S$ gives $-p_- \vn = - p_+ \vn$ in the absence of surface
tension.

In this course we will abandon the assumptions of \S\ref{sec:momc}
and \S\ref{sec:dynbc} and include tangential
frictional forces across material surfaces.

\section{Steady flow past a circular cylinder}

The steady Euler equation with $\vF = 0$ is satisfied if
$\vu = \nabla \phi$ and $p + \tfrac{1}{2} \rho \abs{\vu}^2 = \text{const}$.
The incompressibility condition $\nabla \cdot \vu = 0$ becomes
$\nabla^2 \phi = 0$.

A solution with $\phi \sim U r \cos \theta$ as $r \to \infty$ (uniform
stream with velocity $U$) and $\vu . \vn = 0$ on $r = L$ is

\begin{equation}\label{eq:circcylpot}
\phi = U \left( r + \frac{L^2}{r} \right) \cos \theta,
\end{equation}

with streamfunction

\begin{equation}\label{eq:circcylstr}
\psi = U \left( r - \frac{L^2}{r} \right) \sin \theta.
\end{equation}

The tangential velocity on $r = L$ is $2 U \sin \theta$.

\chapter[The governing equations]%
{The governing equations for a Newtonian fluid}

\section{Viscosity}

Suppose we have two parallel plates a distance $h$ apart, and we put
fluid between them.

\vspace{1in}

What force per unit area on the top plate is needed to keep it moving
at a velocity $U$?  Experiments show that it is proportional to
$\tfrac{U}{h}$, and measuring the flow profile shows that
$\tfrac{U}{h} = \pd{u}{y}$.

The coefficient of proportionality is the \emph{viscosity}, $\mu$.
It has dimensions $M L^{-1} T^{-1}$.

\section{Rate of strain tensor}

Consider the fluid motion near a point $0$.  Then
\[
u_i(\vx) = u_i(0) + x_j \left.\pd{u_i}{x_j}\right|_0
+ \frac{1}{2} x_j x_k \left.\pd{^2 u_i}{x_j x_k}\right|_0 + \dots.
\]

Thus
\[
u_i(\vx) - u_i(0) \approx x_j \left. \pd{u_i}{x_j}\right|_0.
\]

$\pd{u_i}{x_j}$ is called the velocity gradient, and is sometimes written
$\left( \nabla \vu \right)_{j i}$.

The symmetric part of the velocity gradient is the rate of strain
tensor,
\begin{equation}\label{eq:rateofstrain}
e_{i j} = \frac{1}{2} \left( \pd{u_i}{x_j} + \pd{u_j}{x_i}\right)
\end{equation}
and the antisymmetric part is the vorticity tensor
\begin{equation}\label{eq:vort}
\Omega_{i j} = \frac{1}{2} \left( \pd{u_i}{x_j} - \pd{u_j}{x_i}\right)
= -\frac{1}{2} \epsilon_{i j k} \omega_k,
\end{equation}

where the vector $\om$ is the vorticity, $\om = \nabla \times \vu$.

Thus
\[
u_i(\vx) - u_i(0) \approx x_j e_{i j} + \frac{1}{2} \left( \om \times \vx
\right)_i.
\]

The $\om \times \vx$ part of this is a solid body rotation, which we
hope causes no stress.\label{ref:vnostress}

$e_{i j}$ is symmetric and so can be diagonalised.  Its eigenvalues
$\alpha$, $\beta$ and $\gamma$ are the principal rates of strain.
Note that $\alpha + \beta + \gamma = e_{i i} = \nabla \cdot \vu = 0$.

\section{The stress tensor}\label{sec:stress}

The forces acting on a fluid are of two kinds.
\begin{enumerate}
\item Volume or body forces.  These have a long range and are proportional
to the \emph{volume} of a fluid element.  (gravity)
\item Surface tractions.  These have a short range and are proportional
to the surface area of a fluid element.
\end{enumerate}

Let $\vn \ud S$ be an arbitrary element of area drawn in the fluid at
$(\vx,t)$.  We write the force exerted by the fluid on the
$+$ side of $\ud S$ on the fluid on the $-$ side as $\bt \ud S$.  Here
we establish our convention: normals point out of the fluid.

$\bt$ is called the surface traction and depends on $\vx$, $t$ and $\vn$.

\begin{theorem}
We claim that $\bt$ is linearly related to $\vn$, that is
\[
\tau_i = \sigma_{i j} n_j.
\]

$\sigma_{i j}(\vx,t)$ is a second rank tensor called the Cauchy stress
tensor.
\end{theorem}

\begin{proof}
Let $V(t)$ be an arbitrary material volume with surface $S(t)$.  The
momentum of the fluid in $V(t)$ is thus
\[
\int_{V(t)} \rho \vu \ud V,
\]
and so the equation of motion for the fluid in $V(t)$ is
\begin{equation}\label{eq:inteqm}
\diff{}{t} \int_{V(t)} \rho \vu \ud V = \int_{V(t)} \rho \vF \ud V
+ \int_{S(t)} \bt \ud S.
\end{equation}

Now suppose that $V(t)$ is small, with linear dimension $\epsilon$.
As volume integrals are $\cO(\epsilon^3)$ and surface integrals
are $\cO(\epsilon^2)$, and in the limit $\epsilon \to 0$ the
equation of motion must balance at leading order, we have
\[
\lim_{\epsilon \to 0} \int_{S(t)} \bt \ud S = 0.
\]

\vspace{1in}

Now let $V$ be instantaneously a small tetrahedron (as sketched),
with a sloping face having area $\ud S$ and normal $\vn$.  The areas
of the other faces are therefore
$\ud S \vect{i}\cdot \vn$, $\ud S \vect{j} \cdot \vn$ and $\ud S
\vect{k} \cdot \vn$, where $\vect{i}$, $\vect{j}$ and $\vect{k}$ are the
usual unit vectors.

Since the surface forces on the this tetrahedron must balance,
\[
\bt(\vn) \ud S + \bt(-\vect{i}) \vect{i}\cdot \vn \ud S
+ \bt(-\vect{j}) \vect{j}\cdot \vn \ud S
+ \bt(-\vect{k}) \vect{k}\cdot \vn \ud S = 0.
\]

Eliminating $\ud S$ we obtain a linear relationship between $\bt$
and $\vn$.
\end{proof}

With this expression for the surface tractions, the equation of motion
\eqref{eq:inteqm} is
\[
\diff{}{t} \int_{V(t)} \rho u_i \ud V
= \int_{V(t)} \rho F_i \ud V + \int_{S(t)} \sigma_{i j} n_j \ud S.
\]

Using the divergence theorem and the fact that $V(t)$ is a material volume,
\[
\int_{V(t)} \rho \DDt{u_i} \ud V = \int_{V(t)} \rho F_i \ud V
+ \int_{V(t)} \pd{\sigma_{i j}}{x_j} \ud V.
\]

Since $V$ is arbitrary, we obtain the Cauchy momentum equation

\begin{equation}\label{eq:cauchy}
\rho \DDt{u_i} = \rho F_i + \pd{\sigma_{i j}}{x_j}.
\end{equation}

If $\vF = 0$ then momentum must be conserved.  We can rewrite
the Cauchy momentum equation in the form
\[
\pd{}{t} \left( \rho u_i \right)
+ \pd{}{x_j} \left( \rho u_i u_j - \sigma_{i j}\right) = 0.
\]

This is the the form of a conservation equation for the momentum
$\rho \vu$, and we identify the \emph{momentum flux tensor}
as $\rho u_i u_j - \sigma_{i j}$.

\begin{theorem}
Provided no body couples act on the fluid, $\sigma_{i j} = \sigma_{j i}$.
\end{theorem}

\begin{proof}
Taking the origin to lie instantaneously within $V(t)$, the
angular momentum of the fluid in $V$ is
\[
\int_{V(t)} \rho \vx \times \vu\, \ud V,
\]
which is $\cO(\epsilon^4)$.  Conservation of angular momentum
implies that in the absence of body couples%
\footnote{As would arise for a suspension of orientable magnetic
particles in an external magnetic field.}

\[
\diff{}{t} \int_{V(t)} \rho \vx \times \vu\, \ud V
= \int_{V(t)} \rho \vx \times \vF\, \ud V + \int_{S(t)} \vx \times
\bt \ud S.
\]

The last term here is $\cO(\epsilon^3)$ and is therefore of lower order
than the other two terms.  Therefore at leading order it must vanish:
\[
\lim_{\epsilon \to 0} \int_{S(t)} \vx \times \bt \ud S = 0.
\]

Using our result for $\tau$, the $i^{\text{th}}$ component of this
can be written
\begin{align*}
\int_{S(t)} \epsilon_{i j k} x_j \sigma_{k m} n_m\, \ud S
&= \int_{V(t)} \epsilon_{i j k} \pd{}{x_m}\left(x_j \sigma_{k m}\right)
\ud V \\
&= \int_{V(t)} \epsilon_{i j k} x_j \pd{\sigma_{k m}}{x_m}\, \ud V +
\int_{V(t)} \epsilon_{i j k} \sigma_{k j}\, \ud V.
\end{align*}
The integrals here are $\cO(\epsilon^4)$ and $\cO(\epsilon^3)$ respectively,
so letting $\epsilon \to 0$ we obtain $\epsilon_{i j k} \sigma_{k j} = 0$
and so the stress tensor is symmetric.
\end{proof}

\section[The constitutive equation]%
{The constitutive equation for a Newtonian fluid}

We define the pressure $p = -\tfrac{1}{3} \sigma_{i i}$, and so
\[
\sigma_{i j} = - p \delta_{i j} + \sigma'_{i j},
\]

where $\sigma'_{i j}$, the \emph{deviatoric stress}, is tracefree.

We \emph{assume} that $\sigma'$ is \emph{linearly related} to the
\emph{instantaneous} value of $\nabla \vu$ and that the fluid is
isotropic.  Thus we have
\[
\sigma'_{i j} = A_{i j k l} \pd{u_k}{x_l},
\]
with $A$ a property of the fluid, and so isotropic.  Therefore
\[
A_{i j k l} = \lambda \delta_{i j} \delta_{k l}
+ \mu \delta_{i k} \delta_{j k} + \mu' \delta_{i l} \delta_{j k}
\]
and so
\[
\sigma'_{i j} = \lambda \delta_{i j} \nabla \cdot \vu
+ \mu \pd{u_i}{x_j} + \mu' \pd{u_j}{x_i}.
\]

As the stress tensor is symmetric, $\mu = \mu'$.  $\mu$ is the
(shear) viscosity.  Thus ${\sigma'_{i j} = 2 \mu e_{i j}}$ and we obtain
the constitutive equation for a Newtonian fluid,

\begin{equation}\label{eq:constit}
\sigma_{i j} = - p \delta_{i j} + 2 \mu e_{i j}.
\end{equation}

As we hinted on \vpageref{ref:vnostress}, this does not depend on
vorticity.  In general $\mu$ depends on temperature and so can depend
on position.

We can now substitute \eqref{eq:constit} into \eqref{eq:cauchy} to
obtain the Navier-Stokes equations

\begin{equation}\label{eq:ns}
\begin{split}
\rho \DDt{u_i} &= - \pd{p}{x_i} + \rho F_i + \mu \nabla^2 u_i \quad \text{or
in vector notation} \\
\rho \DDt{\vu} & = - \nabla p + \rho \vF + \mu \nabla^2 \vu.
\end{split}
\end{equation}

\section{Boundary conditions}

We can keep the kinematic boundary condition (that only depended on
mass conservation) and so the normal component of $\vu$ is continuous at
a boundary.

At a boundary $S$ no net force can be applied to a pillbox; therefore
the surface tractions must balance.  Thus $\left[ \bs \cdot \vn
\right]_S = 0$.  More generally, if surface tension acts at $S$, a net
force parallel to $\vn$ and proportional to the curvature $\left(
  R_1^{-1} + R_2^{-1} \right)$ (where $R_1$ and $R_2$ are the
principal radii of curvature) appears and we have
\[
\left[\bs \cdot \vn \right]_S = \gamma \left( \frac{1}{R_1}
+ \frac{1}{R_2} \right) \vn.
\]

$\gamma$ is the surface tension coefficient.  We can see that
in the inviscid case $\mu = 0$ we get back the dynamic boundary condition.

We need an extra boundary condition as a $\nabla^2$ term has appeared.
We assume that $\vn \times \vu$ is continuous --- this is the no-slip condition.

\section{The energy equation}

Taking (as before) $V(t)$ to be an arbitrary material volume and
$S(t)$ its surface, the kinetic energy of the fluid inside $V(t)$ is

\[
E = \frac{\rho}{2} \int_{V(t)} \abs{\vu}^2\, \ud V
\]

and the rate of change of this energy is given by the Cauchy equation
\eqref{eq:cauchy} as
\[
\diff{E}{t} = \rho \int_{V(t)} u_i \DDt{u_i}\, \ud V
= \rho \int_{V(t)} u_i F_i\, \ud V + \int_{V(t)} u_i \pd{\sigma_{i j}}{x_j}\,
\ud V.
\]

The final term here may be written
\begin{align*}
\int_{V(t)} u_i \pd{\sigma_{i j}}{x_j}\, \ud V &=
\int_{V(t)} \left\{ \pd{}{x_j} \left(u_i \sigma_{i j}\right)
- \sigma_{i j} \pd{u_i}{x_j} \right\}\ud V \\
&= \int_{S(t)} u_i \sigma_{i j} n_j\, \ud S
- \int_{V(t)} \sigma_{i j} e_{i j}\, \ud V \\
&= \int_{S(t)} u_i \tau_i\, \ud S - 2 \mu \int_{V(t)} e_{ij} e_{ij}\, \ud V.
\end{align*}

We have therefore shown that
\[
\diff{E}{t} = \rho \int_{V(t)} u_i F_i\, \ud V + \int_{S(t)} u_i \tau_i\,
\ud S - 2 \mu \int_{V(t)} e_{i j} e_{i j}\, \ud V.
\]

The first two terms on the right represent the rate of working by body and
surface forces respectively, so the final term must be the rate of energy
dissipation due to viscosity.  The rate of viscous heating
per unit volume $\Phi = 2 \mu e_{i j} e_{i j}$, and the second
law of thermodynamics demands that $\Phi$ and therefore $\mu$ must be
positive.

This heating can change the temperature in the fluid.  If then the
density or viscosity depend on temperature then a further equation involving
the convection and diffusion of heat is needed to determine the temperature.
We shall not persue this (interesting) complication in this course.

Using the momentum flux equation we may alternatively write the energy
equation in the form
\[
\pd{}{t} \left( \tfrac{1}{2} \rho \abs{\vu}^2 \right) + u_i \pd{}{x_j} \left(
\rho u_i u_j - \sigma_{i j} \right) = \rho u_i F_i.
\]

Hence
\[
\pd{}{t} \left( \tfrac{1}{2} \rho \abs{\vu}^2 \right) + \pd{}{x_j} \left(
u_j \tfrac{1}{2} \rho \abs{\vu}^2 - u_i \sigma_{i j} \right) = \rho u_i F_i
- \Phi.
\]

We can thus identify the energy flux vector $\vect{q}$ by,
\[
q_i = \tfrac{1}{2} \rho \abs{\vu}^2 u_i  - \sigma_{i j} u_j,
\]
and the energy flux equation takes the canonical form
\begin{multline*}
\pd{}{t} \left(\text{energy}\right) + \nabla \cdot \left(\text{energy flux} \right)
=\\ \left( \text{rate of doing work by body forces} \right) -
\left( \text{loss of energy due to viscous heating} \right).
\end{multline*}

\section{Scaling estimates}

We would like to know when it is appropriate to treat a real fluid as if
it were incompressible or inviscid.  We can get crude estimates as follows.

Suppose the flow has a typical velocity $U$, lengthscale $L$ and timescale
$T$.  For a sphere in a uniform steady stream, $U$ is the far-field velocity,
$L$ is the radius of the sphere and $T$ is infinite.

If we now wobble this sphere with angular frequency $\omega$ we would keep
$U$ and $L$ the same, but $T$ becomes $\omega^{-1}$.

\subsection{Compressibility}

The pressure $p$ depends on the mass density $\rho$, and it may be shown that
for small variations in density about some ambient level $\rho$ we have
\[
\diff{p}{\rho} = c^2,
\] 
where $c$ is the speed of sound in the fluid.  Thus
\[
\frac{\Delta \rho}{\rho} \sim \frac{p}{\rho c^2}.
\]

The Navier-Stokes equations \eqref{eq:ns} give
\begin{align*}
\frac{p}{L} \sim \nabla p &\sim \max \left\{ \rho \pd{\vu}{t},
\rho \vu \cdot \nabla \vu, \mu \nabla^2 \vu  \right\}
& \sim \max \left\{\frac{\rho U}{T}, \frac{\rho U^2}{L}, \mu frac{U^2}{L^2}
\right\}.
\end{align*}

Thus
\[
\frac{\Delta \rho}{\rho} \sim \max \left\{ \frac{L U}{c^2 T},
\frac{U^2}{L^2}, \frac{\mu}{\rho} \frac{U}{c^2 L}
\right\}.
\]

The fluid is incompressible if $\frac{\Delta \rho}{\rho} \ll 1$,
and in practice the Mach number, $\frac{U}{c} \ll 1$.

\subsection{Viscosity}

For a steady flow $\frac{\rho U^2}{L} \sim \frac{\mu U}{L^2}$,
and the ratio of inertial forces over viscous forces (the Reynolds number)
is
\[
\Rey = \frac{\rho U L}{\mu}.
\]

The ratio $\nu = \frac{\mu}{\rho}$ appears.  This is the kinematic viscosity.

If $\Rey \gg 1$ then the viscosity is negligible and inertia dominates. If
$\Rey \ll 1$ then inertia is negligible.

If we have rectilinear flow, $\vu = (U(y,z),0,0)$ then $\vu \cdot \nabla \vu
= 0$ and $\frac{U^2}{L}$ is a bad estimate of this...

For some unsteady flows the inertial term $\rho \pd{\vu}{t}$ wins,
in this case we get an unsteady Reynolds number $\frac{L^2}{\nu T}$.  This
is also called the Stokes number.

\chapter{Low Reynolds number flows}\label{chap:stokes}

In this case $\Rey \ll 1$ and we neglect inertial terms in the
Navier-Stokes equations \eqref{eq:ns} to obtain

\begin{equation}\label{eq:stokes}
\mu \nabla^2 \vu = \nabla p - \rho \vF \qquad \nabla \cdot \vu = 0.
\end{equation}

If $\vF \equiv 0$ then these are the Stokes equations.  Natural
boundary conditions are that at each point of $S$ either $\vu$ or
$\bs \cdot \vn$ is given.

\eqref{eq:stokes} can also be written

\begin{equation} \label{eq:stokes2}
\pd{\sigma_{i j}}{x_j} = - \rho F_i \qquad \sigma_{i j} = - p \delta_{i j}
+ 2 \mu e_{i j} \qquad \nabla \cdot \vu = 0.
\end{equation}

\section{Properties of the Stokes equations}

\subsection{Instantaneity}

There are no time derivatives in \eqref{eq:stokes}.  Thus $\vu$ responds
instantaneously to the boundary motion (and the force $\vF$).  There
is thus an infinite propagation speed; this situation is sometimes called
``quasi-static''.

For instance, in a sphere falling in an unbounded fluid then the terminal
velocity is acheived at once.  For a sphere falling towards a wall then
the change in velocity is due only to the change in the fluid domain.

\subsection{Linearity}

There is no $\vu \cdot \nabla \vu$ term in \eqref{eq:stokes}; therefore
$\vu$, $p$ and $\bs$ are linearly forced by any boundary motion
(or body force).

For instance; if we have a falling sphere, doubling the velocity will double
$\bs$ and thus double the drag.  More generally,
$\text{force} \propto \text{velocity}$ (as opposed to acceleration).

Another example; if we have a moving ellipsoid the problem can be solved by
superimposing the solutions when the ellipsoid moves along its principal axes.

\subsection{Reversibility}

If the velocity on the boundary of a Stokes flow is reversed then so is
the velocity everywhere in the fluid.  If a prescribed boundary motion is
reversed over time then each material point retraces its history.

\vspace{1in}

Does a sphere falling by a wall migrate towards/away from the wall?
No --- on reversal of $\vect{g}$, $\vu$ must reverse and so if the
sphere migrates to the wall under $\vect{g}$ then it must migrate away
from the wall under $-\vect{g}$ (and similarly for the other case).

\subsection{Uniqueness}

There exists at mosts one Stokes flow in a volume $V$ for which $\vu$
is specified on the boundary.

\begin{proof}
Suppose $\vu_1$ and $\vu_2$ are two such flows.  Let $\vu^\ast = \vu_1 - \vu_2$,
so $\nabla^2 \vu^\ast = 0$.  Also, let $\bs^\ast = \bs_1 - \bs_2$
and $\e^\ast = \e_1 - \e_2$.

Then \eqref{eq:stokes2} gives that $\pd{\sigma^\ast_{i j}}{x_j} = 0$
and $\pd{u^\ast_i}{x_i} = 0$.  Now consider
\begin{align*}
2 \mu \int_V e^\ast_{i j} e^\ast_{i j}\, \ud V &=
\int_V \sigma^\ast_{i j} \pd{u^\ast_i}{x_j}\, \ud V \\
&= \int_V \pd{}{x_j} \left( \sigma^\ast_{i j} u^\ast_i \right)\, \ud V \\
&= \int_S \sigma^\ast_{i j} u_i^\ast n_j\, \ud S = 0.
\end{align*}

Thus since $e^\ast_{i j} e^\ast_{i j} \ge 0$ we must have $e^\ast_{i
  j} = 0$.  Now the most general motion having no rate of strain is a
rigid body motion and so $\vu^\ast = \vu^\ast +
\bo^\ast \times \vx$ for constant $\vu^\ast$ and
$\bo^\ast$.  But since $\vu^\ast = 0$ on $S$ we have
$\vu^\ast = 0$ everywhere.
\end{proof}

A more sophisticated argument proves uniqueness if $\Rey <
\frac{\pi \sqrt{3}}{3}$.

\subsection{Minimum dissipation}

Suppose $\vu(\vx)$ is the unique Stokes flow in $V$ satisfying $\vu = \vu$
on $S$. Let $\Bar{\vu}(\vx)$ be another flow in $V$ such that $
\nabla \cdot \Bar{\vu} = 0$ and $\Bar{\vu} = \vu$ on $S$ (kinematically
admissible).  Then
\[
2 \mu \int_V \Bar{e}_{i j} \Bar{e}_{i j}\, \ud V
\ge 2 \mu \int_V e_{i j} e_{i j}\, \ud V,
\]

with equality only if $\vu = \Bar{\vu}$.

\begin{proof}
Let $\vu^\ast = \vu - \Bar{\vu}$ and $\e^\ast = \e - \Bar{\e}$.
Then
\begin{align*}
2 \mu \int_V e^\ast_{i j} e_{i j}\, \ud V &=
\int_V \sigma_{i j} e^\ast_{i j}\, \ud V \\
&= \int_V \pd{u_i^\ast}{x_j} \sigma_{i j}\, \ud V \\
&= \int_S u^\ast_i \sigma_{i j} n_j\, \ud S = 0.
\end{align*}

Now consider

\begin{align*}
\int_V \left( \Bar{e}_{i j} \Bar{e}_{i j} - e_{i j} e_{i j} \right)\, \ud V
&= - \int_V e^\ast_{ij} \left( \Bar{e}_{i j} + e_{i j} \right)\, \ud V \\
&= \int_V e^\ast_{i j} e^\ast_{i j}\, \ud V
- 2 \int_V e^\ast_{i j} e_{i j}\, \ud V \\
& \ge 0 \qquad \text{as required.}
\end{align*}
\end{proof}

As an example of this we consider the drag on a rigid particle in
unbounded fluid.

\vspace{1in}

\[
\vF = \int_S \bs \cdot \vn\, \ud S
\]

is the force exerted by the particle on the fluid (the drag is
$-\vF$).  We have to solve the Stokes equations in $V$ with the
boundary conditions that $\vu = \vu$ on $S$ and $\vu \to 0$ as
$\vx \to \infty$.

Now the rate of working by the particle on the fluid is
\[
\vu \cdot \vF = \int_S \vu \cdot \bs\cdot \vn\, \ud S
= 2 \mu \int_V e_{i j} e_{i j}\, \ud V \ge 0.
\]

For a ``bar'' problem we choose

\vspace{1in}

with $\Bar{\vu}$ satisfying the Stokes equations in $\Hat{V}$,
$\Bar{\vu} \to 0$ at infinity and $\Bar{\vu} = \vu$ on $S$.
Then $\Bar{\vu}$ is kinematically admissable and $\Bar{\e} = 0$ between
$S$ and $\Hat{S}$.  Therefore
\[
\vu \cdot \Hat{\vF} = 2 \mu \int_{\Hat{V}} \Bar{e}_{i j}
\Bar{e}_{i j}\, \ud V
\ge 2 \mu \int_V e_{i j} e_{i j}\, \ud V =\vu \cdot \vF,
\]

where $\Hat{F}$ is the force exerted by the sphere on the fluid.  The
magnitude of the drag in the direction of motion is therefore less than the
drag on the circumscribing sphere.

\subsection{Solving the Stokes equations}

Taking the divergence of the Stokes equations \eqref{eq:stokes}
(with $\vF = 0$) we see that $p$ is harmonic.  Taking the
curl we see similarly that vorticity is harmonic, and finally taking
the (vector) Laplacian we see that $\nabla^4 \vu = 0$ ---
$\vu$ is \emph{biharmonic}.

For a planar flow $\vu = \nabla \times (0,0,\psi)$ and 
$\om = (0,0,-\nabla^2 \psi)$.  Thus $\nabla^4 \psi = 0$.  We
can solve $\nabla^2 \left( \nabla^2 \psi \right) = 0$ for $\nabla^2 \psi
= f(\vx)$
(Laplace equation) and then solve $\nabla^2 \psi = f(\vx)$ (Poisson
equation).

Alternatively we could use the method of sheet 2, question 11 and write
down a solution.  This is OK in nice geometries.  The final method is
of course numerical solution.

\section{Stokes flow due to a translating sphere}

We consider the inertialess flow generated by a sphere of radius
$a$ and velocity $\vu$ immersed in unbounded fluid of
viscosity $\mu$ which is at rest at infinity.  In particular we want
to calculate the force $\vF$ exerted by the sphere on the fluid.

\subsection{Dimensional analysis}

The linearity of the Stokes equations requires that $F$ is proportional
to both $U$ and $\mu$.  Dimensional considerations therefore give
$F = \alpha \mu a U$, where $\alpha$ is a positive dimensionless constant.
The isotropy of the sphere shape then implies that
$\vF = \alpha \mu a \vu$.

\subsection{Brute force}\label{ref:sphereflow}

As you may have guessed from the title of this section this is an
algebraically unpleasant calculation.  It's probably good for your
soul though...

We take spherical polars $(r,\theta,\phi)$ with $\theta = 0$ parallel
to $\vu$.  The flow is then axisymmetric with no $\phi$ dependence
and so admits a streamfunction $\psi(r,\theta)$ such that the components of
$\vu$ are
\[
u_r = \frac{1}{r^2 \sin \theta} \pd{\psi}{\theta} \qquad
u_\theta = - \frac{1}{r \sin \theta} \pd{\psi}{r}.
\]

It follows from the Stokes equations that $D^2 \left(D^2 \psi \right) = 0$,
where
\[
D^2 \equiv \pd{^2}{r^2} + \frac{\sin \theta}{r^2} \pd{}{\theta}
\left( \frac{1}{\sin \theta} \pd{}{\theta} \right).
\]

The no-slip condition on the sphere surface gives:
\[
\psi = \frac{1}{2} U a^2 \sin^2 \theta \quad \text{and} \quad
\pd{\psi}{r} = U a \sin^2 \theta \quad \text{on } r = a.
\]

Finally, for $r \to \infty$, $\psi = o(r^2)$.  We look for a solution
$\psi = f(r) \sin^2 \theta$ where
\[
f(a) = \frac{1}{2} U a^2, \quad f'(a) = U a, \quad f = o(r^2) \text{ as }
r \to \infty.
\]

We then obtain $D^2 \psi = F(r) \sin^2 \theta$ where $F(r)
= f'' - \frac{2 f}{r^2}$ so that
\[
D^4 \psi = 0 \Leftrightarrow F'' - \frac{2 F}{r^2} = 0.
\]

Integrating these equations we have
$f = A r^4 + B r^2 + C r + \frac{D}{r}$, and the boundary conditions give
$A = B = 0$, $C = \frac{3}{4} U a$ and $D = - \frac{1}{4} U a^3$.

Substituting back we obtain
\[
u_r = 2 \left( \frac{C}{r} + \frac{D}{r^3} \right) \cos \theta
\quad \text{and} \quad u_\theta = \left( - \frac{C}{r} + \frac{D}{r^3}
\right) \sin \theta.
\]

The vorticity $\om$ is necessarily in the $\phi$ direction with magnitude
$\frac{2 C}{r^2} \sin \theta$.  We can also obtain the pressure
(to within an arbitrary constant) as
\[
p - p_\infty = \frac{2 C \mu \cos \theta}{r^2}.
\]

The stress can now be determined from
\[
\bs = - p \vect{I} + \mu \left( \nabla \vu + \left(\nabla \vu\right)^T \right),
\]

although care must be taken in evaluating $\nabla \vu$ in this curvilinear
co-ordinate system.  Recalling that $\vn$ is the normal out of the fluid
the force exerted by the sphere on the fluid is finally given as

\begin{equation}\label{eq:stokeslaw}
\vF = \int_{r = a} \bs \cdot \vn\, \ud S = 6 \pi \mu a \vu,
\end{equation}

a result known as Stokes' law.

\subsection{Comments}

Note the fore and aft symmetry in the streamline pattern, unlike higher
$\Rey$.
\vspace{1in}

$\vu \sim \frac{1}{r}$ as $r \to \infty$, so far field effects are
important and distant boundaries and other particles affect the flow.

We can calculate $\vF$ more easily by moving the integral to a
sphere at infinity using the divergence theorem;
\[
\vF = - \int_{S_\infty} \bs \cdot \vn\, \ud S.
\]
Since $\vF$ is parallel to $\vu$ we only need to calculate
\[
F = -\int_{r = \infty} \sigma_{rr} \cos \theta - \sigma_{r \theta}
\sin \theta\, \ud S.
\]

Only terms of order $\frac{1}{r^2}$ in $\bs$ ($r^{-1}$ in $\vu$ or
$r$ in $\psi$) matter here.  In the far field
$\psi \sim C r \sin^2 \theta$ and $p - p_\infty \sim
\frac{2 \mu C \cos \theta}{r^2}$.  Thus
\[
\sigma_{rr} \sim -p + 2 \mu \pd{u_r}{r} = -p_\infty - 6 C \mu
\frac{\cos \theta}{r^2}
\]
and
\[
\sigma_{r \theta} \sim \nu \left\{ r \pd{}{r} \left( \frac{u_\theta}{r}
\right) + \frac{1}{r} \pd{u_r}{\theta} \right\} = 0.
\]

This simplifies the calculation of $F$.

\subsubsection*{Corollary}

For general shapes of particle in unbounded fluid exerting a force
$\vF$ on the fluid, measuring $\theta =0$ from the direction of
$\vF$, then at large distances
\[
\psi \sim \frac{F r}{8 \pi \mu} \sin^2 \theta \quad
p - p_\infty \sim \frac{F}{4 \pi r^2} \cos \theta \quad
\vu \sim \frac{F}{8 \pi \mu r} \left( 2 \cos \theta\, \e_r -
\sin \theta\, \e_\theta \right),
\]

where $\e_r$ and $\e_\theta$ are the usual unit vectors.

This solution for $\vu$ and $p$ satisfies the Stokes equations everywhere
except at $r = 0$ and corresponds to a point force $\vF$ acting
at $r = 0$.  It is called a \emph{Stokeslet velocity field}.

\subsubsection*{Failure of neglect of inertia}

Near infinity,
\[
\abs{\rho \pd{\vu}{t}} \sim \abs{\rho \vu \cdot \nabla \vu}
\sim \frac{\rho U^2}{r^2} \quad \text{and} \quad
\abs{\mu \nabla^2 \vu} \sim \frac{\mu U}{r^3}.
\]

Thus the Reynolds number (near infinity) is $\frac{\rho U a}{\mu}
\frac{r}{a}$ and even if $\Rey = \frac{\rho U a}{\mu} \ll 1$ inertial
effects still matter at infinity.  This is the Stokes-Oseen paradox.
We need to use the technique of matched asymptotic expansions, but at
leading order the result is unaffected.

\subsubsection*{More elegant techniques}

There are more elegant (and algebraically less complicated) techniques
for solving Stokes' equations.  The Papkovich-Neuber method (as
covered in Part III Slow Viscous Flow) is probably the easiest.

\section{Reciprocal Theorem}

If $(\vu,\bs)$ and $(\vu',\bs')$ are two Stokes flows in $V$ (with different
boundary conditions) then
\[
\int_S u_i \sigma'_{i j} n_j\, \ud S = \int_S u'_i \sigma_{i j} n_j\, \ud S.
\]

\begin{proof}
\begin{align*}
\int_S \left( u_i \sigma'_{i j} - u_j \sigma'_{i j}\right) n_j\, \ud S
&= \int_V \pd{}{x_j} \left( u_i \sigma'_{i j} - u_j \sigma'_{i j}
\right)\, \ud V \\
&= \int_V \sigma'_{i j} \pd{u_i}{x_j} - \sigma_{i j} \pd{u'_i}{x_j}\,
\ud V \\
&= \int_V e_{i j} 2 \mu e'_{i j} - 2 \mu e'_{i j} e_{i j}\, \ud V = 0.
\end{align*}
\end{proof}

\section{Movement of rigid particles in viscous fluids}

Many useful fluids are suspensions (for instance paints, inks,
abrasive cleaners, settling tanks for removal of pollutants).  We want
to know how fast particles sediment under gravity.

Consider a rigid particle of arbitrary shape in an unbounded viscous
fluid.

\vspace{1in}

Let $\rho_p(\vx)$ be the particle density, so that the mass of the
particle is $M = \int_{V_p} \rho_p\, \ud V$.  Take the origin to
be at the centre of buoyancy, that is such that
\[
\int_{V_p} \vx\, \ud V = 0.
\]

The Archimidean upthrust $-\rho V_p \vect{g}$ acts at the origin
($\rho$ is the fluid density).  Let $\vx_g$ be the centre of mass:
\[
\vx_g = \frac{1}{M} \int_{V_p} \rho_p \vx\, \ud V.
\]

Then the total external force acting on the particle is
$\vF = (M - \rho V_p) \vect{g}$
and the total external couple is $\vect{G} = M \vx_g \times \vect{g}$.

What velocity $\vu$ and angular velocity $\bo$ are generated?  It
is easier in practice to solve the inverse problem.  First, though:

Suppose that $\frac{\rho U a}{\mu} \ll 1$ and $\frac{\rho \Omega a^2}{\mu}
\ll 1$ so that the fluid inertia is negligible.  Unless $\rho_p \gg \rho$
the particle inertia is also negligible and therefore
the external force/couple applied to the particle equals the
external force/couple applied to the fluid.

\vspace{1in}

We want to solve the problem $\vu = \vu + \bo \times \vx$ on $S$,
$\vu \to 0$ at infinity and $\vu$ satisfies the Stokes equations in $V$.
The force and couple exerted by the body on the fluid are

\[
\vF = \int_S \bs \cdot \vn\, \ud S \qquad \vect{G} = \int_S
\vx \times \bs \cdot \vn\, \ud S.
\]

This has a unique solution.  By linearity, $\vu$ is linear in $\vu$
and $\bo$, so $\bs$ is linear is $\vu$ and $\bo$, and also proportional
to $\mu$. Thus $\vF$ and $\vect{G}$ are linear in $\vu$ and $\bo$ and
proportional to $\mu$.  Therefore

\begin{align*}
F_i &= \mu \left( a A_{i j} U_j + a^2 B_{i j} \Omega_j \right) \\
G_i &= \mu \left( a^2 C_{i j} U_j + a^3 D_{i j} \Omega_j \right),
\end{align*}

where $\vect{A}$, $\vect{B}$, $\vect{C}$ and $\vect{D}$ are dimensionless
second rank ``resistance'' tensors that depend on the particle shape. (In fact,
$\vect{B}$ and $\vect{C}$ are pseudotensors.)

\vspace{1in}

Now
\[
\int_S \vu^{(1)} \cdot \bs^{(2)} \cdot \vn\, \ud S
= \int_S \vu^{(2)} \cdot \bs^{(1)} \cdot \vn\, \ud S
\]

and so $\vu^{(1)} \cdot \vF^{(2)} = \vu^{(2)} \cdot \vF^{(1)}$.  Thus
$U^{(1)}_i A_{i j} U^{(2)}_j = U^{(2)}_i A_{i j} U^{(1)}_j$ for
all $U^{(1)}$ and $U^{(2)}$, so $\vect{A}$ is symmetric.

The rate of working by external forces is $\vF \cdot \vu
= \mu \vu \cdot \vect{A} \cdot \vu > 0$ if $\vu \neq 0$.  Thus $\vect{A}$
has positive eigenvalues and so is invertible.  It can be proved
(see example sheet 2) that the matrix
\[
\begin{pmatrix}
a \vect{A} & a^2 \vect{B} \\
a^2 \vect{C} & a^3 \vect{D}
\end{pmatrix}
\]
is symmetric and positive definite.

\subsection{Special cases}

\subsubsection*{Sphere}

Take $a$ to be the radius of the sphere.  $\vect{A}$, $\vect{B}$, $\vect{C}$
and $\vect{D}$ are all isotropic.  There exists no force on a rotating
sphere (by reflection symmetry) and so $\vect{B} = \vect{C} = 0$.  Thus
$A_{i j} = \alpha \delta_{i j}$ and $D_{i j} = \beta \delta_{i j}$.

$\alpha$ and $\beta$ are $6 \pi$ and $8 \pi$ respectively.  Thus
\[
\vu = \frac{2}{9} \frac{a^2}{\mu} \left( \rho_p - \rho\right) \vect{g}.
\]
is the sedimentation rate under gravity.

\subsubsection*{Cube}

Take $a$ as half the side length.  The principal axes of $\vect{A}$,
$\vect{B}$, $\vect{C}$ and $\vect{D}$ must coincide with the axes
of the cube and the eigenvalues of each of these
matrices must be equal (by symmetry).  Thus $\vect{A}$,
$\vect{B}$, $\vect{C}$ and $\vect{D}$ are isotropic. $\vect{B} = \vect{C} = 0$
still, and $\alpha$ and $\beta$ are unknown.

Thus a falling cube does not rotate whatever its orientation and also falls
straight down.

\subsubsection*{Ellipsoid}

The best choice for $a$ is the semi-major axis.  The principal axes
of $\vect{A}$ and $\vect{D}$ must coincide with those of the ellipsoid.

$\vect{A}$ and $\vect{D}$ are known and are not isotropic.  $\vect{B}
= \vect{C} = 0$ still.

\subsubsection*{Helix}

$\vect{B}$ and $\vect{C}$ are nonzero.  Note that the helix is
asymmetric on reflection of axes.

\section{Stokes flow in a corner}

\vspace{1in}

We try a local solution $\psi(r,\theta) = r^\lambda f(\theta)$
where $\lambda$ and $f$ are to be determined.  $\lambda > 1$ for
$\vu \to 0$ as $r \to 0$.

We look for solutions even in $\theta$ of
$\nabla^4 \psi = 0$ in $0 < \theta <\alpha$, $f = f' = 0$ on
$\theta = \pm \alpha$ (no slip condition) and $f' = f''' = 0$ at $\theta =0$
(even function).

Now $\nabla^2 \psi = r^{\lambda - 2} F(\theta)$, where $F = f'' + \lambda^2 f$
and $\nabla^4 \psi = 0$ implies $F'' + \left( \lambda - 2\right)^2 F = 0$.

Thus $f = B \cos \lambda \theta + A \cos \left( \lambda - 2\right) \theta$
(restricting to the even solution and taking
$\lambda \neq 1$).  Applying the conditions at
$\theta = \alpha$ we get
$ \sin 2 \left(\lambda - 1\right) \alpha = \left( 1- \lambda \right)
\sin 2 \alpha$.

\vspace{1in}

If the angle of the wedge is too small there are no real solutions.  However,
there are \emph{complex} solutions.  We need $\Re \lambda > 1$ to get
$\vu \to 0$ as $r \to 0$.  If $\lambda - 1 = p + \imath q$ and
$2 \alpha = \frac{\pi}{6}$ then a numerical solution is
$2 \alpha p = 4.2$ and $2 \alpha q = 2.2$.

We can evaluate $\left.u_{\theta}\right|_{\theta = 0}
= - \Re \left( \lambda r^{\lambda - 1} f(\theta)\right) =
C r^p \cos\left( q \log r + \epsilon \right)$ with $C$ and $\epsilon$
real.  We see an infinite sequence of counter-rotating Moffatt eddies.

\vspace{1in}

The eddies are geometrically similar and decrease in intensity by a factor
$e^{\frac{\pi p}{q}}$ (here about $400$) as $r \to 0$. 

\chapter{Flow in a thin layer}

For a rectilinear flow $\vu \cdot \nabla \vu = 0$. What if the flow
is nearly rectilinear?

\section{Lubrication theory}

Suppose fluid is confined in a gap $0 \le y \le h(x)$.

\vspace{1in}

The gap is thin, so $\frac{H}{L} \ll 1$.  We know that
$\DDt{y-h} = 0$ as $y = h(x)$ is a material surface.  Thus we can
either specify $(u,v)$ or $\left(u,\pd{h}{t}\right)$ on $y=h$.

We put $u = \psi_y$ and $v = - \psi_x$.  Then $\frac{u}{v} \sim \frac{L}{H}
\gg 1$ and $v$ is negligible compared with $u$.

If we can neglect inertia then
\begin{gather*}
\abs{\pd{p}{x}} \sim \mu \abs{\nabla^2 u} \sim \mu \abs{\pd{^2 u}{y^2}} \\
\abs{\pd{p}{y}} \sim \mu \abs{\pd{^2 v}{y^2}}.
\end{gather*}

There are therefore large pressure gradients in the $x$ direction, so
$\abs{\pd{p}{x}} \gg \abs{\pd{p}{y}}$ and at leading order
$p$ is a function of $x$ only.  Put $\pd{p}{x} = G(x,t)$.  Then
\[
\pd{^2 u}{y^2} = -\frac{G}{\mu} \quad \Rightarrow \quad
u = - \frac{G}{2 \mu} y \left(y - h\right) + \frac{U y}{h}.
\]

Taking $\psi = 0$ at $y=0$ we obtain
\[
\psi = -\frac{G}{2 \mu} \left( \frac{y^3}{3} - \frac{h y^2}{2} \right)
+ \frac{U y^2}{2 h}.
\]

The total flux in the layer is 
\[
\left.\psi\right|_{y = h} = \frac{U h}{2} + \frac{G h^3}{12 \mu} = Q(x).
\]

Using mass conservation (directly), $\pd{Q}{x} = - \pd{h}{t}$.  This
gives the Reynolds lubrication equation:

\begin{equation}\label{eq:reynolds}
\pd{}{x} \left( h^3 \pd{G}{x}\right) = 6 \mu \left( h \pd{u}{x}
+ u \pd{h}{x} + 2 \pd{h}{t}\right).
\end{equation}

With two boundary conditions we can find the pressure. For flow in
a sheet $0 \le z \le h(x,y,t)$ we have $u,v \gg w$, $p = p(x,y,t)$
and $u$ and $v$ are parabolic in $z$.  The equation is

\begin{equation}\label{eq:reynolds2}
\nabla \cdot \left( h^3 \nabla p \right) = 6 \mu \left(
h \nabla \cdot \vu + \vu \cdot \nabla h + 2 \pd{h}{t}
\right),
\end{equation}

where $\vu = (u,v)$ and $\nabla = \left( \pd{}{x}, \pd{}{y}\right)$.

We must now see if we were justified in neglecting inertia:

\begin{align*}
\abs{ \rho \vu \cdot \nabla \vu } & \sim \frac{\rho U^2}{L} \\
\abs{\mu \nabla^2 \vu} & \sim \frac{\mu U}{H^2}.
\end{align*}

The ratio of inertial forces to viscous forces is thus
$\frac{U H}{\nu} \frac{H}{L}$ and we need this effective Reynolds number
to be small.

\section{Thrust bearing}

\vspace{1in}

Assume that the flow is axisymmetric.  We can immediately apply
\eqref{eq:reynolds2} to get
\[
\nabla^2 p = \frac{1}{r} \pd{}{r} r \pd{p}{r} = \frac{12 \mu}{h^3} \pd{h}{t}.
\]

Thus
\[
p = \frac{3 \mu r^2}{h^3} \pd{h}{t} + C \log r + D.
\]

We know that $C = 0$ (to avoid a singularity at the origin), and we put
$p = p_\infty$ at $r = a$.  We obtain
\[
p - p_\infty = \frac{3 \mu}{h^3} \pd{h}{t} \left( r^2 - a^2 \right).
\]

To obtain the force on the hammer we need $\sigma_{zz} = -p
+ 2 \mu \pd{w}{z}$.  The $\pd{w}{z}$ term is $\cO(\tfrac{H^2}{L^2})$ 
and is smaller than $p$ so we neglect this.  Thus, allowing for
$p_\infty$,

\begin{align*}
F &= \int_0^a \left( p - p_\infty \right) 2 \pi r \, \ud r \\
&=2 \pi \frac{3 \mu}{h^3} \pd{h}{t} \int_0^a r \left(r^2 - a^2 \right)\,
\ud r\\
&= - \frac{3}{2} \frac{\pi \mu}{h^3} \pd{h}{t} a^4.
\end{align*}

For fixed $V_0 = \pi a^2 h$, and if the force applied
to the hammer is fixed, then
\[
F = \frac{3 \mu V_0^2}{8 \pi} \pd{}{t} \frac{1}{h^4} \qquad \Rightarrow
\qquad
h(t) = \left( \frac{3 \mu V_0^2}{8 \pi F}
\right)^{\frac{1}{4}} \left( t + t_0 \right)^{-\frac{1}{4}}
\]

Thus $a(t) \propto \left( t+t_0 \right)^{\frac{1}{8}}$ and as
$t \to \infty$, $h \sim t^{-\frac{1}{4}}$ when $a$ is less than the
radius of the hammer.  $h'' \sim t^{-\frac{9}{4}}$ and so the inertia
of the hammer is negligible.

Lubrication forces are in general big --- hence Sellotape.

\section{Flow in a Hele-Shaw cell}

Suppose we have a flow confined between two accurately parallel
sheets at $z = 0$ and $z=h$.  Then the Reynolds equation \eqref{eq:reynolds2}
gives $\nabla^2 p = 0$ ($\nabla$ is two dimensional) and
\[
(u,v) = \frac{1}{2 \mu} z \left( z - h \right) \nabla p
\]

and the depth averaged velocity $\Bar{\vu}$ is
\[
\Bar{\vu} = \nabla \left( -\frac{h^2 p}{12 \mu} \right).
\]

Thus the velocity field is the gradient of a harmonic potential ---
$\Bar{\vu}$ is an inviscid, irrotational velocity field.

The no-slip condition is accomodated in a small region of size $h$ near
the cylinder surface.  We can't get a circulation as $p$ (and therefore
$\phi$) is single-valued.

\section[Saffman-Taylor instability]%
{Saffman-Taylor instability for a planar interface}

We attempt to analyse the fingering instability observed when a less viscous
fluid advances under a pressure gradient into a more viscous fluid.
For simplicity suppose that one of the fluids is inviscid and consider
a planar interface in a Hele-Shaw cell.  The velocity $\vu(x,y)$ is therefore

\[
\vu = \nabla \phi \quad \text{and} \quad \phi = - \frac{h^2 p}{12 \mu},
\]

where $p$ is the fluid pressure.

Suppose first that the interface at $x = V t$ is planar.  Then the basic
state is

\begin{center}
\begin{tabular}{r l}
Water & Air \\
$\phi_0 = V x + \text{const}$ &  \\
$p_0 = -\tfrac{12 \mu V (x - Vt)}{h^2} + p_\infty$ & $p = p_\infty$.
\end{tabular}
\end{center}

Now imagine that the interface suffers an infinitesimal perturbation
with wavenumber $k$ and amplitude $\epsilon$.  We anticipate that
this perturbation will grow or decay exponentially in time and check
this assumption later.  The position of the interface becomes
\[
x = V t + \epsilon e^{\imath k y + \sigma t},
\]
where $\epsilon$ is arbitrarily small.  At leading order in $\epsilon$
all the perturbation quantities inherit this dependence on $y$
and $t$, so that the velocity potential in the water becomes
\[
\phi = \phi_0 + \epsilon f(x) e^{\imath k y + \sigma t}.
\]

As $\phi$ is harmonic we can find $f(x)$ (picking the solution decaying
as $x \to - \infty$
\[
\phi = V x + \text{const} + \epsilon A e^{\imath k y + \sigma t
+ \abs{k} \left( x - V t \right)},
\]

for some constant $A$.  The corresponding pressure is
\[
p = p_0 + p_1 \quad \text{where} \quad p_1 = - \frac{12 \mu \epsilon}{h^2}
A e^{\imath k y + \sigma t
+ \abs{k} \left( x - V t \right)}.
\]

We therefore have

\begin{center}
\begin{tabular}{r l}
Water & Air \\
$\phi = \phi_0 + \phi_1$ &  \\
$p = p_0 + p_1$ & $p = p_\infty$.
\end{tabular}
\end{center}

Our aim is to find $\sigma$.  If a surface tension $\gamma$ acts between
the fluids then the interface curvature is just $\pd{^2 x}{y^2}$
at leading order, so that
\[
\left[ p \right] = \gamma \pd{^2 x}{y^2} = -k^2 \gamma \epsilon e^{\imath
k y + \sigma t},
\]

where the jump is across the position of the perturbed interface.  At
leading order in $\epsilon$ this gives
\[
p_\infty - \left[ - \frac{12 \mu V}{h^2} \epsilon e^{\imath k y + \sigma
t} + p_\infty - \frac{12 \mu \epsilon}{h^2} A e^{\imath k y + \sigma t} 
\right] = -k^2 \gamma \epsilon e^{\imath k y + \sigma t}.
\]

The first term is the pressure in the air and the second is the pressure
in the water.  Simplifying we find
\[
A = -V - \frac{\gamma h^2 k^2}{12 \mu}.
\]

We can now determine the velocity of the interface as $\pd{\phi}{x}$,
which may again be evaluated (at leading order) at $x = V t$.  This
must correspond to $\pd{x}{t} = V + \epsilon \sigma e^{\imath k y 
+ \sigma t}$. This gives
\[
\sigma = A \abs{k} = - V \abs{k} \left[ 1 + \frac{\gamma h^2 k^2}{12 \mu V}
\right].
\]

The time dependence cancels out, which justifies our initial assumption.
In the absence of surface tension we see that $\sigma > 0$ whenever $V < 0$,
thus the interface perturbation grows (according to linear theory) if
the air moves into the water.  On the other hand $\sigma < 0$ when
$V > 0$ and the water moves into the air.  In the unstable case, the
fastest growing modes are those short waves for which $k \to \infty$.

If $\gamma > 0$ surface tension is predicted to stabilise the shortest
waves.  The interface is still unstable, but the fastest growing mode
has a finite value for $k$ and $\sigma$.

\subsubsection*{Physical mechanism}

\vspace{1in}

Where the interface lags the pressure gradient is bigger, and the disturbance
reduces.  The other way is unstable.

\section[Gravitational spreading]%
{Gravitational spreading on a horizontal surface}

\vspace{1in}

Assume axisymmetry and $\pd{h}{r} \ll 1$.  We will neglect surface tension
(and the contact line).

As this is a free surface problem we can't use the Reynolds equation
directly.

In the layer there is a hydrostatic pressure
$p = p_\infty + \rho g \left( h - z\right)$.  The radial momentum equation
is
\[
\mu \pd{^2 u}{z^2} = \pd{p}{r} = \rho g \pd{h}{r}.
\]

As $u = 0$ at $z =0$ and $\pd{u}{z} = 0$ at $z = h$ we have
\[
u = \frac{\rho g}{2 \mu} \pd{h}{r} z \left( z - 2 h \right).
\]

The flux out of a cylinder of radius $r$ is
\[
Q(r) = 2 \pi r \int_0^h u\, \ud z = - \frac{2 \pi \rho g r h^3}{3 \mu}
\pd{h}{r}.
\]

Mass conservation gives

\begin{align*}
2 \pi r \pd{h}{t} &= - \pd{Q}{r} \qquad \text{and so}\\
\pd{h}{t} &= \frac{g}{3 \nu} \frac{1}{r} \pd{}{r} \left( r h^3 \pd{h}{r}
\right).
\end{align*}

The volume of the drop is fixed, so
\[
\int_0^{a(t)} 2 \pi r h\, \ud r = V.
\]

We now rescale the variables.  The natural lengthscale is $V^{\frac{1}{3}}$
so rescale $h$, $r$ and $a$ by $V^{\frac{1}{3}}$ and the natural timescale
is $\frac{\nu}{g V^{\frac{1}{3}}}$, so we rescale $t$ by this.  Letting
$r^\ast = r V^{-\frac{1}{3}}$ and so on, we get
\[
\pd{h^\ast}{t^\ast} = \frac{1}{3} \frac{1}{r^\ast}
\pd{}{r^\ast} \left( r^\ast h^\ast{}^3 \pd{h^\ast}{r^\ast}\right)
\]
and
\[
\int_0^{a^\ast(t^\ast)} 2 \pi h^\ast r^\ast \ud r^\ast = 1.
\]

We now drop the $\ast$'s.  We need $h(r,0)$ to get a unique solution.

We look for a similarity solution

\begin{enumerate}
\item To aid computer programs,
\item For physical insight,
\item Often the long-time solution, independent of initial conditions.
\end{enumerate}

Try $h = t^{-\alpha} H(\eta)$ where $\eta = \frac{r}{t^\beta}$
and so $a = A t^\beta$.  We get
\[
1 = \int_0^A 2 \pi \eta H(\eta) \, \ud \eta t^{2 \beta - \alpha},
\]

and so $\alpha = 2 \beta$.  Substituting in the differential equation
\[
\left( - \alpha H + \beta \eta H' \right) t^{-\alpha - 1}
= \frac{1}{3 \eta} t^{-4 \alpha - 2 \beta} \diff{}{\eta} \left(
\eta H^3 H'\right).
\]

So $- \alpha - 1 = - 4 \alpha - 2 \beta$. We can solve for $\alpha$
and $\beta$ to obtain $\alpha = \frac{1}{4}$, $\beta = \frac{1}{8}$.  The
ODE for $H$ is
\[
\diff{}{\eta} \left( \eta H^3 H' \right) + \frac{3}{8} \eta
\left( \eta H' + 2 H \right) = 0.
\]

This can be integrated to give
\[
\eta H^3 H' + \frac{3}{8} \eta^2 H = 0,
\]
We can integrate this equation to give
\[
H = \left( \frac{9}{16} \right)^{\frac{1}{3}} \left(A^2 - \eta^2 \right)^
{\frac{1}{3}}.
\]
The constant volume condition gives
\[
\frac{3 \pi}{4} \left( \frac{9}{16} \right)^{\frac{1}{3}} A^{\frac{8}{3}} = 1.
\]

Putting all the dimensions back in, we get
\[
a(t) = \left( \frac{2^{10}}{3^5 \pi^3} \right)^{\frac{1}{8}}
V^{\frac{1}{3}} \left( \frac{g V^{\frac{1}{3}} t}{\nu} \right)^{\frac{1}{8}}.
\]

If surface tension is included then we must consider the
curvature.  We get
\[
p = p_\infty + \rho g\left(h - z\right) - \gamma \pd{^2 h}{r^2}.
\]

The governing equation is
\[
\pd{h}{t} = \frac{g}{3 \nu} \frac{1}{r} \pd{}{r} \left\{
r h^3 \left( \pd{h}{r} - \frac{\gamma}{\rho g} \pd{^3 h}{r^3}\right)
\right\}.
\]

This has no similarity solutions.  We need extra boundary conditions on $h(r)$
as it is now a fourth order equation.  This is an open problem.

\chapter{Vorticity generation and confinement}

\section{Vorticity equation}

Our point of departure is the Navier-Stokes equations with no force term
\[
\rho \left( \pd{\vu}{t} + \nabla \left( \tfrac{1}{2} \abs{\vu}^2 \right)
- \vu \times \om \right) = \rho \DDt{\vu} = - \nabla p + \mu \nabla^2 \vu.
\]

We take the curl of this to get
\[
\rho \pd{\om}{t} = \rho \nabla \times \left( \vu \times \om \right)
+ \mu \nabla^2 \om.
\]

Re-arranged, this gives the vorticity equation

\begin{equation}\label{eq:vorticity}
\DDt{\om} = \om \cdot \nabla \vu + \nu \nabla^2 \om.
\end{equation}

Physical meaning: vorticity moves with fluid particles, is diffused by
viscosity and stretched by $\nabla \vu$.

There are no source terms, so $\om$ originates on boundaries.

For an inviscid fluid ($\nu = 0$) we have $\DDt{\om} = \om \cdot \nabla \vu$.
Recall that for a material line element $\DDt{\vect{dl}} = \vect{dl} \cdot
\nabla \vu$.  Therefore vortex lines stretch and rotate like material line
elements.

We can give an integral form for \eqref{eq:vorticity} in the case
$\nu = 0$.  Define the circulation around $C(t)$ as
\[
\kappa = \oint_{C(t)} \vu \cdot \vect{dl}.
\]

Then
\begin{align*}
\diff{\kappa}{t} &= \oint_{C(t)} \DDt{\vu} \cdot \vect{dl}
+ \vu \cdot \DDt{\vect{dl}} \\
&= \oint_{C(t)} \vect{dl} \cdot \nabla \left( - \frac{p}{\rho}
+ \tfrac{1}{2} \abs{\vu}^2 \right) = 0 \qquad \text{as $p$ and $u$ are single valued.}
\end{align*}

Thus (in the inviscid case), $\kappa$ is constant.  This is 
Kelvin's circulation theorem.

If $\kappa = 0$ at $t=0$ for all contours $C$ then $\kappa = 0$ for all
time. In this case $\vu = \nabla \phi$ and we have an irrotational inviscid
problem.

\subsection{Planar flows}

In the $\nu \neq 0$ case, if we have a planar flow we can write
$\vu = (\psi_y, - \psi_x, 0)$ and then $\om = (0,0,-\nabla^2 \psi)$.

Thus $\om \cdot \nabla \vu = 0$ and the vorticity equation becomes

\begin{equation}\label{eq:2dvort}
\pd{}{t} \nabla^2 \psi - \pd{(\psi,\nabla^2 \psi)}{(x,y)} = \nu \nabla^4 \psi.
\end{equation}

This is the two dimensional vorticity equation.

\section{Vorticity generation}

\vspace{1in}

Suppose that we have the flow $\vu = (u,v)$ with
\[
\left.u\right|_{y = 0} = \begin{cases} U & t > 0 \\ 0 & t < 0.\end{cases}
\]

What is $\vu(y,t)$ for $t > 0$?  We try a solution
$\vu = (u(y,t),0)$ and $p = p(y,t)$.

The $y$ momentum equation gives $\pd{p}{y} = 0$ and the $x$ momentum equation
gives
\[
\pd{u}{t} = \mu \pd{^2 u}{y^2}.
\]

This is a diffusion equation for $u$ (or $\omega$, since
$\omega = -\pd{u}{y}$).  We have the boundary conditions
$u \to 0$ as $y \to \infty$ and $u = 0$ for all $y$ at $t = 0$.  Now
$u \propto U$ by linearity, so $u = U f(y,t,\nu)$ for dimensionless $f$.

As $f$ is dimensionless it depends only on $\frac{y}{2 \sqrt{t \nu}}
= \eta$ --- we have a similarity solution.  The chain rule gives
\[
f'' + 2 \eta f' = 0,
\]
which has the solution
\[
f = A \int_0^\eta e^{-\xi^2}\, \ud \xi + B.
\]

Now $f(0) = 1$ and $f \to 0$ as $\eta \to \infty$, so $A =
\frac{2}{\sqrt{\pi}}$ and $B = 1$.  Thus
\[
u(y,t) = U \left( 1 - \erf \frac{y}{2 \sqrt{\nu t}} \right).
\]

The vorticity $\omega = \frac{U}{2 \sqrt{\pi \nu t}} e^{-\frac{y^2}{4 \nu t}}$.
As $t \to 0$, $\omega(y) \to U \delta(y)$, and as $t$ increases
$\omega$ spreads into a \emph{boundary layer} of thickness $\delta \propto
\left( \nu t \right)^{\frac{1}{2}}$.  This is characteristic of a diffusion
process.

\vspace{1in}

Note that $\int_0^\infty \omega\, \ud y = U\ \forall t$.

\section{Vorticity confinement on a flat plate}

Consider a steady flow past a flat plate with suction.

\vspace{1in}

$\vu = (0,-V)$ at $y=0$, $\vu \to (U,-V)$ as $y \to \infty$.  We try
a solution $\vu = (u(y),v(y))$.  Incompressibility gives $v = -V\ \forall y$.
The $x$ component of the momentum equation is
\[
-V \pd{u}{y} = \nu \pd{^2 u}{y^2}.
\]

Thus the general solution is $u = A + B e^{-\frac{V y}{\nu}}$.  If
$V > 0$ we can apply the boundary conditions to get
$u(y) = U\left( 1- e^{-\frac{V y}{\nu}}\right)$ and
$\omega = -\frac{U V}{\nu} e^{-\frac{V y}{\nu}}$.

The vorticity is confined in a boundary layer near the wall of
thickness $\frac{\nu}{V}$.  Outside the boundary layer the flow is
irrotational.

If $V < 0$ we cannot apply the boundary conditions consistently.

\section{Stagnation point flow}

\vspace{1in}

The irrotational flow $\vu = \alpha (x,-y)$ with streamfunction $\psi
= \alpha x y$ has a stagnation point at the origin.  What does the
flow become if a rigid wall is placed at $y = 0$?  (We look at the
$\alpha > 0$ case first.)

We look for a streamfunction $\psi$ such that $\psi \sim \alpha x y$
as $y \to \infty$, and propose the solution
$\psi = \alpha x f(y)$ with $f(y) \sim y$ as $y \to \infty$.

We nondimensionalise $x$ and $y$ by $\delta = \sqrt{\frac{\nu}{\alpha}}$,
and put
\[
\psi = \alpha \delta^2 \frac{x}{\delta} f(\tfrac{y}{\delta}).
\]

Letting $\eta = \frac{y}{\delta}$, for a steady flow we have
$\vu \cdot \nabla \om = \nu \nabla^2 \om$, $\om = - \nabla^2 \psi$
we get
\[
f' f'' - f f''' = f^{(iv)}
\]

with boundary conditions $f' \to 1$ as $\eta \to \infty$, $f = f' = 0$
at $\eta = 0$.

We can integrate this equation once to get
\[
f'{}^2 - f f'' = f''' + 1.
\]

This equation must be solved numerically; it appears there is a unique
solution.

\vspace{1in}

Far from the plate $\psi \sim \alpha x \left( y - 0.65 \right)$ --- so
the flow is irrotational, with a perturbation as if the stagnation streamline
was at $y = 0.65 \delta$ (this is the displacement thickness).

The vorticity is confined to a layer of thickness proportional to $\delta$.

If $\alpha < 0$ the above analysis carries through, provided we
let $\delta = \sqrt{\frac{\nu}{\abs{\alpha}}}$ and
$\psi = \abs{\alpha} \delta x f(\tfrac{y}{\delta})$.  The ODE
is unchanged but the boundary condition at infinity becomes
$f' \to -1$ as $\eta \to \infty$.  No solution to this equation exists.

\section{The bathtub vortex}

The axisymmetric flow $\vu = v(r,t) \e_\theta$ has vorticity
$\om = \frac{1}{r} \pd{}{r} \left(r v\right) \e_z$.  The vorticity
diffuses according to the equation
\[
\pd{\om}{t} = \nu \nabla^2 \om,
\]
and the local vorticity intensity falls.  To maintain the vorticity
distribution add in a flow $\vu = (-\alpha r,0,2 \alpha z)$.  This
will advect and stretch $\om$ (for $\alpha > 0$).  What is the
steady vorticity?

\begin{align*}
\vu \cdot \nabla \om - \om \cdot \nabla \vu &= \nu \nabla^2 \om \\
- \alpha r \pd{\omega}{r} - 2 \alpha \omega &= \frac{\nu}{r}
\pd{}{t} r \pd{\omega}{r} \\
\Rightarrow \quad
- \alpha r^2 \omega &= \nu r \pd{\omega}{r} + C.
\end{align*}

If $\omega \to 0$ as $r \to \infty$ then $C = 0$ and
\[
\omega = \omega_0 e^{-\frac{\alpha r^2}{2 \nu}}
\qquad \Rightarrow \qquad
v = \frac{\nu \omega_0}{\alpha r} \left( 1 - e^{-\frac{\alpha r^2}{2 \nu}}
\right).
\]

The vorticity is confined to $r \lesssim \sqrt{\frac{\nu}{\alpha}}$.

Now $\diff{p}{r} = \frac{\rho \nu^2}{r} - 2 \alpha^2 r$ and $p$
has a minimum near $r=0$ --- so if we have a free surface a dip
will appear.

\chapter[Boundary layers]{Boundary layer theory at high Reynolds number}

\section{Introduction}

Suppose we have a steady flow past a circular cylinder.

\vspace{1in}

Suppose also that the Reynolds number, $\Rey = \frac{U L}{\nu} \gg 1$.
Near the front stagnation point we expect $\psi \propto x y \frac{U}{L}$
which implies a boundary layer of thickness $\sqrt{\frac{\nu L}{U}}
= \frac{L}{\sqrt{\Rey}} \ll 1$ where the viscosity is important.  Outside
the layer the flow remains irrotational, although it is slightly
modified by a displacement effect of size $\Rey^{-\frac{1}{2}}$, which
we ignore.  Note that $x = L \theta$ and $y = (r - L)$ are co-ordinates 
parallel and perpendicular to the surface.

We expect the boundary layer to continue around the cylinder, and
would like to know what happens in this layer and at the \emph{rear}
stagnation point.

\section[Steady boundary layers]{Steady boundary layer theory at rigid surface}

Recall the steady planar vorticity equation \eqref{eq:2dvort},
\[
- \pd{(\psi,\nabla^2 \psi)}{(x,y)} = \nu \nabla^4 \psi.
\]

In the Euler limit we suppose that $x$ and $y$ have the same scale $L$
and let $\Rey \to \infty$.

Let $\psi = U L \Tilde{\psi}$, $x = L \Tilde{x}$ and $y = L \Tilde{y}$.
Then the steady vorticity equation becomes
\[
- \pd{(\Tilde{\psi},\Tilde{\nabla}^2 \Tilde \psi)}{(\Tilde{x},\Tilde{y})}
= \frac{1}{\Rey} \Tilde{\nabla}^4 \Tilde{\psi}
\]

and as $\Rey \to \infty$ viscosity disappears and we recover the Euler
equation $\rho \vu \cdot \nabla \vu = - \nabla p$.

In the Euler limit we lose the no-slip boundary condition; setting
$\mu = 0$ in the Navier-Stokes equations reduces the order and in general
fewer boundary conditions can be satisfied.

We expect that in the boundary layer viscosity is always important.
In the Prandtl limit we suppose $x = L \Bar{x}$ but $y = \delta \Bar{y}$
where $\delta = \frac{L}{\sqrt{\Rey}}$ and $\psi = U \delta \Bar{\psi}$.
Thus $u$ is scaled by $U$ but $v$ is scaled by $\frac{U}{\sqrt{\Rey}}$.

For any fixed $\vx$, $\lim_{\Rey \to \infty} \vu(\vx) = \left.\vu
\right|_{\text{inviscid}}(\vx)$.  This convergence is \emph{not} uniform;
for any fixed $\Rey$, however big, we can always find $\vx$
at which $\vu(\vx) \neq \left.\vu \right|_{\text{inviscid}}(\vx)$.

Now $\Bar{\omega} = - \Bar{\nabla}^2 \Bar{\psi} = - \pd{^2 \Bar{\psi}}{
\Bar{y}^2}$ which implies that
\[
- \pd{(\Bar{\psi}, \pd{^2 \Bar{\psi}}{
\Bar{y}^2})}{(\Bar{x},\Bar{y})} = - \pd{^4 \Bar{\psi}}{\Bar{y}^4}
\]

to get a balance between inertia and viscosity in the boundary layer.
We can integrate this once with respect to $\Bar{y}$ to get
\[
\Bar{\psi}_{\Bar{y}} \Bar{\psi}_{\Bar{x}\Bar{y}}
- \Bar{\psi}_{\Bar{x}} \Bar{\psi}_{\Bar{y}\Bar{y}}
= \Bar{\psi}_{\Bar{y} \Bar{y} \Bar{y}} + G(\Bar{x})
\]

which may be alternatively written
\[
\Bar{u} \Bar{u}_{\Bar{x}} + \Bar{v} \Bar{u}_{\Bar{y}}
= \Bar{u}_{\Bar{y} \Bar{y}} + G(\Bar{x}).
\]

$G$ is a pressure gradient which does not depend on $\Bar{y}$ in the
layer.

On the wall at $\Bar{y} = 0$ we have $\Bar{\psi} = \Bar{\psi}_{\Bar{y}} = 0$
which gives the no-slip condition.

\subsubsection*{Matching}

As $\Bar{y} \to \infty$ (many distances $\delta$, but for large $\Rey$
still near the boundary) the inner solution must match the outer solution.
Now $\nu = 0$ already matches.  We also need
\[
\lim_{\Bar{y} \to \infty} \Bar{\psi}_{\Bar{y}}
= \lim_{\Tilde{y} \to 0} \Tilde{\psi}_{\Tilde{y}}.
\]

The pressure must also match.  If $\lim_{\Bar{y} \to \infty}
\Bar{\psi}_{\Bar{y}} = U(\Bar{x})$ then by Bernoulli's equation,
$p + \tfrac{1}{2} \rho \abs{\vu}^2 = \text{const}$ and so $G(\Bar{x}) =
U \pd{U}{\Bar{x}}$.

Putting this back into dimensional form we obtain the boundary layer
equation:

\begin{equation}\label{eq:bdylayer}
u u_x + v u_y = U U' + \nu u_{y y}
\end{equation}

with boundary conditions $u = v = 0$ at $y = 0$ and
$u \to U$ as $y \to \infty$.

\subsubsection*{Notes}

\begin{enumerate}
\item This is a parabolic equation.  We need upstream conditions on $\vu$, for
instance the value at $x = 0$.
\item The boundary layer equation is the $x$ momentum equation with
$\nabla^2 = \pd{^2}{y^2}$ and $\pd{p}{x}$ a function of $x$ only.
\item It is a nonlinear equation.  Very few analytic solutions are known,
and those that are known are similarity solutions.
\end{enumerate}

\section[Blasius boundary layer]{Flow past a flat plate: Blasius boundary
layer}

\vspace{1in}

The Euler problem is $\vu = (U,0)$, and the boundary layer equation becomes
\[
\psi_y \psi_{x y} - \psi_x \psi_{y y} = \nu \psi_{y y y}
\]

with $\psi = \psi_y = 0$ at $y = 0$ and $\psi_y \to U$ as $y \to \infty$.

This is like a spatial version of vorticity diffusion.  Vorticity spreads
a distance $\sqrt{\nu t}$ where $t = \frac{x}{U}$ is the time taken for fluid
to reach $x$ starting from $x = 0$.

We try $\delta(x) = \left( \frac{\nu x}{U} \right)^{\frac{1}{2}}$ and
a similarity solution $\eta = \frac{y}{\delta(x)}$,
$\psi = U \delta(x) f(\eta)$.  Substituting into the boundary
layer equation we get
\[
f''' + \tfrac{1}{2} f f'' = 0,
\]
with $f = f' = 0$ at $\eta =0$ and $f' \to 1$ as $\eta \to \infty$. This
must be solved numerically, to give a flow profile:

\vspace{1.5in}

The traction on the plate $y = 0_+$ is $\mu \left. \pd{u}{y} \right|_{y = 0}
= \frac{\mu U}{\delta} f''(0)$, numerical solution gives
$0.3 \rho U^2 \sqrt{\frac{\nu}{U x}}$.

\subsubsection*{Notes}

\begin{enumerate}
\item The displacement thickness is the lateral displacement of streamlines
outside the boundary layer:
\[
\delta_1 = \int_0^\infty \left( 1 - \frac{u}{U}\right)\, \ud y
= 1.7 \sqrt{\frac{\nu x}{U}}.
\]
\item We could get an improved result by modifying the outer Euler flow
to account for the displacement effect (flow past a parabola).  This
is an $\cO(\Rey^{-\frac{1}{2}})$ correction.

\item What is the Reynolds number?  The only available lengthscale is $x$,
so the effective Reynolds number is $\frac{U \delta_1}{\nu}
\propto \sqrt{\frac{U x}{\nu}}$.  There is therefore a small nose region
near $x=0$ of size $\frac{\nu}{U}$ where the theory breaks down.

\item It is found experimentally that if the Reynolds number is big enough
(far enough downstream) the flow becomes unstable.  (At $\Rey \approx 1000$.)
Disturbance to the boundary layer grow, flow becomes unsteady and
ultimately turbulent.  The drag on the plate increases.  In practice,
this Blasius boundary layer agrees with experiment for
$1 \ll \Rey \lesssim 1000$.
\end{enumerate}

\section[Similarity solutions]%
{Similarity solutions of the boundary layer equation}

For geometries with no intrinsic lengthscale it is sensible to try a solution
$\psi(x,y) = U(x) \delta(x) f(\eta)$, where $\eta = \frac{y}{\delta(x)}$,
$U \propto x^p$ and $\delta \propto x^q$.

Then $u = \psi_y = U f'(\eta)$ and so $f' \to 1$ as $\eta \to \infty$.
To get a balance between inertia and viscosity in the boundary layer
we must have $\abs{u u_x} \sim \nu \abs{u_{yy}}$ (in practice
$\abs{u u_x} \sim \abs{v u_y}$).  Thus
\[
U U_x \sim \frac{\nu U}{\delta^2}
\]

and $p+2 q = 1$.  To fix $p$ and $q$ we need extra information.

In the Blasius layer we had $u \to U$ independently of $x$ and so
$p = 0$.  In the stagnation point flow we had $u \to \alpha x$
and so $p = 1$.

Dimensional arguments give the rest.

\section{High Reynolds number flow past a wedge}

\subsection{Outer problem}

\subsubsection*{Symmetric flow}

\vspace{1.5in}

In the outer inviscid, irrotational region we have $\vu = \nabla \phi$
and $\nabla^2 \phi = 0$ with boundary conditions
$\pd{\phi}{\theta} = 0$ on $\theta = 0$ and $\theta = 2 \pi -
\frac{\pi \beta}{2}$.  The last condition is a symmetry requirement.

The trial solutions $\phi = C r^\lambda \cos \lambda \theta$ will work
if $\lambda = \frac{2}{2 - \beta}$.  The outer flow velocity for the
anticipated boundary layer on $\theta = 0$ then has magnitude
$U(x) = A x^m$, where $m = \lambda - 1 = \frac{\beta}{2 - \beta} \ge 0$.

The case $\beta =0$ ($m = 0$) gives the Blasius boundary layer and
the case $\beta = 1$ ($m = 1$) gives stagnation point flow.

\subsubsection*{Antisymmetric flow}

\vspace{1.5in}

This is the same problem as above, but with the symmetry condition replaced
by $\pd{\phi}{r} = 0$ on $\theta = \frac{\pi}{2} + \frac{\pi \beta}{2}$.

The same trial solution works, but now with $\lambda = \frac{1}{1 + \beta}$
and so $U(x) = A x^m$, with $m = - \frac{\beta}{1+\beta} \le 0$.

\subsection{The boundary layer}

The boundary layer equation along $x > 0$ (leaving the apex of the wedge)
becomes
\[
\psi_y \psi_{x y} - \psi_x \psi_{y y} = m A^2 x^{2 m - 1} + \nu \psi_{y y y}.
\]

The only lengthscale for growth of the boundary layer thickness is then
provided by $x$, and since $U(x) = A x^m$ we try a similarity solution
with $p = m$ and $q = \frac{1 - m}{2}$.  Dimensional considerations
dictate a structure of the form
\[
\psi = \sqrt{\nu A x^{m + 1}} f(\eta) \qquad \eta = \frac{y}{\delta(x)}
\quad \text{with} \quad \delta(x) = \sqrt{\frac{\nu x^{1-m}}{A}}.
\]

If $m < 0$ the boundary layer thickness increases faster than the
$\sqrt{x}$ behaviour that would arise from diffusion alone.

Substituting in the boundary layer equation we obtain the Falkner-Skan
equation

\begin{equation}\label{eq:FalSkan}
f''' + \frac{m + 1}{2} f f'' + m \left(1 - f'{}^2\right) = 0,
\end{equation}

with boundary conditions $f = f' = 0$ on $\eta = 0$ (no slip) and
$f' \to 1$ as $\eta \to \infty$ (to match the outer Euler flow).

\subsection{Numerical solution}

The ordinary differential equation may be solved numerically using a
shooting technique.  We find that for $m > 0$ there is a unique for
$f$ qualitatively similar to the Blasius profile having $f' > 0$ for
all $\eta$.  Thus symmetric flows that accelerate away from the apex of
the wedge pose no difficulties for the boundary layer equation.

For antisymmetric flows with $m < 0$ the position is more complicated.

\vspace{1in}

For $m < -0.904$ there is again a unique solution for $f$, but
now $f''(0) < 0$.  There is flow reversal near the wall and the flow
must separate at the apex of the wedge on the downstream side.  This
is unacceptable --- ``upstream infinity'' in the parabolic boundary
layer equation for $u$ is now at $x = \infty$.  Furthermore, for large
$\eta$, $f'$ approaches its asymptotic value of unity from above, so the
presence of the boundary layer apparently speeds up the outer flow
over its inviscid value --- this is unphysical.

For $-0.904 < m < 0$ there are two solutions for $f$, one having
reversed flow and the other not.  The solution without reversed flow
is acceptable.

\subsection{Separation of the boundary layer}

High Reynolds number steady boundary layers on rigid surfaces are commonly
found in experiments, but are not normally observed to contain regions
of reversed flow.%
\footnote{For an exception, see Van Dyke page 26.}
The wedge example above suggests that non-reversed boundary layers
will arise on the rigid boundary $x > 0$ provided the ``imposed'' pressure
gradient $U U'$ is positive --- that is if the external stream accelerates.

If $U' > 0$ then $\pd{V}{y} < 0$ by mass conservation and since $V = 0$
at the boundary, $V < 0$ in the interior of the fluid.  In this case
convection tends to confine the vorticity near the boundary.  If $U' < 0$
vorticity confinement to a thin boundary may be impossible.

If $U U'$ is sufficiently (in fact only slightly) negative, called
an adverse pressure gradient, then the boundary layer thickness grows
more rapidly and flow reversal occurs.  This phenomenon is called
boundary layer separation.  Separation brings into question our entire
method of solution, in particular the imposition of upstream boundary data
on $u$ at or near $x=0$.

Sometimes, worse still, it implies that the outer irrotational Euler
solution is itself incorrect because \emph{gross separation} occurs.  The
classic example here is flow past a circular cylinder, for which we
noted that the outse inviscid irrotational flow has $U U' = 2 \sin 2 x$,
where $x$ measures distance from the front stagnation point.  This
pressure gradient becomes adverse at $x = 90^\circ$, and the boundary
layer equation shows a singularity at $x= 104.5^\circ$.  Experimentally,
the boundary layer is observed to separate and to introduce vorticity into
the wake of the cylinder, changing the leading order outer flow as
sketched below (this happens at $\Rey \approx 20$).  At higher flow rates
still the flow becomes unsteady.\footnote{See Van Dyke pages 28 -- 31.}

\vspace{1in}

This gross separation is characteristic of high Reynolds number flow past
any bluff body and the only way to prevent the separation is to reduce
the adverse pressure gradient by streamlining the body into an aerofoil
shape.

Without separation the magnitude of the ``skin friction'' boundary layer
drag on the body scales as $L \mu \pd{u}{y} = \mu U \sqrt{\Rey}$ and
the contribution from the pressure $\rho U^2$ in the outer inviscid flow
is zero.  With separation the modified external pressure gives a
``form drag'' of magnitude $\rho U^2 L$, which is a (large) factor
of $\sqrt{\Rey}$ bigger than the skin friction.

From a mathematical perspective, note that the limit $\Rey \to \infty$
is in general singular.  The steady flow field for $\Rey \to \infty$, if
it exists, may be completely different from that for an inviscid fluid
with $\Rey = \infty$.

\section{Converging and diverging flow in a wedge}

\vspace{1in}

We will do the source problem first; consider a source with strength
$Q$.  The outer problem has a solution $u_r = \frac{Q}{\pi \beta r}$
and so $U(x) = \frac{A}{x}$, $A = \frac{Q}{\pi \beta} > 0$.

We seek a similarity solution with $p = -1$ and $q = 1$.  Thus
\[
\psi = \sqrt{\nu A} f(\eta) \qquad \eta = \frac{y}{\delta} \quad \text{with}
\quad \delta = \sqrt{\frac{\nu}{A}}.
\]

Substituting into the boundary layer equations we get
\[
f''' + f'{}^2 - 1 = 0, \qquad \text{Falkner-Skan with $m = -1$.}
\]

The boundary conditions are $f = f' = 0$ at $\eta = 0$ and $f' \to 1$
as $\eta \to \infty$.  We can integrate this once to get
\[
\tfrac{1}{2} f''{}^2 + \tfrac{1}{3} f'{}^3 - f' = \text{const}
= -\frac{2}{3} \quad \text{using $\infty$.}
\]

At $\eta = 0$, $f' = 0$ and so $f''{}^2 = -\frac{4}{3}$ --- giving a
contradiction.  There is no steady boundary layer.  Thus the pressure
gradient $U U' = -\frac{A^2}{x^3}$ is too adverse and vorticity must
diffuse into the interior.

In fact there is an exact solution of the full Navier-Stokes equations
(Jeffrey-Hamel flow).  We get rapid oscillations and so viscosity matters
everywhere.  In practice this is very unstable.

\vspace{1in}

We can do the sink problem by sending $Q \mapsto -Q$ in the above.
We obtain the same differential equation, but the boundary condition at
infinity is $f' \to -1$ as $\eta \to \infty$.  We integrate the
differential equation once to get
\[
\tfrac{1}{2} f''{}^2 + \tfrac{1}{3} f'{}^3 - f' = \frac{2}{3},
\]
which implies $f' = 2 - 3 \tanh^2 \left( \frac{\eta}{\sqrt{2}} + C \right)$,
where $\tanh C = \pm \sqrt{\frac{2}{3}}$, one of which has reversed flow
and is no good.  The other is OK.

\chapter{Aerodynamics}

We are interested in flows that do not separate.

\vspace{1in}

We hope (require!) that boundary layers do not separate, are passive and
are dictated by the external inviscid flow, $\vu = (\psi_y,-\psi_x,0)
= \nabla \phi$.

\section{Complex potential}

For an inviscid irrotational flow we have both a streamfunction
$\psi$ and a velocity potential $\phi$, such that
\[
(\psi_y, -\psi_x,0) = \vu = (\phi_x,\phi_y,0).
\]

Thus $\psi_y = \phi_x$ and $-\psi_x = \phi_y$ and if we set the
complex variable $z = x + \imath y$, the function $w(z) = \phi + \imath \psi$
is analytic except at singularities.%
\footnote{You know what I mean --- meromorphic, or something like that...}

We can find the velocity from the complex potential $w$ as
$\diff{w}{z} = \phi_x + \imath \psi_x = u - \imath v$.

\subsubsection*{Examples}

\begin{enumerate}
\item $w = U z$ --- uniform stream.
\item $w = \tfrac{1}{2} \alpha z^2$, giving $\psi = \alpha x y$ ---
stagnation point flow.
\item $w = A z^\lambda$, giving $\phi = A r^\lambda \cos \lambda \theta$,
--- flow past a wedge.
\item $w = - \frac{\imath \kappa}{2 \pi} \log z$ for $\kappa \in \R$ ---
line vortex.
\item $w = U \left(z^2 + a^2\right)^{\frac{1}{2}}$.  This is multivalued
and has branch points at $\pm \imath a$. Put the branch cut along
$[-\imath a, \imath a]$ and choose the square root such that if
$x > 0$ then $\sqrt{x^2 + a^2} > 0$.  As $\abs{z} \to \infty$,
$w(z) \sim U z$.  If $z = \imath y$ and $\abs{y} < a$ then
$w = U \sqrt{a^2 - y^2} \in \R$ and $\psi = 0$.  Near $z = \imath a$,
$w \sim \left(2 \imath a \right)^{\frac{1}{2}} \left( z - \imath a
\right)^{\frac{1}{2}}$, like flow past a wedge.  This is flow past a plate.
We find $u - \imath v = \frac{U z}{\left( z^2 + a^2 \right)} \to \infty$
as $z \to \pm \imath a$ and infinite velocities are predicted at the tip
of the plate.  This flow could be impulsively generated, but viscosity will
act to generate vorticity on the plate and cause the flow to separate.
\end{enumerate}

\begin{theorem}[Milne-Thomson circle theorem]
If $f(z)$ is a complex potential with no singularities in
$\abs{z} < a$ and a cylinder $\abs{z} = a$ is introduced into the
flow then the new potential is
\[
w(z) = f(z) + \Bar{f}(\tfrac{a^2}{\Bar{z}}).
\]
\end{theorem}

\begin{proof}
If $f$ is analytic then so is $\Bar{f}(\tfrac{a^2}{\Bar{z}})$ (use
the Cauchy-Riemann equations).  On the cylinder $z = a e^{\imath \theta}$
and
\[
w = f(a e^{\imath \theta}) + \Bar{f}(a e^{\imath \theta}),
\]
which is real.  Thus $\psi = 0$ on the surface of the cylinder and so
the surface of the cylinder is a streamline.

Outside the cylinder $\Bar{f}(\tfrac{a^2}{\Bar{z}})$ introduces no
new singularities.
\end{proof}

We can use this result to get flow past a cylinder without circulation.
If $\vu = (-U,V)$, then the complex potential for a uniform stream
is $f(z) = - (U + \imath V) z$.  Thus on inserting a cylinder, we get

\begin{equation}\label{eq:cylnocirc}
w(z) = -\left(U + \imath V\right) z - \frac{a^2}{z} \left(U - \imath V\right).
\end{equation}

It is easy to bolt a circulation on to this to get
\begin{equation}\label{eq:cylcirc}
w(z) = -\left(U + \imath V\right) z - \frac{a^2}{z} \left(U - \imath V\right)
- \frac{\imath \kappa}{2 \pi} \log z.
\end{equation}

\section{Conformal mappings}

If $w(\zeta)$ is analytic in $\zeta$ and $\zeta = f(z)$ with $f$ analytic
then $W(z) = w(f(z))$ is an analytic function of $z$.

By judicious choice of $w$, $f$ can generate lots of flows.  At points
$z_0$ where $f$ is analytic and $f'(z_0) \neq 0$, $f$ is a conformal
mapping and a closed curve $C$ in the $z$ plane that doesn't pass through
a singular point of $f$ will become a closed curve $C'$ in the
$\zeta$ plane.

For flow past an aerofoil $C$ in the $z$ plane we choose $f$ to
make $C'$ a circle.  If in addition $f(z) \sim z$ as $\abs{z} \to \infty$
then the flow at $\infty$ is the same in both planes.

Note that if we have $w(\zeta) \sim \frac{m - \imath \kappa}{2 \pi}
\log \left(\zeta - \zeta_0\right)$ as $\zeta \to \zeta_0$ then since
$\zeta - \zeta_0 = f(z) - f(z_0) \sim (z-z_0) f'$ if $f' \neq 0, \infty$
we have $W(z) \sim \frac{m - \imath \kappa}{2 \pi} \log \left(z - z_0\right) +
\text{const}$ --- sources and line vortices are the same in both planes.

\subsection{Flow past an ellipse with circulation}

\vspace{1in}

Consider the inverse map $z = \zeta + \frac{\lambda^2}{\zeta}$ (a
Joukowski map).  A point on $\zeta = c$ is $c e^{\imath \phi}$, which
is mapped to
\[
x = \left( c + \frac{\lambda^2}{c} \right) \cos \phi
\qquad y = \left( c - \frac{\lambda^2}{c} \right) \sin \phi,
\]

which will be the ellipse provided $c = \frac{a + b}{2}$ and
$\lambda^2 = \frac{a^2 - b^2}{4}$.

Solving for $\zeta$ we find $\zeta = \frac{z \pm \sqrt{z^2  - 4
\lambda^2}}{2}$.  We want $\zeta \sim z$ as $z \to \infty$, and so we
choose the $+$ sign.  Thus the complex potential for flow past an ellipse
with circulation is (using \eqref{eq:cylcirc})

\begin{equation}\label{eq:ellcirc}
\begin{split}
w(z) &= - \left( U + \imath V \right) \zeta - \left( U - \imath V \right)
\frac{c^2}{\zeta} - \frac{\imath \kappa}{2 \pi} \log \zeta \\
\text{where} \quad \zeta &= \frac{z + \sqrt{z^2 - 4 \lambda^2}}{2}.
\end{split}
\end{equation}

\subsubsection*{Flow past a flat plate}

The special case of a flat plate has $b = 0$, so $\lambda = c = \frac{a}{2}$.

\vspace{1in}

$\kappa = 0$, $A$ and $B$ are stagnation points.  $L$ and $T$ are the
leading and trailing edges respectively.

As $\kappa$ increases, $A$ and $B$ move to the left until, at
$\kappa = \kappa_c$, $A$ coincides with $T$.

\vspace{1in}

\section{Forces, drag and lift}

To avoid a crisis of notation, we let $q = \abs{\vu}$ (vector norm).

Starting from the Euler equation
\[
\rho \left( \pd{\vu}{t} + \tfrac{1}{2} \nabla q^2 - \vu \times \omega \right)
= - \nabla p
\]
we can, if the flow is steady and irrotational, derive (a form of)
Bernoulli's equation,

\begin{equation}\label{eq:stbern}
p = p_\infty - \tfrac{1}{2} \rho q^2.
\end{equation}

Consider a body in the fluid.

\vspace{1in}

Now $\vect{dl} = (\ud x, \ud y)$ and $\vn \ud l = (-\ud y, \ud x)$.  The
force exerted by the body on the fluid is

\[
\oint_C -p \vn\, \ud l = \tfrac{1}{2} \rho \oint_c \abs{\vu}^2 \vn\, \ud l
- \oint_C p_\infty \vn\, \ud l.
\]

The last term vanishes by the divergence theorem and we see that
\[
F_x - \imath F_y = -\tfrac{1}{2} \rho \int_C \abs{\vu}^2 \left( \ud y
+ \imath \ud x\right).
\]

On $C$, the flow is tangential and so $\ud z = \ud l
\frac{\left(u + \imath v \right)}{q}$ and
$\ud y + \imath \ud x = \imath \ud l \frac{\left(u - \imath v\right)}{q}$.

Thus

\begin{align*}
F_x - \imath F_y &= - \tfrac{1}{2} \imath \rho
\oint_C q^2 \frac{u - \imath v}{u + \imath v}\, \ud z \\
&= -\tfrac{1}{2} \imath \rho \oint_C \left( v - \imath v \right)^2\, \ud z \\
&= - \tfrac{1}{2} \imath \rho \oint_C \left( \diff{w}{z} \right)^2\, \ud z,
\end{align*}

and we have derived Blasius' formula:

\begin{equation}\label{eq:blaform}
F_x - \imath F_y = - \tfrac{1}{2} \imath \rho \oint_C \left( \diff{w}{z}
\right)^2\, \ud z.
\end{equation}

Note that by Cauchy's theorem, if $\diff{w}{z}$ is analytic between
$C$ and $C'$ we may deform the contour $C$ onto $C'$.  In particular,
if there are no singularities in the fluid, we can deform $C$ to $C_\infty$
and then use the calculus of residues to evaluate the integral.

\subsubsection*{Example}

If $w(z) \sim - U z - \frac{\imath \kappa}{2 \pi} \log z$ as $z \to \infty$
then

\[
F_x - \imath F_y = - \tfrac{1}{2} \imath \rho \oint_C  \left(U
+ \frac{\imath \kappa}{2 \pi z}\right)^2\, \ud z.
\]

The residue is $\frac{\imath \kappa U}{\pi}$ and so
$F_x - \imath F_y = \imath \rho U \kappa$.  If $U$ is real (WLOG)
we see that the drag $F_x = 0$ and the lift on the body,
$-F_y = \rho U \kappa$.

\vspace{1in}

\section{The Kutta-Joukowski condition}

\begin{center}
\parbox{4in}{
\itshape
For an aerofoil with a sharp trailing edge, viscosity will cause the
decelerating boundary layer to separate at the edge, modifying the external
Euler flow.  The outer flow will adjust its circulation so as to streamline
the flow at the edge.}
\end{center}

This was proved in 1970 for steady flows.

We wish to find this critical value of the circulation ($\kappa_c$)
for a flat plate (of length $2 a$).

Recall that
\[
\diff{W}{z} = \diff{W}{\zeta} \diff{\zeta}{z}
= \left\{ - \left(U + \imath V \right) + \left(U - \imath V\right) \frac{c^2}
{\zeta^2} - \frac{\imath \kappa}{2 \pi \zeta} \right\}
\left\{ \frac{1}{2} + \frac{z}{2 \sqrt{z^2 - a^2}}\right\}.
\]

There is a singularity at $z = - a$ and so the first bracket must
vanish at $z = - a$ to make the velocity finite.  We can solve
the resulting equation to get $\kappa_c = 2 \pi a V$ and
so the lift is $\rho \sqrt{U^2 + V^2} \kappa_c = 2 \pi \rho a
\sqrt{U^2 + V^2} V$

\vspace{1in}

Equivalently, the lift is $2 \pi \rho a \abs{U}^2 \sin \alpha$.
This result suggests that for a wing of area $A$, the total lift
is proportional to $\rho U^2 A \sin \alpha$.

We have ignored separation at the leading edge.  This can be delayed
by rounding it.

\vspace{1in}

Even in this case, if $\alpha$ is big enough $\gtrsim 10^\circ$, flow
will separate at the leading edge (stall) with a catastrophic decrease in
lift and increase in drag.

\section{Physical mechanism}

How is a circulation established from rest?  At $t = 0$, the picture looks
like:

\vspace{1in}

At $t=0^+$, the flow between $T$ and $A$ decelerates rapidly and there
is a severe adverse pressure gradient on $TA$.  The boundary layer therefore
separates to give a small region of reversed flow in the boundary layer.
This gives a small eddy with circulation $-\kappa_c$.

\vspace{1in}

The eddy is then convected away from the plate (``starting vortex left at
the airport'').  As total circulation at infinity is conserved there
must be a circulation $\kappa_c$ around the wing.

\vspace{1in}

\chapter{Kelvin-Helmholtz instability}

At high Reynolds number many flow profiles are unstable.  We will consider
the easiest case, steady inviscid flow with a discontinuity in velocity.

\vspace{1in}

We have $\vu = (-\tfrac{1}{2} U \sgn y,0,0)$ and hence
$\om = (0,0,U \delta(y))$, a vortex sheet.  Suppose the vortex
sheet is perturbed to $y = \eta(x,t) = f(t) e^{\imath k x}$ and that
the disturbance is small: $\abs{\pd{\eta}{x}} = \abs{k f} \ll 1$.

Now vorticity moves with the fluid, so $y = \eta$ is a material
surface and we have the kinematic boundary condition
$\left.\DDt{}\left(y -\eta\right)\right|_{y = \eta} = 0$.  For $y
\gtrless \eta$ the flow remains irrotational, so

\[
\vu = \begin{cases}
- \tfrac{1}{2} U \e_x + \nabla \phi_> & y > \eta \\
\tfrac{1}{2} U \e_x + \nabla \phi_< & y < \eta,
\end{cases}
\]

with $\nabla^2 \phi_\gtrless = 0$ and $\phi_\gtrless \to 0$ as
$y \to \pm \infty$.  $\phi_\gtrless$ must inherit the
$e^{\imath k x}$ dependence on $x$ as the perturbation is linear; so
\[
\phi_\gtrless = g_\gtrless(t) e^{\imath k y \mp \abs{k} y}.
\]

We now apply the kinematic boundary condition $\left.\DDt{} \left( y- \eta
\right)\right|_{y = \eta} = 0$ to get

\[
\left.\pd{\phi}{y}\right|_{y = \eta} - \pd{\eta}{t}
-u \pd{\eta}{x} - v \pd{\eta}{y} = 0 \qquad \text{at } y = \eta.
\]

We use Taylor's theorem to evaluate this from information at $y=0$
and neglect quadratic terms to get the linearised boundary condition
\[
\pd{f}{t} \mp \tfrac{1}{2} U \imath k f = \mp \abs{k} g_\gtrless.
\]

We still need the pressure to be continuous at $y = \eta$.  To do this
we derive the unsteady form of Bernoulli.

For an inviscid irrotational flow we have
\[
\pd{\vu}{t} = -\tfrac{1}{\rho} \nabla \left( p + \tfrac{1}{2} \rho
\abs{\vu}^2\right)
\]
and so

\begin{equation}\label{eq:unstbern}
p = \rho F(t) - \rho \pd{\phi}{t} - \tfrac{1}{2} \rho \abs{\vu}^2.
\end{equation}

Therefore
\begin{align*}
p_> &= p_\infty + \tfrac{1}{2} \rho \left( \tfrac{1}{2} U \right)^2
- \rho \left( \pd{\phi_>}{t} + \tfrac{1}{2} \abs{-\tfrac{1}{2} U
\e_x + \nabla \phi_>}^2\right) \\
&= C - \rho \left( \pd{\phi_>}{t} - \tfrac{1}{2} U \pd{\phi_>}{x}
\right) + \cO(\eta^2).
\end{align*}

We now apply $p_> = p_<$ at $y = 0$ to get
\[
\dot{g}_> - \tfrac{1}{2} \imath k U g_>
= \dot{g}_< + \tfrac{1}{2} \imath k U g_<.
\]

We have three linear equations with constant coefficients for
$g_>$, $g_<$ and $f$, so each is proportational to $e^{\sigma t}$.  Plugging
this solution in gives $\sigma^2 = \frac{U^2 k^2}{4}$ and so
$\sigma = \pm \frac{U k}{2}$.  Thus there exists a growing mode
with $\sigma = \tfrac{1}{2} U \abs{k}$ and the sheet is unstable to
disturbances of all wavelengths.

\subsubsection*{Notes}

\begin{enumerate}
\item As $k \to \infty$, $\sigma \to \infty$ and thus short waves grow
  infinitely fast.
\item The disturbance rapidly grows out of the linear r\'egime.  We
get roll-up of vortices.%
\footnote{See van Dyke page 85.}
\item Physical mechanism when $\sigma = \tfrac{1}{2} U \abs{k}$.  We
  get
\[
g_\gtrless = \tfrac{1}{2} U \left\{ \mp \sgn k + \imath \right\} f
\]
and so $\left[ u \right]_-^+ = -\tfrac{1}{2} \imath U \abs{k} f - U$
and if $\eta = \eta_0 \cos k x e^{\sigma t}$ we have $\left[ u
\right]_-^+ = -U + \eta_0 U \abs{k} \sin k x e^{\sigma t}$.  Thus the
vortex sheet is stronger at $x = \frac{3 \pi}{2 k}$ and weaker at $z =
\frac{\pi}{2 k}$.

\vspace{1.5in}

\item If viscous effects are included then the vorticity diffuses over
  a distance $\delta \propto \sqrt{\nu t}$ in a time $t$.  We expect
  that if $k^{-1} \lesssim \delta$ the inviscid theory will be wrong
  (and viscosity damps short waves), but long waves with $k^{-1} \gg
  \delta$ should not be affected by vorticity diffusion.  Since the
  growth time $\sigma^{-1} \propto \left( U k \right)^{-1}$ we must
  have $k \ll \frac{U}{\nu}$ for the inviscid theory to work.
  
  If we guess (on dimensional grounds) that short waves are damped at
  a rate $\nu k^2$ then $\sigma = \tfrac{1}{2} U \abs{k} - \nu k^2$
  and there is a most unstable wavelength $k = \frac{U}{4 \nu}$.

  \vspace{1in}
  
\item A long-wave inviscid mechanism will also apply to inviscid
  profiles with inflexion points (eg $\tanh y$).
  
\item The relationship $\sigma^2 = \frac{U^2 k^2}{4}$ is called a
  dispersion relation.  In waves, $\sigma = \imath \omega$ is pure
  imaginary and $\frac{\omega}{k} = c$ is a wavespeed; $\eta = \eta_0
  e^{\imath k \left( x - c t \right)}$.
  
  In some ways this calculation is artificial; in practice we can't
  establish a fully-developed unstable steady state to perturb. (This
  is temporal instability.)  It is more natural to introduce a
  perturbation $e^{\imath \omega t}$ with $\omega \in \R$ at $x = 0$
  and to observe the growth or decay in $x$.  We thus have $\eta =
  \eta_0 e^{\imath \omega t - \imath k x}$ with $k \in \C$.  This is a
  spatial instability problem.  We find that $k = \pm 2 \imath
  \abs{\frac{\omega}{U}}$

  Temporal and spatial analyses are identical at or near marginal stability.
  In general, spatial analysis is harder and so temporal analysis is more
  common.
\end{enumerate}

\chapter{Rising bubbles}

\section{Dimensional analysis}

\vspace{1in}

We want to know both the rise velocity and the shape of the bubble.

The natural lengthscale is $V^{\frac{1}{3}}$ and we set
$a = \left( \frac{3 V}{4 \pi} \right)^{\frac{1}{3}}$, the radius of
a sphere of volume $V$.

For a steady rise, buoyancy is presumably balanced by viscosity.  Define
$U_0 = \frac{a^2 g}{\nu}$ and then the Reynolds number is
\[
\Rey = \frac{U_0 a}{\nu} = \frac{a^3 g}{\nu^2} = \frac{\text{inertia}}
{\text{viscosity}}.
\]

We need a second dimensionless group to indicate the importance of
surface tension.  This is the capillary number
\[
\Ca = \frac{\text{viscous stresses}}{\text{surface tension}}
= \frac{ \mu \frac{U_0}{a}}{\frac{\gamma}{a}} = \frac{\mu U_0}{\gamma}.
\]

Note that in this case, $\Ca = \frac{\rho g a}{\frac{\gamma}{a}} =
\frac{\text{hydrostatic pressure}}{\text{surface tension pressure}}$,
a quantity which is usually called the Bond number which in this case
happens to be equal to the capillary number.

The actual rise speed of the bubble, $U = U_0 f(\Rey,\Ca)$.  The shape
must also depend on $\Rey$ and $\Ca$.

If $\Ca \ll 1$ then surface tension is very large and the shape remains
almost spherical.  This is theoretically tractable, we have
$U = U_0 \Hat{f}(\Rey)$.  If $\Ca \sim 1$ other shapes are possible --- this
is a hard problem.

\section{Low Reynolds number, low capillary number}

This is the $\Rey, \Ca \ll 1$ case.  The bubble shape is a near
sphere $r = a \left( 1+ \cO(\Ca)\right)$ and viscous forces dominate.

We must solve
\begin{align*}
\mu \nabla^2 \vu &= \nabla P \quad r > a \\
\nabla \cdot \vu &= 0 
\end{align*}

where $P = p - \rho g z$ is the modified pressure.

The kinematic boundary condition is $\vu \cdot \vn = \vu \cdot \vn$ on $r = a$.
We also have $\vu \to 0$ as $r \to \infty$.

The tangential stresses must be continuous: $\vn \times \left[ \bs \cdot \vn
\right] = 0$ on $r = a$.  This simplifies to $\vn \times \left[ \e \cdot \vn
\right] = 0$ and thus $e_{r \theta} = 0$ on $r = a$ (in spherical polars).

Finally, $\left[ \bs \cdot \vn \right] = \gamma \kappa \vn$ on $r = a$,
where $\kappa$ is the surface curvature.  Taking the normal component
we get $p_{\text{int}} -p + 2 \mu e_{n n} = \gamma \kappa$ and therefore
\[
\kappa = \frac{1}{\gamma} \left( \text{const} - \left(P + \rho g z
\right) + 2 \mu e_{nn} \right) \quad \text{on } S.
\]

Thus at $\Ca = 0$ the drop is spherical, $\kappa = \frac{2}{a}$ and
surface tension increases the internal pressure to $\frac{2 \gamma}{a}$.

If $\Ca \ll 1$ then at leading order the drop is spherical, but the non-zero
right hand side causes an $\cO(\Ca)$ modification to $\kappa$ and we
can use this to find an $\cO(\Ca)$ modification to the shape at the end.

We can either solve $D^4 \psi = 0$ (see page \pageref{ref:sphereflow})
and apply the boundary conditions to get $C = \frac{U a}{2}$ and
$D = 0$ or use the method of sheet 2:

\newcommand{\bp}{\boldsymbol{\phi}}

\[
\vu = \vect{E} \cdot \vx + 2 \bp - \nabla \left( \bp \cdot \vx \right)
\]

with $\nabla^2 \bp = 0$.   We have $\vect{E} = 0$ and the harmonic potential
$\bp$ must be linear in $\vu$, so
\[
\bp = \frac{\alpha \vu}{r} + \beta \vu \cdot \nabla \nabla \frac{1}{r}.
\]

$\vu$ evaluates to
\[
\vu = \alpha \left( \frac{\vu}{r} + \frac{\vu \cdot \vx\, \vx}{r^3} \right)
+ 4 \beta \left( - \frac{\vu}{r^3} + \frac{3 \vu \cdot \vx\, \vx}{r^5}\right).
\]

Applying the boundary conditions we get $\alpha = \frac{a}{2}$ and $\beta =0$.
We also have $P = - 2 \mu \nabla \cdot \bp = - \mu a \vu \cdot \nabla
\frac{1}{r}$.

For $r > a$ this is a pure Stokeslet field $\abs{u} \propto \frac{1}{r}$.
Integrating over a sphere gives $\vF = 4 \pi \mu a \vu$, the force of the
bubble on the fluid.  Thus $\vu = \frac{a^2 g}{3 \nu}$ (by equating this
to the Archimidean uplift).

To get the shape change, we have
\[
-p + \rho g z + 2 \mu e_{rr} = \rho \vect{g} \cdot \vx - 3 \mu \left.
\frac{\vu \cdot \vx}{r^3} \right|_{r = a} + \text{const} = \text{const on
$r = a$.}
\]

Thus, surprisingly, a spherical drop at low Reynolds number has no
tendency to deform even if $\Ca$ is not small.  Our solution works for
$\Rey \ll 1$ and $\Ca$ arbitrary.

\section{High Reynolds number, low capillary number}

\vspace{1in}

Our first guess is $\vu = \nabla \phi$ and $\nabla^2 \phi = 0$.
We have $\phi = \frac{a^2 \vu \cdot \vx}{r^3}$, which satisfies the Euler
equation and $\vu \cdot \vn = \vu \cdot \vn$ on $r=a$.

The tangential stress on $r=a$ is $\sigma_{r \theta} = 2 \mu e_{r \theta}
= - \frac{3 \mu U}{a} \sin \theta \neq 0$. Also, the drag on the sphere
is zero.

We expect that the outer flow will be modified (at $\cO(\Rey^{-1})$)
and a boundary layer will arise near $r=a$, sweeping vorticity into
the wake, perhaps modifying the outer flow at
$\cO(\Rey^{-\frac{1}{2}})$.

Without gross separation (which is not observed in experiments), we know
the flow almost everywhere and can calculate
\begin{align*}
\vu \cdot \vF &= 2 \mu \int_{r > a} e_{ij} e_{i j}\, \ud V \\
&= 2 \mu \int_{r > a} \pd{^2 \phi}{x_i \partial x_j}
\pd{^2 \phi}{x_i \partial x_j}\, \ud V \\
&= 2 \mu \int_{r > a} \pd{}{x_i} \left\{ \pd{\phi}{x_j}
\pd{^2 \phi}{x_i \partial x_j}\right\}\, \ud V \\
&= 2 \mu \int_{r = a} \vn \cdot \nabla \left(\tfrac{1}{2} \vu^2\right)\,
\ud S \\
&= 12 \pi \mu a U^2.
\end{align*}

Thus $U = \frac{1}{9} \frac{a^2 g}{\nu} = \frac{1}{9} U_0$ and
$\Hat{f}(\Rey) \to \frac{1}{9}$ as $\Rey \to \infty$.

\vspace{1in}

As for the shape change, we see that pressure variations over $r=a$
scale as $\tfrac{1}{2} \rho U^2$ and so $\Delta p = \tfrac{1}{2 \cdot
9^2} \rho \left( \frac{a^2 g}{\nu} \right)^2$.  The drop will remain
spherical if $\frac{\Delta p}{\frac{\gamma}{a}} \ll 1$, or alternatively
$\Rey\, \Ca \ll 160$.

\subsection*{Free surface boundary layers}

Near $r=a$ there is a boundary layer across which not $u_\parallel$
but $\pd{u_\parallel}{n}$ jumps.  There is thus a jump in $\bo$ across
the boundary layer.

The thickness of the boundary layer is still $\delta = a \Rey^{-\frac{1}{2}}$
and the boundary layer equation still applies, but the velocity gradient
in the layer scales as $\frac{U}{a}$ and not $\frac{U}{\delta}$.

The energy dissipation in the layer scales as $4 \pi a^2 \delta
\frac{U^2}{a^2} = \frac{4 \pi a U^2}{\sqrt{\Rey}}$  Thus
\[
\vu \cdot \vF = 12 \pi \mu a U^2 \left( 1 + \beta \Rey^{-\frac{1}{2}} \right).
\]

In fact, $\beta = 2.2$, but we need the dissipation in the wake of the
bubble to find this.

Observation suggests that the boundary layer does not separate unless
the curvature is very high.

\section{The oblate spheroidal bubble}

This is the $\Rey \gg 1$ and $\Ca \sim 1$ case.  It is tough to make
progress.

\vspace{1in}

The first correction will be spheroidal and will modify the rise speed.  As
the capillary number increases we will eventually get boundary layer
separation.

\section{Spherical cap bubble}

This is the $\Rey, \Ca \gg 1$ case.

\vspace{2in}

\[
V = \frac{\pi}{3} \Bar{a}^3 \left\{ 2 - 3 \cos \alpha + \cos^3 \alpha
\right\}.
\]

The outer flow is flow past a sphere of radius $\Bar{a}$,
$\phi = -U \left( r + \frac{\Bar{a}^3}{2 r^2} \right) \cos \theta$.
The tangential velocity on $r = \Bar{a}$ is $u_\theta = \tfrac{3}{2} U
\sin \theta$.  As $\Ca \gg 1$ we have continuity of pressure,
and so $\tfrac{1}{2} \rho u_\theta^2 - \rho g z = \text{const}$.
Near $\theta = 0$, $u_\theta^2 \approx \tfrac{9}{4} U^2 \theta^2$ and
$z \approx \tfrac{1}{2} \Bar{a} \theta^2$.  Thus $U^2 = \tfrac{4}{9}
g \Bar{a}$ and so the rise velocity $U = \tfrac{2}{3} \sqrt{g \Bar{a}}$
independent of $\nu$.

Note that the $\Rey \to \infty$ limit differs from the $\Rey = \infty$
limit.

There are turbulent dissipative processes in the wake that give a
rise velocity independent of $\nu$ as $\nu \to 0$.  Note that this
rise velocity agrees with experiment.  Experimentally,
$\alpha$ is found to be in the range $40^\circ < \alpha < 60^\circ$.

\section{The skirted bubble}

This has $\Rey$ and $\Ca$ ``largish''.  If we decrease surface tension
from the spherical cap bubble a discontinuity appears near the sharp edge.
For suitable parameter values a cusped shaped edge appears and forms a
skirt around the bubble.

\vspace{1.5in}

If surface tension is decreased a little from this, Kelvin-Helmholtz
instability occurs near the cusped edge.

\vspace{1.5in}

This is poorly understood.

\backmatter

\appendix

\chapter{Stress tensor: EJH approach}

Unfortunately, the (standard) derivations given earlier for the
surface traction and symmetry of $\bs$ do not work (by dimensional
arguments).  The following can be inserted in the appropriate places
in \S\ref{sec:stress}.  Discovering the precise places is left as a
challenge to the reader.

\section*{Linearity}

Consider the force balance on this small tetrahedron.  The volume
and acceleration forces are $\cO(\rho g L^3)$, where $L$ is the linear 
size of the tetrahedron.  The surface forces are $\cO(p L^2)$, with a
typical pressure $\rho g H$, where $H$ is the height of the
atmosphere.  Hence for small tetrahedra with $L \ll H$ the surface
forces are $\cO(\tfrac{H}{L})$ larger than the volume and acceleration 
forces and so must balance amongst themselves.

\section*{Symmetry}

The moment of the surface forces is $\cO(p L^3)$ and the moments of
the volume and acceleration forces are $\cO(\rho g L^4)$.  So again
the surface forces must balance amongst themselves.

\begin{thebibliography}{9}
\bibitem{Acheson} D.J. Acheson, \emph{Elementary Fluid Dynamics}, OUP,
  1990.
  
  {\sffamily \small This is an excellent book, easy to read and with
    almost everything in.  Highly recommended. }
  
\bibitem{Batchelor} G.K. Batchelor, \emph{An Introduction to Fluid
    Dynamics}, CUP, 1967.
  
  {\sffamily \small A good reference but somewhat intimidating.  It
    looks like the book of an older version of this course, although
    the content is roughly similiar. Probably not worth spending money
    on... }

\bibitem{SVF} \emph{Slow Viscous Flow}, unpublished, 1998.

  {\sffamily \small These are the course notes for the Part III Slow
    Viscous Flow course, which essentially covers Chapter
    \ref{chap:stokes} in more detail.  Worth a look --- if you can
    find them! }
  
\bibitem{VanDyke} M. van Dyke, \emph{An Album of Fluid Motion}, The
  Parabolic Press, 1982.
  
  {\sffamily \small Lots of pictures of flows.  An excellent book.  Go
    out and buy it.  Now.}

\end{thebibliography}

There are many other books covering this course.  I have heard good things
about Landau and Lifschitz, and Lamb is a standard text for inviscid
fluid mechanics.  I haven't looked at them so I can't personally recommend
them.  If you can, please send me a \emph{brief} review and I will include
it above.

\end{document}

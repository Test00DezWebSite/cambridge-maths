\documentclass{notes}

\usepackage{varioref,pstricks}

\usepackage[matrix,graph,curve,ps]{xy}

\newcommand{\Q}{\mathbb{Q}}
\newcommand{\cO}{\mathcal{O}}
\newcommand{\cA}{\mathcal{A}}
\DeclareMathOperator{\inter}{Int}
\DeclareMathOperator{\Per}{Per}
\DeclareMathOperator{\Tr}{Tr}

\theoremstyle{plain}
\newtheorem{proposition}{Proposition}[chapter]
\newtheorem{theorem}[proposition]{Theorem}
\newtheorem{corollary}[proposition]{Corollary}
%\newtheorem{example}{Example}[chapter]
\newtheorem{definition}[proposition]{Definition}
\newtheorem{lemma}[proposition]{Lemma}

\psset{unit=0.01cm}

\makeatother

\begin{document}

\frontmatter

\title{Dynamical Systems}

\lecturer{Dr.~C.~Baesens}
\maintainer{Paul Metcalfe}
\date{Lent 1998}
\maketitle

\thispagestyle{empty}
\noindent\verb$Revision: 2.2 $\hfill\\
\noindent\verb$Date: 1999/09/17 15:27:35 $\hfill

\vspace{1.5in}

The following people have maintained these notes.

\begin{center}
\begin{tabular}{ r  l}
-- date & Paul Metcalfe
\end{tabular}
\end{center}

\tableofcontents

\chapter{Introduction}

These notes are based on the course ``Dynamical Systems'' given by
Dr.~C.~Baesens in Cambridge in the Lent Term 1998.  These typeset
notes are totally unconnected with Dr.~Baesens.  The recommended books
for this course are discussed in the bibliography.

\alsoavailable
\archimcopyright

\mainmatter

\chapter{Basic concepts}

\section{What is a dynamical system?}

A dynamical system is a system whose evolution in time is uniquely determined
by its current state.  Time can be discrete or continuous, and in this
course we concentrate almost entirely on discrete time, in which the
interesting ideas are reached more easily.  Also, some continuous time
dynamical systems can be reduced to discrete time dynamical systems.

Discrete time dynamical systems are generated by the iteration of maps.
Let $X$ be a topological space and $f \colon X \mapsto X$ be continuous.
Then $x_{n+1} = f(x_n)$ is a dynamical system, where $n$ is the time and $x_n$
is the state at time $n$.  $X$ is called state space or
(for historical reasons) phase space.

For $n > 0$ define the $n^{\text{th}}$ iterate of $f$, $f^n$ as

\[
f^n = \overbrace{f \circ \dots \circ f}^{\text{$n$ times}}.
\]

We also define $f^0 \equiv \text{id}$, and so $x_n = f^n(x_0)$.  If $f$
is invertible then we can also define $f^{-n} = \left(f^{-1}\right)^n$.

A more sophisticated view is that a dynamical system is an action of a
semigroup or a group on a topological space. We have

\begin{align*}
\phi \colon G \times X &\mapsto X \\
\phi(g,x) &\mapsto \phi_g(x) \qquad \text{such that} \\
\phi_g(\phi_h(x)) &= \phi_{g h}(x),
\end{align*}

where $G = (\R,+)$, $(\R_+,+)$, $(\Z,+)$ or $(\Z_+,+)$
$X$ is a topological space and $\phi$ is continuous.  Then the 
discrete time dynamical system is the map $\phi \colon \begin{matrix}
\Z \\
\Z_+ \end{matrix} \times X \mapsto X$
such that $\phi(n,x) = f^n(x)$.

\begin{definition}
  The forward orbit of $x \in X$, which is denoted $O^+(x)$ is the
  sequence $x, f(x), f^2(x), \dots$, that is $\left(f^n(x)\right)_{n \in
    \Z_+}$.  If $f$ is invertible we can define the (full) orbit of
  $x$ as $O(x) = \left( f^n(x) \right)_{n \in \Z}$.  We can also define the
  backwards orbit of $x$ in the obvious way.
\end{definition}

In one dimension there is a graphical representation of iteration.

\begin{center}
\begin{pspicture}(-20,-20)(500,500)
\psline{->}(-10,0)(500,0)
\psline{->}(0,-10)(0,500)
\psline(0,0)(500,500)
\pscurve(0,0)(50,100)(150,400)(200,350)(350,100)(500,500)
\psline[linestyle=dashed](200,0)(200,350)(350,350)(350,0)
\psline[linestyle=dashed](350,100)(100,100)(100,0)
\rput(200,-20){$x_0$}
\rput(350,-20){$x_1$}
\rput(100,-20){$x_2$}
\rput(470,-20){$X$}
\rput(-20,470){$f$}
\end{pspicture}
\end{center}

The apparent generalisation $x_{n+k} = F(x_{n+k-1}, \dots, x_n)$ can
also be viewed as a dynamical system by putting $X = \R^k$ and
\[
f \begin{pmatrix}
x^{(1)} \\
x^{(2)} \\
\vdots \\
x^{(k)}
\end{pmatrix}
= \begin{pmatrix}
x^{(2)} \\
\vdots \\
x^{(k)} \\
F(x^{(k-1)}, \dots, x^{(1)})
\end{pmatrix}.
\]

\section{Dynamical systems viewpoint}

We do \emph{not} aim to find explicit formulae but instead to understand
the ``qualitative features'' of the dynamical system, e.g. fixed points,
periodic orbits.

We say that $x$ is a \emph{fixed point} of $f$ if $f(x) = x$ and $x$
is a \emph{periodic point of (least) period $q$} if $f^q(x) = x$ and
$f^n(x) \neq x$ for $0 < n < q$.

More thoroughly, by ``qualitative features'' we mean properties which are
preserved under change of co-ordinates by homeomorphism.%
\footnote{homeomorphism: continuous map with continuous inverse} We
say that $f \colon X \mapsto X$ and $g \colon Y \mapsto Y$ are
\emph{topologically conjugate} if there exists a homeomorphism $h
\colon X \mapsto Y$ such that $g \circ h = h \circ f$ ($h$ is called
the conjugacy).  Then the qualitative features of $f$ and $g$ are the
same, for instance if $f$ has a fixed point $\Bar{x}$ then $g$ has a
fixed point $h(\Bar{x})$.

\section{Asymptotic behaviour}

The most interesting qualitative features are those to do with the behaviour
of orbits when $t \to \pm \infty$.

\begin{definition}
The $\omega$-limit set of $x \in X$, $\omega(x)$ is the set
\[
\omega(x) = \{ y \in X : \exists (n_i) \to \infty\ \text{such that}\
f^{n_i}(x) \to y \}.
\]

If $f$ is invertible then we can define the $\alpha$-limit set by replacing
$\infty$ with $-\infty$.  Note that $\omega(f(x)) = \omega(x)$.
\end{definition}

For instance, if $\Bar{x}$ is a fixed point or periodic point of $f$ then
$\omega(\Bar{x}) = O^+(\Bar{x})$.%
\footnote{Abuse of notation: $O^+(x)$ is now $\{ f^n(x) : n \in \Z_+ \}$.}

\section{Homeomorphisms of the interval}\label{sec:inthom}

The simplest class of dynamical system is that of homeomorphisms of
a closed interval $I \subset \R$.  There are two possible cases

\begin{itemize}
\item $f$ is orientation preserving.  Then the only possible $\omega$
(resp. $\alpha$) -limit sets are fixed points.

\begin{proof}
  Take $x \in I$.  If $f(x) = x$ there is nothing to prove so assume
  $f(x) > x$, so by orientation preservation $\left(f^n(x)\right)_n$
  is an increasing sequence, bounded above and so tends to a limit
  $\Bar{x}$, which by the continuity of $f$ must be a fixed point.
\end{proof}

We have obtained a complete description of the dynamics.  There is
a closed set of fixed points and in each complementary interval the
orbits move either to the right ($f(x) > x$) or to the left ($f(x) < x$).

\item $f$ is orientation reversing.  Then the only possible
$\omega$ (resp. $\alpha$) -limit sets are fixed points or period 2 points.
To prove this, simply note that $f^2$ is orientation preserving.
\end{itemize}

To get more exciting $\omega$-limit sets we can consider maps of the
circle or add non-invertibility in one dimension, or we can go to
higher dimensions.

\chapter{Maps of the circle}

\section{Generalities}

We consider continuous maps of the circle $S^1$ into itself.  There
are two ways of representing $S^1$,

\begin{enumerate}
\item $S^1 = \{ (x,y) \in \R^2 : x^2 + y^2 = 1 \}$, or equivalently
$S^1 = \{ z \in \C : \abs{z} = 1 \}$.  This is the \emph{multiplicative
notation}.
\item $S^1 = \R / \Z$, the quotient of the reals by integer translation.
This is \emph{additive notation}.
\end{enumerate}

Additive notation will be more useful for our purposes.  $\log$
establishes the isomorphism between the two representations.

We can represent circle maps graphically by cutting
the circle at one point and regarding it as an interval.

\begin{center}
\begin{pspicture}(-20,-20)(500,500)
\psline{->}(-10,0)(500,0)
\psline{->}(0,-10)(0,500)
\psline(0,0)(500,500)
\psline[linestyle=dashed](0,100)(500,100)
\psline[linestyle=dashed](300,0)(300,500)
\rput(10,-20){$0$}
\rput(470,-20){$1$}
\rput(-20,10){$0$}
\rput(-20,480){$1$}
\pscurve(0,100)(100,300)(200,290)(300,500)
\pscurve(300,0)(400,70)(500,100)
\end{pspicture}
\end{center}

The simplest example of circle map is rotation by angle $\beta$%
\footnote{$\beta \in \R$...}, $r_\beta$.  In multiplicative and
additive notation respectively we have

\begin{enumerate}
\item $r_\beta z = e^{2 \pi \imath \beta} z$
\item $r_\beta x = x + \beta \pmod{1}$.
\end{enumerate}

We can find the iterates of $r_\beta$ in both additive and multiplicative
notation:

\begin{enumerate}
\item $r_\beta^n z = e^{2 \pi \imath n \beta} z$
\item $r_\beta^n x = x + n\beta \pmod{1}$.
\end{enumerate}

Represented graphically this is

\begin{center}
\begin{pspicture}(-20,-20)(500,500)
\psline{->}(-10,0)(500,0)
\psline{->}(0,-10)(0,500)
\psline(0,0)(500,500)
\psline[linestyle=dashed](0,100)(500,100)
\psline[linestyle=dashed](400,0)(400,500)
\rput(10,-20){$0$}
\rput(470,-20){$1$}
\rput(-20,10){$0$}
\rput(-20,480){$1$}
\psline(0,100)(400,500)
\psline(400,0)(500,100)
\end{pspicture}
\end{center}

There is a crucial distinction between $\beta \in \Q$ and $\beta \in
\R \setminus \Q$.  If $\beta$ is rational with $\beta = \frac{p}{q}$,
$q>0$ and $p$ and $q$ coprime then $r_\beta^q x = x$ for all $x$ and
all points are periodic with least period $q$.  When $\beta$ is
irrational things are a little more complicated.

\begin{definition}
A subset $S \subset X$ is \emph{invariant} under $f$ if $f S = S$.
It is \emph{positively invariant} if $f S \subset S$ and
\emph{negatively invariant} if $f^{-1} S \subset S$.
\end{definition}

\begin{definition}
  A minimal set $S \subset X$ for $f$ is a closed $f$-invariant subset
  of $X$ with no proper closed invariant subsets.  Equivalently, a
  minimal set is a closed invariant subset of $X$ in which $O^+(x)$ is
  dense in $S$ for all $x \in S$.
\end{definition}

For instance, a periodic orbit is a minimal subset.

\begin{proposition}
If $\beta \in \R \setminus \Q$ then the orbit of every point $x \in S^1$
under rotation $r_\beta$ is dense in $S^1$.
\end{proposition}

\begin{proof}
Given $\beta \in \R \setminus \Q$, $x \in S^1$ and $\epsilon > 0$, the
points $r_\beta^n x$, $n \in \Z_+$ (or $\Z$) are distinct,
else $r^m_\beta x = r^n_\beta x$ for some $m,n$ and $(m-n) \beta \in \Z$
(contradiction). 

As $S^1$ is compact, $\exists n \neq m$ such that
$0 < d(r_\beta^m x, r_\beta^n x) < \epsilon$.  Let $N = \abs{n - m}$
and $\beta_N = d(r_\beta^m x, r_\beta^n x)$.  Now $r_\beta$
preserves orientation and length on $S^1$, so that $r^N_\beta$
is just a rotation by $\beta_N$.  Thus the points
\[
\left\{r^{j N}_\beta x : j = 0, 1,\dots, \left[\beta_N^{-1}\right]\right\}
\]
are equally spaced and come within $\epsilon$ of every point of the circle.
\end{proof}

Do we have similar properties for more general orientation preserving
homeomorphisms of $S^1$?

\section{Lift and degree}

Define $\pi \colon \R \mapsto S^1$ by $\pi(x) = x \pmod{1}$.  Then
given a continuous function $f \colon S^1 \mapsto S^1$ there exists a
(nonunique) continuous function $F \colon \R \mapsto \R$ such that $f
\pi x = \pi F x$.  $\pi$ is \emph{not} a topological conjugacy --- it
is not invertible.

$F$ is called a \emph{lift} of $f$.

\begin{lemma}[Properties of lifts]\label{lem:lifts}\hfill
\begin{enumerate}
\item If $F_i \colon \R \mapsto \R$, $i=1,2$ are lifts of the same
continuous map $f$ then $\exists k \in \Z$ such that
$F_1 x - F_2 x = k\ \forall x$.
\item Given a continuous function $f \colon S^1 \mapsto S^1$
there exists $d$ such that for all lifts $F$ and for all $x \in \R$
such that $F(x+1) = F(x) + d$.  $d$ is called the degree of $f$ and
written $\deg f$.
\item If $F$ is a lift of $f$ then $F^n$ is a lift of $f^n$ for all
$n \in \Z$.
\item $\deg f^n = \left( \deg f\right)^n$.
\end{enumerate}
\end{lemma}

We will prove 1 and 2 leaving 3 and 4 as exercises. 

\begin{proof}\hfill
\begin{enumerate}
\item Take $x \in \R$.  Then $f \pi x = \pi F_1 x = \pi F_2 x$ and
so $F_1 x - F_2 x \in \Z$.  Thus $F_1 x - F_2 x$ is constant
(by connectedness of $\R$ and continuity of $F_1 - F_2$).
\item $\pi(x+1) = \pi x$ and so $
\pi F (x+1) = f \pi (x+1) = f \pi x = \pi F x$.  Thus
$F(x+1) - Fx = d \in \Z$ for all $x$ (argue as before).  Let
$\Bar{F}$ be another lift.  Thus $\Bar{F} = F + k$ and hence
$\Bar{F}(x+1) - \Bar{F}x = d$.
\end{enumerate}
\end{proof}

\begin{center}
\begin{pspicture}(-20,-20)(500,500)
\psline{->}(-10,0)(500,0)
\psline{->}(0,-10)(0,500)
\psline(0,0)(500,500)
\psline[linestyle=dashed](0,100)(500,100)
\psline[linestyle=dashed](300,0)(300,500)
\rput(10,-20){$0$}
\rput(470,-20){$1$}
\rput(-20,10){$0$}
\rput(-20,480){$1$}
\pscurve(0,100)(100,300)(200,290)(300,500)
\pscurve(300,0)(400,70)(500,100)
\rput(100,400){$\deg f = 1$}
\end{pspicture}
\end{center}

Intuitively, $\abs{\deg f}$ measures how many times the circle is
mapped around itself by $f$.

If $f$ is a homeomorphism then $\deg f = \pm 1$.  If $f$ is orientation
preserving then $\deg f = 1$ and if $f$ is orientation reversing
$\deg f = -1$.

The rotation map $r_\beta$ has $\deg r_\beta = 1$.  If we take lifts
$R_{\beta,k} \colon \R \mapsto \R$, $x \mapsto x + \beta + k$ with
$k \in \Z$ then we see that

\[
\lim_{n \to \infty} \frac{R^n_{\beta,k} x - x}{n} =
=\lim_{n \to \infty} \frac{n (\beta + k)}{n} = \beta + k.
\]

We want to generalise this concept to degree 1 maps of the circle.
We restrict to degree 1 maps from now on.

\subsection[Degree one circle maps]%
{Properties of degree one continuous circle maps}

Lemma \ref{lem:lifts} gives that if $f$ is a degree 1 circle map with
lift $F$,

\begin{equation}\label{eq:degoneprop}
\begin{gathered}
F(x+1) = F(x) + 1 \\
F^k(x+1) = F^k(x) + 1 \qquad k \in \N \\
F^k(x+m) = F^k(x) + m \qquad m \in \Z \\
F(x) - x \text{ is periodic with period 1,} \\
F^k(x) - x \text{ is periodic with period 1.}
\end{gathered}
\end{equation}

\section{Rotation number}

\begin{definition}
Let $f \colon S^1 \mapsto S^1$ be a degree 1 continuous map and
$F$ be a lift of $f$.  Then the rotation number of $x \in S^1$ under $F$
is
\[
\Bar{\rho}(F,x) = \lim_{n \to \infty} \frac{F^n(x) - x}{n}
\qquad \text{if this limit exists.}
\]
\end{definition}

\begin{theorem}\label{thm:rotoph}\hfill\\
Let $f \colon S^1 \mapsto S^1$ be an orientation preserving homeomorphism.
Then:

\begin{enumerate}
\item For $x \in \R$, $\Bar{\rho}(F,x)$ exists and is independent of
$x$.  (Denoted $\Bar{\rho}(F)$.)
\item $\rho(f) := \Bar{\rho}(F) \pmod{1}$ does not depend on the lift
used.
\item $\rho(f)$ depends continuously on $f$.
\end{enumerate}
\end{theorem}

$\rho(f)$ is called the rotation number of $f$.  Before we prove
\ref{thm:rotoph} we need two lemmas.

\begin{lemma}\label{lem:rotineq1}
Given $x$, $y \in \R$, $n \in \N$ and $F$ a lift of an orientation
preserving homeomorphism of $S^1$,

\[
F^n(x) - x - 1 < F^n(y) - y < F^n(x) - x + 1.
\]
\end{lemma}

\begin{proof}
$\exists m \in \Z$ such that $x \le y+m < x + 1$.  Then $F^n$ is monotone
(as $F$ is), and
\[
F^n(x) \le F^n(y+m) < F^n(x+1).
\]

Using \eqref{eq:degoneprop} freely,

\begin{align*}
F^n(x) - x - 1 &\le F^n(y+m) - (x+1) \\
& < F^n(y+m) - (y+m)
& < F^n(x+1) - x = F^n(x) -x + 1.
\end{align*}

Now note that $F^n(y+m) - (y+m) = F^n(y) - y$.
\end{proof}

\begin{lemma}\label{lem:rotineq2}
Let $F$ be a lift of an orientation preserving homeomorphism
$f \colon S^1 \mapsto S^1$ and $n \in \N$.  Then
$\exists k(n) \in \Z$ such that
\[
k - 1 < F^n(x) - x < k+1 \qquad \forall x \in \R.
\]
\end{lemma}

The proof of lemma \ref{lem:rotineq2} is left as an exercise.

\begin{proof}[Proof of theorem \ref{thm:rotoph}]
We first prove that $\Bar{\rho}(F,0)$ exists.  For $n$, $k \in \N$,

\begin{align*}
F^{n k}(0) = &\left( F^{nk}(0) - F^{n(k-1)}(0) \right) \\
&+ \left( F^{n(k-1)}(0) - F^{n(k-2)}(0) \right) \\
& \vdots \\
& + \left( F^{2 n}(0) - F^n(0)\right) \\
& + \left(F^n(0) - 0 \right). 
\end{align*}

Using lemma \ref{lem:rotineq1} with $x = 0$ and
$y = F^{n(m-1)}(0)$ for $m = 1,\dots, k$ gives the inequality
\[
k \left(F^n(0) - 1 \right) < F^{n k}(0) < k \left(
F^n(0) + 1 \right).
\]

Thus
\[
\abs{\frac{F^{n k}(0)}{n k} - \frac{F^n(0)}{n} } < \frac{1}{n}.
\]

However, we can exchange the r\^oles of $n$ and $k$ and using the triangle
inequality,
\[
\abs{\frac{F^k(0)}{k} - \frac{F^n(0)}{n}} < \frac{1}{k} + \frac{1}{n}.
\]

Hence the sequence $\left( \frac{F^n(0)}{n} \right)_{n \in \N}$ is
Cauchy and so converges to a limit $\Bar{\rho}(F,0)$.

Now by lemma \ref{lem:rotineq1} we have
\[
\frac{F^n(0) - 1}{n} < \frac{F^n(x) - x}{n} < \frac{F^n(0) + 1}{n}
\]
for all $x \in \R$ and so $\Bar{\rho}(F,x)$ exists and
equals $\Bar{\rho}(F,0)$ for all $x \in \R$.

Now assume that $F_1$ and $F_2$ are two lifts of $f$.  Then
$\exists k \in \Z$ such that $F_2(x) = F_1(x) + k$ and
so $F_2^n(x) = F_1^n(x) + nk$.  Therefore
\[
\lim_{n \to \infty} \frac{F_2^n(x) - x}{n}
= k + \lim_{n \to \infty} \frac{F_1^n(x) - x}{n} \qquad \text{as required.}
\]

We finally need to prove continuous dependence on $f$.  Let $F$
be a lift of $f$.  Lemma \ref{lem:rotineq2} implies that
given $n$, $\exists k$ such that $k - 1 < F^n(x) - x < k+1$ for
all $x \in \R$.  Given $\epsilon > 0$ choose $n \in \N$ such
that $\tfrac{2}{n} < \epsilon$.

For $g$ close enough to $f$ in the $C^0$ topology%
\footnote{$d(f,g) = \sup_{x \in S^1} \abs{f(x) - g(x)}$}
we can choose a lift $G$ of $g$ such that
\[
k-1 < G^n(x) - x < k+1 \qquad \forall x \text{ (same $k$, $n$ as for $F$).}
\]

Now

\[
F^{n l}(0) - 0 = \sum_{j=0}^{l-1} \left(
F^{n (j+1)}(0) - F^{nj}(0) \right)
\]

and so
\[
l (k-1) < F^{nl}(0)  < l(k+1).
\]

We can do the same thing for $G$, and we find that

\begin{gather*}
\frac{k-1}{n} \le \Bar{\rho}(F) \le \frac{k+1}{n} \\
\frac{k-1}{n} \le \Bar{\rho}(G) \le \frac{k+1}{n}
\end{gather*}

and thus $\abs{\Bar{\rho}(F) - \Bar{\rho}(G)} \le \frac{2}{n} < \epsilon$.
\end{proof}

The rotation number of an orientation preserving homeomorphism is
a topological invariant.

\begin{proposition}
Suppose $f \colon S^1 \mapsto S^1$ and $g \colon S^1 \mapsto S^1$ are
orientation preserving homeomorphisms and there exists an
orientation preserving homeomorphism $h$ such that
$h \circ g = f \circ h$.  Then $\rho(f) = \rho(g)$.
\end{proposition}

The proof is left as an exercise.

\section[Rational rotation number]%
{Orientation preserving homeomorphisms with rational rotation number}

\begin{proposition}\label{prop:s1oph}
Let $f$ be an orientation preserving homeomorphism of $S^1$.  Then
$\rho(f) \in \Q$ iff $f$ has a periodic point.  In fact,
$\rho(f) = \tfrac{p}{q}$ with $p$, $q \in \Z$ coprime and $q>0$ iff $f$
has a point of least period $q$.
\end{proposition}

\begin{proof}
$\Leftarrow$  Suppose $f$ has a periodic point $x_0$ with least period
$q$ and let $F$ be a lift of $f$.  Then $\exists k \in \Z$ such that
$F^q(x_0) = x_0 + k$.  Then $F^{nq}(x_0) - x_0 = nk$, so
\[
\rho(F) = \lim_{n \to \infty} \frac{F^{nq}(x_0) - x_0}{nq} = \frac{k}{q}.
\]

Thus $\rho(f) = \frac{k \pmod{q}}{q}$.

$\Rightarrow$  Assume $\rho(f) = \tfrac{p}{q}$ (in lowest terms).  Let
$\Bar{F}$ be a lift of $f$, so $\exists k \in \Z$ such that
$\rho(\Bar{F}) = k + \tfrac{p}{q}$.  Then $F(x) = \Bar{F} - k$ is another
lift of $f$ with $\rho(F) = \tfrac{p}{q}$.  Also,
\[
\rho(F^q -p) = \rho(F^q) - p = q \rho(F) - p = 0.
\]

Let $G(x) = F^q(x) - p$ --- it is enough to prove that $G$ has a fixed
point in $\R$.  We consider $G(0)$, and there are three cases.

\begin{enumerate}
\item $G(0)=0$ --- trivial.
\item $G(0) > 0$.  $G$ is increasing, so $0 < G(0) < \dots < G^n(0) < \dots$.
  This has two subcases:
  \begin{enumerate}
  \item $0 < G^n(0) < 1$ for all $n$.  We have an increasing sequence,
    bounded above. Thus $G^n(0)$ converges to a limit point which by the
    continuity of $G$ is a fixed point.
  \item $\exists k > 0$ such that $G^k(0) > 1$.  Then
    \[
    G^{2k}(0) = G^k(G^k(0)) > G^k(1) = G^k(0) + 1 > 2.
    \]
    By induction, $G^{jk}(0) > j$ and $\tfrac{G^{jk}(0)}{jk} > \tfrac{1}{k}$.
    Thus the rotation number is bounded away from zero and we have a
    contradiction.
  \end{enumerate}
\item $G(0) < 0$.  Similar reasoning applies.
\end{enumerate}
\end{proof}

We can now describe the dynamics of an orientation preserving
homeomorphism of $S^1$ with the following theorem.

\begin{theorem} Let $f \colon S^1 \mapsto S^1$ be an orientation
preserving homeomorphism with rational rotation number $\tfrac{p}{q}$
in lowest terms.  Then every orbit is either periodic of period $q$
or forward asymptotic to a period $q$ orbit and backward asymptotic
to a period $q$ orbit.
\end{theorem}

\begin{proof}
  $f^q$ can be identified with an orientation preserving homeomorphism
  of the closed interval by cutting $S^1$ at a fixed point of $f^q$.
  Then section \ref{sec:inthom} applies.
\end{proof}

The periodic orbits of an orientation preserving homeomorphism of $S^1$
are ordered on $S^1$ like those of a rigid rotation with the
same rotation number.  That is, if $y$ is a periodic point and
$\rho(f) = \tfrac{p}{q}$ that the ordering of
$(y,f(y),\dots,f^n(y),\dots)$ is the same as $(0,\tfrac{p}{q},\dots,
\tfrac{np}{q},\dots)$.

\section[Irrational rotation number]%
{Orientation preserving homeomorphisms with irrational rotation number}

\begin{theorem}\label{thm:irromega}
Assume $f \colon S^1 \mapsto S^1$ is an orientation preserving homeomorphism
and $\rho(f) \in \R \setminus \Q$.  Then
\begin{enumerate}
\item $\omega(x)$ is independent of $x$.
\item $E = \omega(x)$ is the unique minimal set of $f$.
\item $E$ is either $S^1$ or a Cantor subset of $S^1$.
\end{enumerate}
\end{theorem}

\begin{proof}\hfill
  \begin{enumerate}
  \item Take any two points $x$, $y \in S^1$.  We wish to show that
    $\omega(x) \subseteq \omega(y)$.  Since $S^1$ is compact, $\omega(x)$
    is non-empty and since $\rho(f) \in \R \setminus \Q$, all points
    in $O^+(x)$ are distinct.  Take $z \in \omega(x)$, so
    $\exists n_i \to \infty$ such that $\abs{f^{n_i}(x) - z} \to 0$.  Given
    $\epsilon>0$ we can find $n_k > n_j > 0$ such that
    \begin{gather*}
    \abs{f^{n_i}(x) - z} < \epsilon \qquad \text{for $i=k,j$, and} \\
    \abs{f^{n_k}(x) - f^{n_j}(x)} < \epsilon.
    \end{gather*}
    Let $I$ be the closed interval of length $< \epsilon$ with endpoints
    $f^{n_k}(x)$ and $f^{n_j}(x)$ and $N = n_k - n_j$.  Now
    $f^{N}(f^{n_j}(x)) = f^{n_k}(x)$ and $f^{-N}(f^{n_k}(x)) = f^{n_j}(x)$
    and so $f^{-N}(I) \cap I = \{ f^{n_j}(x)\}$.  Arguing similarly,
    $\left(f^{-mN}(I)\right)_{m \in \Z_+}$ is a sequence of closed intervals
    joined end to end.  Either $f^{-mN}(I)$ accumulates to some
    point $p$, which must by continuity satisfy $f^{-N}(p) = p$ --- a 
    fixed point, giving a contradiction (by proposition \ref{prop:s1oph})
    or they cover $S^1$.  Thus for all $y \in S^1$, $\exists l \in \Z_+$
    such that $y \in f^{-lN}(I)$ and so $f^{lN}(y) \in I$ and
    $\abs{f^{lN}(y) - z} < \epsilon$.  Thus $\omega(x) \subseteq w(y)$ and
    by symmetry $\omega(x) = \omega(y)$.

  \item $E$ is closed and invariant (by construction).  Let $A \subseteq S^1$
    be a non-empty, closed invariant set and $x \in A$.  Then $O^+(x)
    \subseteq A$ and so $E = \omega(x) \subseteq A$ as $A$ is closed.
    Thus any non-empty closed invariant subset of $S^1$ contains $E$ and
    $E$ is thus the unique minimal set.

  \item A \emph{Cantor subset} of $\R^n$ is a compact, totally disconnected
    set with no isolated points.  On $S^1$ we can replace ``totally
    disconnected'' with ``empty interior''.

    We know that $\emptyset$ and $E$ are the only closed invariant subsets
    of $E$ and as the boundary $\partial E$ is a closed invariant subset
    of $E$ (exercise), $\partial E = \emptyset$ (and
    $E = S^1$) or $\partial E = E$.

    If $\partial E = E$ then $E$ has an empty interior.  It remains
    to show that $E$ has no isolated point.  Take $x \in E$.  Since
    $E = \omega(x)$, $\exists k_n \to \infty$ such that
    $\lim_{n \to \infty} f^{k_n}(x) = x$.  As $f$ has no periodic point
    $f^{k_n}(x) \neq x$ for all $n$, and so $x$ is an accumulation point
    of $E$ as $f^{k_n}(x) \in E$ by invariance.

  \end{enumerate}
\end{proof}

We have seen examples of maps with $E=S^1$ ($r_\beta$, $\beta \in \R \setminus
\Q$), but do maps exist with $E \neq S^1$?

\begin{theorem}
Assume $f \colon S^1 \mapsto S^1$ is a $C^2$ diffeomorphism and
$\beta = \rho(f) \in \R \setminus \Q$.  Then $f$ is topologically conjugate
to the rotation $r_\beta$.
\end{theorem}

It is actually sufficient to have $f'$ of bounded variation.  In
any case the proof is technical and omitted.  In this case, $E=S^1$
and all orbits are dense in $S^1$.

\begin{proposition}
Let $\beta \in \R \setminus \Q$.  Then there exists a
$C^1$ orientation preserving diffeomorphism $f$ of $S^1$ such that
$\rho(f) = \beta$ and $E \neq S^1$.
\end{proposition}

\begin{proof}[Sketch proof]
We want to find a map with an orbit that is not dense in $S^1$,
so that $E \neq S^1$.  The idea is to start from the rigid rotation
$r_\beta$, $\beta \in \R \setminus \Q$ and to choose an orbit $(x_n)_{n \in
\Z}$ of $r_\beta$ and ``blow it up'' to an orbit of closed intervals
$(I_n)_{n \in \Z}$ with lengths $l_n$ such that $\sum_{n \in \Z} l_n
< \infty$ to obtain a map on a new circle $S^1{}'$.

\vspace{1in}

We extend $r_\beta$ to a map $f \colon S^1{}' \mapsto S^1{}'$ by
choosing for each $n \in \Z$ an orientation preserving homeomorphism
mapping $I_n$ onto $I_{n+1}$.

If we choose $f \colon I_n \mapsto I_{n+1}$ to be affine then we create
a $C^0$ map, but to make $f$ $C^1$ we need $f' = 1$ at the endpoints of
each $I_n$ and $\max_{x \in I_n} \abs{f'(x) - 1} \to 0$ as $n \to \infty$.

We see that $f$ has the same rotation number as $r_\beta$ and that no
point in $I_n$ ever returns to $I_n$ under iteration.  So if
$p \in \inter I_n$ then $f^m(p) \notin \inter I_n$ for $m \neq 0$ and
so $O(p)$ is not dense in $S^1{}'$.

Thus $E$ is not the whole of $S^1{}'$ and so by theorem \ref{thm:irromega}
is a Cantor subset of $S^1{}'$.
\end{proof}

In fact
\[
E = S^1{}' \setminus \bigcup_{n \in \Z} \inter I_n
\]
and the open sets $\inter I_n$ are the gaps in the Cantor set.

$E$ is nowhere dense since $\bigcup_{n \in \Z} \inter I_n$ is dense
in $S^1{}'$.

\begin{definition}
An orbit $O(x)$ is said to be homoclinic to an invariant set $S \subset
S^1 \setminus O(x)$ if $\alpha(x) = \omega(x) = S$.
\end{definition}

\begin{theorem}
  Let $f \colon S^1 \mapsto S^1$ be an orientation preserving
  homeomorphism with $\rho(f) \in \R \setminus \Q$.  Then every orbit
  is either:

\begin{enumerate}
\item dense in $S^1$,
\item dense in a Cantor set or
\item homoclinic to a Cantor set.
\end{enumerate}
\end{theorem}

\section{Families of circle maps}

\begin{proposition}
  Let $f \colon S^1 \mapsto S^1$ be an orientation preserving
  homeomorphism with lift $F$ such that $\rho(F) = \rho(f) =
  \tfrac{p}{q} \in \Q$ and suppose that the graph of $F^q - p$ has
  points on either side of the diagonal.  Then all small enough
  perturbations of $f$ have rotation number $\tfrac{p}{q}$.
\end{proposition}

This phenomenon is called \emph{frequency locking}.

\begin{proof}
  Now $\exists x_0$ such that $F^q(x_0) - x_0 - p > 0$ and $x_1$ such
  that $F^q(x_1) - x_1 - p < 0$.  Then for all small enough
  perturbations $\Bar{f}$ of $f$ with corresponding lift $\Bar{F}$,
  $\Bar{F}^q(x_0) - x_0 - p > 0$ and $\Bar{F}^q(x_1) - x_1 - p < 0$.
  These inequalities give $\tfrac{p}{q} \le \rho(\Bar{F}) \le
  \tfrac{p}{q}$ and the result is thus true.
\end{proof}

\subsection{Monotonicity of rotation number}

Let $F_1$ and $F_2$ be lifts of $f_1$ and $f_2$.  If $F_1(x) < F_2(x)$
for all $x$ then $\rho(F_1) \le \rho(F_2)$.  (Immediate.)

At irrational values the rotation number strictly increases.

\begin{proposition}
  Let $F_1$ and $F_2$ be lifts of the orientation preserving
  homeomorphisms of $S^1$, $f_1$ and $f_2$.  If $F_1(x) < F_2(x)$ for all
  $x \in \R$ then $\rho(F_1) < \rho(F_2)$.
\end{proposition}

\begin{proof}
  By continuity and periodicity, $F_2(x) - F_1(x) > \delta > 0$ for all
  $x \in \R$.  Take $\tfrac{p}{q} \in \Q$ such that
\[
\frac{p}{q} - \frac{\delta}{q} < \rho(F_1) < \frac{p}{q}.
\]

Then $\exists x_0$ such that $F_1^q(x_0) - x_0 > p - \delta$, because
otherwise

\[
\rho(F_1) = \lim_{n \to \infty} \frac{F_1^{nq}(x) - x}{n q} \le
\lim_{n \to \infty} \frac{n (p-\delta)}{n q} = \frac{p}{q} -
\frac{\delta}{q}.
\]

Now
\[
F_2^q(x_0) = F_2(F_2^{q-1}(x_0)) > F_1(F_2^{q-1}(x_0)) + \delta
> F_1^q(x_0) + \delta > x_0 + p
\]
and so $\rho(F_2) \ge \tfrac{p}{q} > \rho(F_1)$.
\end{proof}

\subsection{The Arnold family}

This is a 2 parameter family of circle maps $f_{k,\omega} \colon S^1
\mapsto S^1$ with lifts
\[
F_{k,\omega} \colon x \mapsto x + \omega + \frac{k}{2 \pi} \sin 2 \pi x.
\]

\begin{itemize}
\item $k=0$ : rigid rotation
\item $0 \le k < 1$ : diffeomorphism
\item $k = 1$ : homeomorphism
\item $k > 1$ : not invertible.
\end{itemize}

We consider $k,\omega \in [0,1]$.  First, fix $k$ and vary $\omega$.
If $\omega_1 < \omega_2$ then $F_{k,\omega_1}(x) < F_{k,\omega_2}(x)$
and so $\rho(F_{k,\omega_1}) \le \rho(F_{k,\omega_2})$.  Hence $\rho$
is a non-decreasing function of $\omega$ for fixed $k$.  It is also
continuous (by theorem \ref{thm:rotoph}).

\begin{definition}
  A monotone continuous function $\phi \colon [0,1] \mapsto \R$ is
  called a \emph{devil's staircase} if there exists a collection $\{
  I_\alpha \}_{\alpha \in A}$ of disjoint closed intervals $[0,1]$
  with dense union such that $\phi$ takes distinct constant values on
  these intervals.
\end{definition}

\begin{proposition}
  For $k \in [0,1]$, $\phi \colon \omega \mapsto \rho(f_{k,\omega})$
  is a devil's staircase.  $\phi^{-1}\left( \tfrac{p}{q}\right)$
  is one of the intervals $I_\alpha$ for each rational $\tfrac{p}{q}
  \in [0,1]$.
\end{proposition}

The proof is left as an exercise.
\vspace{2in}

In $(k,\omega)$ parameter space, $\rho(F_{k,\omega}) = 0$ iff
$\exists x$ such that $F_{k,\omega}(x)=x$.  Thus $\sin 2 \pi x
= \tfrac{2 \pi \omega}{k}$ and there exist fixed points if
$\tfrac{2 \pi \omega}{k} \le 1$.

Similarly, $\rho(F_{k,\omega}) = 1$ iff $\tfrac{2 \pi(k-\omega)}{k} \le 1$.
Regions in parameter space where $\rho$ is rational are called
\emph{Arnold tongues}.

\vspace{2in}

We want to know what happens on the boundary of Arnold tongues.

\section{Stability, persistence and bifurcations}

\begin{definition}
In one dimension, a fixed point $x^\ast$ of a differentiable map
$f$ is \emph{hyperbolic} if $\abs{f'(x^\ast)} \neq 1$.  The fixed
point is \emph{stable/attracting/a source} if $\abs{f'(x^\ast)} < 1$.
It is \emph{unstable/repelling/a sink} if $\abs{f'(x^\ast)}>1$.

A period $q$ orbit (cycle) $\{ x_0, x_1,\dots, x_{q-1} \}$ with $x_i =
f^{i}(x_0)$ is stable if $y$ is stable as a fixed point of $f^q$ for
$y$ in the cycle.  It is unstable if $y$ is unstable as a fixed point
of $f^q$ for $y$ in the cycle.
\end{definition}

Note that the $y$ in the cycle used does not matter, as
\[
(f^q)'(y) = \prod_{i=0}^{q-1} f'\left(f^i(y)\right)
= \prod_{i=0}^{q-1} f'(x_i) \qquad \text{by the chain rule.}
\]

\subsection{Persistence and bifurcation}

Consider a 1 parameter family of maps $f \colon \R \times \R$
(or $S^1 \times \R$) $\mapsto \R$, $(x,\lambda) \mapsto f(x,\lambda)
:= f_\lambda(x)$.

We need the implicit function theorem, stated here in a weakened form
(and not proved).

\begin{theorem}\label{thm:IFT}
Let $G \colon R \mapsto \R$ be $C^r$, $r \ge 1$, where
\[
R := \left\{ (x,y) : a < x < b, c < y < d \right\},
\]
with $(x_0,y_0) \in R$.  If $G(x_0,y_0) = 0$ and
$\left.\pd{G}{y}\right|_{(x_0,y_0)} \neq 0$ then there exist open
intervals $I \ni x_0$ and $J \ni y_0$ with $I \times J \subset R$
and a $C^r$ function $p \colon I \mapsto J$ such that $G(x,y) = 0$
if $I \times J$ iff $y=p(x)$.
\end{theorem}

We can now state conditions for the persistence of fixed points.

\begin{theorem}\label{thm:fpp}
Assume $f_\lambda$ is $C^r$, $r \ge 1$ and $f_{\lambda^\ast}(x^\ast)
= x^\ast$ and $f'_{\lambda^\ast}(x^\ast) \neq 1$.  Then there exist
open $I \ni \lambda^\ast$ and $J \ni x^\ast$ and a $C^r$ function
$p \colon I \mapsto J$ such that $p(\lambda^\ast) = x^\ast$ and
$f_\lambda(p(\lambda)) = p(\lambda)$.  Moreover, $f_\lambda$ has no
other fixed points in $I$.
\end{theorem}

\begin{proof}
  Apply the implicit function theorem to $G(x,\lambda) = f_\lambda(x)
  - x$.  Our hypotheses give $G(x^\ast,\lambda^\ast) = 0$ and
  $\left.\pd{G}{x}\right|_{(x^\ast,\lambda^\ast)} \neq 0$.
  
  Thus by the IFT $\exists I \ni \lambda^\ast$ and $J \ni x^\ast$ and
  a $C^r$ function $p \colon I \mapsto J$ such that $p(\lambda^\ast) =
  x^\ast$, $G(p(\lambda),\lambda) = 0$ and $G \neq 0$ in
  $J \times I$ unless $x = p(\lambda)$.
\end{proof}

We have a curve (or ``branch'') of fixed points.

\vspace{1.5in}

We can find $\diff{p}{\lambda}$ by implicit differentiation:
\[
\diff{p}{\lambda} = - \left.\frac{\pd{G}{\lambda}}{\pd{G}{x}}
\right|_{(p(\lambda),\lambda)} = -
\frac{\pd{f_\lambda}{\lambda}(p(\lambda))}{f'_\lambda(p(\lambda)) - 1}.
\]

Intuitively, a bifurcation takes place at $(x_0,\lambda)$ when the
topological nature of the dynamics near $x_0$ changes when $\lambda$ passes
through $\lambda_0$.

Fixed points (dis)appear in a saddle-node/tangent/fold bifurcation.

\begin{theorem}\label{thm:snb}
  Assume $f_\lambda$ is $C^r$ with $r \ge 2$, $f_\lambda(x_0) = x_0$,
  $f_{\lambda_0}'(x_0) = 1$, $f_{\lambda_0}''(x_0) \neq 0$ and
  $\left.\pd{f_\lambda}{\lambda}\right|_{(x_0,\lambda_0)} \neq 0$.
  Then $\exists J \ni x_0$, $I \ni \lambda_0$ and a $C^r$ function $x
  \mapsto g(x)$ such that $g(x_0) = \lambda_0$ and such that
  $f_\lambda(x) = x$ in $J \times I$ iff $\lambda = g(x)$.  Moreover
  $g'(x_0) =0$ and $g''(x_0) \neq 0$.  The fixed points created are
  attracting on one side of $x_0$ and repelling on the other.
\end{theorem}

\begin{proof}
  Let $G(x,\lambda) = f_\lambda(x) - x$. Now $G(x,\lambda) = 0$ iff
  $x$ is a fixed point of $f_\lambda$.
  $\left.\pd{G}{x}\right|_{(x_0,\lambda_0)} = 0$ and
  $\left.\pd{G}{\lambda}\right|_{(x_0,\lambda_0)} \neq 0$, and so, by
  the IFT, there exists a $C^r$ function $g \colon J \mapsto I$
  satisfying $G(x,g(x)) = 0$.
  
  Now $0 = \pd{G}{x} + \pd{G}{\lambda}\pd{g}{x}$ and so $g' = -
  \pd{G}{x} \left( \pd{G}{\lambda} \right)^{-1}$.  In particular,
    $g'(x_0) = 0$.  Differentiating again we get that

\[
g''(x_0) = - \left[\pd{^2 G}{x^2} \left( \pd{G}{\lambda} \right)^{-1}
\right]_{(x_0,\lambda_0)} \neq 0.
\]

The sign of $g''$ determines the direction of the bifurcation.  As for
stability,
\begin{align*}
\pd{f_\lambda}{x} &= 1 + \pd{^2 f_\lambda}{x^2} (x-x_0)
+\pd{^2 f}{x \partial \lambda} (\lambda - \lambda_0) + \text{higher order}\\
&= 1 + \pd{^2 f_\lambda}{x^2} (x-x_0)
\end{align*}
as $g(x)-\lambda_0 = \cO(x-x_0)^2$.  Thus $f'_{g(x)}(x) - 1$
takes opposite signs on either side of $x_0$.
\end{proof}

\newpage

\subsubsection*{Bifurcation diagrams}

\vspace{3in}

As an example, consider the Arnold family of maps,
$f_{k,\omega}(x) = x + \omega + \tfrac{k}{2\pi} \sin 2 \pi x$.

\vspace{2in}

We can generalise these results to periodic orbits of period $q > 1$ by
applying theorems \ref{thm:fpp} and \ref{thm:snb} to $f^q_\lambda$.

An an example consider the Arnold tongue about $\omega = \tfrac{p}{q}$.

\vspace{2in}

The boundaries of the Arnold tongues are lines of saddle-node bifurcation.

\chapter[Chaos and non-invertible maps]%
{Chaos and non-invertible one dimensional maps}

\section{Chaos}

We start with an example.  Consider $f \colon S^1 \mapsto S^1$
given by $f\colon z \mapsto z^2$ in multiplicative notation or $f \colon x
\mapsto 2 x \mod{1}$ in additive notation.

\begin{center}
\begin{pspicture}(-20,-20)(500,500)
\psline{->}(-10,0)(500,0)
\psline{->}(0,-10)(0,500)
\psline(0,0)(500,500)
\psline(0,0)(250,500)
\psline(250,0)(500,500)
\psline[linestyle=dashed](250,0)(250,500)
\rput(100,400){$\deg f = 2$}
\end{pspicture}
\end{center}

We will write $x$ as a binary expansion,
\[
x = \sum_{i=1}^\infty \frac{a_i}{2^i},
\]

with $a_i \in \{0,1\}$.

Note that dyadic rationals $\tfrac{m}{2^n}$ have 2 expansions (although
this will not bother us).  For instance,
\[
\tfrac{1}{2} = 0.1000\ldots = 0.0111\ldots.
\]

It is easy to see what $f(x)$ is.  Using the binary expansion,
\[
f(x) = \sum_{i=1}^\infty \frac{a_{i+1}}{2^i}.
\]

We can write down a list of properties of $f$.

\begin{enumerate}
\item If $x \in \Q$, $x$ is either a periodic
point or eventually periodic.

\begin{definition}
A point $x$ is eventually periodic of period $n$ if $x$ is not periodic,
but $\exists k > 0$ such that $f^k(x)$ is periodic of period $n$.
\end{definition}

\item If $x$ is irrational, $x$ is neither periodic nor eventually periodic
(as the binary expansion never repeats itself).

\item Let $P_n(f)$ be the number of periodic points of $f$ with
(not necessarily least) period $n$.  We can show that $P_n(f) = 2^n - 1$.
($P_n(f)$ is also the number of fixed points of $f^n$.  We require
$z^{2^n} = z$, or $z^{2^n - 1} = 1$.  There are $2^n-1$ such.)

\item If $x \neq y$ then there exists $n \ge 0$ such that
$\abs{f^n(x) - f^n(y)} \ge \tfrac{1}{4}$.

\begin{proof}
Take $x > y$.  If $x-y \ge \tfrac{1}{4}$ then we are done, else
$\exists n \ge 1$ such that
\[
\frac{1}{2^{n+2}} < x - y < \frac{1}{2^{n+1}}.
\]

Now $\abs{f(x) - f(y)} = 2 \abs{x-y}$ and so
\[
\frac{1}{4} \le \abs{f^n(x) - f^n(y)} \le \frac{1}{2}.
\]
\end{proof}

\item For every open interval $J \subset S^1$, $\exists n \ge 0$ such that
$f^n(J) = S^1$. (This follows from property 4.)

\item Periodic points and eventually periodic points are dense in $S^1$.
\end{enumerate}

Some of these properties are specific to this example, but it
has two properties which are of more general interest.

\begin{definition}
A map $f \colon X \mapsto X$ is said to be \emph{topologically transitive}
on an invariant set $\Lambda \subset X$ if the forward orbit of some point
$x \in \Lambda$ is dense in $\Lambda$.
\end{definition}

An equivalent (for most ``reasonable'' topological spaces) definition is:

\begin{definition}
  A map $f \colon X \mapsto X$ ($X$ a topological space) is said to be
  \emph{topologically transitive} on an invariant set $\Lambda \subset
  X$ if for any pair of open sets $U$, $V \subset \Lambda$ $\exists k
  > 0$ such that $f^k(U) \subset V \neq \emptyset$.
\end{definition}

The other interesting property is sensitive dependence on initial conditions
(SDIC).

\begin{definition}
  A map $f \colon X \mapsto X$ ($X$ a metric space) has SDIC on an
  invariant set $\Lambda$ if $\exists \delta > 0$ such that for all $x
  \in \Lambda$ and any neighbourhood $U$ of $x$, $\exists y \in U$ and
  $n > 0$ such that $d(f^n(x),f^n(y))>\delta$.
\end{definition}

\begin{definition}
  A dynamical system $(f,X)$ is \emph{chaotic} if it has a compact
  invariant subset $\Lambda$ on which $f$ is both topologically
  transitive and has SDIC.
\end{definition}

For another example take $X = [-1,1]$ and $g \colon X \mapsto X$ such
that $g(x) = 2x^2 - 1$.  Then $g$ is chaotic.\label{page:2x2m1}

\begin{proof}
Consider $h \colon S^1 \mapsto X$, $h(\theta) = \cos 2 \pi \theta$.
$h$ is continuous and onto. Now
\[
h(f(\theta)) = h(\cos 2 \theta) = \cos 4 \pi \theta
= 2 \cos^2 2 \pi \theta -1 = g(h(\theta)). 
\]

$h$ is \emph{not} one to one, but we don't need that.  We need to prove
both topological transitivity and SDIC.  Given two open sets $I$, $J
\subset X$.  Now $h^{-1}(I)$ and $h^{-1}(J)$ are open in $S^1$ since
$h$ is continuous.  Then $\exists n > 0$ such that
\[
f^n(h^{-1}(I)) \cap h^{-1}(J) \neq \emptyset,
\]
and so $g^n(I) \cap J \neq \emptyset$.  To prove SDIC, given $x \in X$
and open $U \ni x$, $\exists n > 0$ such that $g^n(U) = X$ (as the same is
true for $f$).  Now let $y = 1$ if $g^{n}(x) \le 0$ and $-1$ if
$g^n(x) > 0$.  $\exists z \in U$ such that $g^n(x) = y$ and so
\[
\abs{g^n(x) - g^n(y)} \ge 1.
\]
\end{proof}

This proof has introduced an important notion.

\begin{definition}
Let $f \colon X \mapsto X$ and $g \colon Y \mapsto Y$.  Then $h \colon X
\mapsto Y$ is called a \emph{topological semi-conjugacy} from $f$ to $g$ if
\begin{enumerate}
\item $h$ is continuous,
\item $h$ is onto,
\item $h \circ f = g \circ h$.
\end{enumerate}

We say that $f$ is topologically semi-conjugate to $g$ by $h$.
\end{definition}

Semi-conjugacy means that the dynamics of $f$ are at least as complicated
as the dynamics of $g$.

\section{Sequence spaces}

\renewcommand{\a}{\vect{a}}
\renewcommand{\b}{\vect{b}}

Let
\[
\Sigma_N = \left\{ \a = (a_0,a_1,\dots) : a_i \in \left\{ 0,\dots,N-1
\right\}, i \in \Z_+\right\},
\]

a \emph{sequence space on $n$ symbols}.  We make $\Sigma_N$ a metric
space by defining a distance
\[
d(\a, \b) = \sum_{n=0}^\infty \frac{\gamma(a_n,b_n)}{3^n} \quad
\text{where } \gamma(i,j) = \begin{cases}
0 & i=j \\
1 & i \neq j.
\end{cases}
\]

Two points in $\Sigma_N$ are close if they agree on a long initial
segment, as follows.  Suppose $\a$, $\b \in \Sigma_N$ with
$a_i = b_i$ for $i < m$ and $a_m \neq b_m$.  Then
\[
\frac{1}{3^m} \le d(\a,\b) \le \sum_{n=m}^\infty \frac{1}{3^n}
= \frac{2}{3^{m-1}}.
\]

$\Sigma_N$ is a Cantor set.\footnote{See example sheet.}

\section{Shift map}

We define $\sigma \colon \Sigma_N \mapsto \Sigma_N$ by
\[
\sigma(a_0,a_1,\dots) = (a_1,a_2,\dots).
\]

\begin{proposition}\hfill
  \begin{enumerate}
  \item $\sigma$ is continuous
  \item $P_k(\sigma) = N^k$
  \item $\Per(\sigma)$ (the set of periodic points of $\sigma$) is
    dense in $\Sigma_N$.
  \item There exists a dense (forward) orbit in $\Sigma_N$.
  \item $\sigma$ has SDIC.
  \end{enumerate}
\end{proposition}

\begin{proof}\hfill
  \begin{enumerate}
  \item $d(\a,\b) = \gamma(a_0,b_0) + \tfrac{1}{3}
    d(\sigma(\a),\sigma(\b))$ and so $d(\sigma(\a),\sigma(\b)) \le
    3 d(\a,\b)$.  Given the usual $\epsilon$, pick $N$ such that
    $3^{-N} < \epsilon$, and then $\delta = 3^{-(N+1)}$.
  \item $\sigma^k(\a) = \a$ iff $a_{k+j} = a_j$ for all $j \ge 0$.  Given
    $k$ there are $N^k$ blocks of length $k$.
  \item Given $\a \in \Sigma_N$ and $\epsilon > 0$ take $n$ such that
    $\tfrac{1}{2 \cdot 3^{n-1}} < \epsilon$ and let $\b$ be a periodic
    sequence of the form $(a_0,a_1,\dots,a_n,a_0,a_1,\dots)$.  Then
    $d(\a,\b) \le \epsilon$.
  \item Let $\b$ be a sequence which lists all blocks of length $n$ for each
    successive $n \in \N$.  For instance, for $N = 2$,
    \[
    \b = (0\ 1\ 00\ 01\ 10\ 11\ 000\ \dots).
    \]
    Then given $\a \in \Sigma_N$ and $k \in \N$, $\exists n \in \N$ such
    that $\sigma^n(\b)$ and $\a$ agree on the first $k$ places, and so
    $d(\sigma^n(\b),\a) \le \tfrac{1}{2 \cdot 3^{k-1}}$.
  \item Given $\a \in \Sigma$, choose $\b \in \Sigma$ such that
    $a_i = b_i$ for $i=0,\dots,q$ but $a_i \neq b_i$ for $i > q$.  Then
    \[
    d(\a,\b) < \frac{1}{3^q} \quad \text{but} \quad d(\sigma^q(\a),
    \sigma^q(\b)) = \frac{1}{2}.
    \]
  \end{enumerate}
\end{proof}

Properties 4 and 5 mean that $\sigma$ is chaotic on $\Sigma_N$.

As an example consider the map shown

\vspace{1.5in}

Let
\[
\Lambda = \left\{x : f^n(x) \in I_0 \cup I_2\ \forall n \ge 0 \right\}.
\]

Then $\Lambda$ is the middle third Cantor set,
\[
\Lambda = \left\{ x \in [0,1] : x = \sum_{n=0}^\infty \frac{a_i}{3^{n+1}},
a_i \in \{0,2\}\right\}.
\]

We claim that $\left.f\right|_\Lambda$ is topologically conjugate to
$\sigma$ on $\Sigma_2$, which we prove by exhibiting the conjugacy,
\[
h \colon \sum_{n=0}^\infty \frac{a_n}{3^{n+1}} \mapsto (\tfrac{a_0}{2},
\tfrac{a_1}{2},\dots).
\]

Thus $f$ is chaotic on $\Lambda$.

\section{Subshifts of finite type}

The general setting is with $f \colon I \mapsto I$ continuous,
$I \subset \R$ or $I \subset S^1$.

Suppose that $\{I_0,I_1,\dots,I_{N-1}\}$ are disjoint closed intervals in
$I$.  We say that $I_i$ $f$-covers $I_j$ (and write $I_i \to I_j$
or $i \to j$) if $I_j \subset f(I_i)$.

Let $\Gamma$ be the directed graph with $N$ vertices indicating the
$f$-covering relations.

\vspace{1.5in}

We see that $I_1 \to I_2$, $I_2 \to I_1$ and $I_2 \to I_2$.  The graph
for this is $\xymatrix{1\ar@<1ex>[r] & \ar@<1ex>[l] 2 \ar @(ur,dr)}$

Let $\cA$ be the $N \times N$ matrix defined by
\[
\cA_{ij} = \begin{cases}
1 & \text{if } i \to j \\
0 & \text{otherwise.}
\end{cases}
\]

$\cA$ is called the \emph{transition matrix} associated to
$\{ I_0, \dots,I_{N-1}\}$.  For the example above
\[
\cA = \begin{pmatrix} 0 & 1 \\ 1 & 1 \end{pmatrix}.
\]

Let $\Sigma_{N,A} = \{ \vect{a} \in \Sigma_N : \cA_{a_n a_{n+1}} = 1
\ \forall n \ge 0 \}$.  Note that $\Sigma_{N,A}$ is closed and invariant
under $\sigma$.

\begin{definition}
The restriction $\sigma_A = \left.\sigma\right|_A$ of $\sigma$ to
$\Sigma_{N,A}$ is called a \emph{subshift of finite type} or
\emph{topological Markov chain}.
\end{definition}

\begin{theorem}\label{thm:tcMk}
Let $J = \bigcup_{i=0}^{n-1} I_{a_i}$.  Then there exists a
closed $f$-invariant set $\Lambda \subset J$ such that $\left.f\right|_\Lambda$
is topologically semi-conjugate to $\sigma_A$.
\end{theorem}

Before proving this theorem we need two lemmas.

\begin{lemma}\label{lem:lmk}
If $L$ and $M$ are closed intervals and $L \to M$ then there exists a
closed interval $K \subset L$ such that $f(K) = M$.
\end{lemma}

\begin{proof}
Let $M = [a,b]$.  Then $f^{-1}(a)$ and $f^{-1}(b)$ are closed and non-empty
so we can choose $u \in f^{-1}(a)$ and $v \in f^{-1}(b)$ such that
\[
(u,v) \cap \left( f^{-1}(a) \cup f^{-1}(b) \right) = \emptyset.
\]

WLOG $u < v$.  Set $K = [u,v]$ and use the IVT.
\end{proof}

\begin{lemma}\label{lem:nint}
If $I_{a_0} \to I_{a_1} \to \dots \to I_{a_n}$ then $\bigcup_{i=0}^n
f^{-i}(I_{a_i})$ contains an interval $I_{a_0 a_1 \dots a_n}$ such that
$f^n( I_{a_0 a_1 \dots a_n} ) = I_{a_n}$ and $x \in I_{a_0 a_1 \dots a_n}$
implies that $f^i(x) \in I_{a_i}$ for all $0 \le i \le n$.
\end{lemma}

\begin{proof}
By lemma \ref{lem:lmk} there exists an interval $I_{a_0 a_1} \subset I_{a_0}$
such that $f(I_{a_0 a_1}) = I_{a_0}$.  Now $\exists I_{a_1 a_2} \subset
I_{a_2}$ such that $f(I_{a_1 a_2}) = I_{a_2}$.  Therefore
$\exists I_{a_0 a_1 a_2} \subset I_{a_0 a_1}$ such that
$f(I_{a_0 a_1 a_2}) = I_{a_1 a_2}$ and $f^2(I_{a_0 a_1 a_2}) = I_{a_2}$.
We continue inductively to get $I_{a_0 a_1 \dots a_n}$.
\end{proof}

\begin{proof}[Proof of Theorem \ref{thm:tcMk}]
We first obtain
\[
\Lambda_1 = \left\{ I_{a_0 a_1} \subset I_{a_0} : a_0 \to a_1 \text{ and }
f(I_{a_0 a_1}) = I_{a_1} \right\}.
\]

We define $\Lambda_n$ inductively, assuming that
\[
\Lambda_{n-1} = \left\{ I_{a_0 a_1 \dots a_{n-1}}
: a_0 \to a_1 \to \dots \to a_{n-1} \text{ and }
f^n(I_{a_0 \dots a_{n-1}}) = I_{a_n} \right\}.
\]

For all allowed transitions $a_{n-1} \to a_n$ define
$I_{a_0 \dots a_n} \subset I_{a_0 \dots a_{n-1}}$ such that
$f^{n-1}(I_{a_0 \dots a_n}) = I_{a_{n-1} a_n} \subset \Lambda_1$.
Thus
\[
\Lambda_{n} = \left\{ I_{a_0 a_1 \dots a_n}
: a_0 \to a_1 \to \dots \to a_n \text{ and }
f^n(I_{a_0 \dots a_n}) = I_{a_n} \right\}.
\]

Now define $\Lambda = \bigcup_{n \ge 1} \Lambda_n$.  $\Lambda$ is
non-empty (as it is the intersection of a nested sequence of closed sets).
The connected components of $\Lambda$ are closed intervals or possibly
isolated points.

We can now define $h\colon \Lambda \to \Sigma_{N,A}$ (the
\emph{itinerary map}) as
\[
x \to \vect{a} \text{ where } f^i(x) \in I_{a_i}\ \forall i \in \Z_+.
\]

$h$ is continuous, as given $M$, $\exists \delta > 0$ such that $x,y
\in \Lambda$ with $d(x,y) < \delta$ implies that
$f^i(x)$ and $f^i(y)$ are in the same $I_{a_i}$ for $0 \le i \le M$.

By construction, $h$ is surjective and $h \circ \left.f\right|_\Lambda
= \sigma_A \circ h$.
\end{proof}

If $f$ is monotone on each $I_i$ then $\Lambda$ is uniquely defined.

If $f$ is expanding on $J$ ($\exists \lambda > 1$ such that
for all $i$, $x,y \in I_i$, $d(f(x),f(y)) \ge \lambda d(x,y)$)
then $h$ is a topological conjugacy.

\subsection{Properties of $\sigma_A$}

We call a finite string $\vect w = w_0 w_1 \dots w_k$ a \emph{word}.
Given a transition matrix $\cA$ a word is said to be \emph{allowed} if
the transition $w_i \to w_{i+1}$ is allowed for all $0 \le i \le k-1$,
equivalently
\[
\cA_{w_0 w_1} \cA_{w_1 w_2} \dots \cA_{w_{k-1} w_k} = 1.
\]

\begin{lemma}\label{lem:wordcount}
  Let $N_{ij}^{(n)}$ denote the number of allowed words of length
  $n+1$ starting in $i$ and ending in $j$.  Then
\[
N_{ij}^{(n)} = \left( \cA^n \right)_{ij}.
\]
\end{lemma}

\begin{proof}
The product $\cA_{i w_1} \cA_{w_1 w_2} \dots \cA_{w_{n-1} j}$ equals one
if $i w_1 \dots w_{n-1} j$ is allowed and equals zero otherwise.  Then
\[
N_{ij}^{(n)} = \sum_{w_1,\dots,w_{n-1}}
\cA_{i w_1} \cA_{w_1 w_2} \dots \cA_{w_{n-1} j} = \left( \cA^n \right)_{ij}.
\]
\end{proof}

\begin{proposition}
$P_n(\sigma_A) = \Tr \cA^n$.
\end{proposition}

\begin{proof}
Fixed points of $\sigma_A^n$ are in one-to-one correspondance with
allowed words of length $n+1$ with the same start and finish.  Now
use lemma \ref{lem:wordcount}.
\end{proof}

We can compute $\Tr \cA^n$ by using the Cayley-Hamilton theorem, that
a matrix satisfies its own characteristic equation.  For instance, for
the $\cA$ we considered earlier, we find
$\Tr \cA^{n+2} - \Tr \cA^{n+1} - \Tr \cA^n = 0$, and imposing the
initial conditions $\Tr \cA = 1$ and $\Tr \cA^2 = 3$ we can find $\Tr \cA^n$.

If $N_q$ is the number of periodic cycles of least period $n$, then
\[
P_n = \sum_{q | n} q N_q.
\]

\subsection{Chaos}

\begin{definition}
A matrix $\cA$ is \emph{irreducible} if $\forall i,j$, $\exists n$ such
that $\left(\cA^n\right)_{ij} \neq 0$ (that is there is an allowed path
from $i$ to $j$).
\end{definition}

\begin{proposition}\label{prop:ttirred}
If $\cA$ is irreducible then $\sigma_A$ is topologically transitive.
\end{proposition}

\begin{proof}
  We need to find a dense orbit.  We can do this by choosing a
  sequence which contains every allowed word, with a proper choice of
  transition words between them.  (This is always doable as $\cA$ is
  irreducible.)
\end{proof}

\begin{definition}
  A matrix $\cA$ is \emph{non-trivial} if $\exists i, j_1 \neq j_2$
  such that $i \to j_1$ and $i \to j_2$.
\end{definition}

A permutation matrix (for instance) is trivial.

\begin{proposition}
If $\cA$ is irreducible and non-trivial then $\sigma_A$ is chaotic on
$\Sigma_A$.
\end{proposition}

\begin{proof}
By proposition \ref{prop:ttirred} we just need to show SDIC.  Given
$\vect a = a_0 a_1 \dots$ and $M \in \Z_+$ there exists an allowable
word $a_M w_{M+1}\dots (w_k=i)$ where $i$ can be followed by either
$j_1$ or $j_2$ since $\cA$ is non-trivial.  If
$w_{M+1}\dots w_k = a_{M+1}\dots a_k$ then choose
$\vect{b} = a_0 a_1\dots b_{k+1}\dots$ where $b_{k+1} \neq a_{k+1}$ and
let $n=k$.

If $w_{M+1}\dots w_k \neq a_{M+1} \dots a_k$ then choose $n$ to be the
index of the first non-agreeing character and
$\vect{b} = a_0\dots a_M \dots a_n b_{n+1}\dots$ where $b_{n+1} \neq a_{n+1}$.
Then $d(\vect{a},\vect{b}) \le \tfrac{1}{2 3^n}$ and
$d(\sigma_A^{n+1}(\vect{a}), \sigma_A^{n+1}(\vect{b})) \ge 1$.
\end{proof}

If $\left.f\right|_\Lambda$ is semi-conjugate to an irreducible,
non-trivial subshift $\left.\sigma_A\right|_{\Sigma_A}$, can we deduce
that $\left.f\right|_{\Lambda}$ is chaotic?

This is not true.  We need conjugacy to show that $\left.f\right|_\Lambda$
is chaotic, although we can show that if $\left.f\right|_\Lambda$ is
semi-conjugate to $\sigma_A$ then $\left.f\right|_\Lambda$ has at
least as many periodic cycles as $\sigma_A$.

\begin{proposition}\label{prop:poclosed}
For every closed path $a_0 a_1 \dots (a_k = a_0)$ in $\Gamma$ there
exists a periodic orbit for $f$ in $\Lambda$,
$(x_0\dots x_{k-1})^\infty$ such that $x_n \in I_{a_n}$ for all
$n \ge 0$.
\end{proposition}

We need a lemma before proving this.

\begin{lemma}\label{lem:fpfcov}
If the closed interval $K$ $f$-covers itself then $f$ has a fixed point in $K$.
\end{lemma}

\begin{proof}
Let $K = [a,b]$.  Then $K \to K$ implies that $\exists c,d \in K$
such that $f(c) = a \le c$ and $f(d) =b \ge d$.  Now apply the IVT.
\end{proof}

\begin{proof}[Proof of proposition \ref{prop:poclosed}]
From lemma \ref{lem:nint} $\exists I_{a_0 \dots a_k} \subset I_{a_0}$
such that

\begin{equation}\tag{$*$}\label{eq:unimp1}
f^n(I_{a_0 \dots a_k}) \subset I_{a_n} \qquad \text{and} \qquad
f^k(I_{a_0 \dots a_k}) = I_{a_0}.
\end{equation}

In particular $I_{a_0 \dots a_k} \to I_{a_0 \dots a_k}$ and
so (by lemma \ref{lem:fpfcov}), $f^k$ has a fixed point $x_0$
and by \eqref{eq:unimp1} $f^k(x_0) \in I_{a_n}$ for all $n \ge 0$.
\end{proof}

If the loop $a_0 a_1\dots (a_k = a_0)$ is of least period then
$x_0$ has least period $k$.

If the $I_n$'s are not disjoint but have disjoint interiors then
we can construct $\Gamma$ the same way.  We cannot deduce semiconjugacy
to $\sigma_A$, but we can still get lots of periodic orbits for
$\left. f\right|_\Lambda$ as proposition \ref{prop:poclosed} does not
use semiconjugacy.  The only problem is that multiple paths in $\Gamma$
give the same periodic orbit.

For instance, consider $z \mapsto z^2$ (as a map of $S^1$).

\vspace{1in}

This has a graph:
\[
\xymatrix{\ar @(ul,dl) 1 \ar@<1ex>[r] &
\ar@<1ex>[l] 2 \ar @(ur,dr)}
\]
and the only problem is that $0^\infty = 1^\infty$.

\section{Sharkovsky's theorem}

Our first proposition is the (famous?) ``period 3 implies chaos'' result of
Li and Yorke.

\begin{proposition}\label{prop:liandyorke}
Suppose $f \colon I \mapsto I$ is continuous and has a periodic point
of period 3.  Then $f$ has periodic points of all least periods.
\end{proposition}

\begin{proof}
Let the period 3 cycle be $x < y < z$.  Let $I_0 = [x,y]$ and
$I_1 = [y,z]$.  Suppose $f(y) = z$.  Then $f^2(y) = x$ and so
we have the graph
\[
\xymatrix{I_0 \ar@<1ex>[r] & \ar@<1ex>[l] I_1 \ar @(ur,dr)}
\]

(If $f(y) = x$ then just relabel $I_0$ and $I_1$.)  Now for all
$n \in \N$ there exists a loop
\[
I_0 \to \underbrace{I_1 \to \dots \to I_1}_{n-1} \to I_0,
\]

a least period loop of period $n$.  Proposition \ref{prop:poclosed}
gives a periodic point of exact period $n$.
\end{proof}

This is a special case of a more general result which describes in
a precise way the order in which periodic orbits of different periods
appear.  We first need to define the \emph{Sharkovsky ordering} on the
natural numbers.  This is given by

\begin{gather*}
1 \prec 2 \prec 4 \prec \dots \prec 2^n \prec 2^{n+1} \prec \dots \\
\dots \prec 2^{n+1} \cdot 9 \prec 2^{n+1} \cdot 7 \prec 2^{n+1} \cdot
5 \prec 2^{n+1} \cdot 3 \prec \dots \\
\dots \prec 2^n \cdot 9 \prec 2^n \cdot 7 \prec 2^n \cdot
5 \prec 2^n \cdot 3 \prec \dots \prec 9 \prec 7 \prec 5 \prec 3.
\end{gather*}

We can now state Sharkovsky's theorem.

\begin{theorem}[Sharkovsky's theorem]
Let $I \subset \R$ be a closed interval and $f \colon I \mapsto \R$ be
a continuous map.  If $f$ has a periodic point of least period $k$
then $f$ has a periodic orbit of least period $n$ for all $n \prec k$. 
\end{theorem}

Proof is in a number of stages.  We first prove the case $k > 1$, odd.

\begin{lemma}\label{lem:shar1}
If $f \colon I \mapsto \R$ is continuous and has a periodic point $x$
of least period $k \ge 3$, odd, and no points of odd period $n$ with
$1 < n < k$ then $f$ has a point of least period $n$ for all $n > k$
and all even $n < k$ and period 1.
\end{lemma}

\begin{proof}
Let $J = [\min \theta(x), \max \theta(x)]$.  Make a partition of $J$
by the element of $\theta(x) = \{ p_1 < p_2 < \dots < p_k \}$.  We
define intervals $I_i$ of the form $[p_l,p_{l+1}]$ for $1 \le i \le
k-1$, where $l$ is not necessarily in the same order as $i$.

We aim to show that we can choose the labelling of the $I_i$'s to
obtain the following directed graph (\emph{Stefan graph}) for the
$f$-covering relations.
\vspace{1cm}
\[
\xymatrix{
& \ar@(ul,ur)[] I_1 \ar[dr] & \\
 I_{k-1} \ar[ur] \ar[drr] \ar@<1ex>[d] \ar[]+<2ex,-2ex>;[dddr] & & I_2 \ar[d]  \\
I_{k-2} \ar[u] & & I_3 \ar[d] \\
\text{etc} \ar[u] & & I_4 \ar[dl] \\
& I_5 \ar[ul] & \\
}
\]

That is: a loop $I_1 \to I_2 \to \dots \to I_{k-1} \to I_1$, a loop $I_1
\to I_1$ and directed edges from $I_{k-1}$ to all odd vertices.  Once
this is established we just need to note that there are distinct
closed loops of period
\begin{itemize}
\item $n > k$ : $I_1^{n-k+1} \to I_2 \to \dots \to I_{k-1} \to I_1$
\item even numbers $< k$ : $I_{k-1} \to I_{2 l + 1} \to I_{2 l + 2}
\dots \to I_{k-2} \to I_{k-1}$
\item $1$ : $I_1 \to I_1$.
\end{itemize}

We will prove this in a series of claims.

\emph{Claim 1}: $I_1 \to I_1$.  Note that $f(p_1) > p_1$ and $f(p_k) <
p_k$.  Take $a= \max \{ y \in \theta(x) : f(y) > y\}$ ($a \neq p_k$).
Now let $I_1 = [a,b]$, where $b$ is the closest point of $\theta(x)$
to the right of $a$.  Then $f(a) \ge b$ and $f(b) \le a$ since $b >
a$.  Thus $f(I_1) \supset I_1$ as required.

\emph{Claim 2}: $f^{k-2}(I_1) \supset J$: i.e. there exists a path
from $I_1$ to any other vertex.  To prove this note that $f(I_1)
\supset I_1$ with proper inclusion (else $k=2$) and so $f^{j+1}(I_1)
\supset f^j(I_1)$ (nested iterates).  There are $k-2$ points in
$\theta(x) \setminus \{a,b\}$ and so $p_k \in f^j(I_1)$ for some $0
\le j \le k-2$ and by the nested property $P-k \in f^{k-2}(I_1)$.
Similarly $p_1 \in f^{k-2}(I_1)$ and since $I_1$ is connected
$f^{k-2}(I_1) \supset [p_1,p_k] = J$.

\emph{Claim 3}:  $\exists j \neq 1$ such that $I_1 \subset f(I_j)$
(that is $I_j \to I_1$).  To prove this let $B_l = \{y \in \theta(x)
: y \le a \}$ and $B_r = \{y \in \theta(x): y \ge b\}$.  Now $k$ is
odd, so that $\# B_l \neq \# B_r$.  Let $B$ be the one of $B_l$ and $B_r$
with more elements.  Then $\exists y_1, y_2 \in B$ adjacent with 
$f(y_1) \in B$ and $f(y_2) \in \theta(x) \setminus B$.  Take $I_j =
[y_1,y_2]$, and so $I_1 \subset f(I_j)$.

Now label the intervals such that $I_1 \to I_2 \to \dots \to I_l \to I_1$
is the shortest loop containing $I_1$.

\emph{Claim 4}: The shortest loop with $l \ge 2$ has $l=k-1$.  There
are only $k-1$ distinct intervals, so the shortest loop has $l \le
k-1$.  Assume that $l<k-1$.  Let $q$ be
the odd number out of $\{l, l+1\}$.  So $1 < q < k$. Use the loop
$I_1 \to I_2 \to \dots \to I_l \to I_1$ or $I_1 \to I_2 \to \dots \to
I_l \to I_1 \to I_1$ depending on whether $q=l$ or $q=l+1$.  By
proposition \ref{prop:poclosed}, $\exists y \in I_1$ such that $f^q(y)
= y$.  Now $y \notin \partial I_1$ since points on $\partial I_1$ have
period $k > q$. Hence $y$ has period $q < k$.  This gives a
contradiction: $k$ is the smallest odd number such that $f$ has a
periodic orbit of least period $k$.

\emph{Claim 5}:  \begin{enumerate}
\item If $f(I_i) \supset I_1$ then $i = 1$ or $k-1$.
\item For $j > i+1$, $I_i \not \to I_j$.
\item $I_1$ $f$-covers only $I_1$ and $I_2$.
\end{enumerate}
Claim 4 implies 1 and the shortest loop property implies 2 and 3.

\emph{Claim 6}:  orderings (in terms of $\R$) of the $I_i$'s and of
$\theta(x)$ are either
\[
\begin{cases}
I_{k-1} \le I_{k-3} \le \dots \le I_2 \le I_1 \le I_3 \le \dots \le
I_{k-2} \\
f^{k-1}(a) < f^{k-3}(a) < \dots < f^2(a) < a < f(a) < f^3(a) < \dots <
 f^{k-2}(a)
\end{cases}
\]
or the above exactly reversed.  To prove this note that $I_1=[a,b]$
$f$-covers only $I_1$ and $I_2$, so by connectedness $I_1$ and $I_2$
must be adjacent.  Assume $I_2 \le I_1$ (the other case gives the
reversed order).  Then we must have $f(a) = b$ and $f^2(b)$ the left
point of $I_2$.

Now $f(a) = b$ and $I_2 \not \to I_1$ (by claim 5.1) and so $f(I_2)
\ge a$.  Now $I_2 \to I_3$ but $I_2 \not \to I_j$ for $j > 3$ (by
claim 5.2) so that $I_3$ is adjacent to $I_1$.  We obtain the claimed
order inductively.

\vspace{1.5in}

\emph{Claim 7}: $I_{k-1} \to I_j$ for $j$ odd.  To prove this note
that $I_{k-1} = [f^{k-1}(a),f^{k-3}(a)]$ and $f(f^{k-1}(a)) = a$.
Also $f^{k-3}(a) \subset I_{k-3}$ so $f(f^{k-3}(a)) \subset I_{k-2}$.
Thus $f(I_{k-1}) \supset [a,f^{k-2}(a)] = I_1 \cup I_3 \cup \dots \cup
I_{k-2}$.

We have finally proved the Stefan graph and we can now complete the
proof of lemma \ref{lem:shar1} by using proposition
\ref{prop:poclosed} on each least period loop in the Stefan graph to
get a periodic point of the same least period.
\end{proof}

\begin{lemma}\label{lem:shar2}
If $f$ has a periodic point of least period $k = 2m$, $m \ge 1$ and no
periodic points of odd period greater than or equal to $3$ then $f$
has a fixed point and $f^2$ has two periodic orbits of least period
$m$.  These are $\{p_1,\dots,p_m\}$ and $\{p_{m+1},\dots,p_{2m}\}$.
\end{lemma}

\begin{proof}
Define $a$ and $b$ and set $I_1 = [a,b]$ as before.  Then $I_1 \to
I_1$ and there exists a fixed point in $I_1$.

In lemma \ref{lem:shar1} we used the fact that $k$ was odd only in
claim 3 to show that $\exists I_j$, $j \neq 1$ such that $I_j \to
I_1$.  If we have such a $I_j$ then we get the Stefan graph as before
(but with $k$ even) so there exists a loop of period $k-1$, odd.  This
is a contradiction.  Hence only $I_1$ $f$-covers $I_1$.

Since $f(a) \ge b$ at least one point must change side with respect to
$I_1$, so that all points in $\theta(x)$ must change sides: $f(B_l) =
B_r$ and $f(B_r) = B_l$.  Hence $f^2(B_l) = B_l$ and $f^2(B_r) = B_r$
--- that is $B_l$ and $B_r$ are permuted independently by $f^2$.
\end{proof}

\begin{proof}[Proof of Sharkovsky's theorem]\hfill\\
\begin{itemize}
\item If $k$ is odd then we are done by lemma \ref{lem:shar1}.
\item If $k=2r$, $r$ odd then $f$ has a period $1$ orbit and $f^2$ splits the
  orbit into two components by lemma \ref{lem:shar2}, each of period
  $r$.  Thus $f^2$ has a period $r$ orbit, $r$ odd.  Hence $f$ has
  a period $2m$ orbit for all $m \ge r$, a period $2p$ orbit for all
  even $p < r$, a period 2 orbit and a fixed point.
\item If $k=2^l r$, $r$ odd, $l > 1$ just repeat the argument.
\end{itemize}
\end{proof}

The converse of Sharkovsky's theorem is true: there are examples of
maps with \emph{exactly} the periodic orbits implied by the Sharkovsky
ordering.  For instance, consider the following function.

\vspace{1.5in}

As an exercise prove that this map has no period 3 point.

For all $n \ge 3$ there exists a permutation of $n$ elements such that
there exists a periodic orbit $\{ p_1 < p_2 < \dots < p_n \}$ of least
period $n$ which realises the permutation and forces the existence of
periodic points of all periods.

The Sharkovsky theorem only happens on the line in one dimension: not
on $\C$ or $S^1$.  Consider the map of $S^1$, $r_{\frac{1}{3}} \colon
x \mapsto x+ \tfrac{1}{3} \mod{1}$, which has all orbits of period $3$
and no orbits of other periods.

\section{The quadratic family}

Consider
\[
F_\mu(x) = \mu x (1-x) \text{ with } \mu > 1.
\]

Note that:

\begin{enumerate}
\item $F_\mu(0) = F_\mu(1) = 0$,
\item $F_\mu(P_\mu) = P_\mu$, where $P_\mu = \tfrac{\mu - 1}{\mu}$.
\item If $x < 0$ or $x > 1$ then $F^n(x) \to \infty$ as $n \to \infty$.
\end{enumerate}

\subsection{$1 < \mu < 3$}

\begin{proposition}
If $1 < \mu < 3$ then
\begin{enumerate}
\item $P_\mu$ is attracting and the origin repelling.
\item if $0 < x < 1$ then $\omega(x) = P_\mu$.
\end{enumerate}
\end{proposition}

\begin{proof}
To prove the first part just calculate the derivative $F_\mu'$.

When $1 < \mu \le 2$ then if $0 < x < P_\mu$, $F_\mu(x) > x$,
$F_\mu(x) < P_\mu$. If $P_\mu < x < 1$ then $0 < F_\mu(x) < P_\mu$.

\vspace{2in}

When $2 < \mu < 3$, $F^2_\mu([\tfrac{1}{2},P_\mu]) \subset
[\tfrac{1}{2},P_\mu]$ and $F^2_\mu$ is monotone on
$[\tfrac{1}{2},P_\mu]$.

Let $\Hat{P}_\mu$ be the other preimage of $P_\mu$.  Now
$F_\mu([\Hat{P}_\mu,\tfrac{1}{2}]) = [\tfrac{1}{2},P_\mu]$ and we can
apply the previous result.

If $x_0 \in [0,\Hat{P}_\mu]$, $F_\mu^j(x_0)$ is monotone so long as
the iterates stay in $[0,\Hat{P}_\mu]$.  When the iterates leave
$[0,\Hat{P}_\mu]$ then enter $[\Hat{P}_\mu,P_\mu]$.  Now apply the
previous results.

Finally, if $x_0 \in [P_\mu,1]$, $F_\mu(x_0) \in [0,\Hat{P}_\mu]$.
\end{proof}

\subsection{$\mu = 4$}

$g(x) = 2 x^2 - 1$ is topologically conjugate to $F_4 = 4 x(1-x)$.
Now $g$ is chaotic (see page \pageref{page:2x2m1}), so that
$F_4$ is chaotic.

\subsection{$\mu > 4$}

\vspace{1.5in}

$I_0$ and $I_1$ map on to $I = [0,1]$ so we have the graph
\[
\xymatrix{\ar @(ul,dl) 1 \ar@<1ex>[r] &
\ar@<1ex>[l] 2 \ar @(ur,dr)}
\]

\begin{proposition}\label{prop:fmu1}
If $\mu > 2 + \sqrt{5}$ then
\[
\Lambda_\mu = \{ x : F^n_\mu(x) \in I\ \forall n \ge 0 \}
\]
is a Cantor set and $\left. F_\mu \right|_{\Lambda_\mu}$ is
topologically conjugate to the shift map on $\Sigma_2$.
\end{proposition}

\begin{proof}
Exercise (note that $\mu > 2 + \sqrt{5}$ implies
$\abs{F_\mu'} > 1$).
\end{proof}

Proposition \ref{prop:fmu1} is in fact true for $\mu > 4$
(by negative Schwartzian derivative, not included in course.)

\subsection{$3 \le \mu \le 4$}

When $\mu$ increases through 3, $P_\mu$ becomes repelling and an
attracting period 2 cycle appears.  As $\mu$ is increased further
this period 2 cycle becomes unstable and a stable period 4 cycle appears.
There is a cascade of period-doubling bifurcations at
$\mu_0 < \mu_1 < \dots$, where the period $2^i$ cycle loses stability
at $\mu_i$ and a stable $2^{i+1}$ cycle appears.

As $n \to \infty$, $\mu_n \to \mu_\infty = 3.569942\dots$ and
\[
\lim_{n \to \infty} \frac{\mu_n - \mu_\infty}{\mu_{n+1} - \mu_\infty}
= 4.6692\dots, \quad \text{the Feigenbaum constant.}
\]

\subsection{Period doubling bifurcation}

\vspace{1.5in}

Let $f_\lambda \colon I \mapsto \R$ be a $C^r$ one-parameter
family with $r \ge 3$.

\begin{theorem}
Suppose that
\begin{enumerate}
\item $f_{\lambda_0}(x_0) = x_0$,
\item $f'_{\lambda_0}(x_0) = -1$,
\item $\left. \diff{\phantom{\lambda}}{\lambda} f'_\lambda(P(\lambda))
\right|_{\lambda_0} \neq 0$, or equivalently
\[
\alpha := \left( \pd{f'_\lambda}{\lambda} + \tfrac{1}{2}
\pd{f_\lambda''}{\lambda}\right) \neq 0.
\]
\item the graph of $f^2_{\lambda_0}$ has a non-zero cubic
term in its tangency with the diagonal, or equivalently
\[
f^2_{\lambda_0}(x) - x = - \beta (x-x_0)^3 + \cO(x-x_0)^4,
\]
where
\[
\beta := \tfrac{1}{3} f'''_{\lambda_0}(x_0)
+ \tfrac{1}{2} \left( f''_{\lambda_0}(x_0)\right)^2 \neq 0.
\]
\end{enumerate}

Then there exists a period-doubling bifurcation at $\lambda_0$, that is

\begin{itemize}
\item there exists a differentiable curve of fixed points $P(\lambda)$
of $f_\lambda$ passing through $(x_0,\lambda_0)$, the stability of
which changes at $\lambda_0$, and
\item there exists a differentiable curve $\gamma$
which passes through $(x_0,\lambda_0)$ such that
$\gamma \setminus (x_0,\lambda_0)$ is the union of two hyperbolic
period 2 orbits and $\gamma$ is the graph of a function
$\lambda = h(x)$, $h'(x_0) = 0$ and $h''(x_0) = - \tfrac{\beta}{\alpha}
\neq 0$.
\end{itemize}

If $\beta > 0$ then this period two cycle is stable and if
$\beta < 0$ this period two cycle is unstable.
\end{theorem}

\begin{proof}
$f'_{\lambda_0}(x_0) \neq 1$, so that there exists $P(\lambda)$ such
that
\[
P'(\lambda_0) = -
\frac{\pd{f_\lambda}{\lambda}(p(\lambda))}{f'_\lambda(p(\lambda)) - 1}
= \tfrac{1}{2} \left.\pd{f}{\lambda}\right|_{(x_0,\lambda_0)}.
\]

We change co-ordinates such that the origin is a fixed point
for all $\lambda$ near $\lambda_0$. Let
\[
g(y,\lambda) = f(y+P(\lambda)) - P(\lambda),
\]

so that $\pd{^n g}{y^n} (0,\lambda) = \left.f_{\lambda}^{(n)}
\right|_{P(\lambda)}$ and
\[
\left.\pd{^2 g}{\lambda \partial y}\right|_{(0,\lambda_0)}
= \left.\diff{\phantom{\lambda}}{\lambda} f'_\lambda(P(\lambda))
\right|_{\lambda_0}
= \alpha \neq 0.
\]

Let $G(y,\lambda) =g^2(y,\lambda) - y$ so that when $G(y,\lambda) =
0$, $y$ is either a fixed point or a period two point.
Note that $\left.\pd{G}{\lambda}\right|_{(0,\lambda_0)} = 0$ and so
we cannot naively apply the IFT directly to $G$.

Now if $G(y,\lambda) = 0$ and $y \neq 0$ then $y$ is a period two point.
Define
\[
H(y,\lambda) = \begin{cases}
\frac{G(y,\lambda)}{y} & y \neq 0 \\
\left.\pd{G}{y}\right|_{(0,\lambda)} & y = 0.
\end{cases}
\]

(Exercise: check $H$ is $C^1$)  We want to apply the IFT to $H$, so
we verify the conditions of the IFT:

\begin{itemize}
\item $H(0,\lambda_0) = \left.\pd{g^2}{y}\right|_{(0,\lambda_0)}
- 1 = \left(f'_{\lambda_0}(x_0)\right)^2 -1 = 0$.
\item $\left.\pd{H}{\lambda}\right|_{(0,\lambda_0)}
= \left(\pd{}{\lambda} \left( g'(0,\lambda) \right)^2
\right)_{\lambda=\lambda_0}
= 2 g'(0,\lambda_0) \left.\pd{}{\lambda} g'(0,\lambda)
\right|_{\lambda = \lambda_0} = -2 \alpha \neq 0.$
\end{itemize}

Thus we can use the IFT and so there exists a differentiable function
$h(y)$ with $H(y,h(y)) = 0$.  Now
\[
h'(0) = - \left.\frac{\pd{H}{y}}{\pd{H}{\lambda}}\right|_{(0,\lambda_0)} = 0
\quad \text{and} \quad
h''(0) = - \left.\frac{\pd{^2 H}{y^2}}{\pd{H}{y}}\right|_{(0,\lambda_0)}
 = - \frac{\beta}{\alpha} \neq 0.
\]

To prove the stability result we Taylor expand $\pd{g^2(y,h(y))}{y}$
about $y=0$.  This gives
$\pd{g^2(y,h(y))}{y} = 1 - 2 \beta y^2 + \cO(y^3)$ and so this
orbit is attracting if $\beta > 0$ and repelling if $\beta < 0$.
\end{proof}

We have the following pictures.

\vspace{2in}

\section{Accumulation of period-doublings}

Suppose we have a continuous one parameter family of
unimodal maps $f_\mu \colon I \mapsto I$, $\mu \in [\mu_0,\mu_1]$
with maximum at $c_\mu$ and

\begin{gather*}
f^2(c) \le c < f(c) \\
f^2(c) \le f^3(c)
\end{gather*}

for all $\mu \in [\mu_0,\mu_1]$, with $f^2_{\mu_0}(c) = c$.
$c$ is called a \emph{superstable} period two point.  We also
want $f^3_{\mu_1}(c) = f^2_{\mu_1}(c)$ and so $c$ is eventually fixed
(but unstable).

\vspace{1.5in}

Then there exists an invariant interval $J = [ f^2(c),f(c)] \subset I$
and every orbit starting outside $J$ is either asymptotic to a fixed
point or eventually enters $J$.  

Now the set of periods of $f_{\mu_0}$ is $\{ 1,2\}$ and
the set of periods of $f_{\mu_1}$ is $\N$.

As $\mu$ goes from $\mu_0$ to $\mu_1$ we get a subinterval
$[\Tilde{\mu}_0,\Tilde{\mu}_1]$ for which
\[
f^2(c) < c \le f^4(c) < f^3(c) \le f^5(c) < f(c)
\]

for all $\mu \in [\Tilde{\mu}_0,\Tilde{\mu}_1]$, with
$f^4_{\Tilde{\mu}_0}(c) = c$ ($c$ superstable period 4)
and $f^5_{\Tilde{\mu}_1}(c) = f^3_{\Tilde{\mu}_1}(c)$ ($c$ is
eventually period 2).

\vspace{2in}

Let $J_1 = [f^2(c),f^4(c)]$ and $J_0 = [f^3(c),f(c)]$.  Then
there exists a fixed point in $[f^4(c),f^3(c)]$ and everything
in $[f^4(c),f^3(c)]$ is either asymptotic to a fixed point
or eventually enters $J_0$.  We also have
$f(J_0) = J_1$ and $f(J_1) \subset J_0$ ($f$ exchanges the
intervals $J_0$ and $J_1$) so that $f^2(J_0) \subset J_0$.

Thus the set of periods of $f_\mu$ for $\mu \in
[\Tilde{\mu}_0,\Tilde{\mu}_1]$ is $\{ 1 \} \cup 2 \N$.
Consider $\Tilde{f} = \left.f^2\right|_{J_0}$ for
$\mu \in [\Tilde{\mu}_0,\Tilde{\mu}_1]$.  It is unimodal
and satisfies our original conditions.

Repeat this process to get a nested sequence of subintervals
$[\mu_0^{(i)}, \mu_1^{(i)}]$.

\vspace{1.5in}

Denote the infinite intersection of all these subintervals by $L$.
For $\mu \in L$ the set of periods is the set of all of the powers of $2$.

Denote the interval obtained at the $N^{\text{th}}$ stage of this
process by $J_{0 \dots 0}$ ($N$ zeroes), so that
\[
K_N = \bigcup_{n=0}^{2^N - 1} f^n(J_{0\dots0}) 
\]
is the union of $2^N$ disjoint intervals which are cyclically permuted by
$f$.

\vspace{2in}

Now $K_\infty = \bigcap_N K_N$ is an invariant subset
of $f_\mu$ and is a Cantor set if the maximum lengths of the subintervals
tends to zero as $N \to \infty$.%
\footnote{This can be proved for the quadratic family
using negative Schwartzian derivative.}

\subsection{Dynamics on $K_\infty$}

Label the images of $J_{0\dots0}$ at level $N$ by
\[
f^n(J_{0\dots0}) = J_{\text{base 2 representation of $n$ mod $2^N$
written backwards}}.
\]

For all $a_0, a_1, \dots, a_m$ we have
$J_{a_0, a_1, \dots, a_m} \subset J_{a_0, a_1, \dots, a_{m-1}}$.

Define $h \colon K_\infty \mapsto \Sigma_2$ by
\[
h \colon \bigcap_{m > 0} J_{a_0,\dots,a_m}
\mapsto \vect{a},
\]

so that $h \circ f = A \circ h$, where
$A \colon \Sigma_2 \mapsto \Sigma_2$ is the \emph{adding machine}
defined by
\[
A(a_0 a_1 a_2\dots) = (a_0 a_1 a_2 \dots )
+ (1 0 0 \dots )
\] 

with carrying to the right.  $h$ is a conjugacy if $K_\infty$
is a Cantor set, else $h$ is a semiconjugacy.

\chapter{Two dimensional invertible maps}

The main difference between one dimensional maps and higher dimensional
maps is the possibility of both expansion and contraction at the same
points in higher dimensions.

\section{The horseshoe map}

Consider a map $f \colon \R^2 \mapsto \R^2$ with the following geometric
properties.

\vspace{2in}

$f(S) \cap S = V_0 \cup V_1$.  $S$ is not mapped into itself so we
extend the mapping outside $S$ by considering a stadium $D = S \cup
E_1 \cup E_2$.  We take $f$ to be a contraction on $E_1 \cup E_2$ such
that $f(D) \subset D$.  $f$ is $1-1$ but not onto, so that $f^{-1}$ is
not globally defined.

\subsection{Dynamics on $S$}

If $x \in E_1$ then $f(x) \in E_1$.  Thus by the contraction mapping
principle there exists a unique attracting fixed point $p \in E_1$.
If $x \in E_2$ then $f(x) \in E_1$ and so $f^n(x) \to p$.  For any $x
\in E_1 \cup E_2$, $\omega(x) = p$.

What are the sets remaining in $S$ for all time?  The points of $S$
which are mapped into $S$ are in $f^{-1}(S \cap f(s)) = S \cap
f^{-1}(S) = H_0 \cup H_1$ (the union of two horizontal strips).

\vspace{1.5in}

We want $\left. f\right|_{H_0 \cup H_1}$ to be affine with
derivative
\[
\begin{pmatrix}
\pm \lambda & 0 \\ 0 & \pm \mu
\end{pmatrix} \quad
\text{with a $+$ sign on $H_0$ and a $-$ on $H_1$.}
\]

$f^{-2}(S \cap f(s) \cap f^2(S) )$ is four horizontal strips
$H_{ij}$, $i,j \in \{0,1\}$.

\vspace{2in}

In general
\[
f^{-n}(S \cap f(S) \cap \dots \cap f^n(S)) = \bigcap_{-n \le j \le 0}
f^j(S)
\]

is $2^n$ strips of thickness $\mu^{-n}$.  Define
\[
\Lambda_H = \left\{ x \in S : f^n(x) \in S\ \forall n \ge 0 \right\}
= \bigcap_{j=-\infty}^0 f^j(S)
\]

is a Cantor set of horizontal lines.  Similarly,
$\bigcap_{0 \le j \le n} f^j(S)$ consists of $2^n$ vertical strips
of thickness $\lambda^n$ and
\[
\Lambda_V = \bigcap_{j=0}^\infty f^j(S)
\]

is a Cantor set of vertical lines.  The set of points
which remain in $S$ for all positive and negative times is
\[
\Lambda = \bigcap_{n=-\infty}^\infty f^n(S)
\]

is a Cantor set of points.

Introduce $\Sigma_2 = \{ 0,1\}^\Z$, the set of
\emph{doubly} infinite sequences of $0$'s and $1$'s with metric
\[
d(\a,\b) = \sum_{i \in \Z} \frac{\gamma_i}{4^{\abs{i}}}
\]

with

\[
\gamma_i = \begin{cases}
0 & a_i = b_i \\
1 & a_i \neq b_i.
\end{cases}
\]

Two sequences are close if they agree on a long central block.  We define
the intinerary map $h \colon \Lambda \mapsto \Sigma_2$ by
\[
h(x) = \left( a_i \right)_{i \in \Z} \text{ if }
f^i(x) \in H_{a_i}\ \forall i.
\]

\vspace{2in}

\begin{theorem}
$\left. f\right|_\Lambda$ is topologically conjugate to the
two-sided shift $\left. \sigma \right|_{\Sigma_2}$ by $h$.
\end{theorem}

\begin{proof}
Exercise.
\end{proof}

The hypotheses we took are not the strongest ones possible:

If there exist disjoint horizontal strips $H_i$ such that
$f(H_i) = V_i$ for $i = 1,\dots,N$, $f$ contracts vertical strips
and $f^{-1}$ contracts horizontal strips uniformly then
$f$ possesses an invariant set $\Lambda$ such that
$\left. f\right|_\Lambda$ is topologically conjugate to
$\left. \sigma \right|_{\Sigma_N}$.

It is possible to have geometric horseshoes that are conjugate to
two-sided subshifts of finite type.

\vspace{2in}

\[
\Lambda = \bigcap_{n=-\infty}^\infty f^n(S_1 \cup S_2) \quad
\left. f\right|_\Lambda \sim \left. \sigma \right|_{\Sigma_A} \quad
A = \begin{pmatrix}
1 & 1 \\ 1 & 0
\end{pmatrix}.
\]

\section{The H\'enon map}

This is an example of a nonlinear map with a horseshoe.%
\footnote{A doubly invariant set on which the dynamics are
conjugate to $\sigma$.}

\[
f_{a,b} \colon \begin{pmatrix} x \\ y \end{pmatrix} \mapsto
\begin{pmatrix}
a - b y - x^2 \\
x
\end{pmatrix}.
\]

\begin{theorem}
If $b \neq 0$ and
\begin{align*}
a &\ge \left( 5 + 2 \sqrt{5} \right) \frac{ 1 + \abs{b}^2}{4},\\
R &= \frac{1 + \abs{b} + \sqrt{(1+\abs{b})^2 + 4 a}}{2}, \\
S &= \left\{ (x,y) \in \R^2 : \abs{x}, \abs{y} \le R \right\} \text{ and} \\
\Lambda &= \bigcap_{j \in \Z} f^j(S),
\end{align*}

then $\left.f_{a,b}\right|_\Lambda \sim \left. \sigma \right|_{\Sigma_2}$.
\end{theorem}

\begin{proof} Omitted (very technical).
\end{proof}

This theorem can be made plausible by considering the case when
$a=5$, $b=0.3$ and letting $S = \{ (x,y) : \abs{x}, \abs{y} \le 4\}$.

\vspace{2in}

\section{(Un)stable manifolds and homoclinic points}

In this section we suppose that $f$ is at least $C^1$.

\begin{definition}
  A fixed point $p$ for $f \colon \R^2 \mapsto \R^2$ is
  \emph{hyperbolic} if the Jacobian matrix $Df(p)$ has no eigenvalues
  on the unit circle.

  A periodic point $p$ of least period $n$ is hyperbolic if $Df^n(p)$
  has no eigenvalues on the unit circle.
\end{definition}

There are three types of hyperbolic fixed points.

\begin{itemize}
\item $p$ is a \emph{sink} (or \emph{attracting}) if all the
  eigenvalues of $Df(p)$ lie inside the unit circle.
\item $p$ is a \emph{source} (or \emph{repelling}) if all the
  eigenvalues of $Df(p)$ lie outside the unit circle.
\item $p$ is a saddle otherwise.
\end{itemize}

The following theorem is not proven (but reasonably obvious).

\begin{theorem}
Suppose $f$ has an attracting (repelling) fixed point $p$.  Then there
exists an open set about $p$ in which all points tend to $p$ under
forward (backward) iteration of $f$.
\end{theorem}

The largest such open set is called the \emph{stable set} / \emph{basin of
attraction} (\emph{unstable set} / \emph{basin of repulsion}) 
and is written $W^S(p)$ ($W^U(p)$).

\subsection{Saddle points}

\begin{theorem}{Stable/unstable manifold theorem}
Suppose that $f$ has a saddle point $P$.  Then $\exists \epsilon > 0$
and a $C^1$ curve $\gamma_U \colon (-\epsilon,\epsilon) \mapsto \R^2$
such that:

\begin{enumerate}
\item $\gamma_U(0) = p$,
\item $\gamma_U' \neq 0$,
\item $\gamma_U'(0)$ is an unstable eigenvector of $Df(p)$,
\item $\gamma_U$ is invariant under $f^{-1}$,
\item $f^{-n}(\gamma_U) \to p$ as $n \to \infty$ and
\item if $\abs{f^{-n}(Q) - p} < \epsilon$ for all $n \ge 0$ then
$Q = \gamma_U(t)$ for some $t \in (-\epsilon,\epsilon)$.
\end{enumerate}

$\gamma_U$ is called the local unstable manifold.

The \emph{stable} manifold theorem is the image of the above
theorem under
\[
U \mapsto S \qquad f^{-1} \mapsto f \qquad \text{unstable} \mapsto
\text{stable}.
\]
\end{theorem}

The local stable/unstable manifolds have global counterparts.

\begin{definition}
The unstable manifold of $p$ is
\[
W^U(p) = \bigcup_{n \ge 0} f^n(\gamma_U)
\]

and the stable manifold of $p$ is
\[
W^S(p) = \bigcup_{n \ge 0} f^{-n}(\gamma_S). 
\]
\end{definition}

$W^S(p)$ and $W^U(p)$ frequently cross each other.

\vspace{2in}

A point $q \in W^S(p) \cap W^U(p) \setminus \{ p \}$ is called a
\emph{homoclinic} point to $p$.  $q$ is a transverse homoclinic point
if $W^S(p)$ and $W^U(p)$ intersect transversely.

Since $W^S(p)$ and $W^U(p)$ are invariant then the points
\[
O(q) = \{ f^n(q) : n \in \Z \}
\]

are homoclinic.  $O(q)$ is a \emph{homoclinic orbit} with
$\omega(O(q)) = \alpha(O(q)) = p$.

This leads to very complicated behaviour for the stable and unstable
manifolds.

\vspace{1.5in}

\section{Transverse homoclinics imply chaos}

The next theorem (which we do not prove) states that the existence of
a transverse homoclinic point implies the existence of a horseshoe.

\begin{theorem}[Smale-Birkhoff]\label{thm:smale}
Let $f \colon \R^2 \mapsto \R^2$ be a diffeomorphism with a hyperbolic
fixed point $p$ and suppose that there exists a transverse homoclinic
point $q$ to $p$.  Then $\exists n > 0$ such that $f^n$ has an
invariant set $\Lambda$ on which $f^n$ is conjugate to the two-sided
shift on $\Sigma_2$.
\end{theorem}

The idea of the proof is to find a picture which looks like Smale's
horseshoe for some iterate of $f$.

\vspace{2in}

Take a square $U \ni p$.  Since $f^i(q) \to p$ as $i \to \infty$,
$q \in f^l(p)$ for some $l > 0$.  Similarly $q \in f^{-k}(U)$ for
some $k > 0$.  Let $n = l+k$ so that $f^n$ maps $f^{-k}(U)$ to $f^l(U)$.

\section{The Melnikov method}

We have seen that a horseshoe arises from a transverse homoclinic
point to a hyperbolic periodic point.  We will now obtain a method to
verify that certain classes of ODE's have transverse homoclinic
points.

Consider a Hamiltonian system and add a periodic perturbation.

\begin{equation}\label{eq:hamnon}
\Dot{z} = X_H(z) + \epsilon Y(z,t), \quad z = \begin{pmatrix} x \\ y
\end{pmatrix}, \quad X_H = \begin{pmatrix} \pd{H}{y} \\ - \pd{H}{x}
\end{pmatrix}, \quad Y(z,t+T) = Y(z,t), 
\end{equation}

where $x \in \R \text{ or } S^1$ and $y \in \R$.  Assume that
$\Dot{z} = X_H(z)$ has a saddle equilibrium $P_0$ and that
$\Dot{z} = X_H(z)$ has a homoclinic orbit $\Gamma$ for $p$ given
by $z_0(t)$.

\vspace{1.5in}

We can express \eqref{eq:hamnon} as an autonomous system on
$\R^2 \times S^1_T$:

\begin{equation}\label{eq:hamaut}
\begin{cases}
\Dot{z} = X_H(z) + \epsilon Y(z,t) \\
\Dot{t} = 1.
\end{cases}
\end{equation}

We can also consider the stroboscopic (or Poincar\'e) map
$Q \colon \Sigma \mapsto \Sigma$ given by $z(t) \mapsto z(t+T)$,
where $\Sigma$ is a cross-section.

\subsubsection*{$\epsilon = 0$}

\vspace{2in}

Choose a normal $\Hat{\vect n}$ to $\Gamma$ for \eqref{eq:hamnon}
and let $\pi$ be the corresponding plane for \eqref{eq:hamaut}.  The
unperturbed problem has a one parameter family of homoclinic orbits
$z_\tau(t) = z_0(t+\tau)$: we choose the origin of time for $z_0$ at
its intersection with the plane $\pi$.  We want to find out which
homoclinics survive perturbation --- if any?

\subsubsection*{$\epsilon$ small}

By the IFT a closed orbit $\gamma_\epsilon$ persists in $\R^2 \times
S^1_T$ which is close to $\gamma_0$.  This corresponds to a hyperbolic
saddle for $Q$.

$\gamma_\epsilon$ is hyperbolic and the surfaces
$W^S(\gamma_\epsilon)$ and $W^U(\gamma_\epsilon)$ vary smoothly with
$\epsilon$, so the surfaces $W^S(\gamma_\epsilon)$ and
$W^U(\gamma_\epsilon)$ cross $\pi$ transversely.

\vspace{2in}

We want to measure the separation between $W^U$ and $W^S$ and
see if it changes sign as $\tau$ varies.  $H$ is a good thing to use
to measure this.

\begin{align*}
\Delta H(\tau,\epsilon) &= H(z^U_\epsilon(\tau)) -
H(z^S_\epsilon(\tau)) \\
=& \int_{\tau - n T}^\tau \diff{}{s} \left(z^U_\epsilon(s)\right)\, \ud s
+ H( z^U_\epsilon(\tau - n T)) \\
&+ \int^{\tau + n T}_\tau \diff{}{s} \left(z^S_\epsilon(s)\right)\, \ud s
- H( z^S_\epsilon(\tau + n T)).
\end{align*}

As $n \to \infty$, $z^U_\epsilon(\tau - nT)$ and $z^S_\epsilon(\tau +
n T)$ both tend to $\gamma_\epsilon(\tau)$.  So
\[
\Delta H(\tau,\epsilon) = \lim_{n \to \infty}
\int_{\tau - n T}^{\tau + n T} \diff{}{s} \left( H
  ( \Tilde{z}_\epsilon(s)) \right)\, \ud s, \qquad
\Tilde{z}_\epsilon = \begin{cases}
z^U_\epsilon & s < \tau \\
z^S_\epsilon & s > \tau.
\end{cases}
\]

Now
\[
\diff{}{s} H(\Tilde{z}_\epsilon(s)) = DH\cdot (X_H + \epsilon Y)
(\Tilde{z}_\epsilon(s),s) = \epsilon DH \cdot Y (\Tilde{z}_\epsilon(s),s),
\]

and $\epsilon \to 0$, $\Tilde{z}_\epsilon(s) \to z_0(s+\tau)$.  We
write $\Delta H = \epsilon G(\tau,\epsilon)$ and define the
\emph{Melnikov} function: $M(\tau) = G(\tau,0)$.  Then
\[
\Delta H = \epsilon M(\tau) + \cO(\epsilon^2).
\]

\begin{theorem}
  If $M$ has a zero at $\tau = \tau_0$ and
  $\left. \pd{M}{\tau} \right|_{\tau_0} \neq 0$ then $\gamma_\epsilon$ has
    a transverse homoclinic orbit which is near $z_0(\tau_0)$.
\end{theorem}

\begin{proof}
  We know $\Delta H(\tau,\epsilon) = \epsilon G(\tau,\epsilon)$ and
  that $M(\tau_0) = G(\tau_0,0)$ and
  $\left.\pd{G}{\tau}\right|_{\tau_0} \neq 0$.  We can therefore apply
  the IFT to find a curve $\tau(\epsilon)$ such that
  $G(\tau(\epsilon),\epsilon) = 0$ (for small $\epsilon$).
\end{proof}

In this case we have a horseshoe (by the Smale-Birkhoff theorem
(\ref{thm:smale})) and so chaos.

\subsubsection*{Concluding remarks}

We have found sets (horseshoes) on which the dynamics are chaotic.
However the horseshoe is repelling.  Proving the existence of chaotic
attractors is one of the major challenges of Dynamical Systems.

\backmatter

\begin{thebibliography}{9}
\bibitem{DDEnotes} \emph{Dynamics of Differential Equations},
  unpublished, 1996.
  
  {\sffamily \small Not useful as a reference but a fairly fun related
  course.  The course style has changed since I typed these. }

\bibitem{Glendinning} P.A.~Glendinning, \emph{Stability, Instability
    and Chaos}, CUP, 1994.
  
  {\sffamily \small An easier book than the others I've mentioned
 here.  It is not that useful for discrete time dynamical systems,
 although the bit it does have is explained well. }

\bibitem{GH} Guckenheimer \& Holmes, \emph{Nonlinear Oscillations,
    Dynamical Systems and Bifurcations of Vector Fields}, Springer,
  1983.
  
  {\sffamily \small Worth the read --- I like it but it's probably not
    that good a book for this course. }

\bibitem{Robinson} Clark Robinson, \emph{Dynamical Systems}, CRC
  Press, 1994.
  
  { \sffamily \small A good book for this course, but with some of the
  worst \TeX\ layout I've ever seen.  Still, it is an excellent book.}
\end{thebibliography}

Contrary to most of maths this area appears to be very well served
with excellent readable textbooks.  That means these notes are
probably futile.  Oh well.

\end{document}
